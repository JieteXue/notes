\documentclass{article}
\newcommand{\mydate}{September 25, 2025}
\newcommand{\mytitle}{GP1 HW3}
\title{\textbf{\mytitle}}
\author{Jiete XUE}
\date{\mydate}
\usepackage{fancyhdr}
\pagestyle{fancy}
\fancyhf{}
\fancyhead[C]{\mytitle }
\fancyhead[R]{Jiete Xue}
\fancyhead[L]{\mydate}
\fancyfoot[C]{\thepage}
\usepackage{amsthm}
\usepackage{amsmath}
\usepackage{amssymb}
\usepackage{physics}
\usepackage{tikz}

%% 右矢
%\ket{\psi}          % 输出:|ψ⟩
%\ket{\psi(t)}       % 输出:|ψ(t)⟩
%
%% 左矢
%\bra{\phi}          % 输出:⟨φ|
%
%% 期望值
%\expval{\hat{A}}    % 输出:⟨Â⟩
%\expval{\hat{A}}{\psi}  % 输出:⟨ψ|Â|ψ⟩
%
%% 对易子
%\comm{\hat{A}}{\hat{B}}  % 输出:[Â, B̂]
\newtheoremstyle{1}{}{}{}{}{\bfseries}{}{\newline}{}
\theoremstyle{1}
\newtheorem{problem}{Problem}
\usepackage{chngcntr}
\counterwithin{equation}{problem}
\newcommand{\pa}{\partial}
\newcommand{\rn}[1]{\romannumeral #1\relax}
\newcommand{\Rn}[1]{\expandafter\@slowromancap\romannumeral#1@}

\begin{document}
\maketitle
\begin{problem}[Four springs harmonic oscillation]
    Four fixed points are:
    $$\vec{r}_1=\left(l_0,0\right),\ \vec{r}_2=\left(-l_0,0\right),\ \vec{r}_3=\left(0,l_0\right),\ \vec{r}_4=\left(0,-l_0\right).$$
    The potential energy of the system is:
    \begin{equation}
        V=\sum_{i=1}^{4}\frac{1}{2}k\left[\left(\vec{r}-\vec{r}_i\right)\cdot\left(\vec{r}-\vec{r}_i\right)-l_0^2\right]=2kr^2.
    \end{equation}
    \begin{equation}
        m\ddot{\vec{r}}=-\nabla V=-4k\vec{r}.
    \end{equation}
    So effective force constant is 
    \begin{equation}
        \boxed{k'=4k.}
    \end{equation}
\end{problem}
\begin{problem}[Driven harmonic oscillation]
    Let the mass of the spar buoy be $m$ and the distance moving away from the initial point be $x$, the height of the wave be
    \begin{equation}
        y=h\sin\left(\frac{2\pi t}{T} \right).
    \end{equation}
    Then the extra force after wave come leads to 
    \begin{equation}
        m\ddot{x}=k\left(y-x\right).
    \end{equation}
    where $k$ is a constant that satisfying:
    \begin{equation}
        mg=kL.
    \end{equation}
    Then, we can deduce that 
    \begin{equation}
        \ddot{x}+\frac{g}{L}x=\frac{g}{L}h\sin\left(\frac{2\pi t}{T}\right).
    \end{equation}
    \begin{equation}
        x=\frac{h}{1-\frac{L}{g}\left(\frac{2\pi}{T}\right)^2}\sin\left(\frac{2\pi t}{T}\right).
    \end{equation}
\end{problem}
\begin{problem}[Damped harmonic oscillation I]
    Let $\omega_0=\sqrt{\frac{k}{m}}$.
    \begin{enumerate}
        \item \begin{enumerate}
            \item  For $t<0$, $x(t)=0$. For $t>0$:
            \begin{equation}
                m\ddot{x}=-kx-m\gamma\dot{x}+F_0.
            \end{equation}
            The solution is 
            \begin{equation}
                x(t)=Ae^{\lambda_1t}+Be^{\lambda_2t}+\frac{F_0}{k}.
            \end{equation}
            where
            \begin{equation}
                \lambda_1=-\frac{\gamma}{2}+\sqrt{\frac{\gamma^2}{4}-\omega_0^2}=0,\ \lambda_2=-\frac{\gamma}{2}-\sqrt{\frac{\gamma^2}{4}-\omega_0^2}.
            \end{equation}
            The initial conditions are:
            \begin{equation}
                x(0)=0,\ \dot{x}(0)=0.
            \end{equation}
            Thus,
            \begin{equation}
                A=\frac{F_0}{k}\frac{-\lambda_2}{\lambda_2-\lambda_1},\ B=\frac{F_0}{k}\frac{\lambda_1}{\lambda_2-\lambda_1}.
            \end{equation}
            \begin{equation}
                \boxed{x(t)=\frac{F_0}{k}\left(\frac{-\lambda_2}{\lambda_2-\lambda_1}e^{\lambda_1t}+\frac{\lambda_1}{\lambda_2-\lambda_1}e^{\lambda_2t}+1\right)}
            \end{equation}
            \item \begin{equation}
               m\ddot{x}=-kx-m\gamma\dot{x}+F_0\cos\left(\omega_0 t\right). 
            \end{equation}
            \begin{equation}
                x(t)=\frac{F_0}{m\omega_0\gamma}\sin\left(\omega_0 t\right)+\frac{F_0}{m\gamma\left(\lambda_2-\lambda_1\right)}\left(e^{\lambda_1t}-e^{\lambda_2t}\right).
            \end{equation}
        \end{enumerate}
        \item The amplitude of the oscillation is
        \begin{equation}
        A=\frac{\frac{F_0}{m}}{\sqrt{\left(\omega^2-\omega_0^2\right)^2+\left(\gamma\omega\right)^2}}.
        \end{equation}
        When 
        \begin{equation}
            \boxed{\omega^2=\omega_0^2-\frac{\gamma^2}{2}},\ \text{if it's positive.} 
        \end{equation}
        The amplitude get maximum. If $0>\omega_0^2-\frac{\gamma^2}{2}$, there does not exist a maximum.
    \end{enumerate}
\end{problem}
\begin{problem}[Damped oscillation II]
    We have known that after a long time, the motion of the oscillation is 
    \begin{equation}
        x(t)=A\cos\left(\omega t+\phi\right).
    \end{equation}
    where 
    \begin{equation}
        A=\frac{\frac{F_0}{m}}{\sqrt{\left(\omega^2-\omega_0^2\right)^2+\left(\beta\omega\right)^2}},
    \end{equation}
    and $\phi$ satisfying
    \begin{equation}
        \tan\left(\phi\right)=\frac{\beta\omega}{\omega^2-\omega_0^2}.
    \end{equation}
    \begin{enumerate}
        \item \begin{equation}
            P(t)=F_0\cos\left(\omega t \right)\dot{x}=-F_0A\omega\cos\left(\omega t\right)\sin\left(\omega t+\phi\right).
        \end{equation}
        The average rate is
        \begin{equation}
            \expval{P}=\int_{0}^{T}P(t)\dd{t}/T=-\frac{1}{2}F_0A\omega\sin\left(\phi\right)=\frac{1}{2}mA^2\omega^2\beta.
        \end{equation}
        \item \begin{equation}
            \int_{0}^{T}\beta\dot{x}\dot{x}\dd{t}/T=\frac{1}{2}mA^2\omega^2\beta.
        \end{equation}
        \item \begin{equation}
            \expval{P}=\frac{F_0^2\beta}{2m}\frac{1}{\omega^2+\frac{\omega_0^4}{\omega^2}+\beta^2-2\omega_0^2}
        \end{equation}
        iff. $\omega=\omega_0$, $\expval{P}$ get the maximum:
        \begin{equation}
            \boxed{\expval{P}_{\mathrm{max}}=\frac{F_0^2}{2m\beta}.}
        \end{equation}
    \end{enumerate}
\end{problem}
\end{document}