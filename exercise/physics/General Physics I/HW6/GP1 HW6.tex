\documentclass{article}
\newcommand{\mydate}{October 30, 2025}
\newcommand{\mytitle}{GP1 HW6}
\title{\textbf{\mytitle}}
\author{Jiete XUE}
\date{\mydate}
\usepackage{fancyhdr}
\pagestyle{fancy}
\fancyhf{}
\fancyhead[C]{\mytitle }
\fancyhead[R]{Jiete Xue}
\fancyhead[L]{\mydate}
\fancyfoot[C]{\thepage}
\usepackage{amsthm}
\usepackage{amsmath}
\usepackage{amssymb}
\usepackage{physics}
\usepackage{tikz}

%% 右矢
%\ket{\psi}          % 输出:|ψ⟩
%\ket{\psi(t)}       % 输出:|ψ(t)⟩
%
%% 左矢
%\bra{\phi}          % 输出:⟨φ|
%
%% 期望值
%\expval{\hat{A}}    % 输出:⟨Â⟩
%\expval{\hat{A}}{\psi}  % 输出:⟨ψ|Â|ψ⟩
%
%% 对易子
%\comm{\hat{A}}{\hat{B}}  % 输出:[Â, B̂]
\newtheoremstyle{1}{}{}{}{}{\bfseries}{}{\newline}{}
\theoremstyle{1}
\newtheorem{problem}{Problem}
\usepackage{chngcntr}
\counterwithin{equation}{problem}
\newcommand{\pa}{\partial}
\newcommand{\rn}[1]{\romannumeral #1\relax}
\newcommand{\Rn}[1]{\expandafter\@slowromancap\romannumeral#1@}
\newcommand{\ii}{\mathrm{i}}
\newcommand{\ee}{\mathrm{e}}

\begin{document}
\maketitle
\begin{problem}[Kepler’s problem]
    (1) 
    \begin{equation}
        c=\frac{b^2}{a}.
    \end{equation}
    \begin{equation}
    E=\frac{1}{2}mv_1^2-\frac{\gamma}{a+\sqrt{a^2-b^2}}=\frac{1}{2} mv_2-\frac{\gamma}{a-\sqrt{a^2-b^2}}.   
    \end{equation}
    \begin{equation}
        l=mv_1\left(a+\sqrt{a^2-b^2}\right)=mv_2\left(a-\sqrt{a^2-b^2}\right).
    \end{equation}
    So, 
    \begin{equation}
        E=-\frac{\gamma}{2a}.
    \end{equation}
    \begin{equation}
        l=m\sqrt{\gamma c}.
    \end{equation}
    (2) We find that $a$ is only depends on $E$, and $c$ is only depends on $l$.
    \newline
    (3) If $E$ is fixed, then $a$ is fixed. $l$ reaches its maximum when $b=a$. So it is a circle. If $l$ is fixed, $c$ is fixed. Also when $b=a$, $E$ reaches its minimum. It is also a circle.
    \newline
    (4) The area of the ellipse is
    \begin{equation}
    S=\pi a b.    
    \end{equation}
    \begin{equation}
        \frac{\Delta S}{\Delta t}=\frac{l}{2m}.
    \end{equation}
    
    Thus,
    \begin{equation}
        T=\frac{2m \pi a b }{l}.
    \end{equation}
    
    Hence, 
    \begin{equation}
        \frac{T^2}{a^3}=\frac{4m\pi^2}{\gamma}.
    \end{equation}
\end{problem}
\begin{problem}[Cosmic velocities]
    Let $r$ be the radius of the Earth and $R_e$ is the distance between the Earth and the Sun. $M_e$ is the mass of the Earth, and $M_s$ is the mass of the Sun. 
    \newline
    (1) 
    \begin{equation}
        m\frac{v_1^2}{r}=\frac{GM_e m }{r^2},\ g=\frac{GM_e}{r^2}
    \end{equation}
    So,
    \begin{equation}
        v_1=\sqrt{gr}.
    \end{equation}
    (2) The energy can't less than $0$, if it can escape.
    \begin{equation}
        E=\frac{1}{2}mv_2^2-\frac{GM_em}{r}=0.
    \end{equation}
    So, 
    \begin{equation}
        v_2=\sqrt{2gr}\approx 11.2\mathrm{km/s}.
    \end{equation}
    (3) To escape from the Sun, the velocity after escaping the Earth but around the Earth should be 
    \begin{equation}
        v_{\mathrm{esc}}=\sqrt{\frac{2GM_s}{R_e}}.
    \end{equation}
    
    The velocity of the Earth is 
    \begin{equation}
        v_e=\sqrt{\frac{GM_s}{R_e}}.
    \end{equation}
    
    The relative velocity can least be 
    \begin{equation}
        v_r=(\sqrt{2}-1)\sqrt{\frac{GM_s}{R_e}}.
    \end{equation}
    
    By energy conservation in the Earth system,
    \begin{equation}
        \frac{1}{2}mv_3^2-\frac{GM_em}{r}=\frac{1}{2}m v_r^2.
    \end{equation}
    
    Hence,
    \begin{equation}
        v_3=\sqrt{(3-2\sqrt{2})\frac{GM_s}{R_e}+\frac{2GM_e}{r}}\approx16.7\mathrm{km/s}.
    \end{equation}
    (4) Here we use the notation in the lecture note,

    Initial energy (rocket + Earth system):
\[
E_1 = \frac{1}{2}(m + M)v_0^2 + E_{\text{ch}} - \frac{GMm}{R},
\]
where \( E_{\text{ch}} \) is chemical energy.


After burning chemical fuel, rocket has velocity \( v_0 + v_3 \) near Earth's surface:
\[
E_2 = \frac{m}{2}(v_0 + v_3)^2 + \frac{M}{2}(v_0 + \Delta v)^2 - \frac{GMm}{R},
\]
where \( \Delta v \) is Earth's recoil.

Momentum conservation:
\[
(m + M)v_0 = m(v_0 + v_3) + M(v_0 + \Delta v) \Rightarrow mv_3 + M\Delta v = 0.
\]

Substitute into \( E_2 \):
\[
E_2 = \frac{1}{2}(m + M)v_0^2 + \frac{m}{2}v_3^2 + \frac{M}{2}\Delta v^2 - \frac{GMm}{R}.
\]

Energy conservation \( E_1 = E_2 \) gives:
\[
E_{\text{ch}} = \frac{m}{2}\left(1 + \frac{m}{M}\right)v_3^2 \approx \frac{m}{2}v_3^2.
\]


\[
E_3 = \frac{1}{2}m(\sqrt{2}v_0)^2 + \frac{M}{2}(v_0 + \Delta v')^2,
\]
where \( \Delta v' \) is Earth's final recoil.

Momentum conservation:
\[
(m + M)v_0 = m\sqrt{2}v_0 + M(v_0 + \Delta v') \Rightarrow M\Delta v' = -m(\sqrt{2} - 1)v_0.
\]

Thus:
\[
E_3 = \frac{m}{2}(\sqrt{2}v_0)^2 + \frac{M}{2}v_0^2 - m(\sqrt{2} - 1)v_0^2.
\]

Energy conservation \( E_1 = E_3 \) yields:
\[
v_3^2 = v_0^2(\sqrt{2} - 1)^2 + v_2^2.
\]

With \( v_0 = 30 \, \text{km/s} \) and \( v_2 = 11.2 \, \text{km/s} \):
\[
v_3 = \sqrt{30^2 \times 0.414^2 + 11.2^2} = 16.7 \, \text{km/s}.
\]
(5) $v_3$ is the escape velocity to Earth, for planets at different place in the solar system, they have different escape velocity. So not anything can't get out from the solar system.
\end{problem}
\begin{problem}[Shallow impact - Double Asteroid Redirection Test]
    (1) Since $m1\gg m2$, we suppose $m_1$ is fixed,
    \begin{equation}
        m_2\left(\frac{2\pi}{T}\right)^2R=\frac{Gm_1m_2}{R^2}.
    \end{equation}
    
    So,
    \begin{equation}
        R=\sqrt[3]{\frac{Gm_1T^2}{4\pi^2}}\approx 1180  \mathrm{m}.
    \end{equation}
    \begin{equation}
        v=\frac{2\pi}{T}R=\sqrt[3]{\frac{2\pi Gm_1}{T}}\approx 0.17\mathrm{m/s}.
    \end{equation}
    (2) By Kepler's Third Law, we want to shorten the half-major axis $a$, that is also means to make the total energy get smaller, so we will make the velocity after crashing reaches the least. So we will choose to have a face-to-face collision.
    \begin{equation}
        m_2v-m_0v_0=(m_2+m_0)v'.
    \end{equation}
    \begin{equation}
        \delta v\approx-7.8\times 10^{-4}\mathrm{m/s}.
    \end{equation}
    \begin{equation}
        T=\sqrt{\frac{4\pi^2a^3}{Gm_1}}=2\pi Gm_1\left(-2E\right)^{-\frac{3}{2}}.
    \end{equation}
    So, 
    \begin{equation}
        -2\delta E=-\frac{2}{3}(2\pi Gm_1m_2)^{\frac{2}{3}}\sqrt{m_2}T^{-\frac{5}{3}}\delta T.
    \end{equation}
    \begin{equation}
        \delta E=m_2v\delta v\approx -6.48\times 10^{5}\mathrm{kg\cdot m^2/s^2 }.
    \end{equation}
    \begin{equation}
        \delta T=-14\mathrm{s}.
    \end{equation}
    Maybe it is not a completely inelastic collision.
\end{problem}

\end{document}