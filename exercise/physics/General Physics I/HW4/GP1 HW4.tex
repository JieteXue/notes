\documentclass{article}
\newcommand{\mydate}{October 9, 2025}
\newcommand{\mytitle}{GP1 HW4}
\title{\textbf{\mytitle}}
\author{Jiete XUE}
\date{\mydate}
\usepackage{fancyhdr}
\pagestyle{fancy}
\fancyhf{}
\fancyhead[C]{\mytitle }
\fancyhead[R]{Jiete Xue}
\fancyhead[L]{\mydate}
\fancyfoot[C]{\thepage}
\usepackage{amsthm}
\usepackage{amsmath}
\usepackage{amssymb}
\usepackage{physics}
\usepackage{tikz}

%% 右矢
%\ket{\psi}          % 输出:|ψ⟩
%\ket{\psi(t)}       % 输出:|ψ(t)⟩
%
%% 左矢
%\bra{\phi}          % 输出:⟨φ|
%
%% 期望值
%\expval{\hat{A}}    % 输出:⟨Â⟩
%\expval{\hat{A}}{\psi}  % 输出:⟨ψ|Â|ψ⟩
%
%% 对易子
%\comm{\hat{A}}{\hat{B}}  % 输出:[Â, B̂]
\newtheoremstyle{1}{}{}{}{}{\bfseries}{}{\newline}{}
\theoremstyle{1}
\newtheorem{problem}{Problem}
\usepackage{chngcntr}
\counterwithin{equation}{problem}
\newcommand{\pa}{\partial}
\newcommand{\rn}[1]{\romannumeral #1\relax}
\begin{document}
\maketitle
\begin{problem}[The hyperbolic orbit]
    (1)\begin{equation}
        \frac{\dd{\vec{S}}}{\dd{t}}=\frac{1}{2}\vec{v}\times\vec{r}=\frac{\vec{L}}{2m}=\text{const.}
    \end{equation}
    (2)
        \begin{equation}
            E=\frac{1}{2}mv^2-\frac{GMm}{r}.
        \end{equation}
        Let $\vec{v}=\vec{u}+\vec{C}$, where $\vec{C}$ is a constant vector and $\left|\vec{u}\right|=\frac{GM}{L}$ Then,
    \begin{equation}
        \left|\vec{v}-\frac{GM}{L}\vec{e}\right|=\frac{GM}{L}.
    \end{equation}    
    $\vec{v}$ can only be a part of circle, so $E>0$.
    
\end{problem}

\begin{problem}[A parabolic orbit]
    By Problem 1 we can know that when $E=0$, the orbit is a parabola.
\end{problem}
\begin{problem}[Repulsive inverse-square force field]
    (1) Same as Problem 1.
    \newline
    (2)Let 
    \begin{equation}
        \frac{\dd{\vec{v}}}{\dd{t}}=\frac{k}{mr^2}\hat{r},\ k>0.
    \end{equation}
    We have 
    \begin{equation}
        \dot{\theta}=\frac{L}{mr^2}.
    \end{equation}
    So 
    \begin{equation}
        \frac{\dd{\vec{v}}}{\dd{\theta}}=\frac{k}{L}\hat{r}=-\frac{k}{L}\frac{\dd{\hat{\theta}}}{\dd{\theta}}.
    \end{equation}
    Thus, 
    \begin{equation}
        \vec{v}=\vec{v_0}-\frac{k}{L}\vec{\theta}.
    \end{equation}
    $E=\frac{1}{2}mv^2+\frac{k}{r}>0$, so the velocity circle does not contain the origin. So the orbit is a hyperbola.
\end{problem}
\begin{problem}[Proof of Kepler's First Law]
    We start from Binet's equation:
    \begin{equation}
        h^2u^2\left(\frac{\dd^2{u}}{\dd{\theta}^2}+u\right)=-\frac{F}{m}.
    \end{equation}
    Where $h=r^2\dot{\theta}$ is a constant, $u=\frac{1}{r}$, $F=-\frac{GMm}{r^2}=-GMmu^2$. Hence,
    \begin{equation}
        \frac{\dd^2{u}}{\dd{\theta}^2}+u=\frac{Gm}{h^2}.
    \end{equation}
    \begin{equation}
        u=A\cos\left(\theta+\theta_0\right)+\frac{GM}{h^2}.
    \end{equation}
    \begin{equation}
        r=\frac{1}{A\cos\left(\theta+\theta_0\right)+\frac{GM}{h^2}}.
    \end{equation}
    We can use the initial condition to determine $A$ and $\theta_0$. Anyway, we know it is a quadratic curve.
\end{problem}
\end{document}