\documentclass{article}
\newcommand{\mydate}{October 11, 2025}
\newcommand{\mytitle}{QM HW5}
\title{\textbf{\mytitle}}
\author{Jiete XUE}
\date{\mydate}
\usepackage{fancyhdr}
\pagestyle{fancy}
\fancyhf{}
\fancyhead[C]{\mytitle }
\fancyhead[R]{Jiete Xue}
\fancyhead[L]{\mydate}
\fancyfoot[C]{\thepage}
\usepackage{amsthm}
\usepackage{amsmath}
\usepackage{amssymb}
\usepackage{physics}
\usepackage{tikz}

%% 右矢
%\ket{\psi}          % 输出:|ψ⟩
%\ket{\psi(t)}       % 输出:|ψ(t)⟩
%
%% 左矢
%\bra{\phi}          % 输出:⟨φ|
%
%% 期望值
%\expval{\hat{A}}    % 输出:⟨Â⟩
%\expval{\hat{A}}{\psi}  % 输出:⟨ψ|Â|ψ⟩
%
%% 对易子
%\comm{\hat{A}}{\hat{B}}  % 输出:[Â, B̂]
\newtheoremstyle{1}{}{}{}{}{\bfseries}{}{\newline}{}
\theoremstyle{1}
\newtheorem{problem}{Problem}
\usepackage{chngcntr}
\counterwithin{equation}{problem}
\newcommand{\pa}{\partial}
\newcommand{\rn}[1]{\romannumeral #1\relax}
\newcommand{\Rn}[1]{\expandafter\@slowromancap\romannumeral#1@}

\begin{document}
\maketitle
\begin{problem}[Coherent states]
    Let $l=\sqrt{\frac{\hbar}{m\omega}}$, then
    \begin{equation}
        x=\frac{l}{\sqrt{2}}\left(a+a^\dagger\right),\ p=\frac{\hbar}{il}\frac{a-a^\dagger}{\sqrt{2}}.
    \end{equation}
    \begin{equation}
        \expval{x}=\frac{l}{\sqrt{2}}\expval{a+a^\dagger}{\alpha}=\frac{l}{\sqrt{2}}\left(\alpha+\alpha^*\right).
    \end{equation}
    \begin{equation}
        \expval{p}=\frac{\hbar}{il}\frac{\expval{a-a^\dagger}{\alpha}}{\sqrt{2}}=\frac{\hbar}{il}\frac{\alpha-\alpha^*}{\sqrt{2}}.
    \end{equation}
\begin{equation}
    x^2=\frac{l^2}{2}\left[a^2+{a^\dagger}^2+\{a,a^\dagger\}\right]=\frac{l^2}{2}\left[a^2+{a^\dagger}^2+2a^\dagger a+1\right].
\end{equation}
\begin{equation}
    p^2=\frac{\hbar^2}{2l^2}[2a^\dagger a+1-(a^2+{a^\dagger}^2)].
\end{equation}
\begin{equation}
    \expval{x^2}=\frac{l^2}{2}\left[(\alpha+\alpha^*)^2+1\right].
\end{equation}
\begin{equation}
    \expval{p^2}=\frac{\hbar^2}{2l^2}\left[-(\alpha-\alpha^*)^2+1\right].
\end{equation}
Thus, 
\begin{equation}
    \sqrt{\overline{\Delta x^2}}\sqrt{\overline{\Delta p^2}}=\frac{\hbar}{2}.
\end{equation}
It reaches the minimum uncertainty, so we call it the most classical quantum state.
\end{problem}
\begin{problem}[Wavefunctions of Harmonic Oscillator]
    \begin{equation}
        0=\bra{x}a\ket{0}=\bra{x}\frac{1}{\sqrt{2}}\left[\frac{x}{l}+\frac{ipl}{\hbar}\right]\ket{0}=\frac{1}{\sqrt{2}}\left(\frac{x}{l}\psi_0+l\frac{\dd{\psi_0}}{\dd{x}}\right).
    \end{equation}
    Hence, 
    \begin{equation}
        \psi_0(x)=Ae^{-\frac{1}{2}\frac{x^2}{l^2}}.
    \end{equation}
    Normalize,
    \begin{equation}
        \left|A\right|^2\int_{-\infty}^{+\infty}e^{-\frac{x^2}{l^2}}\dd{x}=\left|A\right|^2l \Gamma \left(\frac{1}{2}\right)=1.
    \end{equation}
    Therefore,
    \begin{equation}
        \boxed{\psi_0(x)=\frac{e^{-\frac{x^2}{2l^2}}}{l^{1/2}\pi^{1/4}}.}
    \end{equation}
    By $a^\dagger\ket{n}=\sqrt{n+1}\ket{n+1}$,
    \begin{equation}
        \psi_{n+1}(x)=\frac{1}{\sqrt{2(n+1)}}\left(\frac{x}{l}-l\frac{\dd}{\dd{x}}\right)\psi_n(x).
    \end{equation}
    So, 
    \begin{equation}
        \boxed{\psi_n(x)=\frac{1}{\sqrt{2^nn!}}\left(\frac{x}{l}-l\frac{\dd}{\dd{x}}\right)^n\psi_0(x).}
    \end{equation}
\end{problem}
\begin{problem}[High dimensional Oscillator]
    (1) We have $[x_i,p_j]=i\hbar\delta_{ij}$, so
    \begin{equation}
        [a^i,a^\dagger_j]=\delta^i_{\ j},\ [a^i,a_i^\dagger]=D.
    \end{equation}
    \begin{equation}
        x^ix_i=\frac{l^2}{2}\left(a^ia_i+{a^i}^\dagger a_i^\dagger+2a_i^\dagger a^i+[a^i,a_i^\dagger]\right).
    \end{equation}
    \begin{equation}
        p^ip_i=-\frac{\hbar^2}{2l^2}\left(a^ia_i+{a^i}^\dagger a_i^\dagger-2a_i^\dagger a^i-[a^i,a_i^\dagger]\right).
    \end{equation}
    Let $N=a_i^\dagger a_i$, then 
    \begin{equation}
        \boxed{H=\hbar \omega (N+D/2).}
    \end{equation}
    \begin{equation}
        a'=Ua, \ a'^\dagger=a^\dagger U^\dagger,\ N'=a^\dagger \,U^\dagger U a=N.
    \end{equation}
    Hence, $H$ is invariant under the transformation.
\newline
(2) We only need to check $[Q_{ij},N]=0$. We have
\begin{equation}
    [a_i,a_k^\dagger a^k]=a^k\delta_{i k}=a_i,\  [a_i^\dagger,a_k^\dagger a^k]=-a^k\delta_{i k}=-a_i^\dagger.
\end{equation}
Since $A_{ij}$ is a number, $[A_{ij},N]=0$, thus 
\begin{align}
    [a_i^\dagger A_{ij}a_j,N]&=a_i^\dagger A_{ij} [ a_j,N]+[a_i^\dagger,N] A_{ij}a_j \notag\\
    &=a_i^\dagger A_{ij}a_j-a_i^\dagger A_{ij} a_j \notag\\
    &=0.
\end{align}
(I'm confused about why we need a $A_{ij}$. It is just a number.)
\newline
(3) Anything commutable with $H$ is a conservation. $a_ia_j^\dagger$ is conserved, which means the angular momentum is conserved.
\end{problem}
\begin{problem}[Quantum Virial Theorem]
    (1) We use the Schr\"odinger picture. Let $\ket{\alpha,t}$ be a state\footnote{Shorten as $\ket{\alpha}$}. The Schr\"odinger equation is
    \begin{equation}
        i\hbar \frac{\pa}{\pa t}\ket{\alpha}=H\ket{\alpha}.
    \end{equation}
    Then,
    \begin{equation}
        \frac{\dd}{\dd{t}}\bra{\alpha}\mathbf{x}\cdot\mathbf{p}\ket{\alpha}=\frac{1}{i\hbar}\expval{[\mathbf{x}\cdot\mathbf{p},H]}{\alpha}.
    \end{equation}
    \begin{equation}
        [x^ip_i,\frac{p^jp_j}{2m}+V(\mathbf{x})]=[x^i,\frac{p^jp_j}{2m}]p_i+x^i[p_i,V(\mathbf{x})]=i \hbar\left(\frac{p^ip_i}{m}-x^i\pa _i V\right).
    \end{equation}
    Thus,
    \begin{equation}
         \frac{\dd}{\dd{t}}\bra{\alpha}\mathbf{x}\cdot\mathbf{p}\ket{\alpha}=\expval{\frac{p^2}{m}-\mathbf{x}\cdot \nabla V}{\alpha}.
    \end{equation}
    Therefore,
    \begin{equation}
        \boxed{ \frac{\dd}{\dd{t}}\bra{\alpha}\mathbf{x}\cdot\mathbf{p}\ket{\alpha}=\expval{\frac{p^2}{m}}-\expval{\mathrm{x}\cdot\nabla V}.}
    \end{equation}
    (2) Since $H$ is Hermitian and $E$ is real, we have 
    \begin{equation}
      (H-E)  \ket{n,\lambda}=0,\ \bra{n,\lambda}(H-E)=0.
    \end{equation}
    Thus, 
    \begin{equation}
        \left(\frac{\pa}{\pa \lambda}\bra{n,\lambda}\right)(H-E)\ket{n,\lambda}=0,\ \bra{n,\lambda}(H-E)\left(\frac{\pa}{\pa \lambda}\ket{n,\lambda}\right)=0.
    \end{equation}
    Let $\frac{\pa}{\pa \lambda}$ act on the following equation,
    \begin{equation}
        \expval{(H-E)}{n,\lambda}=0,
    \end{equation}
    we obtain, 
    \begin{equation}
        \expval{\frac{\pa}{\pa \lambda}\left(H-E\right)}{n,\lambda}.
    \end{equation}
    Exactly,
    \begin{equation}
        \boxed{\frac{\pa E}{\pa \lambda}=\expval{\frac{\pa H}{\pa \lambda}}{n,\lambda}}.
    \end{equation}
    (3) 
    \newline
    (4) For harmonic oscillator, $V=\frac{1}{2}m\omega^2x^2$, $\mathbf{x}\cdot\nabla V=m\omega^2x^2$.
    \begin{equation}
        \mathbf{x}\cdot\mathbf{p}=\frac{\hbar}{2}\left(a^2-{a^\dagger}^2-3\right).
    \end{equation}
    So,
    \begin{equation}
        \expval{\mathbf{x}\cdot\mathbf{p}}=-\frac{3\hbar}{2}
    \end{equation}
    \begin{equation}
        \expval{p^2}{n}=\left(n+\frac{3}{2}\right)\frac{\hbar m \omega}{2},\ \expval{x^2}{n}=\left(n+\frac{3}{2}\right)\frac{\hbar}{2m\omega}.
    \end{equation}
Quantum Viral theorem holds. We can find that $\expval{\mathbf{x}\cdot\mathbf{p}}$ is not dependent on $n$.
\end{problem}
\begin{problem}[Operator normal product]
    Since $\ket{0}$ is the ground state and $H=\omega a^\dagger a$, $a\ket{0}=0$.
    \begin{equation}
        e^{\alpha a}\ket{0}=\sum_{k=1}^{\infty} \frac{\alpha^k}{k!}a^k\ket{0}=\ket{0}.
    \end{equation}
    So,
    \begin{equation}
        \expval{:e^A:}{0}=\expval{e^{\alpha' a^\dagger}e^{\alpha a}}{0}=1.
    \end{equation}
    \begin{equation}\label{5.3}
        \expval{AB}{0}=\alpha\beta'.
    \end{equation}
    By Baker-Hausdorff lemma, we can deduce that: If $[[A,B],A]=[[A,B],B]=0$, then
    \begin{equation}
        \exp(AB)=\exp(BA)\exp([A,B]).
    \end{equation}
    Thus,
    \begin{equation}
        e^{\alpha a}e^{\beta' a^\dagger}=e^{\beta' a^\dagger}e^{\alpha a}e^{\alpha \beta'}.
    \end{equation}
    \begin{equation}
        e^{\alpha' a}e^{\alpha a}e^{\beta' a^\dagger}e^{\beta a}=e^{(\alpha'+\beta')a^\dagger}e^{(\alpha+\beta)a}e^{\alpha \beta'}.
    \end{equation}
    That is 
    \begin{equation}\label{5.7}
        :e^A::e^B=:e^{A+B}:e^{\expval{AB}{0}}.
    \end{equation}
    \begin{equation}\label{5.8}
        e^A=:e^A:e^{\frac{1}{2}\alpha\alpha'}.
    \end{equation}
    Let $A=B$ in \eqref{5.3}, we get $\alpha\alpha'=\expval{A^2}{0}$. Then Plug \eqref{5.8} into \eqref{5.7}, we obtain
    \begin{equation}
        e^Ae^B=:e^{A+B}:e^{\expval{AB+\frac{A^2}{2}+\frac{B^2}{2}}{0}}.
    \end{equation}
    Therefore,
    \begin{equation}
    \boxed{
        \expval{e^Ae^B}{0}=\expval{:e^{A+B}:}{0}e^{\expval{AB+\frac{A^2}{2}+\frac{B^2}{2}}{0}}=e^{\expval{AB+\frac{A^2}{2}+\frac{B^2}{2}}{0}}.}
    \end{equation}
\end{problem}
\end{document}