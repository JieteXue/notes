\documentclass{article}
\title{QM HW1}
\author{Jiete XUE}
\date{2025/09/04}
\usepackage{amsthm}
\usepackage{amsmath}
\newtheoremstyle{t}{10pt}{5pt}{}{}{\bfseries}{}{\newline}{}
\theoremstyle{t}
\newtheorem{problem}{Problem}
\usepackage{chngcntr}
\counterwithin{equation}{problem}
\newcommand{\dd}{\mathrm{d}}
\newcommand{\pa}{\partial}
\begin{document}

\maketitle
\begin{problem}[Canonical Transformation]
(1)\begin{equation}
   \dd \Phi(q,P,t)=p\dd q+Q\dd P+(H'-H)\dd t 
\end{equation}
\begin{equation}
    \frac{\partial \Phi}{\partial q}=p,\frac{\partial \Phi}{\partial P}=Q,\frac{\partial \Phi}{\partial t}+H=H'.
\end{equation}
(2) We have proved:
\begin{equation}
    \frac{\partial S}{\partial q_t}=-p_t,\frac{\partial S}{\partial q_{t+\tau}}=p_{t+\tau},
\end{equation}
First, we can check:
\begin{equation}
    \frac{\partial \Psi}{\partial q_t}=-\frac{\partial S}{\partial q_t}=p_t
\end{equation}
Second ,by chain rule,we obtain
\begin{equation}
    \frac{\partial \Psi }{\partial p_{t+\tau}}=q_{t+\tau}+\left( p_{t+\tau}-\frac{\partial S}{\partial p_{t+\tau}}\right)\frac{\partial q_{t+\tau}}{\partial p_{t+\tau}}=q_{t+\tau}
\end{equation}
(3) \begin{equation}
    \dd p_t\wedge\dd q_t=\dd p_{t+\tau}\wedge\dd q_{t+\tau},\frac{\partial(q_{t+\tau},p_{t+\tau})}{\partial(q_t,p_t)}=1
\end{equation}
\end{problem}
\begin{problem}[Hamilton-Jacobi equation]
    (1) \begin{equation}
        \frac{\pa S}{\pa t}=\frac{\dd S}{\dd t}-\frac{\pa S}{\pa q}\dot{q}=L-p\dot{q}=-H
    \end{equation}
    (2) \begin{equation}
        \beta=\frac{\pa S}{\pa \alpha}
    \end{equation}
    \begin{equation}
        H'=H+\frac{\pa S}{\pa t}=0
    \end{equation}
    \textbf{Remark.} Another way to calculate $\frac{\pa S}{\pa q_f}$ and $\frac{\pa S}{\pa t_f}$.
    \begin{equation}
        q=q(q_f,t_f,t),\dd q=\frac{\pa q}{\pa q_f}\dd q_f+\frac{\pa q}{\pa t_f}\dd t_f+\frac{\pa q}{\pa t}\dd t
    \end{equation}
    We have $q|_{t=t_0}=q_0,q|_{t=t_f}$ , hence,
    \begin{equation}
      \left. \left( \frac{\pa q}{\pa q_f}\right)_{t_f,t}\right|_{t=t_0}=0,\space\left.\left( \frac{\pa q}{\pa q_f}\right)_{t_f,t}\right|_{t=t_f}=1
    \end{equation}
    Now,let $t\equiv t_0,\dd q=\dd q_0=0$.Thus,
    \begin{equation}
        \left.\frac{\pa q}{\pa t_f}\right|_{t=t_0}=0
    \end{equation}
    Let $t$ be a function of $t_f$ and take the form $t=t_f$.$\dd q=\dd q_f$,so 
    \begin{equation}
        \left(\frac{\pa q}{\pa t_f}+\frac{\pa q}{\pa t}\right)\dd t_f=0
    \end{equation}
    \begin{equation}
        \left.\frac{\pa q}{\pa t_f}\right|_{t=t_f}=-\left.\frac{\pa q}{\pa t}\right|_{t=t_f}=\left.-\dot{q}\right|_{t=t_f}
    \end{equation}
\begin{equation}
\begin{split}
    \frac{\pa S}{\pa q_f}&=\int_{t_0}^{t_f}\left[\frac{\pa L}{\pa \dot{q}}\frac{\pa}{\pa q_f}\left(\frac{\pa q}{\pa t}\right)+\frac{\pa L}{\pa q}\frac{\pa q}{\pa q_f}\right]\,\dd t\\
    &=\int_{t_0}^{t_f}\left[\frac{\pa L}{\pa \dot{q}}\frac{\pa}{\pa t}\left(\frac{\pa q}{\pa q_f}\right)+\frac{\pa L}{\pa q}\frac{\pa q}{\pa q_f}\right]\,\dd t\\
    &=\left.\frac{\pa L}{\pa \dot{q}}\frac{\pa q}{\pa q_f}\right|_{t_0}^{t_f}+\int_{t_0}^{t_f}\left[\frac{\pa L}{\pa q}-\frac{\pa}{\pa t}\left(\frac{\pa L}{\pa \dot{q}}\right)\right]\frac{\pa q}{\pa q_f}\,\dd t\\
    &=p(t_f)
\end{split}
\end{equation}
\begin{equation}
    \begin{split}
        \frac{\pa S}{\pa t_f}&=L(t_f)+\int_{t_0}^{t_f}\left[\frac{\pa L}{\pa \dot{q}}\frac{\pa}{\pa t_f}\left(\frac{\pa q}{\pa t}\right)+\frac{\pa L}{\pa q}\frac{\pa q}{\pa t_f}\right]\,\dd t\\
        &=L(t_f)+\left.p\frac{\pa q}{\pa t_f}\right|_{t_0}^{t_f}\\
        &=L-p\dot{q}=-H
    \end{split}
\end{equation}
\end{problem}

\begin{problem}[Harmonic Oscillator]
    (1) easy to check:
    \begin{equation}
        \frac{\pa S}{\pa x}=p,\frac{\pa S}{\pa t}=-E
    \end{equation}
    (2)\begin{equation}
        \frac{\pa S}{\pa t_f}=-E
    \end{equation}
    \begin{equation}
        \frac{\pa S}{\pa x_f}=p=\pm m \omega \sqrt{A^2-x^2}
    \end{equation}
    (3)
    \begin{equation}
        S=\pm m \omega \int \sqrt{\frac{2E}{m \omega^2}-x^2}\, \dd x-E t+\mathrm{const.}
    \end{equation}
    \begin{equation}
        \frac{\pa S}{\pa E}=\pm \frac{\arcsin (\sqrt{\frac{m \omega^2}{2E}}x)}{\omega}-t
    \end{equation}
    New Hamitonian:
    \begin{equation}
        H'=H+\frac{\pa S}{\pa t}=0
    \end{equation}
    By Hamilton equation
    \begin{equation}
        \dot{\beta}=\frac{\pa H'}{\pa E}=0
    \end{equation}
    Therefore $\beta$ is a constant.It means the initial phase of  oscillator.
\end{problem}
\begin{problem}[Planck’s derivation of black body radiation]
    (1) $\omega \rightarrow +\infty$:
    \begin{equation}
        \ln u=-\frac{\hbar \omega}{k_B T}+\mathrm{const.}
    \end{equation}
    \begin{equation}
        \frac{\pa (\ln u)}{\pa\left(\frac{1}{T}\right)}=-\frac{\hbar \omega}{k_B}
    \end{equation}
    \qquad $\omega\rightarrow 0$:
    \begin{equation}
        \frac{1}{u}\frac{\pa u}{\pa\left(\frac{1}{T}\right)}=-T=-\frac{u}{k_B}
    \end{equation}
    We assume that 
    \begin{equation}
        \frac{\pa(\ln u)}{\pa \left(\frac{1}{T}\right)}=-\frac{\hbar \omega +u}{k_B}
    \end{equation}
    That leads to 
    \begin{equation}
        \frac{u}{\hbar \omega +u}=Ce^{-\frac{\hbar \omega}{k_B T}}
    \end{equation}
    By the expression at low frequency,we can deduce that $C=1$.
    \begin{equation}
        u=\frac{\hbar \omega}{e^{\frac{\hbar \omega}{k_B T}}-1}
    \end{equation}
\end{problem}
\begin{problem}[Heisenberg’s magic]
    Quantization condition:
    \begin{equation}
        \oint p\dd q=\int_{0}^{T}m\dot{x}^2\dd t=nh
    \end{equation}
    \begin{equation}
        x_{n\leftarrow m}(t)=x_{n \leftarrow m}e^{i \omega_{n\leftarrow m}t}
    \end{equation}
    \begin{equation}
        \dot{x}^2_{n\leftarrow m }(t)=\sum_{k=-\infty}^{+\infty}x_{n \leftarrow k}x_{k \leftarrow m}\omega_{n\leftarrow k}\omega_{k\leftarrow m}e^{i \omega_{n\leftarrow m}t}
    \end{equation}
    Let $m=n+l$, we obtain f-sum rule
    \begin{equation}
        \sum_{l=0}^{+\infty} \left( \omega_{n \to n + l} \left| x_{n + l, n} \right|^2 - \omega_{n-l \to n} \left| x_{n,n-l} \right|^2 \right) = \frac{\hbar}{2m} 
    \end{equation}
    In classic mechanics ,we have
    \begin{equation}
        \ddot{x}+\omega_0^2 x=0
    \end{equation}
    Plug in $x_{n\pm l\leftarrow n}(t)=x_{n\pm l\leftarrow n}e^{i \omega_{n\pm l\leftarrow n}t}$
    \begin{equation}
        (\omega_0^2-\omega_{n\pm l\leftarrow n}^2)x_{n\pm l\leftarrow n}=0
    \end{equation}
    Only when $l=\pm 1,\omega_0^2-\omega_{n\pm l\leftarrow n}^2=0,x_{n\pm l\leftarrow n}\not=0$
    \begin{equation}
        \frac{\hbar}{2m} = \omega_0 \left( |x_{n+1,n}|^2 - |x_{n,n-1}|^2 \right)
    \end{equation}
    There must exists a lower bound ,or $\left|x_{n,n-1}\right|^2\ge 0$ can not hold.Thus,
    \begin{equation}
        x_{n+1,n}=\sqrt{\frac{\hbar}{2m\omega_0}(n+1)}
    \end{equation}
    Therefore,
    \begin{equation}
        (\dot{x})_{nn}^{2} = (n + 1/2) \frac{\hbar \omega_0}{m}
    \end{equation}
\end{problem}

\end{document}