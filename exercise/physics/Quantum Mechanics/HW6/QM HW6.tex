\documentclass{article}
\newcommand{\mydate}{October 26, 2025}
\newcommand{\mytitle}{QM HW6}
\title{\textbf{\mytitle}}
\author{Jiete XUE}
\date{\mydate}
\usepackage{fancyhdr}
\pagestyle{fancy}
\fancyhf{}
\fancyhead[C]{\mytitle }
\fancyhead[R]{Jiete Xue}
\fancyhead[L]{\mydate}
\fancyfoot[C]{\thepage}
\usepackage{amsthm}
\usepackage{amsmath}
\usepackage{amssymb}
\usepackage{physics}
\usepackage{tikz}

%% 右矢
%\ket{\psi}          % 输出:|ψ⟩
%\ket{\psi(t)}       % 输出:|ψ(t)⟩
%
%% 左矢
%\bra{\phi}          % 输出:⟨φ|
%
%% 期望值
%\expval{\hat{A}}    % 输出:⟨Â⟩
%\expval{\hat{A}}{\psi}  % 输出:⟨ψ|Â|ψ⟩
%
%% 对易子
%\comm{\hat{A}}{\hat{B}}  % 输出:[Â, B̂]
\newtheoremstyle{1}{}{}{}{}{\bfseries}{}{\newline}{}
\theoremstyle{1}
\newtheorem{problem}{Problem}
\usepackage{chngcntr}
\counterwithin{equation}{problem}
\newcommand{\pa}{\partial}
\newcommand{\rn}[1]{\romannumeral #1\relax}
\newcommand{\Rn}[1]{\expandafter\@slowromancap\romannumeral#1@}
\newcommand{\ii}{\mathrm{i}}
\newcommand{\ee}{\mathrm{e}}

\begin{document}
\maketitle
\begin{problem}[Current, gauge transformation]
    (1) Use the substitution:
    \begin{equation}
        -i\hbar \nabla\longrightarrow -i\hbar \nabla -\frac{q\vec{A}}{c},
    \end{equation}
    the probability current will be:
    \begin{equation}
        \vec{j}=\frac{\hbar}{m}\mathrm{Im}\left(\psi^*\nabla \psi\right)\longrightarrow\frac{\hbar}{m}\mathrm{Im}\left(\psi^*\nabla \psi\right)-\frac{q}{mc} \vec{A}\left|\psi\right|^2.
    \end{equation}
    Hence,
    \begin{equation}
        \boxed{\vec{j}=\frac{\hbar}{m}\mathrm{Im}\left(\psi^*\nabla \psi\right)-\frac{q}{mc} \vec{A}\left|\psi\right|^2.}
    \end{equation}
    (2) 
    \begin{equation}
        A_\mu'=A_\mu+\pa_\mu f.
    \end{equation}
    Since the commutator is antisymmetric and the derivative is commutative,
    \begin{equation}
        \pa_{[\mu}\pa_{\nu]}f=0.
    \end{equation}
    Thus,
    \begin{equation}
        F_{\mu\nu}'=\pa_{[\mu}A'_{\nu]}=\pa_{[\mu}A_{\nu]}+\pa_{[\mu}\pa_{\nu]}f=F_{\mu\nu}.
    \end{equation}
    So, 
    \begin{equation}
        \mathbf{E}'=\mathbf{E},\ \mathbf{B}'=\mathbf{B}.
    \end{equation}
    (3) 
    \begin{equation}
        \frac{\pa \psi'}{\pa t}=i\frac{\pa \varphi}{\pa t}\psi'+e^{i\varphi}\frac{\pa \psi}{\pa t}
    \end{equation}
    Cancelate  $e^{i\varphi}$, and plug in Schrödinger equation, we should take
    \begin{equation}
        \boxed{\varphi=\frac{q}{\hbar c}f.}
    \end{equation}
    (4)
    \begin{equation}
        \rho'=e^{i\varphi}\psi e^{-i\varphi}\psi^*=\psi\psi^*=\rho.
    \end{equation}
    If we let 
    \begin{equation}
        \psi=\sqrt{\rho}e^{\frac{iS}{\hbar}},
    \end{equation}
    then,
    \begin{equation}
        \mathbf{j}=\frac{\rho}{m}\left(\nabla S-\frac{q\mathbf{A}}{c}\right).
    \end{equation}
    \begin{equation}
       S'=S+\hbar \varphi,\quad  \nabla S'-\frac{q\mathbf{A}'}{c}=\nabla S-\frac{q\mathbf{A}}{c}.
    \end{equation}
    So the probability current  is invariant under the gauge transformation.
    
\end{problem}
\begin{problem}[Landau gauge]
    (1)
    \begin{equation}
        \left(p_x-\frac{qB}{c}y\right)^2e^{ik_x x}=\left(\hbar k_x-\frac{qB}{c}y\right)^2 e^{ik_x x}.
    \end{equation}
    \begin{equation}
        \left[\frac{\left(\hbar k_x-\frac{qBy}{c}\right)^2}{2m}+(V(y)-E)\right]\phi_{k_x}(y)=\frac{\hbar^2}{2m}\phi_{k_x}''(y).
    \end{equation}
    We can define $H_y(k_x)$ as 
    \begin{equation}
        H_y(k_x)=\frac{\left(\hbar k_x-\frac{qBy}{c}\right)^2}{2m}+\frac{p_y^2}{2m}+V(y).
    \end{equation}
    (2) Note that 
    \begin{equation}
        \frac{\pa H_y(k_x)}{\pa k_x}=\frac{\hbar k_x-qBy/c}{m}.
    \end{equation}
    \begin{equation}
        I_x(x,k_x)=qJ=\frac{q\rho}{m}\left(\hbar k_x-\frac{qBy}{c}\right)=\frac{q}{L_x}\frac{\pa E_n}{\hbar \pa k_x}.
    \end{equation}
    (3) 
    \begin{equation}
    I_{x,n}=\frac{1}{2\pi}\int\dd{k_x}I_x(n,k_x)  =\frac{q^2}{h}\Delta V/L_x. 
    \end{equation}
    \begin{equation}
        \sigma_{xy}=\frac{I_x}{E_y}=\frac{q^2}{h}\nu.
    \end{equation}
    This result is not related to $V_{\text{imp}}$.
\end{problem}
\begin{problem}[Spherical coordinates]

    I will use Einstein summation convention if there's neither special announcement nor summation symbol.

    In spherical coordinates, Lamé coefficients are
    \begin{equation}
        A_r=1,\quad A_\theta=r,\quad A_\phi=r\sin\theta.
    \end{equation}
    (1)
    \begin{equation}
        \nabla f=\mathbf{g}^i\pa_i f=\sum_i \frac{1}{A_i}\mathbf{e}^i\pa_i f.
    \end{equation}
    So,
    \begin{equation}
        \nabla f=\frac{\pa f}{\pa r}\mathbf{e}_r+\frac{1}{r}\frac{\pa f}{\pa \theta}\mathbf{e}_\theta+\frac{1}{r\sin\theta}\frac{\pa f}{\pa \phi}\mathbf{e}_\phi.
    \end{equation}
    (2) Note that
    \begin{equation}
        \frac{\pa \sqrt{g}}{\pa x^i}=\Gamma_{ji}^j\sqrt{g}.
    \end{equation}
    We have,
    \begin{equation}
        \nabla\cdot\mathbf{V}=\pa_i V^i+V^m\Gamma^i_{im}=\pa_iV^i+V^m\frac{1}{\sqrt{g}}\pa_m\sqrt{g}=\frac{1}{\sqrt{g}}\pa_i\left(\sqrt{g}V^i\right).
    \end{equation}
    That is 
    \begin{equation}
        \nabla\cdot\mathbf{V}=\sum_i\frac{1}{A_rA_\theta A_\phi}\pa_i\left(\frac{A_rA_\theta A_\phi}{A_i}V\langle i\rangle \right).
    \end{equation}
    \begin{equation}
        \nabla\cdot \mathbf{V}=\frac{1}{r^2}\frac{\pa}{\pa r}\left(r^2 V \langle r\rangle\right)+\frac{1}{r\sin\theta}\frac{\pa}{\pa \theta}\left(\sin\theta V\langle \theta\rangle\right)+\frac{1}{r\sin\theta}\frac{\pa V\langle \phi\rangle}{\pa \phi}.
    \end{equation}
    (3)
    \begin{equation}
        \nabla\times\mathbf{V}=\epsilon^{ijk}\nabla_i V_j\mathbf{g}_k=\epsilon^{ijk}\left(\pa_i V_j-V_m\Gamma_{ij}^m\right)\mathbf{g}^k=\epsilon^{ijk}\pa_iV_j\mathbf{g}^k.
    \end{equation}
    \begin{equation}
        \nabla\times \mathbf{V}=\frac{1}{r^2\sin\theta}\begin{vmatrix}
  \mathbf{e}_r & r\mathbf{e}_\theta  & r\sin\theta\mathbf{e}_\phi \\
  \partial _r& \partial _\theta  & \partial _\phi \\
 V\langle r\rangle  &rV\langle\theta \rangle   &r\sin\theta V\langle \phi\rangle 
\end{vmatrix}.
    \end{equation}
    (4) By (1) and (2),
    \begin{equation}
        \nabla^2 f=
\nabla^2 = \frac{1}{r^2} \frac{\partial }{\partial r} \left( r^2 \frac{\partial f}{\partial r} \right)
+ \frac{1}{r^2 \sin\theta} \frac{\partial}{\partial \theta} \left( \sin\theta \frac{\partial f}{\partial \theta} \right)
+ \frac{1}{r^2 \sin^2\theta} \frac{\partial^2 f}{\partial \phi^2}.
    \end{equation}

\end{problem}
\begin{problem}[Angular momentum operators]
    (1) 
    \begin{equation}
        \mathbf{l}=\mathbf{r}\times\mathbf{p}:=-\ii\hbar\mathbf{r}\times\nabla=\pdv{\theta}\hat{\mathbf{\phi}}-\frac{1}{\sin\theta}\pdv{\phi}\hat{\mathbf{\theta}}.
    \end{equation}
    Then, project on $\hat{\mathbf{x}},\hat{\mathbf{y}},\hat{\mathbf{z}}$, we obtain
    \begin{eqnarray}
    &l_z=&-\ii \hbar \pdv{\phi},\\
    &l_x=&-\ii \hbar \left(-\sin \phi \pdv{\theta}-\cot\theta\cos\phi \pdv{\phi}\right),\label{4.3}\\
    &l_y=&-\ii \hbar \left(\cos \phi \pdv{\theta}-\cot\theta\sin\phi \pdv{\phi}\right).\label{4.4}
\end{eqnarray}
(2) \begin{equation}
    l_x=\left(l_++l_-\right)/2,\quad l_y=\left(l_+-l_-\right)/2\ii,
\end{equation}
\begin{equation}
    l^2=l_x^2+l_y^2+l_z^2=l_z^2+l_+l_-+l_-l_+.
\end{equation}
By \eqref{4.3} and \eqref{4.4}, 
\begin{equation}
    l^2=-\hbar^2\left[ \frac{1}{ \sin\theta} \frac{\partial}{\partial \theta} \left( \sin\theta \frac{\partial }{\partial \theta} \right)
+ \frac{1}{ \sin^2\theta} \frac{\partial^2 }{\partial \phi^2}\right].
\end{equation}
(3) To avoid confusion, we use $\mathbf{L}$ and $\mathbf{x}$ to represent $\mathbf{l}$ and $\mathbf{r}$ respectively.
\begin{equation}
\begin{aligned}
\mathbf{L}^2 &= \sum_{ijlmk} \varepsilon_{ijk}x_i p_j \varepsilon_{lmk}x_l p_m \\
&= \sum_{ijlm} (\delta_{il}\delta_{jm} - \delta_{im}\delta_{jl})x_i p_j x_l p_m \\
&= \sum_{ijlm} [\delta_{il}\delta_{jm}x_i (x_l p_j - \ii\hbar\delta_{jl})p_m - \delta_{im}\delta_{jl}x_i p_j(p_m x_l + \ii\hbar\delta_{lm})]  \\
&= \mathbf{x}^2 \mathbf{p}^2 - \ii\hbar \mathbf{x} \cdot \mathbf{p} - \sum_{ijlm} \delta_{im}\delta_{jl}[x_i p_m(x_l p_j - \ii\hbar\delta_{jl}) + \ii\hbar\delta_{lm}x_i p_j] \\
&= \mathbf{x}^2 \mathbf{p}^2 - (\mathbf{x} \cdot \mathbf{p})^2 + \ii\hbar \mathbf{x} \cdot \mathbf{p}.
\end{aligned}
\end{equation}
That is 
\begin{equation}
    -\frac{\hbar ^2}{2m}\nabla^2=-\frac{\hbar^2}{2m}\frac{1}{r^2} \frac{\partial }{\partial r} \left( r^2 \frac{\partial }{\partial r} \right)+\frac{\mathbf{L}^2}{2mr^2}.
\end{equation}
\end{problem}
\begin{problem}[Associated Legendre Polynomials]
    (1)
    \begin{equation}
        \frac{\dd}{\dd{z}}P^{|m|} (z)=\left(1-z^2\right)^{\frac{|m|}{2}-1}\left[-z|m|G(z)+\left(1-z^2\right)G'(z)\right].
    \end{equation}
    \begin{equation}
        \begin{aligned}
           \frac{\dd}{\dd{z}}\left[(1-z^2)\frac{\dd}{\dd{z}}P^{| m| }\right] =\left(1-z^2\right)^{\frac{| m| }{2}-1} \left(\left(z^2-1\right) | m|  \left(2 z
       G'(z)+G(z)\right)\right.\\
       \left.
       +z^2 | m| ^2 G(z)+\left(z^2-1\right) \left(\left(z^2-1\right)
       G''(z)+2 z G'(z)\right)\right).
        \end{aligned}
    \end{equation}
    Hence,
    \begin{equation}
        (1 - z^2)G'' - 2(|m| + 1)zG' + \left[\beta - |m|(|m| + 1)\right] G = 0.
    \end{equation}
    (2) 
    \begin{equation}
        (1-z^2)G''=\sum_{z=2}^{+\infty}n(n-1)a_n\left(z^{n-2}-z^n\right),
    \end{equation}
    \begin{equation}
        zG'=\sum_{n=1}^{+\infty}na_nz^n.
    \end{equation}
    Thus, 
    \begin{equation}
        \begin{aligned}
            \sum_{n=0}^{+\infty}\left\{(n+2)(n+1)a_{n+2}-n(n+1)a_n-2(|m|+1)na_n\right.\\
            \left.+\left[\beta-|m|(|m|+1)\right]a_n\right\}z^n=0.
        \end{aligned}
    \end{equation}
    Therefore,
    \begin{equation}
        a_{n+2}=\frac{\left(n+|m|\right)\left(n+|m|+1\right)-\beta}{(n+1)(n+2)}a_n.
    \end{equation}
    (3) If $\beta=l(l+1)$, then, when $n+\left|m\right|\ge l$, $a_{n+2}=0$. So $G$ becomes a polynomial.
\end{problem}
\begin{problem}[Generation function of Legendre Polynomials]
    (1) One has 
    \begin{equation}
        \frac{\pa T}{\pa t}=\sum_{n=0}^{+\infty}lP_l\left(z\right)t^{l-1}.
    \end{equation}
    \begin{equation}
        \frac{\pa}{\pa t}\left(\frac{1}{\sqrt{1-2tz+t^2}}\right)=\left(z-t\right)\left(1-2z+t^2\right)^{\frac{3}{2}}=\frac{z-t}{1-2zt+t^2}T.
    \end{equation}
    Compare the coefficients of $t^l$, we obtain,
    \begin{equation}\label{6.3}
        (l+1)P_{l+1}\left(z\right)-(2l+1)zP_l(z)+lP_{l-1}(z)=0.
    \end{equation}
    (2) \begin{equation}
        \frac{\pa T}{\pa z}=\sum_{n=0}^{+\infty}P_{l}'(z)t^l.
    \end{equation}
    \begin{equation}
        \frac{\pa }{\pa z}\left(\frac{1}{\sqrt{1-2tz+t^2}}\right)=\frac{t}{1-2zt+t^2}T.
    \end{equation}
    Hence,
    \begin{equation}\label{6.6}
        P'_{l+1}-2zP'_l+P'_{l-1}=P_l.
    \end{equation}
    (3) $l\times$\eqref{6.6}$-\frac{\dd}{\dd{z}}$ \eqref{6.3}:
    \begin{equation}\label{6.7}
        zP'_l-P'_{l-1}=lP_l.
    \end{equation}
    \eqref{6.6}$+$\eqref{6.7}:
    \begin{equation}\label{6.8}
        P'_{l+1}-zP'_l=(l+1)P_l.
    \end{equation}
    (4) \eqref{6.7}:$l\rightarrow l+1$, \eqref{6.8}$\times z$:
    \begin{equation}
        (z^2-1)P_l'=(l+1)(P_{l+1}-zP_l).
    \end{equation}
    So, 
    \begin{equation}
        \frac{\dd}{\dd{z}}\left[(1-z^2)\frac{\dd}{\dd{z}}P_l\right]=(l+1)\frac{\dd}{\dd{z}}\left(zP_l-P_{l+1}\right)=(l+1)(P_l+zP_l'-P_{l+1}').
    \end{equation}
    Plug in \eqref{6.8}, we obtain
    \begin{equation}\label{6.11}
        \frac{d}{dz} \left[ (1 - z^2) \frac{\dd P_l(z)}{\dd z} \right] + l(l + 1)P_l(z) = 0
    \end{equation}
    By \eqref{6.11}
    \begin{equation}\label{6.12}
        \dv{z} \left[ (1 - z^2) \left( P_n \dv{P_m}{z} - P_m \dv{P_n}{z} \right) \right] + \left[m(m+1) - n(n+1)\right] P_m P_n = 0
    \end{equation}
    Since $\left[1-z^2\right]_{z=\pm 1}=0$, the first term in \eqref{6.12} will be zero after integrating. So the second term must be zero after integrating if $n\not=m$, exact,
    \begin{equation}
        \int_{-1}^{1}P_n(z)P_m(z)=0,\quad n\not=m.
    \end{equation}
    (5) By \eqref{6.3}, 
    \begin{equation}
        lP_l-(2l-1)zP_{l-1}+(l-1)P_{l-2}=0,\quad zP_l=\frac{(l+1)P_l+lP_{l-1}}{2l+1}.
    \end{equation}
    Multiplying both sides by $P_l$, and integrate, we obtain,
    \begin{equation}
        l\int_{-1}^{1}P_l^2\dd{z}=(2l-1)\int_{-1}^{1}zP_[l-1]P_{l}\dd{z}=\frac{l(2l-1)}{2l+1}\int_{-1}^{1}P_{l-1}^2\dd{z}.
    \end{equation}
    Thus,
    \begin{equation}
        (2l+1)\int_{-1}^{1}P_l^2(z)\dd{z}=(2(l-1)+1)\int_{-1}^{1}P_{l-1}^2(z)\dd{z}=\int_{-1}^{1}P_0^2(z)\dd{z}=2.
    \end{equation}
    Therefore,
    \begin{equation}
        \int_{-1}^{1}P_{l}^2(z)\dd{z}=\frac{2}{2l+1}.
    \end{equation}
    
\end{problem}
\begin{problem}[Associated Legendre Polynomials]
    (1) 
    \begin{equation}
        \dv{z} P^{|m|}_l=\left(1-z^2\right)^{\frac{|m|}{2}-1}\left[(1-z^2)\frac{\dd^{|m|+1}}{\dd{z^{|m|+1}}}P_l-z|m|\frac{\dd^{|m|}}{\dd{z}^{|m|}}P_l\right].
    \end{equation}
    {\small
    \begin{equation}
        \frac{\dd^{|m|}}{\dd{z}^{|m|}}\left[(1-z^2)\frac{\dd{P_l}}{\dd{z}}\right]=(1-z^2)\frac{\dd^{|m|}}{\dd{z}^{|m|}}P_l-2|m|z\frac{\dd^{|m|-1}}{\dd{z}^{|m|-1}}\frac{\dd{P_l}}{\dd{z}}-|m|(|m|-1)\frac{\dd^{|m|-2}}{\dd{z}^{|m|-2}}\frac{\dd{P_l}}{\dd{z}}.
    \end{equation}
    }
    By \eqref{6.11},
    \begin{equation}
        -l(l+1)\frac{\dd^{|m|}}{\dd{z}^{|m|}}P_l=\dv{z}\frac{\dd^{|m|}}{\dd{z}^{|m|}}\left[(1-z^2)\frac{\dd{P_l}}{\dd{z}}\right].
    \end{equation}
    So,
    \begin{equation}
        \frac{\dd}{\dd z } \left[ (1 - z^2) \frac{\dd}{\dd z} P_l^{|m|}(z) \right] + \left[ l(l + 1) - \frac{m^2}{1 - z^2} \right] P_l^{|m|}(z) = 0.
    \end{equation}
    (2) {\small
    \begin{equation}
        \dv{z}\left[(1-z^2)\left(\dv{z}P_l^{|m|}P_{l'}^{|m|}-P_l^{|m|}\dv{z}P_{l'}^{|m|}\right)\right]+\left[l(l+1)-l'(l'+1)\right]P_l^{|m|}P_{l'}^{|m|}=0.
    \end{equation}}
    So,
    \begin{equation}
        \int_{-1}^{1}P_l^{|m|}P_{l'}^{|m|}\dd{z}=0.
    \end{equation}
    (4) Let $\frac{\dd^{|m|}}{\dd{z}^{|m|}}$ map at \eqref{6.3}, we obtain, 
    \begin{equation}
        z\frac{\dd^{|m|}}{\dd{z}^{|m|}}P_l+|m|\frac{\dd^{m-1}}{\dd{z}^{m-1}}P_l=\frac{l+1}{2l+1}\frac{\dd^{|m|}}{\dd{z}^{|m|}}P_{l+1}+\frac{1}{2l+1}\frac{\dd^{|m|}}{\dd{z}^{|m|}}P_{l-1}.
    \end{equation}
    Map $\frac{\dd^{|m|-1}}{\dd{z}^{|m|-1}}$ at $P_{l+1}'-P_{l-1}=(2l+1)P_l$, and deduce the term with $\frac{\dd^{|m|-1}}{\dd{z}^{|m|-1}}$, we obtain,
    \begin{equation}
        zP_l^{|m|}=\frac{l+|m|}{2l+1}P_{l-1}^{|m|}+\frac{l-|m|+1}{2l+1}P_{l+1}^{|m|}.
    \end{equation}
\end{problem}
\begin{problem}[Laguerre polynomials]
    (1) 
    \begin{equation}
        \sum_{\nu=0}^{+\infty} \left[ a_{\nu+1}\nu(\nu+1) +2(l+1)a_{\nu+1}(\nu+1)-a_{\nu}\nu+\left(\lambda-l-1\right)a_{\nu} \right]\xi^{\nu}=0.
    \end{equation}
    \begin{equation}
        a_{\nu+1}=\frac{\nu+l+1-\lambda}{(\nu+1)(2l+2+\nu)}a_{\nu}.
    \end{equation}
    \begin{equation}
        u(\xi)=\sum_{\nu=0}^{+\infty}\prod_{m=0}^{\nu-1}\frac{m+l+1-\lambda}{(m+1)(2l+2+m)}\xi^\nu.
    \end{equation}
    (2) When $\nu$ gets large, we have
    \begin{equation}
        \frac{\nu+l+1-\lambda}{(\nu+1)(2l+2+\nu)}\sim O(\frac{1}{\nu}).
    \end{equation}
    Its action is like $\frac{1/(n+1)!}{1/n!}$, so in the general case,
    \begin{equation}
        u(\xi)\sim \ee^\xi.
    \end{equation}
    When there exists $\nu$, such that $\nu+l+1-\lambda=0$, then the following term will be zero, and $u$ becomes a polynomial. So $\lambda$ should be a integer lager than $l+1$.
    (3) One has 
    \begin{equation}
        \frac{\pa U}{\pa u}=\sum_{m=0}^{+\infty}\frac{L_{m+1}(\xi)}{m!}u ^m.
    \end{equation}
    \begin{equation}
        \frac{\pa}{\pa u}\left[\frac{1}{1-u}e^{-\frac{\xi u}{1-u}}\right]=\frac{1}{1-u}e^{-\frac{\xi u}{1-u}}\frac{\pa}{\pa u}\ln\left[\frac{1}{1-u}e^{-\frac{\xi u}{1-u}}\right]=U\frac{1-u-\xi}{(1-u)^2}.
    \end{equation}
    Hence,
    \begin{equation}\label{8.3}
        L_{m+1}(\xi)+\left(\xi-2m-1\right)L_m(\xi)+m^2L_{m-1}(\xi)=0.
    \end{equation}
    \begin{equation}
        \frac{\pa U}{\pa \xi}=U\frac{\pa}{\pa \xi}\ln U=U\frac{-u}{1-u},
    \end{equation}
    so,
    \begin{equation}\label{8.5}
        L'_m(\xi)-mL_{m-1}'(\xi)+mL_{m-1}(\xi)=0.
    \end{equation}
    (4) Derivative of \eqref{8.3} shows that 
    \begin{equation}
        L_{m+1}'+L_m+(\xi -1-2m)L_m'+m^2L_{m-1}'=0.
    \end{equation}
    By \eqref{8.5} and \eqref{8.3}
    \begin{equation}
        L_{m+1}'-L_{m+1}+(2m+2-\xi)L_m+(\xi -1-m)L_m'=0,
    \end{equation}
    \begin{equation}
        L''_{m+1}-L_{m+1}'+L_m'(2m+2-\xi)+(\xi-1-m)L_m''=0.
    \end{equation}
    Plug in \eqref{8.5} after $m\rightarrow m+1$:
    \begin{equation}
        \xi L_m''+(1-\xi)L_m'+(m+1)L_m=0.
    \end{equation}
    (5) 
    \begin{equation}
        \frac{\dd^s}{\dd{\xi}^s}\left[\xi L_m''(\xi)\right]=s\frac{\dd^{s-1}}{\dd{\xi}^{s-1}}\left[L_m''(\xi)\right]+\xi\frac{\dd^s}{\dd{\xi}^s}\left[\xi L_m''(\xi)\right].
    \end{equation}
    \begin{equation}
        \frac{\dd^s}{\dd{\xi}^s}\left[(1-\xi) L_m'(\xi)\right]=-s\frac{\dd^{s-1}}{\dd{\xi}^{s-1}}\left[L_m'(\xi)\right]+(1-\xi)\frac{\dd^s}{\dd{\xi}^s}\left[\xi L_m'(\xi)\right].
    \end{equation}
    Therefore,
    \begin{equation}
        \xi {L_m^s}''(\xi)+\left(s+1-\xi\right){L_m^s}'(\xi)+\left(m-s\right) L_m^s(\xi)=0.
    \end{equation}



\end{problem}
\end{document}