\documentclass{article}
\newcommand{\mydate}{October 14, 2025}
\newcommand{\mytitle}{QM HW6}
\title{\textbf{\mytitle}}
\author{Jiete XUE}
\date{\mydate}
\usepackage{fancyhdr}
\pagestyle{fancy}
\fancyhf{}
\fancyhead[C]{\mytitle }
\fancyhead[R]{Jiete Xue}
\fancyhead[L]{\mydate}
\fancyfoot[C]{\thepage}
\usepackage{amsthm}
\usepackage{amsmath}
\usepackage{amssymb}
\usepackage{physics}
\usepackage{tikz}

%% 右矢
%\ket{\psi}          % 输出:|ψ⟩
%\ket{\psi(t)}       % 输出:|ψ(t)⟩
%
%% 左矢
%\bra{\phi}          % 输出:⟨φ|
%
%% 期望值
%\expval{\hat{A}}    % 输出:⟨Â⟩
%\expval{\hat{A}}{\psi}  % 输出:⟨ψ|Â|ψ⟩
%
%% 对易子
%\comm{\hat{A}}{\hat{B}}  % 输出:[Â, B̂]
\newtheoremstyle{1}{}{}{}{}{\bfseries}{}{\newline}{}
\theoremstyle{1}
\newtheorem{problem}{Problem}
\usepackage{chngcntr}
\counterwithin{equation}{problem}
\newcommand{\pa}{\partial}
\newcommand{\rn}[1]{\romannumeral #1\relax}
\newcommand{\Rn}[1]{\expandafter\@slowromancap\romannumeral#1@}

\begin{document}
\maketitle
\begin{problem}[Current, gauge transformation]
    (1) Use the substitution:
    \begin{equation}
        -i\hbar \nabla\longrightarrow -i\hbar \nabla -\frac{q\vec{A}}{c},
    \end{equation}
    the probability current will be:
    \begin{equation}
        \vec{j}=\frac{\hbar}{m}\mathrm{Im}\left(\psi^*\nabla \psi\right)\longrightarrow\frac{\hbar}{m}\mathrm{Im}\left(\psi^*\nabla \psi\right)-\frac{q}{mc} \vec{A}\left|\psi\right|^2.
    \end{equation}
    Hence,
    \begin{equation}
        \boxed{\vec{j}=\frac{\hbar}{m}\mathrm{Im}\left(\psi^*\nabla \psi\right)-\frac{q}{mc} \vec{A}\left|\psi\right|^2.}
    \end{equation}
    (2) 
    \begin{equation}
        A_\mu'=A_\mu+\pa_\mu f.
    \end{equation}
    Since the commutator is antisymmetric and the derivative is commutative,
    \begin{equation}
        \pa_{[\mu}\pa_{\nu]}f=0.
    \end{equation}
    Thus,
    \begin{equation}
        F_{\mu\nu}'=\pa_{[\mu}A'_{\nu]}=\pa_{[\mu}A_{\nu]}+\pa_{[\mu}\pa_{\nu]}f=F_{\mu\nu}.
    \end{equation}
    So, 
    \begin{equation}
        \mathbf{E}'=\mathbf{E},\ \mathbf{B}'=\mathbf{B}.
    \end{equation}
    (3) 
    \begin{equation}
        \frac{\pa \psi'}{\pa t}=i\frac{\pa \varphi}{\pa t}\psi'+e^{i\varphi}\frac{\pa \psi}{\pa t}
    \end{equation}
    Cancelate  $e^{i\varphi}$, and plug in Schrödinger equation, we should take
    \begin{equation}
        \boxed{\varphi=\frac{q}{\hbar c}f.}
    \end{equation}
    (4)
    \begin{equation}
        \rho'=e^{i\varphi}\psi e^{-i\varphi}\psi^*=\psi\psi^*=\rho.
    \end{equation}
    If we let 
    \begin{equation}
        \psi=\sqrt{\rho}e^{\frac{iS}{\hbar}},
    \end{equation}
    then,
    \begin{equation}
        \mathbf{j}=\frac{\rho}{m}\left(\nabla S-\frac{q\mathbf{A}}{c}\right).
    \end{equation}
    \begin{equation}
       S'=S+\hbar \varphi,\quad  \nabla S'-\frac{q\mathbf{A}'}{c}=\nabla S-\frac{q\mathbf{A}}{c}.
    \end{equation}
    So the probability current  is invariant under the gauge transformation.
    
\end{problem}
\begin{problem}[Landau gauge]
    (1)
    \begin{equation}
        \left(p_x-\frac{qB}{c}y\right)^2e^{ik_x x}=\left(\hbar k_x-\frac{qB}{c}y\right)^2 e^{ik_x x}.
    \end{equation}
    \begin{equation}
        \left[\frac{\left(\hbar k_x-\frac{qBy}{c}\right)^2}{2m}+(V(y)-E)\right]\phi_{k_x}(y)=\frac{\hbar^2}{2m}\phi_{k_x}''(y).
    \end{equation}
    We can define $H_y(k_x)$ as 
    \begin{equation}
        H_y(k_x)=\frac{\left(\hbar k_x-\frac{qBy}{c}\right)^2}{2m}+\frac{p_y^2}{2m}+V(y).
    \end{equation}
    (2) Note that 
    \begin{equation}
        \frac{\pa H_y(k_x)}{\pa k_x}=\frac{\hbar k_x-qBy/c}{m}.
    \end{equation}
    \begin{equation}
        I_x(x,k_x)=qJ=\frac{q\rho}{m}\left(\hbar k_x-\frac{qBy}{c}\right)=\frac{q}{L_x}\frac{\pa E_n}{\hbar \pa k_x}.
    \end{equation}
    (3) 
    \begin{equation}
    I_{x,n}=\frac{1}{2\pi}\int\dd{k_x}I_x(n,k_x)  =\frac{q^2}{h}\Delta V/L_x. 
    \end{equation}
    \begin{equation}
        \sigma_{xy}=\frac{I_x}{E_y}=\frac{q^2}{h}\nu.
    \end{equation}
    This result is not related to $V_{\text{imp}}$.
\end{problem}


\end{document}