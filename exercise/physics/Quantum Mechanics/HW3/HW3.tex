\documentclass{article}
\newcommand{\mydate}{September 23, 2025}
\newcommand{\mytitle}{QM HW3}
\title{\textbf{\mytitle}}
\author{Jiete XUE}
\date{\mydate}
\usepackage{fancyhdr}
\pagestyle{fancy}
\fancyhf{}
\fancyhead[C]{\mytitle }
\fancyhead[R]{Jiete Xue}
\fancyhead[L]{\mydate}
\fancyfoot[C]{\thepage}
\usepackage{amsthm}
\usepackage{amsmath}
\usepackage{amssymb}
\usepackage{physics}

%% 右矢
%\ket{\psi}          % 输出:|ψ⟩
%\ket{\psi(t)}       % 输出:|ψ(t)⟩
%
%% 左矢
%\bra{\phi}          % 输出:⟨φ|
%
%% 期望值
%\expval{\hat{A}}    % 输出:⟨Â⟩
%\expval{\hat{A}}{\psi}  % 输出:⟨ψ|Â|ψ⟩
%
%% 对易子
%\comm{\hat{A}}{\hat{B}}  % 输出:[Â, B̂]
\newtheoremstyle{1}{}{}{}{}{\bfseries}{}{\newline}{}
\theoremstyle{1}
\newtheorem{problem}{Problem}
\usepackage{chngcntr}
\counterwithin{equation}{problem}
\newcommand{\pa}{\partial}
\newcommand{\rn}[1]{\romannumeral #1\relax}
\newcommand{\Rn}[1]{\expandafter\@slowromancap\romannumeral#1@}

\begin{document}

\maketitle
\begin{problem}[A quantum particle in an infinitely deep potential well]
    (1) When $\left| x\right|>\frac{L}{2}$, $\psi(x)=0$. Now devoted exclusively to the case where $\left| x\right|<\frac{L}{2}$.By Schrödinger equation,
    \begin{equation}
       E\psi+\frac{\hbar^2}{2m}\frac{\dd^2 \psi}{\dd x^2}=0.
    \end{equation}
    Let $k=\sqrt{2mE/\hbar^2}$,
    \begin{equation}
        \psi(x)=A\sin(kx)+B\cos(kx).
    \end{equation}
    For odd parity, $B=0$. For even parity, $A=0$.
    Take the boundary condition
    \begin{equation}
        \psi(\pm \frac{L}{2})=0.
    \end{equation}
    \begin{equation}
        \cos(\frac{k^+L}{2})=0,\sin(\frac{k^-L}{2})=0.
    \end{equation}
    So, we have
    \begin{equation}
        k_n^+=\frac{\left(2n-1\right)\pi}{L},\ k_n^-=\frac{2n\pi}{L}.
    \end{equation}
    That is 
    \begin{equation}
        \boxed{E_n^+=\frac{\hbar^2 \pi^2(2n-1)^2}{2mL^2},\ E_n^-=\frac{\hbar^2 \pi^2 (2n)^2}{2mL^2}.}
    \end{equation}
    \begin{equation}
        \psi_n^+=A_n^+\cos\left(\frac{\left(2n-1\right)\pi x}{L}\right),\ \psi_n^-=A_n^-\sin\left(\frac{2n\pi x}{L}\right).
    \end{equation}
    Normalize the wave functions,
    \begin{equation}
       \boxed{ \psi_n^+=\sqrt{\frac{2}{L}}\cos\left(\frac{\left(2n-1\right)\pi x}{L}\right),\ \psi_n^-=\sqrt{\frac{2}{L}}\sin\left(\frac{2n\pi x}{L}\right)}.
    \end{equation}
    (2) \begin{eqnarray}
        \Psi(x,t)=\sqrt{\frac{1}{L}}\left[e^{-iE_1^+t}\cos(\frac{\pi x}{L})+e^{-iE_1^-t}\sin(\frac{2\pi x}{L})\right]\\
        =\boxed{\sqrt{\frac{1}{L}}\left[e^{-i \frac{\hbar^2 \pi^2}{2mL^2}t}\cos(\frac{\pi x}{L})+e^{-i \frac{2\hbar^2 \pi^2 }{mL^2}t}\sin(\frac{2\pi x}{L})\right]}.
    \end{eqnarray}
    (3) By symmetry,
        \begin{equation}
        \left \langle x \right \rangle =0 ,\ \left \langle p \right \rangle =0.
        \end{equation}
        \begin{equation}
            \left \langle x^2 \right \rangle =\int_{-\frac{L}{2} }^{\frac{L}{2} } \psi^*x^2\psi\ \dd x=\frac{L^2}{12}-\frac{L^2}{2n^2\pi ^2} .
        \end{equation}
        \begin{equation}
        \left \langle p^2 \right \rangle =\footnote{By Schrödinger equation: $\frac{p^2}{2m}\ket{\psi}=E\ket{\psi} $, then $\expval{p^2}=\expval{p^2}{\psi}=2mE$.}2mE= \frac{n^2\pi^2\hbar^2}{L^2}.
        \end{equation}
        So,
        \begin{equation}
            \boxed{\sqrt{\Delta x^2}=L\sqrt{\frac{1}{12}-\frac{1}{2n^2\pi^2}}},
        \end{equation}
        \begin{equation}
            \boxed{\sqrt{\Delta p^2}=\frac{n\pi \hbar}{L}},
        \end{equation}
        \begin{equation}
            \boxed{\sqrt{\Delta x^2}\sqrt{\Delta p^2}=\hbar \sqrt{\frac{n^2\pi^2}{12}-\frac{1}{2}}\ge\hbar\sqrt{\frac{\pi^2}{12}-\frac{1}{2}}>\frac{\hbar}{2}}.
        \end{equation}
\end{problem}
\begin{problem}[$\delta$-function]
    (1) Easy to check that 
    \begin{equation}
        \lim_{a\rightarrow0}\frac{1}{\pi}\frac{a}{x^2+a^2}=\lim_{a\rightarrow0}\frac{1}{a\sqrt{\pi}}e^{-\frac{x^2}{a^2}}=\left\{\begin{matrix}
            0&,x\not=0\\
            +\infty&,x=0
        \end{matrix}\ .\right.
    \end{equation}
    \begin{equation}
        \int_{-\epsilon}^{\epsilon}\frac{a}{x^2+a^2}\, \dd{x}=2\arctan(\frac{\epsilon}{a}).
    \end{equation}
    \begin{equation}
        \lim_{a\rightarrow0}\frac{2}{\pi}\arctan(\frac{\epsilon}{a})=1.
    \end{equation}
    \begin{equation}
        \lim_{a\rightarrow0}\frac{1}{a\sqrt{\pi}}\int_{-\epsilon}^{\epsilon}e^{-\frac{x^2}{a^2}}\, \dd{x}=\frac{\Gamma(\frac{1}{2})}{\sqrt{\pi}}=1.
    \end{equation}
    (2) \begin{equation}
        \frac{1}{x\pm i\epsilon}=\frac{x}{x^2+\epsilon^2}\mp\frac{i\epsilon}{x^2+\epsilon^2}.
    \end{equation}
    $\frac{x}{x^2+\epsilon^2}$ is an odd function, in the interval where $\left|x\right|<\epsilon$, it contributes $O(\epsilon)$. Once $\lim_{\epsilon\rightarrow0}\frac{x}{x^2+\epsilon^2}=\frac{1}{x}$ and by (1), we obtain
    \begin{equation}
        \boxed{\frac{1}{x\pm i\epsilon}=\mathcal{P}\left(\frac{1}{x}\right)\mp i\pi\delta(x).}
    \end{equation}
    (3) Let $A:=\{x_n\mid g(x_n)=0,g'(x_n)\not=0\}$. Around $x=x_n$,
    \begin{equation}
        g(x)=g(x_n)+g'(x_n)(x-x_n)=g'(x_n)(x-x_n).
    \end{equation}
    Thus, 
    \begin{equation}
        \delta[g(x)]=\delta(\sum_{x_n\in A}g'(x_n)(x-x_n))=\sum_{x_n\in A}\delta[g'(x_n)(x-x_n)].
    \end{equation}
    By $\delta(ax)=\frac{\delta(x)}{\left|a\right|}\ (a\not=0)$,
    \begin{equation}
        \boxed{\delta[g(x)]=\sum_{x_n\in A}\frac{\delta(x-x_n)}{\left|g'(x_n)\right|}.}
    \end{equation}
    In particular, 
    \begin{equation}
        \boxed{\delta(x^2-a^2)=\frac{\delta(x-a)+\delta(x+a)}{2a}.}
    \end{equation}
    (4) \begin{align}
        &\int_{-\infty}^{\infty}\frac{\dd}{\dd{x}}\left[\delta(x-a)\right]\left(x f(x)\right)\, \dd{x}\\
        =&\left.\delta(x-a)xf(x)\right|_{-\infty}^{\infty}-\int_{-\infty}^{\infty}\delta(x-a)\left[f(x)+xf'(x)\right]\, \dd{x}\\
        =&f(a)+af'(a).
    \end{align}
    \begin{align}
        &\int_{-\infty}^{\infty}\frac{\dd^2}{\dd{x}^2}\left[\delta(x-a)\right]\left(x^2 f(x)\right)\, \dd{x}\\
        =&\int_{-\infty}^{\infty}-\frac{\dd}{\dd{x}}\left[\delta(x-a)\right]\frac{\dd}{\dd{x}}\left[x^2 f(x)\right]\, \dd{x}\\
        =&\int_{-\infty}^{\infty}\delta(x-a)\frac{\dd^2}{\dd{x}^2}\left[x^2 f(x)\right]\, \dd{x}\\
        =&a^2f''(a)+4af'(a)+2f(a).
    \end{align}
\end{problem}
\begin{problem}[Momentum representation]
    (1)\begin{equation}
        \bra{x}\hat{p}^2\ket{\psi}=\bra{x}\hat{p}\left(\hat{p}\ket{\psi}\right)=-i\hbar\bra{x}\hat{p}\ket{\psi}=-\hbar^2\psi(x).
    \end{equation}
    (2) \begin{equation}
        \bra{x}\hat{H}\ket{\psi}=\bra{x}\hat{p}^2/2m+\hat{V}(x)\ket{\psi}=\left(-\frac{\hbar^2}{2m}\frac{\dd^2}{\dd{x}^2}+\frac{1}{2}m\omega^2\right)\psi(x).
    \end{equation}
    (3) \begin{equation}
        \bra{p}\hat{x}\ket{\psi}=\int\dd{p'}\bra{p}\hat{x}\ket{p'}\braket{p'}{\psi}
    \end{equation}
    \begin{equation}
        \bra{p}\hat{x}\ket{p'}=i\hbar\frac{\pa}{\pa p}\delta(p-p')=-i\hbar\frac{\pa}{\pa p'}\delta(p-p').
    \end{equation}
    Similar to Problem 2-(4),
    \begin{equation}
        \bra{p}\hat{x}\ket{\psi}=i\hbar\frac{\pa}{\pa p}\psi(p).
    \end{equation}
    (4) \begin{align}
        \bra{p}H\ket{\psi}=&\int\dd{p'}\bra{p}\frac{\hat{p^2}}{2m}\ket{p'}\bra{p'}\ket{\psi}+\frac{1}{2}m\omega^2\bra{p}\hat{x}^2\ket{\psi}\notag\\
        =&\int{\dd{p'}}\frac{p^2}{2m}\delta(p-p')\psi(p)-\frac{1}{2}m\omega^2\hbar^2\frac{\dd^2}{\dd{p}^2}\psi(p)\notag\\
        =&\left[\frac{p^2}{2m}-\frac{1}{2}m\omega^2\hbar^2\frac{\dd^2}{\dd{p}^2}\right]\psi(p).
    \end{align}
\end{problem}
\begin{problem}[Orbital angular momentum]\label{4}
    (1) \begin{align}
        [L_i,L_j]=&[\epsilon_{imn}x_mp_n,\epsilon_{jkl}x_kp_l]\notag\\
        =&\epsilon_{imn}\epsilon_{jkl}\left(x_mx_k[p_n,p_l]+x_m[p_n,x_k]p_l+x_k[x_m,p_l]p_n+[x_m,x_k]p_lp_n\right)\notag\\
        =&i\hbar\epsilon_{imn}\epsilon_{jln}(x_mp_l-x_lp_m)\notag\\
        =&i\hbar (x_ip_j-x_jp_i)\notag\\
        =&i\hbar \epsilon_{ijk}\epsilon_{kmn}x_mp_n\notag\\
        =&i\hbar \epsilon_{ijk}L_k.
    \end{align}
    \begin{equation}
        [L_jL_j,L_i]=\{L_j,[L_j,L_i]\}=i\hbar\epsilon_{jik}\{L_j,L_k\}=0.
    \end{equation}
    (2) \begin{equation}
        [L_i,x_j]=\epsilon_{imn}\left(x_m[p_n,x_j]+[x_m,x_j]p_n\right)=\epsilon_{imn}\left(-i\hbar x_m\delta_{nj}\right)=i\hbar\epsilon_{ijk}x_k.
    \end{equation}
    \begin{equation}
        [L_i,p_j]=\epsilon_{imn}\left(x_m[p_n,p_j]+[x_m,p_j]p_n\right)=\epsilon_{imn}\left(i\hbar p_n\delta_{mj}\right)=i\hbar\epsilon_{ijk}p_k.
    \end{equation}
    (3) \begin{equation}
        [L_i,x_j x_j]=\{x_j,[L_i,x_j]\}=i\hbar\epsilon_{ijk}\{x_j,x_k\}=0.
    \end{equation}
    The last equality holds since $\epsilon_{ijk}$ is antisymmetric and $\{x_j,x_k\}$ is symmetric. Similarly,
    \begin{equation}
        [L_i,p_j p_j]=\{p_j,[L_i,p_j]\}=i\hbar\epsilon_{ijk}\{p_j,p_k\}=0.
    \end{equation}
    (4) In coordinate representation, 
    \begin{equation}
        L_i=i\hbar\epsilon_{ijk}x_j\frac{\dd}{\dd{x_k}}.
    \end{equation}
    In momentum representation,
    \begin{equation}
        L_i=i\hbar\epsilon_{ijk}p_k\frac{\dd}{\dd{p_j}}.
    \end{equation}
\end{problem}
\begin{problem}[Complete set of Mechanical variables]
    We have the fact that 
    \begin{equation}
        [A,f(B)]=\frac{\pa f(B)}{\pa B}[A,B].
    \end{equation}
    if $[A,B]$ is communicative with any operator.
    \begin{equation}
        H=\frac{p^2}{2m}-\frac{e^2}{r}
    \end{equation}
    By Problem ~\ref{4}, we have
    \begin{equation}
        [L_i,H]=\frac{[L_i,p^2]}{2m}-e^2\left[L_i,\frac{1}{\sqrt{r^2}}\right]=0.
    \end{equation}
    \begin{align}
        [L^2,H]=&\frac{[L^2,p^2]}{2m}-e^2\left[L^2,\frac{1}{\sqrt{r^2}}\right]\notag\\
        =&\frac{\{L_i,[L_i,p^2]\}}{2m}-e^2\{L_i,[L_i,\frac{1}{\sqrt{r^2}}]\}\notag\\
        =&0.
    \end{align}
    Therefore $L_i,L^2$ are compatible with $H$.
\end{problem}
\begin{problem}[Gaussian and uncertainty principle]
     Uniform probability distribution:
    \begin{equation}
        \int_{-\infty}^{+\infty}\dd{x}\psi^*(x)\psi(x)=AA^*\frac{\Gamma\left(\frac{1}{2}\right)}{\left(l^{-2}\right)^{\frac{1}{2}}}=1.
    \end{equation}
    So,
    \begin{equation}
        \boxed{A=\sqrt{\frac{1}{\sqrt{\pi l}}},}
    \end{equation}
    if we choose a positive real number as the coefficient.
    \newline
    (1) By the symmetry, easy to find 
    \begin{equation}
        \expval{x}=\expval{p}=0.
    \end{equation}
    Then,
    \begin{align}
        \expval{(\Delta x)^2}=\expval{x^2}=&\int_{-\infty}^{+\infty}\dd{x}\psi^*(x)x^2\psi(x)\notag\\
        =&\int_{-\infty}^{+\infty}A^2e^{-\frac{x^2}{l^2}}x^2\dd{x}\notag\\
        =&A^2l^3\Gamma\left(\frac{3}{2}\right)\notag\\
        =&\boxed{\frac{1}{2}l^2}.
    \end{align}
    \begin{align}
        \expval{\left(\Delta p\right)^2}=\expval{p^2}=&\int_{-\infty}^{+\infty}\dd{x}\psi^*(x)p^2\psi(x)\notag\\
        =&\int_{-\infty}^{+\infty}A^2e^{-\frac{x^2}{2l^2}}\left(-\hbar^2\frac{\dd^2}{\dd{x^2}}\right)e^{-\frac{x^2}{2l^2}}\dd{x}\notag\\
        =&\frac{A^2\hbar^2}{l}\Gamma\left(\frac{3}{2}\right)\notag\\
        =&\boxed{\frac{\hbar^2}{2l^2}}.
    \end{align}
    Hence, 
    \begin{equation}
        \boxed{\sqrt{\expval{\left(\Delta x\right)^2}}\sqrt{\expval{\left(\Delta p\right)^2}}=\frac{\hbar}{2}.}
    \end{equation}
    It reaches the lower bound of the uncertainty principle.
    \newline
    (2)\begin{equation}
        \psi(p)=\frac{1}{\sqrt{2\pi\hbar}}\int_{-\infty}^{+\infty}\dd{x}\psi(x)e^{-i\frac{px}{\hbar}}=\frac{Al}{\sqrt{\hbar}}e^{-\frac{p^2l^2}{2\hbar^2}}.
    \end{equation}
    \begin{align}
       \expval{\left(\Delta p\right)^2}=\expval{p^2}=&\int_{-\infty}^{+\infty}\dd{p}\psi^*(p)p^2\psi(p)\notag\\ 
       =&\frac{A^2l^2}{\hbar}\frac{\hbar^3}{l^3}\Gamma\left(\frac{3}{2}\right)\notag\\
       =&\boxed{\frac{\hbar^2}{2l^2}}.
    \end{align}
    \begin{align}
        \expval{\left(\Delta x\right)^2}=\expval{x^2}=&\int_{-\infty}^{+\infty}\dd{p}\psi^*(p)x^2\psi(p)\notag\\
        =&\int_{-\infty}^{+\infty}\frac{A^2l^2}{\hbar}e^{-\frac{p^2l^2}{2\hbar^2}}\left(-\hbar^2\frac{\dd^2}{\dd{p^2}}\right)´e^{-\frac{p^2l^2}{2\hbar^2}}\dd{p}\notag\\
        =&\boxed{\frac{1}{2}l^2}.
    \end{align}
    Still,
    \begin{equation}
        \boxed{\sqrt{\expval{\left(\Delta x\right)^2}}\sqrt{\expval{\left(\Delta p\right)^2}}=\frac{\hbar}{2}.}
    \end{equation}
\end{problem}
\end{document}