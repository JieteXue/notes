\documentclass{article}
\newcommand{\mydate}{September 16, 2025}
\newcommand{\mytitle}{QM HW3}
\title{\textbf{\mytitle}}
\author{Jiete XUE}
\date{\mydate}
\usepackage{fancyhdr}
\pagestyle{fancy}
\fancyhf{}
\fancyhead[C]{\mytitle }
\fancyhead[R]{Jiete Xue}
\fancyhead[L]{\mydate}
\fancyfoot[C]{\thepage}
\usepackage{amsthm}
\usepackage{amsmath}
\usepackage{amssymb}
\usepackage{physics}

%% 右矢
%\ket{\psi}          % 输出:|ψ⟩
%\ket{\psi(t)}       % 输出:|ψ(t)⟩
%
%% 左矢
%\bra{\phi}          % 输出:⟨φ|
%
%% 期望值
%\expval{\hat{A}}    % 输出:⟨Â⟩
%\expval{\hat{A}}{\psi}  % 输出:⟨ψ|Â|ψ⟩
%
%% 对易子
%\comm{\hat{A}}{\hat{B}}  % 输出:[Â, B̂]
\newtheoremstyle{1}{}{}{}{}{\bfseries}{}{\newline}{}
\theoremstyle{1}
\newtheorem{problem}{Problem}
\usepackage{chngcntr}
\counterwithin{equation}{problem}
\newcommand{\pa}{\partial}
\newcommand{\rn}[1]{\romannumeral #1\relax}
\newcommand{\Rn}[1]{\expandafter\@slowromancap\romannumeral#1@}

\begin{document}

\maketitle
\begin{problem}[A quantum particle in an infinitely deep potential well]
    (1) When $\left| x\right|>\frac{L}{2}$, $\psi(x)=0$. Now devoted exclusively to the case where $\left| x\right|<\frac{L}{2}$.By Schrödinger equation,
    \begin{equation}
       E\psi+\frac{\hbar^2}{2m}\frac{\dd^2 \psi}{\dd x^2}=0.
    \end{equation}
    Let $k=\sqrt{2mE/\hbar^2}$,
    \begin{equation}
        \psi(x)=A\sin(kx)+B\cos(kx).
    \end{equation}
    For odd parity, $B=0$. For even parity, $A=0$.
    Take the boundary condition
    \begin{equation}
        \psi(\pm \frac{L}{2})=0.
    \end{equation}
    \begin{equation}
        \cos(\frac{k^+L}{2})=0,\sin(\frac{k^-L}{2})=0.
    \end{equation}
    So, we have
    \begin{equation}
        k_n^+=\frac{\left(2n-1\right)\pi}{L},\ k_n^-=\frac{2n\pi}{L}.
    \end{equation}
    That is 
    \begin{equation}
        \boxed{E_n^+=\frac{\hbar^2 \pi^2(2n-1)^2}{2mL^2},\ E_n^-=\frac{\hbar^2 \pi^2 (2n)^2}{2mL^2}.}
    \end{equation}
    \begin{equation}
        \psi_n^+=A_n^+\cos\left(\frac{\left(2n-1\right)\pi x}{L}\right),\ \psi_n^-=A_n^-\sin\left(\frac{2n\pi x}{L}\right).
    \end{equation}
    Normalize the wave functions,
    \begin{equation}
       \boxed{ \psi_n^+=\sqrt{\frac{2}{L}}\cos\left(\frac{\left(2n-1\right)\pi x}{L}\right),\ \psi_n^-=\sqrt{\frac{2}{L}}\sin\left(\frac{2n\pi x}{L}\right)}.
    \end{equation}
    (2) \begin{eqnarray}
        \Psi(x,t)=\sqrt{\frac{1}{L}}\left[e^{-iE_1^+t}\cos(\frac{\pi x}{L})+e^{-iE_1^-t}\sin(\frac{2\pi x}{L})\right]\\
        =\boxed{\sqrt{\frac{1}{L}}\left[e^{-i \frac{\hbar^2 \pi^2}{2mL^2}t}\cos(\frac{\pi x}{L})+e^{-i \frac{2\hbar^2 \pi^2 }{mL^2}t}\sin(\frac{2\pi x}{L})\right]}.
    \end{eqnarray}
    (3) By symmetry,
        \begin{equation}
        \left \langle x \right \rangle =0 ,\ \left \langle p \right \rangle =0.
        \end{equation}
        \begin{equation}
            \left \langle x^2 \right \rangle =\int_{-\frac{L}{2} }^{\frac{L}{2} } \psi^*x^2\psi\ \dd x=\frac{L^2}{12}-\frac{L^2}{2n^2\pi ^2} .
        \end{equation}
        \begin{equation}
        \left \langle p^2 \right \rangle =\footnote{By Schrödinger equation: $\frac{p^2}{2m}\ket{\psi}=E\ket{\psi} $, then $\expval{p^2}=\expval{p^2}{\psi}=2mE$.}2mE= \frac{n^2\pi^2\hbar^2}{L^2}.
        \end{equation}
        So,
        \begin{equation}
            \boxed{\sqrt{\Delta x^2}=L\sqrt{\frac{1}{12}-\frac{1}{2n^2\pi^2}}},
        \end{equation}
        \begin{equation}
            \boxed{\sqrt{\Delta p^2}=\frac{n\pi \hbar}{L}},
        \end{equation}
        \begin{equation}
            \boxed{\sqrt{\Delta x^2}\sqrt{\Delta p^2}=\hbar \sqrt{\frac{n^2\pi^2}{12}-\frac{1}{2}}\ge\hbar\sqrt{\frac{\pi^2}{12}-\frac{1}{2}}>\frac{\hbar}{2}}.
        \end{equation}
\end{problem}
\end{document}