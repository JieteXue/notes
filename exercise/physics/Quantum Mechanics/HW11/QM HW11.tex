\documentclass{article}
\newcommand{\mydate}{December 18, 2025}
\newcommand{\mytitle}{QM HW11}
\title{\textbf{\mytitle}}
\author{Jiete XUE}
\date{\mydate}
\usepackage{fancyhdr}
\pagestyle{fancy}
\fancyhf{}
\fancyhead[C]{\mytitle }
\fancyhead[R]{Jiete Xue}
\fancyhead[L]{\mydate}
\fancyfoot[C]{\thepage}
\usepackage{amsthm}
\usepackage{amsmath}
\usepackage{amssymb}
\usepackage{bm}
\usepackage{enumitem}
\usepackage{physics}
\usepackage{tikz}
\usetikzlibrary{arrows.meta, decorations.markings}
\usepackage{float}

%% 右矢
%\ket{\psi}          % 输出:|ψ⟩
%\ket{\psi(t)}       % 输出:|ψ(t)⟩
%
%% 左矢
%\bra{\phi}          % 输出:⟨φ|
%
%% 期望值
%\expval{\hat{A}}    % 输出:⟨Â⟩
%\expval{\hat{A}}{\psi}  % 输出:⟨ψ|Â|ψ⟩
%
%% 对易子
%\comm{\hat{A}}{\hat{B}}  % 输出:[Â, B̂]
\newtheoremstyle{1}{}{}{}{}{\bfseries}{}{\newline}{}
\theoremstyle{1}
\newtheorem{problem}{Problem}
\newtheorem{solution}{Solution}
\usepackage{chngcntr}
\counterwithin{equation}{problem}

\setlist[enumerate]{label=(\arabic*), leftmargin=*, align=left}

\newcommand{\pa}{\partial}
\newcommand{\rn}[1]{\romannumeral #1\relax}
\newcommand{\Rn}[1]{\expandafter\@slowromancap\romannumeral#1@}
\newcommand{\ii}{\mathrm{i}}
\newcommand{\ee}{\mathrm{e}}

\begin{document}
\maketitle
\begin{problem}
    We need to solve the equation:
    \begin{equation}
        \begin{cases}
            i \frac{\pa}{\pa t} G(x,t)=-\frac{\hbar^2}{2m}\frac{\pa^2}{\pa x^2}G(x,t)\\
            G(x,0)=-i \delta(x).
        \end{cases}
    \end{equation}

Let the Laplace transform of \( G(x,t) \) with respect to \( t \) be

\begin{equation}
    \widetilde{G}(x,s) = \mathcal{L}\{ G(x,t) \} = \int_{0}^{\infty} e^{-s t} \, G(x,t) \dd{t},
\end{equation}
where \( s \) is a complex parameter with \( \Re(s) > 0 \).

\begin{equation}
    \mathcal{L}\left\{ i \frac{\partial G}{\partial t} \right\} 
    = i \big[ s \widetilde{G}(x,s) - G(x,0) \big]=i s \widetilde{G}(x,s) - \delta(x).
\end{equation}
The right-hand side transforms as:

\begin{equation}
    \mathcal{L}\left\{ -\frac{\hbar^2}{2m} \frac{\partial^2 G}{\partial x^2} \right\}
    = -\frac{\hbar^2}{2m} \frac{\partial^2 \widetilde{G}(x,s)}{\partial x^2}.
\end{equation}
Thus the transformed equation is:
\begin{equation}
    \frac{\partial^2 \widetilde{G}}{\partial x^2} - k^2 \widetilde{G} = \frac{2m}{\hbar^2} \, \delta(x).
\end{equation}
where $k = \sqrt{\frac{2m i s}{\hbar^2}}$. We choose the branch such that \( \Re(k) > 0 \).

The solution decaying as \( |x| \to \infty \) is:

\begin{equation}
    \widetilde{G}(x,s) = -\frac{m}{\hbar^2 k} e^{-k |x|}= -\sqrt{\frac{m}{2 i \hbar^2 s}} \, e^{-\sqrt{\frac{2m i s}{\hbar^2}} \, |x|}.
\end{equation}

We recognize the Laplace transform pair (from tables):
\begin{equation}
    \mathcal{L}^{-1}\left\{ \frac{e^{-a\sqrt{s}}}{\sqrt{s}} \right\} = \frac{1}{\sqrt{\pi t}} \, e^{-a^2/(4t)}, \quad a > 0.
\end{equation}
So,

\begin{equation}
    \boxed{ G(x,t) = - e^{-i\pi/4} \sqrt{\frac{m}{2\pi \hbar^2 t}} \; \exp\left( \frac{i m x^2}{2\hbar^2 t} \right) }.
\end{equation}
\end{problem}
\begin{problem}

We start from the definition of the propagator in quantum mechanics:
\begin{equation}
iG(x_b, t_b; x_a, t_a) = \langle x_b | e^{-iH(t_b - t_a)} | x_a \rangle,
\end{equation}
with \( H = \frac{p^2}{2m} \) for a free particle.

Divide the time interval into \( N \) equal segments:
\begin{equation}
\Delta t = \frac{t_b - t_a}{N}, \quad t_j = t_a + j\Delta t, \quad j = 0,1,\dots,N,
\end{equation}
where \( t_0 = t_a,\; t_N = t_b \), and correspondingly \( x_0 = x_a,\; x_N = x_b \).

Insert \( N-1 \) completeness relations \( \int dx_j |x_j\rangle\langle x_j| = 1 \):
\begin{equation}
iG(x_b, t_b; x_a, t_a) = \int dx_1 \dots dx_{N-1} \prod_{j=1}^N \langle x_j | e^{-iH\Delta t} | x_{j-1} \rangle.
\end{equation}

For small \(\Delta t\), we use the approximation \( e^{-iH\Delta t} \approx e^{-i\frac{p^2}{2m}\Delta t} \) (since \( V=0 \)):
\begin{equation}
\langle x_j | e^{-i\frac{p^2}{2m}\Delta t} | x_{j-1} \rangle
= \int \frac{dp}{2\pi} \, e^{ip(x_j - x_{j-1})} e^{-i\frac{p^2}{2m}\Delta t}.
\end{equation}

This is a Gaussian integral. Using the formula
\[
\int_{-\infty}^{\infty} dp \, e^{-ap^2 + bp} = \sqrt{\frac{\pi}{a}} \, e^{b^2/(4a)}, \quad \text{Re}(a) > 0,
\]
with \( a = \frac{i\Delta t}{2m},\; b = i(x_j - x_{j-1}) \), we get:
\begin{equation}
\langle x_j | e^{-i\frac{p^2}{2m}\Delta t} | x_{j-1} \rangle
= \sqrt{\frac{m}{2\pi i \Delta t}} \,
\exp\left[ i\frac{m}{2} \frac{(x_j - x_{j-1})^2}{\Delta t} \right].
\end{equation}

Thus the propagator becomes:
\begin{equation}
iG(x_b, t_b; x_a, t_a) = \left( \frac{m}{2\pi i \Delta t} \right)^{\!N/2}
\int \prod_{j=1}^{N-1} dx_j \,
\exp\left[ i\sum_{j=1}^N \frac{m}{2} \frac{(x_j - x_{j-1})^2}{\Delta t} \right].
\end{equation}

We now separate the classical path from fluctuations. Let
\begin{equation}
x_j = x_{\text{cl}}(t_j) + \delta x_j,
\end{equation}
where the classical path for a free particle is linear:
\begin{equation}
x_{\text{cl}}(t) = x_a + \frac{x_b - x_a}{t_b - t_a}(t - t_a).
\end{equation}

The boundary conditions are \(\delta x_0 = \delta x_N = 0\). The discretized action is:
\begin{equation}
S_N = \sum_{j=1}^N \frac{m}{2} \frac{(x_j - x_{j-1})^2}{\Delta t}
= S_{\text{cl}} + S_{\text{fl}},
\end{equation}
where
\begin{equation}
S_{\text{cl}} = \frac{m}{2} \frac{(x_b - x_a)^2}{t_b - t_a},
\end{equation}
and the fluctuation part is
\begin{equation}
S_{\text{fl}} = \sum_{j=1}^N \frac{m}{2} \frac{(\delta x_j - \delta x_{j-1})^2}{\Delta t}.
\end{equation}

The linear cross term vanishes because the classical path satisfies the equation of motion.

The propagator now factorizes:
\begin{equation}
iG(x_b, t_b; x_a, t_a) = \left( \frac{m}{2\pi i \Delta t} \right)^{\!N/2}
e^{iS_{\text{cl}}} \int \prod_{j=1}^{N-1} d(\delta x_j) \,
e^{i S_{\text{fl}}}.
\end{equation}

The fluctuation integral is Gaussian:
\begin{equation}
\int \prod_{j=1}^{N-1} d(\delta x_j) \,
\exp\left[ i \frac{m}{2\Delta t} \sum_{j=1}^N (\delta x_j - \delta x_{j-1})^2 \right]
= \left( \frac{2\pi i \Delta t}{m} \right)^{\!(N-1)/2} \frac{1}{\sqrt{N}}.
\end{equation}
This result follows from evaluating the determinant of the tridiagonal matrix
\[
M_{jk} = \frac{m}{i\Delta t}
\begin{pmatrix}
2 & -1 & & \\
-1 & 2 & \ddots & \\
& \ddots & \ddots & -1 \\
& & -1 & 2
\end{pmatrix}_{(N-1)\times(N-1)},
\]
with \(\det M = N\left( \frac{m}{i\Delta t} \right)^{\!N-1}\).

Putting everything together:
\begin{equation}
iG(x_b, t_b; x_a, t_a) =
\left( \frac{m}{2\pi i \Delta t} \right)^{\!N/2}
\left( \frac{2\pi i \Delta t}{m} \right)^{\!(N-1)/2} \frac{1}{\sqrt{N}}
e^{iS_{\text{cl}}}.
\end{equation}

Using \( N\Delta t = t_b - t_a \), we simplify:
\begin{equation}
iG(x_b, t_b; x_a, t_a) =
\sqrt{\frac{m}{2\pi i (t_b - t_a)}} \,
\exp\left[ i \frac{m}{2} \frac{(x_b - x_a)^2}{t_b - t_a} \right].
\end{equation}

Thus the free propagator obtained via the path integral method is
\begin{equation}
\boxed{ G(x_b, t_b; x_a, t_a) = -i\sqrt{\frac{m}{2\pi i (t_b - t_a)}} \,
\exp\left[ i \frac{m}{2} \frac{(x_b - x_a)^2}{t_b - t_a} \right] }.
\end{equation}
This matches the result from Problem 1 (solving the Schrödinger equation directly).
\end{problem}
\begin{problem}
    For infinitesimal $\varepsilon$, we have
    \begin{equation}
        e^{-\varepsilon H}=e^{-\frac{\varepsilon V(x)}{2}}e^{- \frac{\varepsilon p^2}{2m}}e^{-\frac{\varepsilon V(x)}{2}} +\mathcal{O}(\varepsilon^3).
    \end{equation}
    \begin{equation}
        \langle x_{k+1} | e^{-\varepsilon H} | x_k \rangle
\approx \sqrt{\frac{m}{2\pi \varepsilon}} 
\exp\left[ -\frac{m}{2\varepsilon} (x_{k+1} - x_k)^2 
- \varepsilon V(x_k) \right].
    \end{equation}
    So,
    \begin{equation}
        G(x_N, x_0; \beta) = 
\left( \frac{m}{2\pi \varepsilon} \right)^{\frac{N}{2}}
\int \prod_{k=1}^{N-1} dx_k \;
\exp\left[ -\sum_{k=0}^{N-1} 
\left( \frac{m}{2\varepsilon} (x_{k+1} - x_k)^2 
+ \varepsilon V(x_k) \right) \right].
    \end{equation}
    Taking the limit when $\varepsilon \rightarrow 0$, $N\rightarrow +\infty$, and let $x_N$ to be same as $x_0$, we get
    \begin{equation}
        G(x_0, x_0; \beta) = 
\int_{x(0)=x_0}^{x(\beta)=x_0} \mathcal{D}x(\tau) \; e^{-S[x]}.
    \end{equation}
    Where,
    \begin{equation}
        S[x] = \int_0^\beta \dd{t} \; \left[ \frac{m}{2} \dot{x}^2 + V(x) \right].
    \end{equation}
Then, by definition,
\begin{equation}
    Z(\beta) = \int \dd{x} \, G(x, x; \beta)
= \int \dd{x}
\int_{x(0)=x}^{x(\beta)=x} \mathcal{D}x(\tau) \; e^{-S[x]}.
\end{equation}

\end{problem}
\begin{problem}
    The matrices of spin-1 operators are:
    \begin{equation}
        S_x = \frac{1}{\sqrt{2}} 
\begin{pmatrix}
0 & 1 & 0 \\
1 & 0 & 1 \\
0 & 1 & 0
\end{pmatrix}
,\quad
S_y = \frac{1}{\sqrt{2}}
\begin{pmatrix}
0 & -i & 0 \\
i & 0 & -i \\
0 & i & 0
\end{pmatrix}
,\quad
S_z = 
\begin{pmatrix}
1 & 0 & 0 \\
0 & 0 & 0 \\
0 & 0 & -1
\end{pmatrix}
    \end{equation}
Let 
\begin{equation}
    \ket{\hat{n},m}=e^{-i \phi S_z} e^{-i \theta S_y}\ket{\hat{z},m}.
\end{equation}
Then,
\begin{equation}
    \begin{split}
        \nabla_{\hat{n}}\ket{\hat{n},m}&=\left(\frac{\pa}{\pa \theta}\ket{\hat{n},m},\frac{1}{\sin \theta}\frac{\pa }{\pa \phi} \ket{\hat{n},m}\right)\\
        &=\left(-i S_y \ket{\hat{n},m},-\frac{i }{\sin \theta}S_z \ket{\hat{n},m}\right).
    \end{split}
\end{equation}
Since
\begin{equation}
    [S_z, S_y]= - i S_x, [S_x , S_y]= i S_z.
\end{equation}
By Baker-Hausdorff lemma,
\begin{equation}
    e^{i \theta S_y} S_z e^{-i \theta S_y}= \cos \theta S_z - i \sin \theta S_x.
\end{equation}
So,
\begin{equation}
    A_\theta=-i\expval{-i S_y}{\hat{n},m}=-i\expval{e^{i \phi S_z} (-i S_y ) e^{-i \phi S_z}}{\hat{z},m}=0.
\end{equation}
\begin{equation}
    A_\phi =-i \expval{-\frac{i}{\sin \theta}S_z}{\hat{n},m}= -\frac{1}{\sin \theta} \expval{S_z}{\hat{n},m}=-m \cot \theta.
\end{equation}
Thus,
\begin{equation}
    \hat{\mathbf{r}}\cdot(\nabla \times \mathbf{A})=-m.
\end{equation}
Therefore,
\begin{equation}
    q=-4\pi m.
\end{equation}
\end{problem}
\end{document}



