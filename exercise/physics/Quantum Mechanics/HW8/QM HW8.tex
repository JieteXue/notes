\documentclass{article}
\newcommand{\mydate}{November 7, 2025}
\newcommand{\mytitle}{QM HW8}
\title{\textbf{\mytitle}}
\author{Jiete XUE}
\date{\mydate}
\usepackage{fancyhdr}
\pagestyle{fancy}
\fancyhf{}
\fancyhead[C]{\mytitle }
\fancyhead[R]{Jiete Xue}
\fancyhead[L]{\mydate}
\fancyfoot[C]{\thepage}
\usepackage{amsthm}
\usepackage{amsmath}
\usepackage{amssymb}
\usepackage{physics}
\usepackage{tikz}
\usepackage{float}

%% 右矢
%\ket{\psi}          % 输出:|ψ⟩
%\ket{\psi(t)}       % 输出:|ψ(t)⟩
%
%% 左矢
%\bra{\phi}          % 输出:⟨φ|
%
%% 期望值
%\expval{\hat{A}}    % 输出:⟨Â⟩
%\expval{\hat{A}}{\psi}  % 输出:⟨ψ|Â|ψ⟩
%
%% 对易子
%\comm{\hat{A}}{\hat{B}}  % 输出:[Â, B̂]
\newtheoremstyle{1}{}{}{}{}{\bfseries}{}{\newline}{}
\theoremstyle{1}
\newtheorem{problem}{Problem}
\usepackage{chngcntr}
\counterwithin{equation}{problem}
\newcommand{\pa}{\partial}
\newcommand{\rn}[1]{\romannumeral #1\relax}
\newcommand{\Rn}[1]{\expandafter\@slowromancap\romannumeral#1@}
\newcommand{\ii}{\mathrm{i}}
\newcommand{\ee}{\mathrm{e}}

\begin{document}
\maketitle
\begin{problem}[Derivative $x^H,p^H$ in harmonic oscillator case]
    We have 
    \begin{equation}
        [H,x]=-\ii \hbar \frac{p}{m},\ \ [H,p]=\ii \hbar m\omega^2{x}.
    \end{equation}
    Thus,
    \begin{equation}
        \underset{n\text{ copies}}{\underbrace{[H,\ldots[H,x]]}}=\begin{cases}
            -\ii \hbar^{2k-1}\omega^{2k-2} \frac{p}{m}&,n=2k-1\\
            \hbar^{2k} \omega^{2k} x&,n=2k\\
        \end{cases}
    \end{equation}
    \begin{equation}
        \underset{n\text{ copies}}{\underbrace{[H,\ldots[H,p]]}}=\begin{cases}
           \ii \hbar^{2k-1} m \omega^{2k}\frac{p}{m}&,n=2k-1\\
             \hbar^{2k} \omega^{2k}x&,n=2k\\
        \end{cases}
    \end{equation}
    By Baker-Hausdorff lemma, we have
    \begin{equation}
        \begin{aligned}
            x(t)=\ee^\frac{\ii H t}{\hbar}x\ee^{-\frac{\ii H t}{\hbar}}&=\sum_{n=0}^{+\infty}\frac{1}{n!}\left(\frac{\ii}{\hbar}\right)^n\underset{n\text{ copies}}{\underbrace{[H,\ldots[H,x]]}}\\
            &=\frac{p}{m\omega}\sum_{k=0}^{+\infty}\frac{1}{(2k-1)!}(-1)^{k-1}\omega^{2k-1}+\sum_{k=0}^{+\infty}\frac{1}{(2k)!}\omega^{2k}x\\
            &=x\cos(\omega t)+\frac{p}{m\omega}\sin(\omega t).
        \end{aligned}
    \end{equation}
    Similarly,
    \begin{equation}
        p(t)=\ee^\frac{\ii H t}{\hbar}p\ee^{-\frac{\ii H t}{\hbar}}=-m\omega x\sin(\omega t)+p\cos(\omega t).
    \end{equation}
\end{problem}
\begin{problem}
    We decompose the Hamiltonian \( H^S \) of the Schr\"{o}dinger picture into the free part \( H_0 \) and the perturbative part \( V \) as
\[
H^S = H_0 + V,
\]
where \( H_0 \) is independent of time; \( V \) may depend on time. We define the state vector evolution with time as
\[
\ket{\Psi^I(t)} = e^{iH_0 t/\hbar} \ket{\Psi^S(t)} = e^{iH_0 t/\hbar} T(t,0)\ket{\Psi^S(0)},
\]
and correspondingly the operator
\[
F^I(t) = e^{iH_0 t/\hbar} F^S e^{-iH_0 t/\hbar}.
\]
In such a convention, we keep the inner product invariant:
\[
\bra{\Psi_A^I(t)} F^I(t) \ket{\Psi_B^I(t)} = \bra{\Psi_A^S(t)} F^S(t) \ket{\Psi_B^S(t)}.
\]

Now let us derive the equation of motion. We have
\[
\frac{d}{dt} F^I(t) = \frac{1}{i\hbar} [F^I(t), H_0] + e^{iH_0 t/\hbar} \frac{\partial F^S(t)}{\partial t} e^{-iH_0 t/\hbar}.
\]

For the state vector, we have
\[
\begin{aligned}
\frac{\partial}{\partial t} \ket{\Psi^I(t)} &= \frac{i}{\hbar} H_0 e^{iH_0 t/\hbar} \ket{\Psi^S(t)} + e^{iH_0 t/\hbar} \frac{1}{i\hbar} H^S \ket{\Psi^S(t)} \\
&= e^{iH_0 t/\hbar} \frac{i}{\hbar} (H_0 - H^S) e^{-iH_0 t/\hbar} \ket{\Psi^I(t)} \\
&= \frac{1}{i\hbar} V^I(t) \ket{\Psi^I(t)}.
\end{aligned}
\]

From the above equation, we can derive the time-evolution operator \( U(t, t_0) \) in the interaction picture as
\[
\ket{\Psi^I(t)} = U(t, t_0) \ket{\Psi^I(t_0)},
\]
\[
U(t, t_0) = T \exp\left\{-\frac{i}{\hbar} \int_{t_0}^t dt' V^I(t') \right\}.
\]
\end{problem}

\end{document}



