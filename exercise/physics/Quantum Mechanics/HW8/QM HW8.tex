\documentclass{article}
\newcommand{\mydate}{November 14, 2025}
\newcommand{\mytitle}{QM HW8}
\title{\textbf{\mytitle}}
\author{Jiete XUE}
\date{\mydate}
\usepackage{fancyhdr}
\pagestyle{fancy}
\fancyhf{}
\fancyhead[C]{\mytitle }
\fancyhead[R]{Jiete Xue}
\fancyhead[L]{\mydate}
\fancyfoot[C]{\thepage}
\usepackage{amsthm}
\usepackage{amsmath}
\usepackage{amssymb}
\usepackage{physics}
\usepackage{tikz}
\usepackage{float}

%% 右矢
%\ket{\psi}          % 输出:|ψ⟩
%\ket{\psi(t)}       % 输出:|ψ(t)⟩
%
%% 左矢
%\bra{\phi}          % 输出:⟨φ|
%
%% 期望值
%\expval{\hat{A}}    % 输出:⟨Â⟩
%\expval{\hat{A}}{\psi}  % 输出:⟨ψ|Â|ψ⟩
%
%% 对易子
%\comm{\hat{A}}{\hat{B}}  % 输出:[Â, B̂]
\newtheoremstyle{1}{}{}{}{}{\bfseries}{}{\newline}{}
\theoremstyle{1}
\newtheorem{problem}{Problem}
\usepackage{chngcntr}
\counterwithin{equation}{problem}
\newcommand{\pa}{\partial}
\newcommand{\rn}[1]{\romannumeral #1\relax}
\newcommand{\Rn}[1]{\expandafter\@slowromancap\romannumeral#1@}
\newcommand{\ii}{\mathrm{i}}
\newcommand{\ee}{\mathrm{e}}

\begin{document}
\maketitle
\begin{problem}[Derivative $x^H,p^H$ in harmonic oscillator case]
    We have 
    \begin{equation}
        [H,x]=-\ii \hbar \frac{p}{m},\ \ [H,p]=\ii \hbar m\omega^2{x}.
    \end{equation}
    Thus,
    \begin{equation}
        \underset{n\text{ copies}}{\underbrace{[H,\ldots[H,x]]}}=\begin{cases}
            -\ii \hbar^{2k-1}\omega^{2k-2} \frac{p}{m}&,n=2k-1\\
            \hbar^{2k} \omega^{2k} x&,n=2k\\
        \end{cases}
    \end{equation}
    \begin{equation}
        \underset{n\text{ copies}}{\underbrace{[H,\ldots[H,p]]}}=\begin{cases}
           \ii \hbar^{2k-1} m \omega^{2k}\frac{p}{m}&,n=2k-1\\
             \hbar^{2k} \omega^{2k}x&,n=2k\\
        \end{cases}
    \end{equation}
    By Baker-Hausdorff lemma, we have
    \begin{equation}
        \begin{aligned}
            x(t)=\ee^\frac{\ii H t}{\hbar}x\ee^{-\frac{\ii H t}{\hbar}}&=\sum_{n=0}^{+\infty}\frac{1}{n!}\left(\frac{\ii}{\hbar}\right)^n\underset{n\text{ copies}}{\underbrace{[H,\ldots[H,x]]}}\\
            &=\frac{p}{m\omega}\sum_{k=0}^{+\infty}\frac{1}{(2k-1)!}(-1)^{k-1}\omega^{2k-1}+\sum_{k=0}^{+\infty}\frac{1}{(2k)!}\omega^{2k}x\\
            &=x\cos(\omega t)+\frac{p}{m\omega}\sin(\omega t).
        \end{aligned}
    \end{equation}
    Similarly,
    \begin{equation}
        p(t)=\ee^\frac{\ii H t}{\hbar}p\ee^{-\frac{\ii H t}{\hbar}}=-m\omega x\sin(\omega t)+p\cos(\omega t).
    \end{equation}
\end{problem}
\begin{problem}[Interaction picture]
    We decompose the Hamiltonian \( H^S \) of the Schrödinger picture into the free part \( H_0 \) and the perturbative part \( V \) as
\[
H^S = H_0 + V,
\]
where \( H_0 \) is independent of time; \( V \) may depend on time. We define the state vector evolution with time as
\[
\ket{\Psi^I(t)} = e^{iH_0 t/\hbar} \ket{\Psi^S(t)} = e^{iH_0 t/\hbar} T(t,0)\ket{\Psi^S(0)},
\]
and correspondingly the operator
\[
F^I(t) = e^{iH_0 t/\hbar} F^S e^{-iH_0 t/\hbar}.
\]
In such a convention, we keep the inner product invariant:
\[
\bra{\Psi_A^I(t)} F^I(t) \ket{\Psi_B^I(t)} = \bra{\Psi_A^S(t)} F^S(t) \ket{\Psi_B^S(t)}.
\]

Now let us derive the equation of motion. We have
\[
\frac{d}{dt} F^I(t) = \frac{1}{i\hbar} [F^I(t), H_0] + e^{iH_0 t/\hbar} \frac{\partial F^S(t)}{\partial t} e^{-iH_0 t/\hbar}.
\]

For the state vector, we have
\[
\begin{aligned}
\frac{\partial}{\partial t} \ket{\Psi^I(t)} &= \frac{i}{\hbar} H_0 e^{iH_0 t/\hbar} \ket{\Psi^S(t)} + e^{iH_0 t/\hbar} \frac{1}{i\hbar} H^S \ket{\Psi^S(t)} \\
&= e^{iH_0 t/\hbar} \frac{i}{\hbar} (H_0 - H^S) e^{-iH_0 t/\hbar} \ket{\Psi^I(t)} \\
&= \frac{1}{i\hbar} V^I(t) \ket{\Psi^I(t)}.
\end{aligned}
\]

From the above equation, we can derive the time-evolution operator \( U(t, t_0) \) in the interaction picture as
\[
\ket{\Psi^I(t)} = U(t, t_0) \ket{\Psi^I(t_0)},
\]
\[
U(t, t_0) = T \exp\left\{-\frac{i}{\hbar} \int_{t_0}^t dt' V^I(t') \right\}.
\]
\end{problem}
\begin{problem}[Rotation Operator]
    (1) Consider the commutator
    \begin{equation}
        [\hat{n}\cdot \vec{J},r_i]=\left. -\frac{\ii}{\alpha}\frac{\pa g}{\pa \theta}\right|_{\theta=0;i,j}r_j.
    \end{equation}
    Take $\hat{n}$ to be $z$-axis, it should be $-\frac{\ii}{\hbar}\frac{\pa g}{\pa \theta}_{\theta=0;,i,j}r_j$. So, $\alpha=\hbar$.
    \newline
    (2) For a infinitesimal rotation
    \begin{equation}
        D(g(\hat{n},\theta))=1-\ii \frac{\varepsilon}{\hbar}\hat{n}\cdot\vec{J},\ g_{ij}=\delta_{ij}-\varepsilon\epsilon_{ijk}\hat{n}_k.
    \end{equation} 
    So, 
    \begin{equation}
        D^\dagger S_i D=S_i+\ii\frac{\varepsilon}{\hbar}[\hat{n}\cdot\vec{J},S_i],\ g_{ij}S_j=S_i-\varepsilon\epsilon_{ijk}\hat{n}_kS_j.
    \end{equation}
    Take angular momentum to be $\vec{S}$, then
    \begin{equation}
        [S_i,S_j]=\ii \hbar \epsilon_{ijk}S_k.
    \end{equation}
    (3)  
    \begin{equation}
        \begin{aligned}
            \left[L_i,L_j\right]=&[\epsilon_{imn}x_mp_n,\epsilon_{jkl}x_kp_l]\\
            =&\epsilon_{imn}\epsilon_{jkl}\left(x_mx_k[p_n,p_l]+x_m[p_n,x_k]p_l+x_k[x_m,p_l]p_n+[x_m,x_k]p_lp_n\right)\\
            =&i\hbar\epsilon_{imn}\epsilon_{jln}(x_mp_l-x_lp_m)\\
            =&i\hbar (x_ip_j-x_jp_i)\\
            =&i\hbar \epsilon_{ijk}\epsilon_{kmn}x_mp_n\\
            =&i\hbar \epsilon_{ijk}L_k.
        \end{aligned}
    \end{equation}
    (4) Take $\vec{J}$ as $\vec{L}$, Similar to (2).
\end{problem}
\begin{problem}[Pauli Matrices]
    (1)
    \begin{equation}
        \sigma_i^2=I,\ \sigma_i\sigma_j=-\sigma_j\sigma_i\ (i\neq j).
    \end{equation}
    So, 
    \begin{equation}
        \{\sigma_i,\sigma_j\}=2\delta_{ij}.
    \end{equation}
    (2)
    \begin{equation}
        \sigma_i\sigma_j=\delta_{ij}I+\ii \epsilon_{ijk} \sigma_k.
    \end{equation}
    Thus,
    \begin{equation}
        \left(\vec{\sigma}\cdot\vec{n}\right)^2=I.
    \end{equation}
    Hence,
    \begin{equation}
        \exp\left[-\frac{\ii}{2}\theta\vec{\sigma}\cdot\vec{n}\right]=\sum_{k=0}^\infty\frac{1}{k!}(-\frac{\ii}{2}\theta\vec{\sigma}\cdot\vec{n})^k=\cos\left(\frac{\theta}{2}\right)-\ii\frac{\vec{\sigma}\cdot\vec{n}}{2}\sin\left(\frac{\theta}{2}\right).
    \end{equation}
\end{problem}
\begin{problem}[Anti-unitary transformation]
    (1) 
    \begin{equation}
        R^{-1}=(UK)^{-1}=K^{-1}U{-1}=K U^{-1}=K U^\dagger.
    \end{equation}
    So,
    \begin{equation}
        \braket{R\psi}{R\phi}=\bra{\psi^*}U^\dagger U\ket{\phi^*}=\bra{\phi}\ket{\psi}.
    \end{equation}
    (2)\begin{equation}
        \braket{R^{-1}\psi}{R^{-1}\phi}=\braket{U^\dagger\psi}{U^\dagger \phi}^*=\braket{\phi}{\psi}
    \end{equation}
\end{problem}
\begin{problem}[Time-reversal transformation]
    (1) 
    \begin{equation}
        \ii\epsilon_{ijk}L_k=[L_i,L_j]=T[L_i,L_j]T^{-1}=T\ii\epsilon_{ijk}L_kT^{-1}
    \end{equation}
    Since $L_k$ is arbitrary, 
    \begin{equation}
        T\ii T^{-1}=-\ii.
    \end{equation}
    (2) Let $\Pi$ be the mechanical momentum, we expect $T\Pi T^{-1}=-\Pi$. So $H=\frac{\Pi^2}{2m}$ should not change under time-reversal transformation.
    \begin{equation}
        H^T=H.
    \end{equation}
    (3) No, consider integer spin system, it is not always an energy level degeneracy.
\end{problem}
\begin{problem}[Parity Transformation]
    (1) We expect any $\ket{\psi}$ satisfies:
    \begin{equation}
        \bra{P\psi}\ket{P \psi}=\bra{\psi}\ket{\psi}.
    \end{equation}
    Let $\ket{\psi}=\ket{\alpha}+\ket{\beta}$, then 
    \begin{equation}
        \mathrm{Re}\left(\bra{P\alpha}\ket{P\beta}\right)=\mathrm{Re}\left(\bra{\alpha}\ket{\beta}\right).
    \end{equation}
    Let $\ket{\psi}=\ket{\alpha}+\ii \ket{\beta}$, then
    \begin{equation}
        \mathrm{Im}\left(\bra{P\alpha}\ket{P\beta}\right)=\mathrm{Im}\left(\bra{\alpha}\ket{\beta}\right).
    \end{equation}
    So,
    \begin{equation}
        \bra{P\alpha}\ket{P\beta}=\bra{\alpha}\ket{\beta}.
    \end{equation}
    By Wigner theorem, $P$ is unitary, we choose $P^2=1$. But $T^2=\pm 1$ is depend on the spin of system.
    \newline
    (2) (a) Momentum eigenstate

Let the momentum eigenstate be:
\[
\psi_p(x,t) = e^{i(p x - \omega t)}
\]

Time reversal transformation $T$:
\[
\psi_p^T(x,t) = T \psi_p(x,t) = \psi_p^*(x,-t) = e^{-i(p x + \omega t)} = e^{i[(-p)x - \omega t]} = \psi_{-p}(x,t)
\]

Parity transformation $P$:
\[
\psi_p^P(x,t) = P \psi_p(x,t) = \psi_p(-x,t) = e^{i[p(-x) - \omega t]} = e^{i[(-p)x - \omega t]} = \psi_{-p}(x,t)
\]

Both $T$ and $P$ reverse the momentum: $p \to -p$.

(b) Angular momentum eigenstate

Let the angular momentum eigenstate be:
\[
\psi_m(\varphi, t) = e^{i m \varphi - i \omega t}
\]
Considering only the spatial part at $t=0$:
\[
\psi_m(\varphi) = e^{i m \varphi}
\]

Time reversal transformation $T$:
\[
\psi_m^T(\varphi) = T \psi_m(\varphi) = \psi_m^*(\varphi) = \psi_{-m}(\varphi)
\]
Time reversal changes $m \to -m$.

Parity transformation $P$:
In spherical coordinates, parity acts as $\varphi \to \varphi + \pi$:
\[
\psi_m^P(\varphi) = P \psi_m(\varphi) = \psi_m(\varphi + \pi)  = (-1)^m \psi_m(\varphi)
\]
Parity gives a phase factor $(-1)^m$ but does not change $m$.
\end{problem}

\end{document}



