\documentclass{article}
\newcommand{\mydate}{November 26, 2025}
\newcommand{\mytitle}{QM HW9}
\title{\textbf{\mytitle}}
\author{Jiete XUE}
\date{\mydate}
\usepackage{fancyhdr}
\pagestyle{fancy}
\fancyhf{}
\fancyhead[C]{\mytitle }
\fancyhead[R]{Jiete Xue}
\fancyhead[L]{\mydate}
\fancyfoot[C]{\thepage}
\usepackage{amsthm}
\usepackage{amsmath}
\usepackage{amssymb}
\usepackage{enumitem}
\usepackage{physics}
\usepackage{tikz}
\usepackage{float}

%% 右矢
%\ket{\psi}          % 输出:|ψ⟩
%\ket{\psi(t)}       % 输出:|ψ(t)⟩
%
%% 左矢
%\bra{\phi}          % 输出:⟨φ|
%
%% 期望值
%\expval{\hat{A}}    % 输出:⟨Â⟩
%\expval{\hat{A}}{\psi}  % 输出:⟨ψ|Â|ψ⟩
%
%% 对易子
%\comm{\hat{A}}{\hat{B}}  % 输出:[Â, B̂]
\newtheoremstyle{1}{}{}{}{}{\bfseries}{}{\newline}{}
\theoremstyle{1}
\newtheorem{problem}{Problem}
\newtheorem{solution}{Solution}
\usepackage{chngcntr}
\counterwithin{equation}{problem}

\setlist[enumerate]{label=(\arabic*), leftmargin=*, align=left}

\newcommand{\pa}{\partial}
\newcommand{\rn}[1]{\romannumeral #1\relax}
\newcommand{\Rn}[1]{\expandafter\@slowromancap\romannumeral#1@}
\newcommand{\ii}{\mathrm{i}}
\newcommand{\ee}{\mathrm{e}}

\begin{document}
\maketitle
\begin{problem}

(1) The energy of three lowest-lying states are 
    \begin{equation}
        \begin{split}
            E_0 &= \hbar\omega\\
            E_1 &= \frac{3\hbar\omega}{2}\\
            E_2 &= 2\hbar\omega.
        \end{split}
    \end{equation}
    $E_0$ is not degenerate, while $E_1$ and $E_2$ have degeneracy of $2,3$ respectively.
\newline
(2) $E_0$ is non-degenerate, $\expval{V}{0}=0$, so $\Delta_0^{(1)}=0$.
\begin{equation}
    V=\delta m\omega^2 l^2\frac{(a_1+a_1^\dagger)(a_2+a_2^\dagger)}{2}.
\end{equation}
\begin{equation}
    \frac{\delta m \omega^2 l^2}{2}\begin{pmatrix}
        0 & 1 \\
        1 & 0
    \end{pmatrix}\begin{pmatrix}
    \braket{\psi}{1_1}\\
    \braket{\psi}{1_2}
    \end{pmatrix}=\Delta_1^{(1)}\begin{pmatrix}
    \braket{\psi}{1_1}\\
    \braket{\psi}{1_2}
    \end{pmatrix}.
\end{equation}
So,
\begin{equation}
    \Delta_1^{(1)}=\pm \frac{\delta m \omega^2 l^2}{2},
\end{equation}
with respect to eigen-kets $\frac{\ket{1,0}+\ket{0,1}}{\sqrt{2}}$ and $\frac{\ket{1,0}-\ket{0,1}}{\sqrt{2}}$.
Similarly,
\begin{equation}
    \Delta_2^{(1)}=\pm \frac{\delta m \omega^2 l^2}{\sqrt{2}} \text{ or }0.
\end{equation}
(3) 
\end{problem}

\begin{problem}
    Let
    \begin{equation}
        l=\sqrt{\frac{\hbar}{m \omega}},\ a=\frac{1}{\sqrt{2}}\left(\frac{x}{l}+\frac{\ii l p}{\hbar}\right),\ N=a^\dagger a.
    \end{equation}
    Then,
    \begin{equation}
        H_0=\hbar \omega \left(N + \frac{1}{2}\right),\ H'=\frac{\epsilon}{4} \hbar \omega \left((a^\dagger)^2+a^2 +2N+ 1\right).
    \end{equation}
    Perturbation:
            \begin{equation}
                \Delta_0^{(1)}=\expval{H'}{0}=\frac{\epsilon}{4}\hbar \omega.
            \end{equation}
            \begin{equation}
                \Delta_0^{(2)}=\sum_{k=1}^{+\infty}\frac{\left|\bra{k}H'\ket{0}\right|^2}{E_0-E_k}=-\frac{\epsilon^2}{16}\hbar \omega.
            \end{equation}
            So, 
            \begin{equation}
                \tilde{E_0}\approx \hbar \omega\left(\frac{1}{2}+\frac{\epsilon}{4}-\frac{\epsilon^2}{16}\right).
            \end{equation}
            \begin{equation}
                \ket*{\tilde{0}^{(1)}}=\sum_{k=1}^{+\infty}\frac{\ket{k}\bra{k}H'\ket{0}}{E_0-E_k}=-\frac{\epsilon\sqrt{2}}{8}\ket{2}.
            \end{equation}
            Thus,
            \begin{equation}
                \tilde{\psi}(x)\approx \bra{x}\left(\ket{0}-\frac{\epsilon\sqrt{2}}{8}\ket{2}\right)\sim \ee^{-\frac{x^2}{2l^2}}\left(H_0\left(\frac{x}{l}\right)-\frac{\epsilon\sqrt{2}}{8}H_2\left(\frac{x}{l}\right)\right)
            \end{equation}
        Exact:
            
            Let 
            \begin{equation}
                m'=m\sqrt{1+\epsilon},\ l'=\sqrt{\frac{\hbar}{m' \omega}},\ a'=\frac{1}{\sqrt{2}}\left(\frac{x}{l'}+\frac{\ii l' p}{\hbar}\right).
            \end{equation}
            Then,
            \begin{equation}
                H=\sqrt{1+\epsilon}\hbar \omega \left(N' + \frac{1}{2}\right).
            \end{equation}
            \begin{equation}
                \tilde{E_0}= \frac{1}{2}\sqrt{1+\epsilon}\hbar \omega\approx \hbar \omega\left(\frac{1}{2}+\frac{\epsilon}{4}-\frac{\epsilon^2}{16}\right).
            \end{equation}
            \begin{equation}
                \tilde{\psi}(x)\sim \ee^{-\frac{x^2}{2l^2}}\left[1-\frac{\epsilon}{8}-\frac{m\omega}{4\hbar}\epsilon x^2\right].
            \end{equation}
    
\end{problem}
\begin{problem}
    (1) 
    \begin{equation}
        V=\lambda r^2 \cdot2\sqrt{\frac{2\pi}{15}}\left(Y_2^2+Y_2^{-2}\right).
    \end{equation}
\end{problem}

\end{document}



