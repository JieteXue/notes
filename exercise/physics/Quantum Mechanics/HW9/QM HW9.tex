\documentclass{article}
\newcommand{\mydate}{November 26, 2025}
\newcommand{\mytitle}{QM HW9}
\title{\textbf{\mytitle}}
\author{Jiete XUE}
\date{\mydate}
\usepackage{fancyhdr}
\pagestyle{fancy}
\fancyhf{}
\fancyhead[C]{\mytitle }
\fancyhead[R]{Jiete Xue}
\fancyhead[L]{\mydate}
\fancyfoot[C]{\thepage}
\usepackage{amsthm}
\usepackage{amsmath}
\usepackage{amssymb}
\usepackage{enumitem}
\usepackage{physics}
\usepackage{tikz}
\usepackage{float}

%% 右矢
%\ket{\psi}          % 输出:|ψ⟩
%\ket{\psi(t)}       % 输出:|ψ(t)⟩
%
%% 左矢
%\bra{\phi}          % 输出:⟨φ|
%
%% 期望值
%\expval{\hat{A}}    % 输出:⟨Â⟩
%\expval{\hat{A}}{\psi}  % 输出:⟨ψ|Â|ψ⟩
%
%% 对易子
%\comm{\hat{A}}{\hat{B}}  % 输出:[Â, B̂]
\newtheoremstyle{1}{}{}{}{}{\bfseries}{}{\newline}{}
\theoremstyle{1}
\newtheorem{problem}{Problem}
\newtheorem{solution}{Solution}
\usepackage{chngcntr}
\counterwithin{equation}{problem}

\setlist[enumerate]{label=(\arabic*), leftmargin=*, align=left}

\newcommand{\pa}{\partial}
\newcommand{\rn}[1]{\romannumeral #1\relax}
\newcommand{\Rn}[1]{\expandafter\@slowromancap\romannumeral#1@}
\newcommand{\ii}{\mathrm{i}}
\newcommand{\ee}{\mathrm{e}}

\begin{document}
\maketitle
\begin{problem}[2D Coupled Harmonic Oscillator]

(1) The energy of three lowest-lying states are 
    \begin{equation}
        \begin{split}
            E_0 &= \hbar\omega\\
            E_1 &= \frac{3\hbar\omega}{2}\\
            E_2 &= 2\hbar\omega.
        \end{split}
    \end{equation}
    $E_0$ is not degenerate, while $E_1$ and $E_2$ have degeneracy of $2,3$ respectively.
\newline
(2) $E_0$ is non-degenerate, $\expval{V}{0}=0$, so $\Delta_0^{(1)}=0$.
\begin{equation}
    V=\delta m\omega^2 l^2\frac{(a_1+a_1^\dagger)(a_2+a_2^\dagger)}{2}.
\end{equation}
\begin{equation}
    \frac{\delta m \omega^2 l^2}{2}\begin{pmatrix}
        0 & 1 \\
        1 & 0
    \end{pmatrix}\begin{pmatrix}
    \braket{\psi}{1_1}\\
    \braket{\psi}{1_2}
    \end{pmatrix}=\Delta_1^{(1)}\begin{pmatrix}
    \braket{\psi}{1_1}\\
    \braket{\psi}{1_2}
    \end{pmatrix}.
\end{equation}
So,
\begin{equation}
    \Delta_1^{(1)}=\pm \frac{\delta m \omega^2 l^2}{2},
\end{equation}
with respect to eigen-kets $\frac{\ket{1,0}+\ket{0,1}}{\sqrt{2}}$ and $\frac{\ket{1,0}-\ket{0,1}}{\sqrt{2}}$.
Similarly,
\begin{equation}
    \Delta_2^{(1)}=\pm \delta m \omega^2 l^2 \text{ or }0.
\end{equation}
(3) Let $u=\frac{x+y}{\sqrt{2}}$, $v=\frac{u-v}{\sqrt{2}}$, then,
\begin{equation}
    H=\frac{p_u^2}{2m}+\frac{1}{2}m \omega^2(1+\delta)u^2+\frac{p_u^2}{2m}+\frac{1}{2}m\omega^2(1-\delta)v^2.
\end{equation}
Let,
\begin{equation}
    \omega_u=\omega\sqrt{1+\delta},\ \omega_v=\omega\sqrt{1-\delta}.
\end{equation}
Then,
\begin{equation}
    E_{n_u,n_v}=\hbar \omega_u(n_u+1/2)+\hbar \omega_v(n_v+1/2)\approx\hbar \omega\left(n_u+n_v+1+\frac{\delta}{2}(n_u-n_v)\right).
\end{equation}
We get the same result as perturbation theory.
\end{problem}

\begin{problem}[Quadratic Perturbation]
    Let
    \begin{equation}
        l=\sqrt{\frac{\hbar}{m \omega}},\ a=\frac{1}{\sqrt{2}}\left(\frac{x}{l}+\frac{\ii l p}{\hbar}\right),\ N=a^\dagger a.
    \end{equation}
    Then,
    \begin{equation}
        H_0=\hbar \omega \left(N + \frac{1}{2}\right),\ H'=\frac{\epsilon}{4} \hbar \omega \left((a^\dagger)^2+a^2 +2N+ 1\right).
    \end{equation}
    Perturbation:
            \begin{equation}
                \Delta_0^{(1)}=\expval{H'}{0}=\frac{\epsilon}{4}\hbar \omega.
            \end{equation}
            \begin{equation}
                \Delta_0^{(2)}=\sum_{k=1}^{+\infty}\frac{\left|\bra{k}H'\ket{0}\right|^2}{E_0-E_k}=-\frac{\epsilon^2}{16}\hbar \omega.
            \end{equation}
            So, 
            \begin{equation}
                \tilde{E_0}\approx \hbar \omega\left(\frac{1}{2}+\frac{\epsilon}{4}-\frac{\epsilon^2}{16}\right).
            \end{equation}
            \begin{equation}
                \ket*{\tilde{0}^{(1)}}=\sum_{k=1}^{+\infty}\frac{\ket{k}\bra{k}H'\ket{0}}{E_0-E_k}=-\frac{\epsilon\sqrt{2}}{8}\ket{2}.
            \end{equation}
            Thus,
            \begin{equation}
                \tilde{\psi}(x)\approx \bra{x}\left(\ket{0}-\frac{\epsilon\sqrt{2}}{8}\ket{2}\right)\sim \ee^{-\frac{x^2}{2l^2}}\left(H_0\left(\frac{x}{l}\right)-\frac{\epsilon\sqrt{2}}{8}H_2\left(\frac{x}{l}\right)\right)
            \end{equation}
        Exact:
            
            Let 
            \begin{equation}
                m'=m\sqrt{1+\epsilon},\ l'=\sqrt{\frac{\hbar}{m' \omega}},\ a'=\frac{1}{\sqrt{2}}\left(\frac{x}{l'}+\frac{\ii l' p}{\hbar}\right).
            \end{equation}
            Then,
            \begin{equation}
                H=\sqrt{1+\epsilon}\hbar \omega \left(N' + \frac{1}{2}\right).
            \end{equation}
            \begin{equation}
                \tilde{E_0}= \frac{1}{2}\sqrt{1+\epsilon}\hbar \omega\approx \hbar \omega\left(\frac{1}{2}+\frac{\epsilon}{4}-\frac{\epsilon^2}{16}\right).
            \end{equation}
            \begin{equation}
                \tilde{\psi}(x)\sim \ee^{-\frac{x^2}{2l^2}}\left[1-\frac{\epsilon}{8}-\frac{m\omega}{4\hbar}\epsilon x^2\right].
            \end{equation}
    
\end{problem}
\begin{problem}[Quadrupole Perturbation]
    (1) 
    \begin{equation}
        V=\lambda r^2 \cdot2\sqrt{\frac{2\pi}{15}}\left(Y_2^2+Y_2^{-2}\right).
    \end{equation}
    By Wigner-Eckart Theorem, 
    \begin{equation}
        V \bumpeq A \begin{pmatrix}
            0 & 0 & 1 \\
            0 & 0 & 0 \\
            1 & 0 & 0
        \end{pmatrix},
    \end{equation}
    where $A$ is a constant. Eigen-value of $V$ is $E_1=0$, $E_2$, $E_3=0$. And eigen-kets are 
    \begin{equation}
        \ket{+}=\frac{1}{\sqrt{2}}\left(\ket{1}+\ket{-1}\right),\ket{-}=\frac{1}{\sqrt{2}}\left(\ket{1}-\ket{-1}\right),\ket{0}=\ket{m=0}
    \end{equation}
    respectively. Threefold degeneracy has been removed completely.
    \newline
    (2) 
\end{problem}
\begin{problem}[Second-order lifting of degeneracy]
\noindent(1) \textbf{Nondegenerate perturbation theory (Wrong)}

Unperturbed states: $\ket{1}, \ket{2}$ with energy $E_1$, $\ket{3}$ with energy $E_2$.

Using $E_n = E_n^{(0)} + \bra{n}V\ket{n} + \sum_{k\neq n} \frac{|\bra{k}V\ket{n}|^2}{E_n^{(0)}-E_k^{(0)}}$:

For $\ket{1}$: $\bra{1}V\ket{1} = 0$, only coupling to $\ket{3}$: $\bra{3}V\ket{1} = a^*$, so
\[
E_1^{(2)} = \frac{|a|^2}{E_1 - E_2}
\]
For $\ket{2}$:
\[
E_2^{(2)} = \frac{|b|^2}{E_1 - E_2}
\]
For $\ket{3}$: couplings to $\ket{1}, \ket{2}$:
\[
E_3^{(2)} = \frac{|a|^2 + |b|^2}{E_2 - E_1}
\]

Thus:
\[
E_A \approx E_1 + \frac{|a|^2}{E_1 - E_2}, \quad
E_B \approx E_1 + \frac{|b|^2}{E_1 - E_2}, \quad
E_C \approx E_2 + \frac{|a|^2 + |b|^2}{E_2 - E_1}
\]

\noindent(2) Exact

Characteristic equation:
\[
\det\begin{pmatrix}
E_1 - E & 0 & a \\
0 & E_1 - E & b \\
a^* & b^* & E_2 - E
\end{pmatrix} = 0
\]
Expansion gives:
\[
(E_1 - E)\left[(E_1 - E)(E_2 - E) - |b|^2\right] - |a|^2(E_1 - E) = 0
\]
\[
(E_1 - E)\left[(E_1 - E)(E_2 - E) - (|a|^2 + |b|^2)\right] = 0
\]
So one root is $E = E_1$, others satisfy:
\[
(E_1 - E)(E_2 - E) - S = 0, \quad S = |a|^2 + |b|^2
\]
Solving:
\[
E = \frac{E_1 + E_2 \pm \sqrt{(E_2 - E_1)^2 + 4S}}{2}
\]
For small $S$:
\[
E \approx E_1 - \frac{S}{\Delta}, \quad E_1, \quad E_2 + \frac{S}{\Delta}
\]

\noindent(3) Degenerate perturbation theory

Degenerate subspace: $\{\ket{1}, \ket{2}\}$, unperturbed energy $E_1$.

Effective Hamiltonian:
\[
(H_{\text{eff}})_{ij} = E_1 \delta_{ij} + V_{ij} + \sum_{k \notin \text{deg}} \frac{V_{ik}V_{kj}}{E_1 - E_k}
\]
Only $k=3$ contributes, with $E_3^{(0)} = E_2$:

\[
H_{\text{eff}} = E_1 - \frac{1}{\Delta} \begin{pmatrix}
|a|^2 & ab^* \\
a^*b & |b|^2
\end{pmatrix}
\]
Eigenvalues:
\[
\lambda = E_1, \quad E_1 - \frac{S}{\Delta}
\]
Third eigenvalue from nondegenerate PT on $\ket{3}$:
\[
E_3 \approx E_2 + \frac{S}{\Delta}
\]

\noindent(4) Comparison

\begin{itemize}
\item Nondegenerate PT (wrong): \\
$E_1 + \frac{|a|^2}{E_1 - E_2}$, $E_1 + \frac{|b|^2}{E_1 - E_2}$, $E_2 + \frac{S}{E_2 - E_1}$

\item Exact \\
$E_1$, $E_1 - \frac{S}{\Delta}$, $E_2 + \frac{S}{\Delta}$

\item Degenerate PT (correct): \\
Same as exact to second order
\end{itemize}

The nondegenerate treatment fails because it doesn't account for degeneracy lifting between $\ket{1}$ and $\ket{2}$.
\end{problem}

\end{document}



