\documentclass{article}
\newcommand{\mydate}{December 22, 2025}
\newcommand{\mytitle}{QM HW12}
\title{\textbf{\mytitle}}
\author{Jiete XUE}
\date{\mydate}
\usepackage{fancyhdr}
\pagestyle{fancy}
\fancyhf{}
\fancyhead[C]{\mytitle }
\fancyhead[R]{Jiete Xue}
\fancyhead[L]{\mydate}
\fancyfoot[C]{\thepage}
\usepackage{amsthm}
\usepackage{amsmath}
\usepackage{amssymb}
\usepackage{bm}
\usepackage{enumitem}
\usepackage{physics}
\usepackage{tikz}
\usetikzlibrary{arrows.meta, decorations.markings}
\usepackage{float}

%% 右矢
%\ket{\psi}          % 输出:|ψ⟩
%\ket{\psi(t)}       % 输出:|ψ(t)⟩
%
%% 左矢
%\bra{\phi}          % 输出:⟨φ|
%
%% 期望值
%\expval{\hat{A}}    % 输出:⟨Â⟩
%\expval{\hat{A}}{\psi}  % 输出:⟨ψ|Â|ψ⟩
%
%% 对易子
%\comm{\hat{A}}{\hat{B}}  % 输出:[Â, B̂]
\newtheoremstyle{1}{}{}{}{}{\bfseries}{}{\newline}{}
\theoremstyle{1}
\newtheorem{problem}{Problem}
\newtheorem{solution}{Solution}
\usepackage{chngcntr}
\counterwithin{equation}{problem}

\setlist[enumerate]{label=(\arabic*), leftmargin=*, align=left}

\newcommand{\pa}{\partial}
\newcommand{\rn}[1]{\romannumeral #1\relax}
\newcommand{\Rn}[1]{\expandafter\@slowromancap\romannumeral#1@}
\newcommand{\ii}{\mathrm{i}}
\newcommand{\ee}{\mathrm{e}}

\begin{document}
\maketitle
\begin{problem}[Resonance scattering]
    We have potential:
    \begin{equation}
        V(r)=\frac{\gamma \hbar^2}{2m}\delta(r-R).
    \end{equation}
    Since we only consider the case where $l=0$, the Schrödinger equation becomes
    \begin{equation}
        -\frac{\hbar^2}{2m} \cdot\frac{1}{r}\frac{\dd^2}{\dd{r}^2}(rR_0)+(V(r)-E)R_0=0.
    \end{equation}
    $R_0$ is the radial contribution.

    Let $u(r)=rR_0(r)$, then,
    \begin{equation}
        -\frac{\hbar^2}{2m}u''+(V-E)u=0.
    \end{equation}
    We suppose the solution at $r<R$ and $r>R$ are $u_1$ and $u_2$ respectively. Both of them are of the form
    \begin{equation}
        u_i(r)=A_i \ee^{ k r} +B_i\ee^{- k r},\ i\in \{1,2\},\ k=\sqrt{-\frac{2mE}{\hbar^2}}>0.
    \end{equation}
\begin{enumerate}
    \item We suppose the bound state exists, then $E<0$, and $A_2=0$, or the integral of wavefunction will diverge. The derivative continuous at $r=0$, thus equals zero, it require $A_1+B_1=0$.
        Then the connection condition becomes
        \begin{equation}
            u(R)=A_1\left(\ee^{k R}-\ee^{-kR}\right)=B_2\ee^{-kR},
        \end{equation}
        \begin{equation}
            -\frac{\hbar^2}{2m}\left[\left(-B_2 k \ee^{-kR}\right)-A_1 k\left(\ee^{kR}+\ee^{-k R}\right)\right]+\frac{\gamma \hbar^2}{2m}u(R)=0.
        \end{equation}
        Then we can deduce that 
        \begin{equation}
            \frac{2kR}{\gamma R}=\left(\ee^{-2kR}-1\right).
        \end{equation}
        The derivative of function $\ee^{-x}-1$ is monotone, and is $-1$ at $x=0$, so $|\gamma_c|=R^{-1}$.
    \item Suppose $k=\sqrt{\frac{2mE}{\hbar^2}}$ and 
        \begin{equation}
            u(r)=\begin{cases}
                A \sin(k r), & r<R,\\
                C \sin(k r+\delta_0), & r>R.
            \end{cases}
        \end{equation}
        The connection conditions give
        \begin{equation}
            A\sin (kR)= C[\sin(kR)\cos\delta_0+\cos(kR)\sin\delta_0],
        \end{equation}
        \begin{equation}
            Ak\cos(kR)=Ck[\cos(kR)\cos\delta_0-\sin(kR)\sin\delta_0] +\gamma A\sin(kR).
        \end{equation}
        So,
        \begin{equation}
            \tan\delta_0= \frac{\gamma \sin^2(kR)}{k{-\gamma \sin(kR)\cos(kR)}}=\frac{1-\cos(2kR)}{2k-\gamma \sin(2kR)}.
        \end{equation}
        It is odd function of $k$.
        \begin{equation}
            2k-\gamma \sin(2kR)=(1-\gamma R)(2k)+ \frac{\gamma}{6} (2kR)^3+ \mathcal{O}(k^5).
        \end{equation}
        \begin{equation}
            1-\cos(2kR)=\frac{1}{2} (2kR)^2 -\frac{1}{24}(2kR)^4+\mathcal{O}(k^6).
        \end{equation}
        So,
        \begin{equation}
            a_0=\left(\frac{1}{R}-\frac{1}{\gamma R^2}\right)^{-1},\ r_0=\frac{1}{3}\left(R+\frac{1}{\gamma}\right).
        \end{equation}
\end{enumerate}
\end{problem}
\end{document}



