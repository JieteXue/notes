\documentclass[12pt]{article}
\usepackage{interval}
\usepackage{amssymb}
\usepackage{enumitem}
\usepackage[top=1.2in, bottom=1.2in, left=1.5in, right=1.5in]{geometry}
\pagestyle{headings}
\setlist[enumerate,2]{label=(\arabic*), leftmargin=*}    % 二级:(1) (2) (3)
\setlist[enumerate,3]{label=(\alph*), leftmargin=*}
\newcommand{\NN}{\mathbb{N}}
\newcommand{\ZZ}{\mathbb{Z}}
\newcommand{\RR}{\mathbb{R}}
\newcommand{\CC}{\mathbb{C}}
\begin{document}
\section{Basic Logic}
\begin{enumerate}
    \item truth value:
\begin{table}[htpb]
\centering
\begin{tabular}{|c|c|c|c|c|c|}
\hline
$P$ & $Q$ & $P\wedge \neg P$ & $P \vee \neg P$ & $(P\vee Q)\Rightarrow(P\wedge Q) $ & $(P\Rightarrow Q) \Rightarrow(Q\Rightarrow P)$\\ \hline
T & T &  F &   T&   T&   T\\ \hline
F & T &  F &   T&   F&   F\\ \hline
T & F &  F &   T&   F&   T\\ \hline
F & F &  F &   T&   T&   T\\ \hline
\end{tabular}
\caption{truth value table}
\end{table}
\item (1)$Q\wedge \neg Q=$F,$P\Rightarrow (Q \wedge \neg Q)=\neg P \vee $F$=\neg P$
\newline(2)$(P\wedge \neg Q)\Rightarrow Q=\neg P \vee Q\vee Q=\neg P\vee Q=P\Rightarrow Q$
\item (1)$P\wedge Q\Rightarrow R$
\newline
(2)$Q\Rightarrow P$
\newline
(3)$P\Leftarrow Q$
\item We denote that "bear is smart" as $P$,"bear is lazy" as $Q$, then "bear is not smart" can be denoted as $\neg P$. We have $(P\wedge Q\vee( \neg P))\wedge P$,it's equivalent to $P\wedge Q$,then $Q$ must be true .
\setcounter{enumi}{5}
\item We denote "At door 1,2,3" as $P,Q,R$ ,one of them is true ,while we can get another information:one of $\neg P,\neg Q,Q$ is true. Due to "not $Q$ then $\neg Q$",we can infer that $\neg P$ is false.(We can confirm while $Q=R=$false,it can satisfies the requirements of the question)
\newline so the treasure is behind the Door 1! 
\item We denote $\dots$can leads to the capital as $P,Q,R$, then $P\wedge (R\Rightarrow Q)=(\neg P)\wedge (\neg R)=P\wedge(\neg Q)=$False. Combine the first and the third formula $P\wedge(\neg R\vee Q\vee \neg Q)=P=$False, then from the second $\neg R=$False. We are not sure about the stone path ,but we are sure that the dirt path can lead to capital.
\item Denote "$a+1==0$" as $P$ ,$b+1==0$ as $Q$,then $ab +a+b\not=-1=(a+1)(b+1)==0=\neg P\wedge\neg Q$
\item (1)Use the proof by contradiction.Not losing generality ,we assume that $a=1$,
\end{enumerate}
\setcounter{section}{3}
\section{Ordering}
\begin{enumerate}
    \item $\frac{7}{13}<\frac{6}{11}$
    \item If $ab<0,a^2+b^2>0>ab$.If $ab\ge 0,a^2+b^2\ge 2ab\ge ab $.Thus, $a^2+b^2\ge ab$.
    \item Let $c=1000000001$,then $a=(c+1)^2,b=(c-7)(c+7),a-b=2c+50>0$.So $a>b$.
    \item $\frac{2+\sqrt{3}}{2-\sqrt{3}}=7+4\sqrt{3}$
    \item 
        \begin{enumerate}
            \item $x\in \interval[open]{-8}{2}$
            \item $x\in \interval[open]{\frac{2}{3}}{6}$
            \item $x\in \interval[open]{-2}{4}$
        \end{enumerate}
    \item $x\in \interval{-2}{\frac{3+\sqrt{13}}{2}}$
    \item 
        \begin{enumerate}
            \item $0$.
            \item $-1$.
            \item No.
        \end{enumerate}
    \item $$A^\mathrm{u}=\{x\in \mathbb{R} |\sqrt{2}\le x\},A^\mathrm{l}=\{x\in \mathbb{R}|-\sqrt{2}\ge x\}$$
       $$\sup A=\sqrt{2},\inf A=-\sqrt{2}$$
       $$B^\mathrm{u}=\{x\in \mathbb{R}|x\ge 1\},B^\mathrm{l}=\{x\in \mathbb{R}|x\le 0\}$$ 
       $$\sup B=1,\inf B=0$$
    \item $2$.
    \item  Cauchy's inequality.$n^2$
    \item 
        \begin{enumerate}
            \item 
                \begin{enumerate}
                    \item reflexive:$A\subseteq A$
                    \item transitive $A\subseteq B\wedge B\subseteq C\Rightarrow A\subseteq C$
                    \item antisymmetric $A\subseteq B\wedge B\subseteq A\Rightarrow A=B$
                \end{enumerate}
            \item Denote $\bigcup_{i\in I}A_i$ as $A$
            \newline$\forall i\in I,A_i\subseteq A$,so $A\in (A_i)_{i\in I}^\mathrm{u}$.$\forall B\in  (A_i)_{i\in I}^\mathrm{u},\forall i \in I,A_i\subseteq B$, so $A\subseteq B,A=\min  (A_i)_{i\in I}^\mathrm{u},\sup  (A_i)_{i\in I}=A$.Similarly,$\inf  (A_i)_{i\in I}=\cap_{i\in I}A_i$
        \end{enumerate}
    \item The following is about induction,we skip it.
    \setcounter{enumi}{21}
    \item 
        \begin{enumerate}
            \item 
                \begin{enumerate}
                    \item reflexive:$\forall n\in \mathbb{N},n|n$
                    \item transitive:If $a|b,b|c$, where $(a,b,c)\in \mathbb{N}^3$,then $\exists (m,n)\in \mathbb{N}^2$ such that $b=am,c=nb$, so $c=(nm)a$, which leads to $a|c$.
                    \item antisymmetric:Let $a=mb,b=na,(m,n)\in \mathbb{N}^2$
                    \newline
                    Then $1=mn,m=n=1$.Hence $a=b$
                \end{enumerate}
                Therefore $(\mathbb{N},|)$ is a partially ordered set.
            \item Obvious.
            \item $\forall n\in \mathbb{N},1|n$.$1$ is the least element.
            \item $\forall n \in \mathbb{N},n|0$.$0$ is the greatest element.
            \item If there exists a $n\in \mathbb{N},n\not=0$, such that $\forall a\in A,a|n$, then $a\le n $.That contradicts to $A$ is infinite.Thus $n$ can only be $0$.$\sup _{(\mathbb{N},|)}A=0$
            \item 
                \begin{enumerate}
                    \item $\forall a\in A, a|n$,where,$\displaystyle n=\prod_{x\in A}x$,so $n\in M(A)$.
                    \item Suppose $\exists n\in M(A),n_0\nmid n$ we can write $n=dn_0+r$,where $d,r\in \NN,0<r<n_0$.Claim $r\in M(A)$:Take $x\in A$, since $n,n_0\in M(A),\exists s,s_0\in \NN,xs=n,xs_0=n_0$,then $xs=dxs_0+r,x|r$,so$r\in M(A)$.That contradicts to the fact that $n_0$ is the least number in $M(A)$.
                    \item $\sup A=n_0$
                \end{enumerate}
            \item 
                \begin{enumerate}
                    \item Let $x=\sum_{i=1}^{k}a_in_i,y=\sum_{j=1}^{t}b_jm_j,\sum_{i=1}^{k}a_in_i+\sum_{j=1}^{t}b_jm_j\in A\mathbb{Z}$.
                    \item $\sum_{i=1}^{k}a_i (yn_i)\in A\mathbb{Z}$
                    \item $\forall a\in A$, let $k=1,a_1=a,n_1=1$,we have $a\in A\mathbb{Z}$.$A\cap (\mathbb{N}\backslash\{0\})\not=\varnothing$,hence,$(A\mathbb{Z})\cap(\mathbb{N}\backslash\{0\})\not=\varnothing$.
                    \item $\{d\}\subseteq A\ZZ$.By (b), we have $d\ZZ\subseteq A\ZZ$.If $A\ZZ\nsubseteq d\ZZ$,then $\exists x=\sum a_ix_i\notin d\ZZ$,i.e. $d\nmid x$.Write $x=dm+r$,where $m,n\in \NN,0<r<d$.$r=x-dm=\sum a_ix_i+(-m)d\in A\ZZ$. But that's impossible.Hence $A\ZZ\subseteq d\ZZ,A\ZZ=d\ZZ$.
                    \item By (d),$A\ZZ= d\ZZ$,by (c),$A\subseteq A\ZZ\Rightarrow A\subseteq d\ZZ$,i.e. $d|a,\forall a\in A\Rightarrow d$ is a lower bound of $A$.Take another lower bound $d'$ of $A$.$d'|a,\forall a \in A\Rightarrow d|y,\forall y\in A\ZZ=d\ZZ\Rightarrow d'|d\Rightarrow d$ is the greatest lower bound of $A$.i.e.$\inf A=d$.        
                \end{enumerate}
            \item If $A$ is empty, it is easy to check $\gcd(A)=0,\mathrm{lcm}(A)=1$.Assume $A=\not=\varnothing$.If $A=\{0\}$, then easy to check $\gcd(A)=\mathrm{lcm}(A)=0$.Set $A'=\{a\in A|a\not=0\}\subseteq A,A'\not=\varnothing$.By (7)-(e),$A'$ has infimum $d$. $d$ is also the infimum of $A$.By (5),(6)-(c),$A'$ has a supremum $D$.$d$ is also the supremum of $A$.
            \item $A=\{a,b\}$,by (7)-(d)(e),$A\ZZ=d\ZZ\Rightarrow d\in A\ZZ\Rightarrow\exists m,n$ such that $d=ma+nb$ (Bézout Lemma )
            \item $\frac{ab}{\gcd(a,b)}=a\frac{b}{\gcd(a,b)}=b\frac{a}{\gcd(a,b)}\Rightarrow \frac{ab}{\gcd(a,b)}$ is an upper bound of $A=\{a,b\}$ under $(\NN,|)$.Since $\mathrm{lcm}(a,b)$ is the least upper bound of $A$ , $\mathrm{gcd}(a,b)|\frac{ab}{\gcd(a,b)}$$$a=\frac{ab}{\mathrm{lcm}(a,b)}\frac{\mathrm{lcm(a,b)}}{b},b=\dots$$ $ \frac{ab}{\mathrm{lcm}(a,b)}$ is a lower bound of $A=\{a,\}$ under $(\NN,|)$,gcd is the greatest $\dots$\newline
                $ \frac{ab}{\mathrm{lcm}(a,b)}|\gcd(a,b),ab=\gcd(a,b)\mathrm{lcm}(a,b)$.
        \end{enumerate}
        
    \item 
        \begin{enumerate}
            \item Obvious.
            \item $\forall x\in \varnothing,P(x)$ is true.There is no non-empty set can be the subset of $\varnothing$,$(\varnothing,\underline{\in})$ is true.
            \item  $(\alpha,\underline {\in})$ is a well-ordered set since it is a subset of $\left(\alpha\cup \{\alpha\},\underline{\in}\right)$.
                $\forall x\in \alpha\cup\{\alpha\}$,if $x=\alpha,x\subseteq(\alpha\cup\{\alpha\})$;if $x\in \alpha ,x\subseteq \alpha\subseteq(\alpha\cup\{\alpha\})$.So $\alpha$ is ordinal.
            \item $\forall x\in \alpha,x\underline{\in }\alpha,\forall A\subseteq\alpha,\min(A)\in \alpha\subseteq(\alpha\cup\{\alpha\})$,so $\left(\alpha\cup\{\alpha\},\underline{\in}\right)$ is well ordered.$\forall x\in \alpha\cup \{\alpha\}$,if $x=\alpha$,$\alpha\subseteq\alpha\cup\{\alpha\}$;If $x\in \alpha$,since $\alpha$ is ordinal, $x\subseteq \alpha\subseteq\alpha\cup\{\alpha\}$.Thus $\alpha\cup\{\alpha\}$ is an ordinal.
                \newline
                Obviously,$$\alpha\subseteq\bigcup_{x\in \alpha\cup\{\alpha\}}x$$ 
                Conversely,$\forall y\in x,x\in \alpha\cup\{\alpha\}$,if $x=\alpha$,then $y\in \alpha$.If $x\in \alpha$,since $\alpha$ is ordinal,$y\in x\subseteq \alpha,y\in \alpha$.Hence,
                $$\alpha\supseteq\bigcup_{x\in \alpha\cup\{\alpha\}}x$$
                Therefore,
                $$\alpha=\bigcup_{x\in \alpha\cup\{\alpha\}}x$$   
            \item $$\alpha=\bigcup_{x\in \alpha\cup\{\alpha\}}x=\bigcup_{x\in \beta\cup\{\beta\}}x=\beta$$
            \item If $x=\alpha\vee y=\alpha$,easy.If $x,y\in \alpha$,since $(\alpha,\underline{\in})$ is well ordered, consider $\{x,y\}\subseteq \alpha,x\underline{\in}y\vee y\underline{\in}x$.
            \item $\forall x\in \alpha,x\subseteq \alpha$, since $(\alpha,\underline{\in})$ is well ordered, $(x,\underline{\in})$ is well ordered.$\forall y\in x,z\in x$,by transitive $z\in x,y\subseteq x$.Therefore,all elements of $\alpha$ are ordinals.
            \item Take $x\in \beta$,denote $X:=\{\,y\in \alpha|y\underline{\in }x\,\}$.Take $y\in X$,since $y\underline{\in}x\underline{\in }\beta$ ,by transitivity,$y\underline{\in }\beta$.If $y=\beta,\beta\in x\wedge x\in \beta$,contradicts to axiom of foundation.So $y\in \beta,X\subseteq \beta$.
            \item If $\beta\in \alpha\cup\{\alpha\}$ and $\beta\not=\alpha,\beta \subseteq\alpha$.By (8),$\beta$ is an initial segment of $\alpha$. If $\beta$ is an initial segment of $\alpha$    
        \end{enumerate}
    \item 
        \begin{enumerate}
            \item $\Rightarrow$:Let $\alpha=A\cup\{A\}$ for an ordinal $A$.By (4) of 23. $$A=\bigcup_{x\in A\cup\{A\}}x=\bigcup_{x\in \alpha}x\subseteq \alpha$$
            $\Leftarrow $: Let $U=\cup_{x\in \alpha}x$,claim that $\alpha=U\cup\{U\}$ (to be continue to check)
            \item -
            \item N.T.S. $\forall x\in \varnothing\cup\{\varnothing\}$,$x$ is not a limit ordinal.$\Rightarrow x=\varnothing$, which is not a limit ordinal by definition.
            \item $\alpha=n$ is a natural number $\Leftrightarrow \forall x\in \alpha\cup\{\alpha\},x$ is not limit.N.T.S $\alpha+1$ is not $\NN$, i.e. $\forall x\in \alpha\cup \{\alpha\}\cup\{\alpha\cup\{\alpha\}\},x$ is not limit.Whether $x\in \alpha\cup\{\alpha\}$ or $x=\alpha\cup\{\alpha\}$ , it's right.
            \item -
            \item $\alpha=n$ natural number .$\forall x\in \alpha+1,x$ is not limit. N.T.S $\forall y\in \alpha,\forall z\in y+1 ,z$ is not limit.$z\in y+1\nsubseteq \alpha+1\Rightarrow z\in \alpha+1\Rightarrow z $ is not a limit ordinal.
            \item -
            \item -
            \item  $f$ increasing $\Leftrightarrow \forall x_1,x_2\in \NN,f(x_1)\le f(x_2)$.Prove by induction.Claim $f(0)=0$.Pf.:If not ,then $f(0)\not=0\Rightarrow f(0)\ge 1$.By increasing,$\forall n>0,f(n)\ge f(0)\ge 1.\forall n\in \NN ,f(n)\not=0,f$ is not surjective.Claim:If $f(n)=n,\forall n\ge m $, then $f(m+1)=m+1$.Pf. $f(m+1)\ge f(m)=m$ .If $f(m+a)=m=f(m)\Rightarrow f $ is not injective.If $f(m+1)>m+1$,then $\forall i >m+1,f(i)\ge f(m+1)>m+1$.     
       \end{enumerate}
\end{enumerate}


\section{Group}
\begin{enumerate}
    \item It is communicative and associative.
    \item It's communicative, but not associative.
    \item 
        \begin{enumerate}
            \item $1+3(x*y)=1+3x+3y+9xy=(1+3x)(1+3y)$
            \item  Easy to prove it's communicative. $(x*y)*z=x+y+z+3xy+3yz+3zx+9xyz$ , $x,y,z$ are in the same place. Then it's associative.
            \item $\forall x\in \RR,(x*0)=(0*x)=x$, so $e=0$ is the neutral element in the semigroup.
            \item $\forall x\not=-\frac{1}{3},y=-\frac{x}{1+3x}$ satisfies $(x*y)=0=e$.
        \end{enumerate}
    \item 
        \begin{enumerate}
            \item Easy. $e=0$.
            \item $\forall (x,y)\in \RR_{>0}^2,\sqrt{x^2+y^2}>0=e$. So none of the non-zero element is invertible.
        \end{enumerate}
    \item 
        \begin{enumerate}
            \item Easy to check it is close.
            \item Composition of mapping is associative, so it's a semigroup.
            \item $\forall i\in \{1,2,3,4\}, f_1\circ f_i=f_i=f_i \circ f_1$. So it is a monoid.
            \item $\forall i \in \{1,2,3,4\}, f_i\circ f_i=f_1$. So it is a group.
        \end{enumerate}
    \item 
        \begin{enumerate}
            \item $e=(1,0), (\frac{1}{a},-\frac{x}{a})$ is the inverse of $(a,x)$.
            \item Not communicative.
            \item Easy.
        \end{enumerate}
    \item 
        \begin{enumerate}
            \item Not close.
            \item Not close.
            \item $e=1$ is the neutral element. $\forall (x,y)\in H^2$, let $x=\frac{q}{p},y=\frac{t}{s},\iota(y)=\frac{s}{t}$, then $x\cdot\iota(y)=\frac{qs}{pt}\in H$. So $(H,\cdot)$ is a group.
            \item $\forall \sigma\in H,\sigma(x)=x\Rightarrow x=\sigma^{-1}(\sigma(x))=\sigma^{-1}(x)$, so $\sigma^{-1}\in H$. Since we've known $H$ is monoid, $H$ is a group.
        \end{enumerate}
    \item We denote $G:=\{a+b\sqrt{2}\mid(a,b)\in \ZZ^2\}$. Take two elements $x=a+b\sqrt{2},y=c+d\sqrt{2}$ from $G$, $x\cdot y=(ac+2bd)+(ad+bc)\sqrt{2}\in G$. The neutral element $e=1$ also in $G$, so it is a submonoid of $(\RR,\cdot)$.
    \item $\forall z\in \mu_n(\CC),\iota(z)=z^{-1}$. $\forall (z_1,z_2)\in \mu_n(\CC)^2,(z_1 z_2^{-1})^n=z_{1}^{n}(z_{2}^{n})^{-1}=1$, thus $z_1\iota(z_2)\in \mu_{n}(\CC)$. Therefore $\mu_{n}(\CC)$ is a subgroup of $(\CC^\times,\cdot)$.
    \item 
        \begin{enumerate}
            \item Neutral element $e=1$ is in $G:=\{x+y\sqrt{3}\mid x\in \NN,y\in \ZZ,x^2-3y^2=1\}$. If $x+y\sqrt{3}$ is an element of $G$, then $(x+y\sqrt{3})(x-y\sqrt{3})=1$, since $x\ge 0,x+y\sqrt{3}$ and $x-y\sqrt{3}$ can not both be negative. Then they are both positive, so they are both in $\RR_{>0}$. Moreover, They are inverse of each other. $(x+y\sqrt{3})(z-w\sqrt{3})=xz-3yw+(zy-xw)\sqrt{3},x>\sqrt{3}y,z>\sqrt{3}w\Rightarrow xz-3yw>0$. So $xz-3yw\in \NN$. $(x+y\sqrt{3})(z-w\sqrt{3})\in G$. Therefore, it is a subgroup of $(\RR_{>0},\times)$.
            \item Easy.
            \item $\frac{97}{56}-\sqrt{3}=\frac{1}{(97+56\sqrt{3})56}$.
        \end{enumerate}
    \item 
        \begin{enumerate}
            \item $\forall (n,m)\in \ZZ^2,(-1)^n(-1)^m=(-1)^{n+m}$.
            \item Easy.
            \item Easy.
        \end{enumerate}
    \item 
        \begin{enumerate}
            \item Easy to check $e\in \mathrm{Stab}(x)$. $\forall g\in \mathrm{Stab}(x), x=(g^{-1}g)x=g^{-1}(gx)=g^{-1}x$. So $g^{-1}\in \mathrm{Stab}(x)$. Moreover, $\forall (g_1,g_2)\in \mathrm{Stab}(x)^2,g_1g_2^{-1}x=g_1 x=x$, so $g_1g_2^{-1}\in \mathrm{Stab}(x)$. Therefore, $\mathrm{Stab}(x)$ is a subgroup of $G$.
            \item Claim that: if $\exists g\in G$, $g\in g_1\mathrm{Stab}(x)\wedge g\in g_2\mathrm{Stab}(x)$, then $g_1\mathrm{Stab}(x)=g_2\mathrm{Stab}(x)$. Let $g=g_1s_1=g_2s_2$, then $g_2=g_1s_1\iota(s_2)$. Thus, for any $s\in \mathrm{Stab}(x),g_2s=g_1s_1\iota(s_2)s\in g_1\mathrm{Stab}(x)$.
                \newline So $g_2\mathrm{Stab}(x)\subseteq g_1\mathrm{Stab}(x)$. resp. we have $g_2\mathrm{Stab}(x)\supseteq g_1\mathrm{Stab}(x)$. Hence $g_2\mathrm{Stab}(x)=g_1\mathrm{Stab}(x)$. If $g_1s_1=g_2s_2, g_1x=g_1s_1x=g_2s_2x=g_2x$. Therefore, they map at the same $gx$.
            
        \end{enumerate}
\end{enumerate}
\end{document}