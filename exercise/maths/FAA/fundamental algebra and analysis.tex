\documentclass[12pt]{article}
\usepackage{interval}
\usepackage{amssymb}
\usepackage{amsmath}
\usepackage{enumitem}
\usepackage{dsfont}
\usepackage[top=1.2in, bottom=1.2in, left=1.5in, right=1.5in]{geometry}
\pagestyle{headings}
\setlist[enumerate,2]{label=(\arabic*), leftmargin=*}    % 二级:(1) (2) (3)
\setlist[enumerate,3]{label=(\alph*), leftmargin=*}
\newcommand{\NN}{\mathbb{N}}
\newcommand{\ZZ}{\mathbb{Z}}
\newcommand{\RR}{\mathbb{R}}
\newcommand{\CC}{\mathbb{C}}
\newcommand{\QQ}{\mathbb{Q}}
\begin{document}
\section{Basic Logic}
\begin{enumerate}
    \item truth value:
\begin{table}[htpb]
\centering
\begin{tabular}{|c|c|c|c|c|c|}
\hline
$P$ & $Q$ & $P\wedge \neg P$ & $P \vee \neg P$ & $(P\vee Q)\Rightarrow(P\wedge Q) $ & $(P\Rightarrow Q) \Rightarrow(Q\Rightarrow P)$\\ \hline
T & T &  F &   T&   T&   T\\ \hline
F & T &  F &   T&   F&   F\\ \hline
T & F &  F &   T&   F&   T\\ \hline
F & F &  F &   T&   T&   T\\ \hline
\end{tabular}
\caption{truth value table}
\end{table}
\item (1)$Q\wedge \neg Q=$F,$P\Rightarrow (Q \wedge \neg Q)=\neg P \vee $F$=\neg P$
\newline(2)$(P\wedge \neg Q)\Rightarrow Q=\neg P \vee Q\vee Q=\neg P\vee Q=P\Rightarrow Q$
\item (1)$P\wedge Q\Rightarrow R$
\newline
(2)$Q\Rightarrow P$
\newline
(3)$P\Leftarrow Q$
\item We denote that "bear is smart" as $P$,"bear is lazy" as $Q$, then "bear is not smart" can be denoted as $\neg P$. We have $(P\wedge Q\vee( \neg P))\wedge P$,it's equivalent to $P\wedge Q$,then $Q$ must be true .
\setcounter{enumi}{5}
\item We denote "At door 1,2,3" as $P,Q,R$ ,one of them is true ,while we can get another information:one of $\neg P,\neg Q,Q$ is true. Due to "not $Q$ then $\neg Q$",we can infer that $\neg P$ is false.(We can confirm while $Q=R=$false,it can satisfies the requirements of the question)
\newline so the treasure is behind the Door 1! 
\item We denote $\dots$can leads to the capital as $P,Q,R$, then $P\wedge (R\Rightarrow Q)=(\neg P)\wedge (\neg R)=P\wedge(\neg Q)=$False. Combine the first and the third formula $P\wedge(\neg R\vee Q\vee \neg Q)=P=$False, then from the second $\neg R=$False. We are not sure about the stone path ,but we are sure that the dirt path can lead to capital.
\item Denote "$a+1==0$" as $P$ ,$b+1==0$ as $Q$,then $ab +a+b\not=-1=(a+1)(b+1)==0=\neg P\wedge\neg Q$
\item (1)Use the proof by contradiction.Not losing generality ,we assume that $a=1$,
\end{enumerate}
\setcounter{section}{3}
\section{Ordering}
\begin{enumerate}
    \item $\frac{7}{13}<\frac{6}{11}$
    \item If $ab<0,a^2+b^2>0>ab$.If $ab\ge 0,a^2+b^2\ge 2ab\ge ab $.Thus, $a^2+b^2\ge ab$.
    \item Let $c=1000000001$,then $a=(c+1)^2,b=(c-7)(c+7),a-b=2c+50>0$.So $a>b$.
    \item $\frac{2+\sqrt{3}}{2-\sqrt{3}}=7+4\sqrt{3}$
    \item 
        \begin{enumerate}
            \item $x\in \interval[open]{-8}{2}$
            \item $x\in \interval[open]{\frac{2}{3}}{6}$
            \item $x\in \interval[open]{-2}{4}$
        \end{enumerate}
    \item $x\in \interval{-2}{\frac{3+\sqrt{13}}{2}}$
    \item 
        \begin{enumerate}
            \item $0$.
            \item $-1$.
            \item No.
        \end{enumerate}
    \item $$A^\mathrm{u}=\{x\in \mathbb{R} |\sqrt{2}\le x\},A^\mathrm{l}=\{x\in \mathbb{R}|-\sqrt{2}\ge x\}$$
       $$\sup A=\sqrt{2},\inf A=-\sqrt{2}$$
       $$B^\mathrm{u}=\{x\in \mathbb{R}|x\ge 1\},B^\mathrm{l}=\{x\in \mathbb{R}|x\le 0\}$$ 
       $$\sup B=1,\inf B=0$$
    \item $2$.
    \item  Cauchy's inequality.$n^2$
    \item 
        \begin{enumerate}
            \item 
                \begin{enumerate}
                    \item reflexive:$A\subseteq A$
                    \item transitive $A\subseteq B\wedge B\subseteq C\Rightarrow A\subseteq C$
                    \item antisymmetric $A\subseteq B\wedge B\subseteq A\Rightarrow A=B$
                \end{enumerate}
            \item Denote $\bigcup_{i\in I}A_i$ as $A$
            \newline$\forall i\in I,A_i\subseteq A$,so $A\in (A_i)_{i\in I}^\mathrm{u}$.$\forall B\in  (A_i)_{i\in I}^\mathrm{u},\forall i \in I,A_i\subseteq B$, so $A\subseteq B,A=\min  (A_i)_{i\in I}^\mathrm{u},\sup  (A_i)_{i\in I}=A$.Similarly,$\inf  (A_i)_{i\in I}=\cap_{i\in I}A_i$
        \end{enumerate}
    \item The following is about induction,we skip it.
    \setcounter{enumi}{21}
    \item 
        \begin{enumerate}
            \item 
                \begin{enumerate}
                    \item reflexive:$\forall n\in \mathbb{N},n|n$
                    \item transitive:If $a|b,b|c$, where $(a,b,c)\in \mathbb{N}^3$,then $\exists (m,n)\in \mathbb{N}^2$ such that $b=am,c=nb$, so $c=(nm)a$, which leads to $a|c$.
                    \item antisymmetric:Let $a=mb,b=na,(m,n)\in \mathbb{N}^2$
                    \newline
                    Then $1=mn,m=n=1$.Hence $a=b$
                \end{enumerate}
                Therefore $(\mathbb{N},|)$ is a partially ordered set.
            \item Obvious.
            \item $\forall n\in \mathbb{N},1|n$.$1$ is the least element.
            \item $\forall n \in \mathbb{N},n|0$.$0$ is the greatest element.
            \item If there exists a $n\in \mathbb{N},n\not=0$, such that $\forall a\in A,a|n$, then $a\le n $.That contradicts to $A$ is infinite.Thus $n$ can only be $0$.$\sup _{(\mathbb{N},|)}A=0$
            \item 
                \begin{enumerate}
                    \item $\forall a\in A, a|n$,where,$\displaystyle n=\prod_{x\in A}x$,so $n\in M(A)$.
                    \item Suppose $\exists n\in M(A),n_0\nmid n$ we can write $n=dn_0+r$,where $d,r\in \NN,0<r<n_0$.Claim $r\in M(A)$:Take $x\in A$, since $n,n_0\in M(A),\exists s,s_0\in \NN,xs=n,xs_0=n_0$,then $xs=dxs_0+r,x|r$,so$r\in M(A)$.That contradicts to the fact that $n_0$ is the least number in $M(A)$.
                    \item $\sup A=n_0$
                \end{enumerate}
            \item 
                \begin{enumerate}
                    \item Let $x=\sum_{i=1}^{k}a_in_i,y=\sum_{j=1}^{t}b_jm_j,\sum_{i=1}^{k}a_in_i+\sum_{j=1}^{t}b_jm_j\in A\mathbb{Z}$.
                    \item $\sum_{i=1}^{k}a_i (yn_i)\in A\mathbb{Z}$
                    \item $\forall a\in A$, let $k=1,a_1=a,n_1=1$,we have $a\in A\mathbb{Z}$.$A\cap (\mathbb{N}\backslash\{0\})\not=\varnothing$,hence,$(A\mathbb{Z})\cap(\mathbb{N}\backslash\{0\})\not=\varnothing$.
                    \item $\{d\}\subseteq A\ZZ$.By (b), we have $d\ZZ\subseteq A\ZZ$.If $A\ZZ\nsubseteq d\ZZ$,then $\exists x=\sum a_ix_i\notin d\ZZ$,i.e. $d\nmid x$.Write $x=dm+r$,where $m,n\in \NN,0<r<d$.$r=x-dm=\sum a_ix_i+(-m)d\in A\ZZ$. But that's impossible.Hence $A\ZZ\subseteq d\ZZ,A\ZZ=d\ZZ$.
                    \item By (d),$A\ZZ= d\ZZ$,by (c),$A\subseteq A\ZZ\Rightarrow A\subseteq d\ZZ$,i.e. $d|a,\forall a\in A\Rightarrow d$ is a lower bound of $A$.Take another lower bound $d'$ of $A$.$d'|a,\forall a \in A\Rightarrow d|y,\forall y\in A\ZZ=d\ZZ\Rightarrow d'|d\Rightarrow d$ is the greatest lower bound of $A$.i.e.$\inf A=d$.        
                \end{enumerate}
            \item If $A$ is empty, it is easy to check $\gcd(A)=0,\mathrm{lcm}(A)=1$.Assume $A=\not=\varnothing$.If $A=\{0\}$, then easy to check $\gcd(A)=\mathrm{lcm}(A)=0$.Set $A'=\{a\in A|a\not=0\}\subseteq A,A'\not=\varnothing$.By (7)-(e),$A'$ has infimum $d$. $d$ is also the infimum of $A$.By (5),(6)-(c),$A'$ has a supremum $D$.$d$ is also the supremum of $A$.
            \item $A=\{a,b\}$,by (7)-(d)(e),$A\ZZ=d\ZZ\Rightarrow d\in A\ZZ\Rightarrow\exists m,n$ such that $d=ma+nb$ (Bézout Lemma )
            \item $\frac{ab}{\gcd(a,b)}=a\frac{b}{\gcd(a,b)}=b\frac{a}{\gcd(a,b)}\Rightarrow \frac{ab}{\gcd(a,b)}$ is an upper bound of $A=\{a,b\}$ under $(\NN,|)$.Since $\mathrm{lcm}(a,b)$ is the least upper bound of $A$ , $\mathrm{gcd}(a,b)|\frac{ab}{\gcd(a,b)}$$$a=\frac{ab}{\mathrm{lcm}(a,b)}\frac{\mathrm{lcm(a,b)}}{b},b=\dots$$ $ \frac{ab}{\mathrm{lcm}(a,b)}$ is a lower bound of $A=\{a,\}$ under $(\NN,|)$,gcd is the greatest $\dots$\newline
                $ \frac{ab}{\mathrm{lcm}(a,b)}|\gcd(a,b),ab=\gcd(a,b)\mathrm{lcm}(a,b)$.
        \end{enumerate}
        
    \item 
        \begin{enumerate}
            \item Obvious.
            \item $\forall x\in \varnothing,P(x)$ is true.There is no non-empty set can be the subset of $\varnothing$,$(\varnothing,\underline{\in})$ is true.
            \item  $(\alpha,\underline {\in})$ is a well-ordered set since it is a subset of $\left(\alpha\cup \{\alpha\},\underline{\in}\right)$.
                $\forall x\in \alpha\cup\{\alpha\}$,if $x=\alpha,x\subseteq(\alpha\cup\{\alpha\})$;if $x\in \alpha ,x\subseteq \alpha\subseteq(\alpha\cup\{\alpha\})$.So $\alpha$ is ordinal.
            \item $\forall x\in \alpha,x\underline{\in }\alpha,\forall A\subseteq\alpha,\min(A)\in \alpha\subseteq(\alpha\cup\{\alpha\})$,so $\left(\alpha\cup\{\alpha\},\underline{\in}\right)$ is well ordered.$\forall x\in \alpha\cup \{\alpha\}$,if $x=\alpha$,$\alpha\subseteq\alpha\cup\{\alpha\}$;If $x\in \alpha$,since $\alpha$ is ordinal, $x\subseteq \alpha\subseteq\alpha\cup\{\alpha\}$.Thus $\alpha\cup\{\alpha\}$ is an ordinal.
                \newline
                Obviously,$$\alpha\subseteq\bigcup_{x\in \alpha\cup\{\alpha\}}x$$ 
                Conversely,$\forall y\in x,x\in \alpha\cup\{\alpha\}$,if $x=\alpha$,then $y\in \alpha$.If $x\in \alpha$,since $\alpha$ is ordinal,$y\in x\subseteq \alpha,y\in \alpha$.Hence,
                $$\alpha\supseteq\bigcup_{x\in \alpha\cup\{\alpha\}}x$$
                Therefore,
                $$\alpha=\bigcup_{x\in \alpha\cup\{\alpha\}}x$$   
            \item $$\alpha=\bigcup_{x\in \alpha\cup\{\alpha\}}x=\bigcup_{x\in \beta\cup\{\beta\}}x=\beta$$
            \item If $x=\alpha\vee y=\alpha$,easy.If $x,y\in \alpha$,since $(\alpha,\underline{\in})$ is well ordered, consider $\{x,y\}\subseteq \alpha,x\underline{\in}y\vee y\underline{\in}x$.
            \item $\forall x\in \alpha,x\subseteq \alpha$, since $(\alpha,\underline{\in})$ is well ordered, $(x,\underline{\in})$ is well ordered.$\forall y\in x,z\in x$,by transitive $z\in x,y\subseteq x$.Therefore,all elements of $\alpha$ are ordinals.
            \item Take $x\in \beta$,denote $X:=\{\,y\in \alpha|y\underline{\in }x\,\}$.Take $y\in X$,since $y\underline{\in}x\underline{\in }\beta$ ,by transitivity,$y\underline{\in }\beta$.If $y=\beta,\beta\in x\wedge x\in \beta$,contradicts to axiom of foundation.So $y\in \beta,X\subseteq \beta$.
            \item If $\beta\in \alpha\cup\{\alpha\}$ and $\beta\not=\alpha,\beta \subseteq\alpha$.By (8),$\beta$ is an initial segment of $\alpha$. If $\beta$ is an initial segment of $\alpha$    
        \end{enumerate}
    \item 
        \begin{enumerate}
            \item $\Rightarrow$:Let $\alpha=A\cup\{A\}$ for an ordinal $A$.By (4) of 23. $$A=\bigcup_{x\in A\cup\{A\}}x=\bigcup_{x\in \alpha}x\subseteq \alpha$$
            $\Leftarrow $: Let $U=\cup_{x\in \alpha}x$,claim that $\alpha=U\cup\{U\}$ (to be continue to check)
            \item -
            \item N.T.S. $\forall x\in \varnothing\cup\{\varnothing\}$,$x$ is not a limit ordinal.$\Rightarrow x=\varnothing$, which is not a limit ordinal by definition.
            \item $\alpha=n$ is a natural number $\Leftrightarrow \forall x\in \alpha\cup\{\alpha\},x$ is not limit.N.T.S $\alpha+1$ is not $\NN$, i.e. $\forall x\in \alpha\cup \{\alpha\}\cup\{\alpha\cup\{\alpha\}\},x$ is not limit.Whether $x\in \alpha\cup\{\alpha\}$ or $x=\alpha\cup\{\alpha\}$ , it's right.
            \item -
            \item $\alpha=n$ natural number .$\forall x\in \alpha+1,x$ is not limit. N.T.S $\forall y\in \alpha,\forall z\in y+1 ,z$ is not limit.$z\in y+1\nsubseteq \alpha+1\Rightarrow z\in \alpha+1\Rightarrow z $ is not a limit ordinal.
            \item -
            \item -
            \item  $f$ increasing $\Leftrightarrow \forall x_1,x_2\in \NN,f(x_1)\le f(x_2)$.Prove by induction.Claim $f(0)=0$.Pf.:If not ,then $f(0)\not=0\Rightarrow f(0)\ge 1$.By increasing,$\forall n>0,f(n)\ge f(0)\ge 1.\forall n\in \NN ,f(n)\not=0,f$ is not surjective.Claim:If $f(n)=n,\forall n\ge m $, then $f(m+1)=m+1$.Pf. $f(m+1)\ge f(m)=m$ .If $f(m+a)=m=f(m)\Rightarrow f $ is not injective.If $f(m+1)>m+1$,then $\forall i >m+1,f(i)\ge f(m+1)>m+1$.     
       \end{enumerate}
\end{enumerate}


\section{Group}
\begin{enumerate}
    \item It is communicative and associative.
    \item It's communicative, but not associative.
    \item 
        \begin{enumerate}
            \item $1+3(x*y)=1+3x+3y+9xy=(1+3x)(1+3y)$
            \item  Easy to prove it's communicative. $(x*y)*z=x+y+z+3xy+3yz+3zx+9xyz$ , $x,y,z$ are in the same position. Then it's associative.
            \item $\forall x\in \RR,(x*0)=(0*x)=x$, so $e=0$ is the neutral element in the semigroup.
            \item $\forall x\not=-\frac{1}{3},y=-\frac{x}{1+3x}$ satisfies $(x*y)=0=e$.
        \end{enumerate}
    \item 
        \begin{enumerate}
            \item Easy. $e=0$.
            \item $\forall (x,y)\in \RR_{>0}^2,\sqrt{x^2+y^2}>0=e$. So none of the non-zero element is invertible.
        \end{enumerate}
    \item 
        \begin{enumerate}
            \item Easy to check it is close.
            \item Composition of mapping is associative, so it's a semigroup.
            \item $\forall i\in \{1,2,3,4\}, f_1\circ f_i=f_i=f_i \circ f_1$. So it is a monoid.
            \item $\forall i \in \{1,2,3,4\}, f_i\circ f_i=f_1$. So it is a group.
        \end{enumerate}
    \item 
        \begin{enumerate}
            \item $e=(1,0), (\frac{1}{a},-\frac{x}{a})$ is the inverse of $(a,x)$.
            \item Not communicative.
            \item Easy.
        \end{enumerate}
    \item 
        \begin{enumerate}
            \item Not close.
            \item Not close.
            \item $e=1$ is the neutral element. $\forall (x,y)\in H^2$, let $x=\frac{q}{p},y=\frac{t}{s},\iota(y)=\frac{s}{t}$, then $x\cdot\iota(y)=\frac{qs}{pt}\in H$. So $(H,\cdot)$ is a group.
            \item $\forall \sigma\in H,\sigma(x)=x\Rightarrow x=\sigma^{-1}(\sigma(x))=\sigma^{-1}(x)$, so $\sigma^{-1}\in H$. Since we've known $H$ is monoid, $H$ is a group.
        \end{enumerate}
    \item We denote $G:=\{a+b\sqrt{2}\mid(a,b)\in \ZZ^2\}$. Take two elements $x=a+b\sqrt{2},y=c+d\sqrt{2}$ from $G$, $x\cdot y=(ac+2bd)+(ad+bc)\sqrt{2}\in G$. The neutral element $e=1$ also in $G$, so it is a submonoid of $(\RR,\cdot)$.
    \item $\forall z\in \mu_n(\CC),\iota(z)=z^{-1}$. $\forall (z_1,z_2)\in \mu_n(\CC)^2,(z_1 z_2^{-1})^n=z_{1}^{n}(z_{2}^{n})^{-1}=1$, thus $z_1\iota(z_2)\in \mu_{n}(\CC)$. Therefore $\mu_{n}(\CC)$ is a subgroup of $(\CC^\times,\cdot)$.
    \item 
        \begin{enumerate}
            \item Neutral element $e=1$ is in $G:=\{x+y\sqrt{3}\mid x\in \NN,y\in \ZZ,x^2-3y^2=1\}$. If $x+y\sqrt{3}$ is an element of $G$, then $(x+y\sqrt{3})(x-y\sqrt{3})=1$, since $x\ge 0,x+y\sqrt{3}$ and $x-y\sqrt{3}$ can not both be negative. Then they are both positive, so they are both in $\RR_{>0}$. Moreover, They are inverse of each other. $(x+y\sqrt{3})(z-w\sqrt{3})=xz-3yw+(zy-xw)\sqrt{3},x>\sqrt{3}y,z>\sqrt{3}w\Rightarrow xz-3yw>0$. So $xz-3yw\in \NN$. $(x+y\sqrt{3})(z-w\sqrt{3})\in G$. Therefore, it is a subgroup of $(\RR_{>0},\times)$.
            \item Easy.
            \item $\frac{97}{56}-\sqrt{3}=\frac{1}{(97+56\sqrt{3})56}$.
        \end{enumerate}
    \item 
        \begin{enumerate}
            \item $\forall (n,m)\in \ZZ^2,(-1)^n(-1)^m=(-1)^{n+m}$.
            \item Easy.
            \item Easy.
        \end{enumerate}
    \item 
        \begin{enumerate}
            \item Easy to check $e\in \mathrm{Stab}(x)$. $\forall g\in \mathrm{Stab}(x), x=(g^{-1}g)x=g^{-1}(gx)=g^{-1}x$. So $g^{-1}\in \mathrm{Stab}(x)$. Moreover, $\forall (g_1,g_2)\in \mathrm{Stab}(x)^2,g_1g_2^{-1}x=g_1 x=x$, so $g_1g_2^{-1}\in \mathrm{Stab}(x)$. Therefore, $\mathrm{Stab}(x)$ is a subgroup of $G$.
            \item Claim that: if $\exists g\in G$, $g\in g_1\mathrm{Stab}(x)\wedge g\in g_2\mathrm{Stab}(x)$, then $g_1\mathrm{Stab}(x)=g_2\mathrm{Stab}(x)$. Let $g=g_1s_1=g_2s_2$, then $g_2=g_1s_1\iota(s_2)$. Thus, for any $s\in \mathrm{Stab}(x),g_2s=g_1s_1\iota(s_2)s\in g_1\mathrm{Stab}(x)$.
                \newline So $g_2\mathrm{Stab}(x)\subseteq g_1\mathrm{Stab}(x)$. resp. we have $g_2\mathrm{Stab}(x)\supseteq g_1\mathrm{Stab}(x)$. Hence $g_2\mathrm{Stab}(x)=g_1\mathrm{Stab}(x)$. If $g_1s_1=g_2s_2, g_1x=g_1s_1x=g_2s_2x=g_2x$. Therefore, they map at the same $gx$.
            \item By definition, $\forall g\in G,\left|\mathrm{Stab}(x)\right|=\left|g \mathrm{Stab}(x)\right|$.(Lagrange Theorem)
        \end{enumerate}
    \item 
        \begin{enumerate}
            \item 1.
            \item By definition, $n\in N(a)$. Hence, $\min(N(a))\le n$.
            \item Let $p,q\le \mathrm{ord}(a),0\le p<q$. Suppose that $a^p=a^q$, then $e=a^{q-p},(q-p)\in N(a),q-p<\mathrm{ord}(N(a))$, contradiction. Thus they are distinct.
            \item Let $f:(\ZZ,+)\rightarrow (G,*)$ be the homomorphism, $f(1)=a$, then $\forall n\in \ZZ, a*f(n)=f(n+1)$.
                \begin{enumerate}
                    \item Suppose $\left \langle a \right \rangle $ is finite. If $\forall n,m\in \ZZ,f(n)\not=f(m)$, then the image is not finite, contradiction. Take $f(n)=f(m), n<m$, then $a^{m-n}=1$. Thus $\mathrm{ord}(a)\le m-n$ is finite.
                    \item Suppose $\mathrm{ord}(a)$ is finite. Then $\forall n\in \ZZ, f(n+\mathrm{ord}(a))=f(n)\in\{f(i)\mid i\in \NN,1\le i\le \mathrm{ord}(a) \}$
                \end{enumerate}
            \item By (4)(b), $\left|\left \langle a \right \rangle \right|\le \mathrm{ord}(a)$. By (4)(a),$\left|\left \langle a \right \rangle \right|\ge \mathrm{ord}(a)$.
        \end{enumerate}
    \item 
        \begin{enumerate}
            \item By comm. law $(ab)^{N}=a^N b^N=e, \mathrm{ab}\le N$ is finite.
            \item -
            \item -
        \end{enumerate}
    \item 
        \begin{enumerate}
            \item We know that the composition of mapping is associative. And easy to check that in this case, the composition is closed. $e=\mathrm{Id}_E,f\circ f^{-1}=\mathrm{Id}_E$. Hence $\mathcal{S}_E$ equipped with composition of mapping forms a group.
            \item $\sigma^0(x)=x$. $\phi_\sigma(n+m,x)=\sigma^{(n+m)}(x)=\sigma^n\circ\sigma^m(x)=\phi_\sigma(n,x)\circ\phi_{\sigma}(m,x)$. So $\phi_\sigma$ defines a left action of $\ZZ$ on $E$.
            \item $\forall\sigma^n(x)\in \mathrm{Orb}_\sigma(x), \sigma(\sigma^n(x))=\sigma\circ\sigma^n(x)=\sigma^{n+1}(x)\in \mathrm{Orb}_\sigma(x)$. Hence $\sigma(\mathrm{Orb}_\sigma(x))\subseteq \mathrm{Orb}_\sigma(x)$.
            \item We claim that $x,y$ both in a same orbit is a equivalence relation. Reflexivity: $x\in \mathrm{Orb}_\sigma(x)\Leftrightarrow x\in \mathrm{Orb}_\sigma(x) $. Transitivity: $x\sim y\Rightarrow \exists n\in \ZZ, \sigma^n(x)=y, y\sim z\Rightarrow \exists m\in \ZZ, \sigma^m(y)=z$. Thus $\sigma^{n+m}(x)=z, x\sim z$. Symmetry: $x\sim y\Rightarrow \exists n\in \ZZ, \sigma^n(x)=y, \sigma^{-n}(y)=x$. Hence $y\sim x$. Therefore, if $x\in O_i$, then $x\notin O_j,i\not=j$. So $\sigma_i(x)=\sigma(x),\sigma_j(x)=x,i\not=j$. $\forall x\in E, \sigma_1\dots\sigma_n(x)=\sigma(x)$, hence $\sigma=\sigma_1\dots\sigma_n$.
        \end{enumerate}
    \item 
        \begin{enumerate}
            \item By definition.
            \item Let $n$ be the largest cardinal of its orbits and $O$ be the orbit that has more than one element. Then for any element $x$ in any other orbit,  $\sigma(x)=x$. Moreover, $\forall m\in \ZZ, \sigma^m(x)=x$. While $n$ is the order of $\sigma$ on $O$, for any $x\in E$, $\sigma^n(x)=x,\  n$ is the order of $\sigma$. This relation is NOT hold generally. If there exists two orbits $O_1,O_2$ , there cardinal are $n,m$ and $m>n>1,\mathrm{gcd}(n,m)=1 $, then for the element $x\in O_1$,  $\sigma^m(x)\not= x$. So $m$ is not the order of $\sigma$.
            \item For any $y\notin \mathrm{Orb(x)},\sigma(y)=y=\tau_{x_i,x_{i+1}},\ i\in \{0,\dots,p-1\}$.
                $$\tau_{x_i,x_{i+1}}(\tau_{x_{i+1},x_{i+2}}(\dots(x_i)))=\tau_{x_i,x_{i+1}}(x_i)=x_{i+1},$$
                $$\tau_{x_1,x_2}(\dots(\tau_{x_{i-1},x_i}(x_{i+1})))=x_{i+1}.$$
                Hence, $\forall i\in \{0,\dots,p-1\},\ \sigma(x_i)=\tau_{x_1,x_2}\dots \tau_{x_{p-2},x_{p-1}}(x_i).$ Therefore, 
                $$\sigma=\tau_{x_1,x_2}\dots\tau_{x_{p-2},x_{p-1}}.$$
            \item Take $O_i$ from $ \left \langle \sigma \right \rangle \backslash E$, let 
                $$\sigma_i(x):= \left\{ \begin{matrix}
                    \sigma(x) & \text{if } x\in O_i\\
                    x & \text{if } x\notin O_i
                \end{matrix} \ . \right. $$
                Similarly to (3), we can get $\sigma=\sigma_1\dots\sigma_n$, where $n=\mathrm{Card}[\left \langle \sigma \right \rangle \backslash E]$. Since $\sigma_i$ is the composition of transpositions, any $\sigma\in \mathcal{S}_E$ can be written in the form of composition of transpositions.
        \end{enumerate}
    \item Let $E=\{1,2,\dots,n\}$, $\forall \sigma\in \mathfrak{S}_E,$
            $$\prod_{i\not=j\in E}[\sigma(i)-\sigma(j)]=\prod_{i\not=j\in E}(i-j)$$
            $$\prod_{i\not=j\in E}(i-j)[(\sigma\circ \pi) (i)-(\sigma\circ\pi)(j)]=\prod_{i\not=j\in E}[\sigma(i)-\sigma(j)][\pi(i)-\pi(j)]$$
            $$\prod_{i<j\in E}(i-j)[(\sigma\circ \pi) (i)-(\sigma\circ\pi)(j)]=\prod_{i<j\in E}[\sigma(i)-\sigma(j)][\pi(i)-\pi(j)]$$
            $$\prod_{i<j\in E}\frac{[(\sigma\circ \pi) (i)-(\sigma\circ\pi)(j)]}{i-j}=\prod_{i<j\in E}\frac{\sigma(i)-\sigma(j)}{i-j}\prod_{i<j\in E}\frac{\pi(i)-\pi(j)}{i-j}$$
            Hence, $\mathrm{sgn}$ is a homomorphism.
    \item By 16. $\forall \sigma \in \mathfrak{S}_E$, it can be represented by the composition of transpositions $\tau_i$, where,
        $$\tau_i: E\longrightarrow E,$$$$ \tau_i(x)=
        \left\{\begin{matrix}
        x&,&x\in E\backslash \left \{ i,\mathrm{mod} (i ,n)+1\right \} \\
        \mathrm{mod}(x,n)+1&,&x=i \\
        \mathrm{mod}( x-2,n)+1&,&x=\mathrm{mod}(i,n)+1
        \end{matrix}\right.$$
        Easy to check that $\tau_i\circ\tau_i=\mathrm{Id}_E$. So $\forall f \text{ be a homomorphism, },f(\tau_i)f(\tau_i)=1,f(\tau_i)=\pm 1$. Then $\forall \sigma \in \mathfrak{S}_E, f(\sigma)=\pm1$.
\end{enumerate}
\textbf{Remark.  }$f=-1$ corresponds to the $\mathrm{sgn}$ in 17.
\section{Rings and Modules}
\subsection{Rings and Modules}
\begin{enumerate}
    \item 
        \begin{enumerate}
            \item First, we check that it is a monoid. \newline One has it is closed. For any $(x_1,x_2)\in \ZZ^2, (x_1,x_2)*(1,0)=(x_1,x_2)$. So $(1,0)$ is the neutral element.
            \item Second, check that it is commutative. \newline For any $(x_1,x_2),(y_1,y_2)\in \ZZ^2, (x_1,x_2)*(y_1,y_2)=(x_1y_1+rx_2y_2,x_1y_2+x_2y_1)=(y_1x_1+ry_2x_2,y_1x_2+y_2x_1)=(y_1,y_2)*(x_1,x_2)$.
            \item Third, check that it is distributive. \newline For any $(x_1,x_2),(y_1,y_2),(z_1,z_2)\in \ZZ^3, (x_1,x_2)*((y_1,y_2)+(z_1,z_2))=(x_1(y_1+z_1)+rx_2(y_2+z_2),x_1(y_2+z_2)+x_2(y_1+z_1))=(x_1,x_2)*(y_1,y_2)+(x_1,x_2)*(z_1,z_2)$. 
        \end{enumerate}
    \item
        \begin{enumerate}
            \item Associativity is NOT valid in general.
            \item Just verify.
        \end{enumerate}
    \item Let $e_i=(0,\dots,1,\dots,0)$
    \item 
        \begin{enumerate}
            \item $2x=(2x)^2=2x^2+2x=2x+2x\Rightarrow x=-x.\ a+b=(a+b)^2=a^2+ab+ba+b^2=a+b+ab+ba.\Rightarrow 0=ab+ba=ab-ba\Rightarrow ab=ba.$
            \item $x(x-1)=0$. If $x\not=0$ and $x\not=1$, then $x$ is a zero divisor.
            \item Assume there is a boolean ring that has three elements $0,1,x$. If $x-1\not=1$, then $x-1=x$, contradicts to $2x=0$. Hence, $x-1=1$, this contradicts to $1+1=0$. Therefore, $A$ cannot have exactly $3$ elements.
            \item $1+x=y,1+y=x,x+y=1,xy=0.$
        \end{enumerate}
    \item Let $$f:\QQ\longrightarrow \QQ$$ be a automorphism. Then $$f(1)=1.$$
        For any $n\in\NN$, 
        $$f(n)=f\left(\sum_{i=1}^{n}1\right)=\sum_{i=1}^{n}f(1)=nf(1)=n.$$
        $$0=f(0)=f(n+(-n))=f(n)+f(-n)=n+f(-n).$$
        So, $f(-n)=-n$. Let $(n,m)\in \ZZ$,
        $$f(n)=f(m)f(\frac{n}{m}),$$
        $$f(\frac{n}{m})=\frac{n}{m}.$$
        Therefore, for any $x\in \QQ$, $f(x)=x$, which means 
        $$f=\mathrm{Id}_{\QQ}.$$
    \item  $0$ is the zero element and $1$ is the unit element. Then all the natural number are in the subfield. It is a group with the composition law $+$, by inverse law $-$, and the identity element $0$, all the integers are in the subfield. Similarly, all the rational numbers are in the subfield. So the only subfield of $\QQ$ is $\QQ$.
    \item 
        \begin{enumerate}
            \item $0\in M$. Assume $M\cap\NN_{>0}=\varnothing$, then for any non-zero elements $x,y\in M$, $x+y<0$, that contradicts to $x$ is invertible.
            \item If there exists a element equals $kd+m,k\in \ZZ,m\in\interval{1}{d-1} $, then $kd+m+k(-d)=m<d\in M$. Contradiction!
            \item By (2), $M\subseteq d\ZZ$. For any $nd\in d\ZZ, nd=\underset{n \text{ copies}}{\underbrace{d+d+\dots+d}}\in d\ZZ$. So $d\ZZ\subseteq M$. Therefore, $M=d\ZZ$.
        \end{enumerate}
    \setcounter{enumi}{8}
    \item 
        \begin{enumerate}
            \item We claim it is a left-$A$-module, then right-$A$-module follow the similar proof.First, it is a left action. For any $a\in A$, $ea=a$, where $e$ is neutral element. By definition, $a(bc)=(ab)c$. Second, check it is a left $A$-module. For any $(a,b)\in A^2, (c,d)\in A^2, ac+bc=(a+b)c, ac+ad=a(c+d).$
            \item If $[a_1]=[a_2],[b_1]=[b_2]$, then $a_1a_2^{-1}\in I,\ b_1b_2^{-1}\in I$. Hence $(a_1b_1)(a_2b_2)^{-1}=a_1(b_1b_2^{-1})a_2^{-1}\in I$. So it is well defined. 
            \newline
            By definition it is closed. For any $a,b,c\in A$, $[a]([b][c])=[a][bc]=[a(bc)]=[(ab)c]=([a][b])[c]$. So it is associative. For any $a\in A, [a][1]=[a]=[1][a]$. So it determines a structure of monoid on $A/I$.
            \item Similarly, we can prove $A/I$ equipped with additive law is well defined and forms a monoid. Then we only need to check that $A/I$ is a group under induced $+$. For any $a\in A, [a]+[-a]=[a+(-a)]=[0]$, where $0$ is the neutral element of $(A,+)$. Therefore, $A/I$ becomes a unitary ring. For ant $(a,b)\in A , \pi(a)\pi(b)=[a][b]=[ab]=\pi(ab), \pi(1)=[1]$. So $\pi:A\rightarrow A/I$ is a homomorphism of monoids. $\pi(a)+\pi(b)=[a]+[b]=[a+b]\pi(a+b).$ So it is also a homomorphism of groups. Therefore, $\pi$ is a homomorphism of rings.
            \item 
                \begin{enumerate}
                    \item For any $a\in A,x\in I,\ f(ax)=f(a)f(x)=f(a)0=0.$ So $ax\in I$. Similarly, $xa\in I$. So $I$ is an ideal of $A$.
                    \item $\forall (x,y,z)\in A^3, f(x)\left(f(y)f(z)\right)=f(x)f(yz)=f(x(yz))=f((xy)z)=f(xy)f(z)=\left(f(x)f(y)\right)f(z)$. $f(x)f(1)=f(x\cdot1)=f(x)$. So $\left(\mathrm{Im}f,\cdot\right)$ is a monoid. Similarly, we can deduce that $\left(\mathrm{Im}(f),+\right)$ is a monoid, in addition, $f(x)+f(-x)=f(x-x)=f(0)=0$. So $f(-x)$ is the inverse of $f(x)$. So $\mathrm{Im}(f)$ is the subring of $B$.
                    \item $\tilde{f}\left([x][y]\right)=f(xy)=f(x)f(y)=\tilde{f}\left([x]\right)\tilde{f}\left([y]\right)$. So it is a homomorphism. If $x,y\in A$ satisfy $x -y\in \ker{f}$, then $f(x)+f(-y)=f(x-y)=0.$ Thus $f(x)=f(y)$. So $\tilde{f}$ is a injective homomorphism of unitary rings. Therefore, it forms a isomorphism.
                \end{enumerate}
            \item Let $f:\ZZ\rightarrow A$ be a mapping. $f(n)=n1_A$ is a homomorphism. If $g$ is a homomorphism, then $f(1)=1_A,\ f(0)=0_A$. For any $n\in \NN, f(n)=f\left(\sum_{i=1}^{n}1\right)=\sum_{i=1}^{n}f(1)=\sum_{i=1}^{n}1_A$. For any $\dots$ 
        \end{enumerate}
    \item 
        \begin{enumerate}
            \item $$\left(\sum_{n\in \NN}a_nT^n\right)\dagger \left(\sum_{n\in \NN}b_nT^n\right)=\sum_{n\in \NN}(a_n+b_n)T^n$$
            $$=\sum_{n\in \NN}(b_n+a_n)T^n=\left(\sum_{n\in \NN}b_nT^n\right)\dagger\left(\sum_{n\in \NN}a_nT^n\right).$$
            So $\dagger$ is a communitative composition law.
            \newline
            For any $\sum_{n\in \NN}a_nT^n\in k[[T]]$,
            $$\left(\sum_{n\in \NN}a_nT^n\right)\dagger\sum_{n\in \NN}0T^n=\left(\sum_{n\in \NN}a_nT^n\right).$$
            So $\sum_{n\in\NN}0T^n$ is the neutral element of $k[[T]]$.
            \newline
            For any $\sum_{n\in \NN}a_nT^n\in k[[T]]$,
            $$\left(\sum_{n\in \NN}a_nT^n\right)\dagger\left(\sum_{n\in \NN}-a_nT^n\right)=\sum_{n\in \NN}0T^n,$$
            $$\left(\sum_{n\in \NN}-a_nT^n\right)\dagger\left(\sum_{n\in \NN}a_nT^n\right)=\sum_{n\in \NN}0T^n.$$
            Therefore, $k[[T]]$ equipped with $\dagger$ forms a communitative group.
            \item Note that, for any $\sum_{n\in \NN}a_nT^n\in k[[T]]$,
            $$\left(\sum_{n\in \NN}a_nT^n\right)*\mathds{1}=\sum_{n\in \NN}\left(\sum_{i=0}^{n}a_i\delta_{i,n}T^{n}\right)=\sum_{n\in \NN}a_nT^n.$$
            Hence $\mathds{1}$ is the neutral element of $k[[T]]$. One has 
            $$\sum_{i=0}^{n}a_ib_{n-i}=\sum_{t=n}^{0}a_{n-t}b_t=\sum_{t=0}^{n}b_ta_{n-t}.$$
            Thus, $*$ is communitative. Therefore, what given is a communitative monoid.
            \item $$a_i=a\delta_{i,n}, b_i=b\delta_{i,m}.$$
                $$(aT^n)(bT^m)=\sum_{k\in \NN}\sum_{i=0}^{k}ab\delta_{i,n}\delta_{k-i,m}T^k=abT^{n+m}.$$
            \item We only need to check it's distributive.
                \begin{align*}
                     &\left(\sum_{n\in \NN}a_nT^n\right)*\left[\left(\sum_{n\in \NN}b_nT^n\right)\dagger\left(\sum_{n\in \NN}c_nT^n\right)\right]\\
                    =&\left(\sum_{n\in \NN}\left(\sum_{i=0}^{n}a_i(b_{n-i}+c_{n-i})\right)T^n\right)\\
                    =&\left(\sum_{n\in \NN}\left(\sum_{i=0}^{n}a_ib_{n-i}T^n+\sum_{i=0}^{n}a_ic_{n-i}T^n\right)\right)\\
                    =&\left(\sum_{n\in \NN}\sum_{i=0}^{n}a_ib_{n-i}T^n\right)\dagger\left(\sum_{n\in \NN}\sum_{i=0}^{n}a_ic_{n-i}T^n\right)\\
                    =&\left(\sum_{n\in \NN}a_nT^n\right)*\left(\sum_{n\in \NN}b_nT^n\right)\dagger\left(\sum_{n\in \NN}a_nT^n\right)*\left(\sum_{n\in \NN}c_nT^n\right).
                \end{align*}
            \item 
                \begin{enumerate}
                    \item Suppose $f$ is invertible, and $\displaystyle g=\sum_{n\in \NN}b_nT^n$ be its inverse, then by (2), $(b_i)_{i\in \NN}$ satisfies:
                        $$\sum_{i=0}^{n}a_ib_{n-i}=\mathds{1},\forall n\in \NN.$$
                        Take $n=0$, we obtain $a_0$ must be invertible.
                    \item Suppose $a_0$ is invertible. For any $n\in \NN$, let 
                        $$b_{n+1}=\left(\mathds{1}-\sum_{i=1}^{n+1}(a_ib_{n+1-i})\right)a_0^{-1},$$
                        then,
                        $$\sum_{i=0}^{n+1}(a_ib_{n+1-i})=\mathds{1}.$$
                        Hence $\displaystyle g=\sum_{n\in \NN}b_nT^n$ is the inverse of $f$.
                \end{enumerate}
                \item Follow the algorithm in (5), we can easily get the result.
                    $$(1-aT)^{-1}=\sum_{n\in \NN}a^nT^n.$$
                \item -
                \item $k$ is communitative. We claim that $D$ is a homomorphism. 
                    \begin{align*}
                        D(f_1)\dagger D(f_2)=&\left(\sum_{n\in \NN}(n+1)a_{1,(n+1)}T^n\right)\dagger \left(\sum_{n\in \NN}(n+1)a_{2,(n+1)}T^n\right)\\
                        =&\sum_{n\in \NN}(n+1)(a_{1,(n+1)}+ a_{2,(n+1)})T^n\\
                        =&D\left(f'_1\dagger f'_2\right).
                    \end{align*}
                    $$
                        D\left(\sum_{n\in\NN}0T^n\right)=\sum_{n\in \NN}(n+1)0T^n=\sum_{n\in \NN}0T^n.
                    $$
                    Then we prove it is surjective. For any $f'=\sum_{n\in \NN}b_nT^n$, let $a_n=b_{n-1}(n-1)^{-1},\ n\not=0$, $D[\sum_{n\in\NN}a_nT^n]=f'$. Therefore $D$ is a surjective k-linear mapping.
                \item Let $\displaystyle f=\sum_{n\in \NN}a_nT^n\in \ker(D)$, then for any $n\in\NN,a_{n+1}=0.$ Thus, 
                    $$\ker(D)=k.$$
                \item $$a_{n+1}=a_n(n+1)^{-1}.$$
                    $$a_n=a_0\prod_{i=0}{n}(i+1)^{-1}.$$
                    $$f=\sum_{n\in\NN}a_0\prod_{i=0}{n}(i+1)^{-1}T^n,\ \forall a_0\in k.$$
        \end{enumerate}
    \item 
        \begin{enumerate}
            \item $$(a_n,b_n)\longmapsto \sum_{i=1}^{n}a_ib_{n-i}.$$
                If $n>\deg(F)$, then 
                $$\sum_{i=1}^{n}a_ib_{n-i}=\sum_{i=\deg(F)}^{n}a_ib_{n-i}.$$
                If $n<\deg(F)+\deg(G)$, then it will be $0$.
                If $n=\deg(F)+\deg(G)$, then
                $$\sum_{i=1}^{n}a_ib_{n-i}=a_{\deg(F)}b_{\deg(G)}\not=0.$$
                So $\deg(FG)=\deg(F)+\deg(G).$
            \item   Existence: If $\deg(F)<\deg(P)$ let $Q=0,R=F$. If $\deg(F)>\deg(P)$, let $F_{i+1}=F_{i}-a_{\deg(F)}T^{\deg(F)-\deg(P)}P, \ (F_0=F)$, then $\deg(F_{i+1})<\deg(F_i).$ Then come to the first case. 
                    \newline
                    Uniqueness: If $F=Q_1 P+R_1=Q_2P+R_2$. Then $(Q_1-Q_2)P=R_2-R_1$. If $Q_1\not=Q_2$, then $\deg(Q_1-Q_2)>0, \deg((Q_1-Q_2)P)>\deg(P)>\deg(R_1-R_2).$ This contradicts to $\deg((Q_1-Q_2)P)=\deg(R_2-R_1).$ Thus $Q_1=Q_2,\ R_1=R_2.$
            \item By (2) $F=PQ+R$. $R$ must be $0$, or it will contradicts to $\deg(P)$ is the least. Let $\deg(P_1)=\deg(P_2)$ be the least. Then there exists $Q\in I\backslash\{0\},\ P_1Q=P_2.$ Then $\deg(P_2)=\deg(P_1)+\deg(Q).$ So $\deg(Q)$ must be zero. Hence $Q=1.$, we proved the uniqueness. 
            \item Let $P'$ be the minimal polynomial, since $I$ is an ideal, $P=a_{\deg(P')}^{-1}P'\in I$. It is a monic polynomial.
            \item Easy.
        \end{enumerate} 
    \item
        \begin{enumerate}
            \item Let $Q=\sum_{n\in \NN}b_n T^n$, then $(T-x)Q=-b_0x+\sum_{n\in\NN^*}\left(-xb_n+b_{n-1}\right)T^n.$ 
                $$a_0=-b_0 x,\ a_n=-xb_n+b_{n-1}.$$
                It satisfies $P(0)=0$ automatically.
            \item By (1), then can write in the form. $\deg(T-x_i)=1$. By 11.(1), 
                $$d=\deg(P)=d_1+d_2+\dots +d_n+\deg(P)\ge d_1+\dots+d_n.$$
            \item Suppose it has more the $d$ root. Write it into (2) form. Then leads to a contradiction.
        \end{enumerate}
    \item
        \begin{enumerate}
            \item $[T][T]=[P-1]=[P]-I=I-I=0.$
            \item Let $F=P_0Q_0+R_0, Q_i=PQ_{i+1}+R_{i+1}$, $ [F]=[PQ_0]+[R_0]=[P][Q_0]+[c_0+b_0T]=[PQ_1]+[R_1]+b_0i=\left([Q_n]+\sum_{i\in \NN}^{n}b_ii\right)$ (Since $k[T]$ is formal, $n$ is finite.)
            \item Let $f$ be the mapping, $f$ is injective. By (2), $f$ is surjective. So it is a bijection. $[Q][R]=[QR]=[RQ]=[R][Q]$, hence it's communicative. $f(a,b)+f(c,d)=f(a+c,b+d)$, thus it is a homomorphism of group. $cf(a,b)=f(ca,cb)$. Therefore, $f$ is a $\RR$-linear bijection.
            \item -
            \item Easy. $\iota\circ\iota=\mathrm{Id}\Rightarrow $ bijection $\Rightarrow$ isomorphism.
            \item Easy
            \item $i=[T]=[-\iota(T)]=-i.$
            \item A bit confused.
            \item By communicative law.
            \item Find the inverse, and it is communicative.
            \item Easy.
            \item Emm.
            \item $\ZZ[i]^\times=\{1,-1,i,-i\}.$
        \end{enumerate}
    \item
        \begin{enumerate}
            \item Addition: group, $0$, Multiplication: monoid, $1$.
            \item Let $P=T^2-2$, $I$ be the ideal defined as 
                $$I:=P\RR[T].$$
                Then we can prove: If we denote $[T]$ as $\sqrt{2}$, then $\left(\sqrt{2}\right)^2=2.$ Then similar to 13.
            \item If $a^2-2b^2=\pm 1$, then $\pm(a-b\sqrt{2})$ is the inverse of $a+b\sqrt{2}.$ If $a+b\sqrt{2}$ is invertible, let $c+d\sqrt{2}$ be its inverse. Then $\left(ac+2bd+(ad+bc)\sqrt{2}\right)=\pm 1.$ Hence $ad+bc=0$ \textbf{TO BE CONTINUE.}
        \end{enumerate}
        \setcounter{enumi}{14}
    \item 
        \begin{enumerate}
            \item 
                \begin{enumerate}
                    \item (i)$\Rightarrow$ (ii): Let $\bar{b}$ be the inverse of $\bar{a}$. If $\bar{a}\bar{c}=0$, then
                        $$0=\bar{b}0=\bar{b}(\bar{a}\bar{c})=(\bar{b}\bar{a})\bar{c}=\bar{c}.$$
                        Hence $\bar{a}$ is not a zero divisor.
                    \item (ii)$\Rightarrow$ (iii): We prove by contradiction. Assume $\mathrm{gcd}(a,n)=k,1<k<n$.Then
                        $$\bar{a}\bar{\frac{n}{k}}=0.$$
                        That is contradicts to the fact that $\bar{a}$ is not a zero divisor.
                    \item (iii)$\Rightarrow$ (i): 
                \end{enumerate}
                \item By (1) (i)$\Rightarrow$ (iii), $(\ZZ/n\ZZ)^\times\subseteq\{k\mid k\in [0,n-1], \mathrm{gcd}(k,n)=1\}$.
                    \newline
                    By (1) (iii)$\Rightarrow$ (i), $\{k\mid k\in [0,n-1],\mathrm{gcd}(n,k)=1\}\subseteq (\ZZ/n\ZZ)^\times$.
                    \newline Hence $\{k\mid k\in [0,n-1],\mathrm{gcd}(n,k)=1\}=(\ZZ/n\ZZ)^\times$. 
                    $$\phi(n)=\#\{k\mid k\in [0,n-1],\mathrm{gcd}(n,k)=1\}.$$
                \item  Suppose $\bar{\alpha}$ is invertible and let $\bar{\beta}$ be its inverse. Then, 
                    $$\forall k\in \NN, \bar{k}=k\bar{\beta}\bar{\alpha}=(k\beta)\bar{\alpha}.$$
                    So $\ZZ/n\ZZ=\{k\alpha\}_{k\in\ZZ}$.
                    \newline
                    Conversely, if $\ZZ/n\ZZ=\{k\alpha\}_{k\in\ZZ}$, then there exists $k\in\ZZ$ such that $\bar{k}\bar{\alpha}=1$, which means $\bar{k}$ is $\bar{\alpha}$'s inverse. Thus, $\bar{\alpha}$ is invertible.
                \item -
                \item $\{x\mid x=a^n,n\in\ZZ\}$ forms a subgroup of $(\ZZ/n\ZZ)^\times$. By Lagrange theorem, its order is a divisor of $n$. So $\bar{a}^{\phi(n)}=1, a^{\phi(n)}\equiv 1 (\mathrm{\mod} n )$.
                \item There are $\frac{n}{p_i}$ elements in $\{k\in\NN^{*}\mid k\le n\}$ satisfies $\mathrm{gcd}(k,p_i)=p_i\not=1$. So, there are $n(1-\frac{1}{p_i})$ elements in $\{k\in\NN^{*}\mid k\le n\}$ satisfies $\mathrm{gcd}(k,p_i)=1$. By (4),
                    $$\phi(n)=n\prod_{i=1}^{k}(1-\frac{1}{p_i}).$$
                \item By definition of prime number, for any $n\in \NN^{*}, n<p, \mathrm{gcd}(n,p)=1$, so $\phi(p)=p-1.$ By (1), any element in $\ZZ/p\ZZ$ except $0$ is invertible. For any $\bar{a},\bar{b}\in \ZZ/p\ZZ, \bar{a}\bar{b}=\bar{ab}=\bar{ba}=\bar{b}\bar{a}.$ So $\ZZ/p\ZZ$ is commutative. Therefore $\ZZ/p\ZZ$ is a field. 
        \end{enumerate}
\end{enumerate}
\section{Linear Algebra: Vectors and Matrices}
\begin{enumerate}
    \setcounter{enumi}{5}
    \item 
        \begin{enumerate}
            \item $A=\begin{pmatrix}
                0&1\\
                0&0
            \end{pmatrix}$.
            \item $A=\begin{pmatrix}
                0&1\\
                -1&0
            \end{pmatrix}$.
            \item $A=
            \begin{pmatrix}
                \cos\left(\frac{2\pi}{3}\right)&\sin\left(\frac{2\pi}{3}\right)\\
                -\sin\left(\frac{2\pi}{3}\right)&\cos\left(\frac{2\pi}{3}\right)
            \end{pmatrix}.$
        \end{enumerate}
    \setcounter{enumi}{7}
    \item \begin{enumerate}
        \item Easy
        \item Easy
        \item $B$ is a non-zero solution. If $X$ is the inverse of $A$, then $AX=I$. But $B(AX)=(BA)X=0$. that contradicts to $BI=B$, for any non-zero matrix.
    \end{enumerate}
    \item \begin{enumerate}
        \item $J^2=\begin{pmatrix}
            -1&0\\
            0&-1
        \end{pmatrix}$. Easy
        \item Easy.
        \item If $(a,b)=0$, it is not invertible. If $(a,b)\not=0$, easy to check $M\left(\frac{a}{a^2+b^2},-\frac{b}{a^2+b^2}\right)$ is its inverse.
    \end{enumerate}
    \setcounter{enumi}{10}
    \item \begin{enumerate}
        \item Easy.
        \item Let $x$ be $1$ and $2$, then we obtain $1=a_n+b_n,\ 2^n=2a_n+b_n.$. Hence $a_n=2^n-1,\  b_n=2-2^n.$
        \item $A^n=\left(A^2-3A+2\right)Q(A)+(a_n A+b_n)=\dots.$
    \end{enumerate}
    \item For any $i,k,[b_ij],\sum_{j}a_{ij}b_{jk}=\sum_{j}b_{ij}a_{jk}$. Let $b_{ij}=\delta_{i,m}\delta_{j,n}$, then $a_{im}\delta_{k,n}=\delta_{i,m}a_{nk}$. So $a_{mn}=0$, if $n\not=m$. $a_{mm}=a_{nn}$, for any $m,n$. So $a_{ij}=\lambda \delta_{i,j}$.
    \item \begin{enumerate}
        \setcounter{enumii}{1}
        \item Easy.
        \item Easy.
    \end{enumerate}
    \item Easy.
    \setcounter{enumi}{16}
    \item Easy.
    \item \begin{enumerate}
        \item $a^{-1}=A$.
        \item $A_\lambda\left(A+\lambda I_n\right)/\left(1-\lambda^2\right)=I_n$
    \end{enumerate}
    \item \begin{enumerate}
        \item Easy. $(I+A)(I-\frac{1}{2}A)=I+\frac{1}{2}A-\frac{1}{2}A^2=I.$
        \item $A(A-I)=0$. If $C$ is the inverse of $A$, then $0=BA(A-I)=A-I$. So $A=I$. If $A=I$, then $I$ is its inverse.
    \end{enumerate}
\end{enumerate}
\end{document}