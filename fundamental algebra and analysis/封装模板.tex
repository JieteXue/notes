\usepackage{fix-cm}
\usepackage{lmodern}
\usepackage{newtxtext, geometry, amsmath, amssymb}
\usepackage{amsthm} % 设置定理环境,要在amsmath后用
\usepackage{bbm}
\usepackage{interval}
\usepackage{xpatch} % 用于去除amsthm定义的编号后面的点
\usepackage[strict]{changepage} % 提供一个 adjustwidth 环境
\usepackage{multicol} % 用于分栏
\usepackage{cancel}
\usepackage[fontsize=14pt]{fontsize} % 字号设置
\usepackage{esvect} % 实现向量箭头输入(显示效果好于\vec{}和\overrightarrow{}),格式为\vv{⟨向量符号⟩}或\vv*{⟨向量符号⟩}{⟨下标⟩} 
\usepackage[
            colorlinks, % 超链接以颜色来标识,而并非使用默认的方框来标识
            linkcolor=mlv,
            anchorcolor=mlv,
            citecolor=mlv % 设置各种超链接的颜色
            ]
            {hyperref} % 实现引用超链接功能
\usepackage{tcolorbox} % 盒子效果
\tcbuselibrary{breakable}
\tcbuselibrary{most} % tcolorbox宏包的设置,详见宏包说明文档

\geometry{a4paper,centering,scale=0.8}

% ————————————————————————————————————自定义符号————————————————————————————————————
\newcommand{\。}{.} % 示例:将指令\。定义为全角句点。
\newcommand{\zte}{\mathrm{e}} % 正体e
\newcommand{\ztd}{\mathrm{d}} % 正体d
\newcommand{\zti}{\mathrm{i}} % 正体i
\newcommand{\ztj}{\mathrm{j}} % 正体j
\newcommand{\ztk}{\mathrm{k}} % 正体k
\newcommand{\CC}{\mathbb{C}} % 黑板粗体C
\newcommand{\QQ}{\mathbb{Q}} % 黑板粗体Q
\newcommand{\RR}{\mathbb{R}} % 黑板粗体R
\newcommand{\ZZ}{\mathbb{Z}} % 黑板粗体Z
\newcommand{\NN}{\mathbb{N}} % 黑板粗体N
% 该文件包含了自定义符号的设置,格式为\newcommand{\⟨符号名⟩}{⟨符号内容⟩},如\newcommand{\R}{\mathbb{R}}表示实数集
% ————————————————————————————————————自定义颜色————————————————————————————————————
\definecolor{mlv}{RGB}{40, 137, 124} % 墨绿
\definecolor{qlv}{RGB}{240, 251, 248} % 浅绿
\definecolor{slv}{RGB}{33, 96, 92} % 深绿
\definecolor{qlan}{RGB}{239, 249, 251} % 浅蓝
\definecolor{slan}{RGB}{3, 92, 127} % 深蓝
\definecolor{hlan}{RGB}{82, 137, 168} % 灰蓝
\definecolor{qhuang}{RGB}{246, 250, 235} % 浅黄
\definecolor{shuang}{RGB}{77, 82, 59} % 深黄
\definecolor{qzi}{RGB}{240, 243, 252} % 浅紫
\definecolor{szi}{RGB}{49, 55, 70} % 深紫
% 将RGB换为rgb,颜色数值取值范围改为0到1

% ————————————————————————————————————盒子设置————————————————————————————————————
% tolorbox提供了tcolorbox环境,其格式如下:
% 第一种格式:\begin{tcolorbox}[colback=⟨背景色⟩, colframe=⟨框线色⟩, arc=⟨转角弧度半径⟩, boxrule=⟨框线粗⟩]   \end{tcolorbox}
% 其中设置arc=0mm可得到直角;boxrule可换为toprule/bottomrule/leftrule/rightrule可分别设置对应边宽度,但是设置为0mm时仍有细边,若要绘制单边框线推荐使用第二种格式
% 方括号内加上title=⟨标题⟩, titlerule=⟨标题背景线粗⟩, colbacktitle=⟨标题背景线色⟩可为盒子加上标题及其背景线
\newenvironment{box1}{
    \begin{tcolorbox}[colback=hlan, colframe=hlan, arc = 1mm]
    }
    {\end{tcolorbox}}
\newenvironment{box2}{
    \begin{tcolorbox}[colback=hlan!5!white, colframe=hlan, arc = 1mm]
    }
    {\end{tcolorbox}}
% 第二种格式:\begin{tcolorbox}[enhanced, colback=⟨背景色⟩, boxrule=0pt, frame hidden, borderline={⟨框线粗⟩}{⟨偏移量⟩}{⟨框线色⟩}]   {\end{tcolorbox}}
% 将borderline换为borderline east/borderline west/borderline north/borderline south可分别为四边添加框线,同一边可以添加多条
% 偏移量为正值时,框线向盒子内部移动相应距离,负值反之
\newenvironment{box3}{
    \begin{tcolorbox}[enhanced, colback=hlan!5!white, boxrule=0pt, frame hidden,
        borderline south={0.5mm}{0.1mm}{hlan}]
    }
    {\end{tcolorbox}}
























%                             颜色










    \newenvironment{lvse}{
    \begin{tcolorbox}[breakable,enhanced, colback=qlv, boxrule=0pt, frame hidden,
        borderline west={0.7mm}{0.1mm}{slv}]
    }
    {\end{tcolorbox}}
\newenvironment{lanse}{
    \begin{tcolorbox}[breakable,enhanced, colback=qlan, boxrule=0pt, frame hidden,
        borderline west={0.7mm}{0.1mm}{slan}]
    }
    {\end{tcolorbox}}
\newenvironment{huangse}{
    \begin{tcolorbox}[breakable,enhanced, colback=qhuang, boxrule=0pt, frame hidden,
        borderline west={0.7mm}{0.1mm}{shuang}]
    }
    {\end{tcolorbox}}
\newenvironment{zise}{
    \begin{tcolorbox}[breakable,enhanced, colback=qzi, boxrule=0pt, frame hidden,
        borderline west={0.7mm}{0.1mm}{szi}]
    }
    {\end{tcolorbox}}
% tcolorbox宏包还提供了\tcbox指令,用于生成行内盒子,可制作高光效果
\newcommand{\hl}[1]{
    \tcbox[on line, arc=0pt, colback=hlan!5!white, colframe=hlan!5!white, boxsep=1pt, left=1pt, right=1pt, top=1.5pt, bottom=1.5pt, boxrule=0pt]
{\bfseries \color{hlan}#1}}
% 其中on line将盒子放置在本行(缺失会跳到下一行),boxsep用于控制文本内容和边框的距离,left、right、top、bottom则分别在boxsep的参数的基础上分别控制四边距离












\newenvironment{box_blueness}{\begin{tcolorbox}
[colback=Emerald!10,colframe=cyan!40!black,title=\textbf{Definition}]
\end{tcolorbox}}

%挂链接(懒)https://zhuanlan.zhihu.com/p/336171630

% ————————————————————————————————————定理类环境设置————————————————————————————————————
\newtheoremstyle{t}{0pt}{}{}{\parindent}{\bfseries}{}{1em}{} % 定义新定理风格。格式如下:
%\newtheoremstyle{⟨风格名⟩}
%                {⟨上方间距⟩} % 若留空,则使用默认值
%                {⟨下方间距⟩} % 若留空,则使用默认值
%                {⟨主体字体⟩} % 如 \itshape
%                {⟨缩进长度⟩} % 若留空,则无缩进;可以使用 \parindent 进行正常段落缩进
%                {⟨定理头字体⟩} % 如 \bfseries
%                {⟨定理头后的标点符号⟩} % 如点号、冒号
%                {⟨定理头后的间距⟩} % 不可留空,若设置为 { },则表示正常词间间距;若设置为 {\newline},则环境内容开启新行
%                {⟨定理头格式指定⟩} % 一般留空
% 定理风格决定着由 \newtheorem 定义的环境的具体格式,有三种定理风格是预定义的,它们分别是:
% plain: 环境内容使用意大利斜体,环境上下方添加额外间距
% definition: 环境内容使用罗马正体,环境上下方添加额外间距
% remark: 环境内容使用罗马正体,环境上下方不添加额外间距
\theoremstyle{t} % 设置定理风格
\newtheorem{definitin}{\color{slv} Definition}[section] % 定义定义环境,格式为\newtheorem{⟨环境名⟩}{⟨定理头文本⟩}[⟨上级计数器⟩]或\newtheorem{⟨环境名⟩}[⟨共享计数器⟩]{⟨定理头文本⟩},其变体\newtheorem*不带编号
\newtheorem{theorem}{\color{shuang} Theorem}[section]
\newtheorem{example}{\color{szi} Example}[section]
\newtheorem*{solutipon}{\color{slan} Solution} % 定义无编号的解答环境
\newtheorem*{prof}{\color{slan} Proof} % 定义无编号的证明环境
\newtheorem{proposition}{\color{shuang}Proposition}[section]
\newtheorem{corollary}{\color{shuang}Corollary}[section]
\newtheorem{remark}{Remark}[section]
\newtheorem{axiom}{Axiom}
\newtheorem{notation}{\color{slv}Notation}[section]
\newtheorem{lemma}{Lemma}[section]
\newenvironment{definitionenv}{\begin{lvse}\begin{definitin}}{\end{definitin}\end{lvse}}
\newenvironment{notationenv}{\begin{lvse}\begin{notation}}{\end{notation}\end{lvse}}
\newenvironment{proofenv}{\begin{lanse}\begin{prof}}{$\hfill \square$\end{prof}\end{lanse}}
\newenvironment{theoremenv}{\begin{huangse}\begin{theorem}}{\end{theorem}\end{huangse}}
\newenvironment{exampleenv}{\begin{zise}\begin{example}}{\end{example}\end{zise}}
\newenvironment{propositionenv}{\begin{huangse}\begin{proposition}}{\end{proposition}\end{huangse}}
\newenvironment{corollaryenv}{\begin{huangse}\begin{corollary}}{\end{corollary}\end{huangse}}
\newenvironment{axiomenv}{\begin{zise}\begin{axiom}}{\end{axiom}\end{zise}}
% ————————————————————————————————————目录设置————————————————————————————————————
\setcounter{tocdepth}{4} % 设置在 ToC 的显示的章节深度
\setcounter{secnumdepth}{3} % 设置章节的编号深度
% 数值可选选项:-1 part 0 chapter 1 section 2 section 3 subsubsection 4 paragraph 5 subparagraph
%\renewcommand{\contentsname}{\bfseries 目\hspace{1em}录}
% ————————————————————————————————————目录设置————————————————————————————————————
