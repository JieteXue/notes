\chapter{Limit}
\section{Filters}
\begin{definitionenv}
    Let $X$ be a set. We call \textbf{filter} on $X$ any non-empty subset $\mathcal{F}$ of $\wp(X)$ this satisfies:
    \newline
    (1) $\forall(V_1,V_2)\in \mathcal{F}^2, V_1\cap V_2\in \mathcal{F}$.
    \newline
    (2) $ \forall V\in \mathcal{F}, \forall W\in \wp(X)$, if $V\subseteq W$, then $W\in \mathcal{F}$.
\end{definitionenv}
\begin{remark}
    \quad\newline
    If $\varnothing\in \mathcal{F}$, then $\mathcal{F}=\wp(X)$, we say that $\mathcal{F}$ is degenerate.

\begin{exampleenv}
    If $Y\subseteq X$, then 
    $$\mathcal{F}_Y:=\{V\in \wp(X)\mid Y\subseteq V\}$$
    is a filter, called the principal filter of $Y$.
\end{exampleenv}
If $\mathcal{F}$ is a non-degenerate filter such that, for any non-degenerate filter $\mathcal{G}$, one has $\mathcal{F}\nsubseteq \mathcal{G}$. We say that $\mathcal{F}$ is an \textbf{ultrafilter}.
\end{remark}
\begin{propositionenv}
    Let $I$ be a non-empty set and $(\mathcal{F}_i)_{i\in I}$ is a family of filters on $X$, then $\displaystyle \mathcal{F}:=\bigcap_{i\in I}\mathcal{F}_i$ is also a filter on $X$.
\end{propositionenv}
\begin{proofenv}
    \quad
    \newline
    (1) $\forall (V_1,V_2)\in \mathcal{F}^2$, one has 
    $$\forall i\in I, (V_1,V_2)\in \mathcal{F}_i^2,$$
    so $V_1\cap V_2\in \mathcal{F}_i$. This leads to $V_1\cap V_2\in \mathcal{F}$.
    \newline
    (2) $\forall V\in \mathcal{F}$, one has $\forall i\in I, V\in \mathcal{F}_i$. If $W\in \wp(X), W\supseteq V $, then $\forall i\in I, W\in \mathcal{F}_i$.
\end{proofenv}
\begin{definitionenv}
    Let $S$ be a subset of $\wp(X)$. We denote by $\mathcal{F}_S$ the intersection of all filters containing $S$. It is thus the least filter containing $S$. We call it the filter generated by $S$.
\end{definitionenv}
\begin{remark}
    If $Y\subseteq X$, then the principal filter $\mathcal{ F}_Y$ is generated by $\{Y\}$.
\end{remark}
\begin{propositionenv}
    Let $X$ be a set and $S$ be a non-empty subset of  $\wp(X)$, then 
    $$\mathcal{F}_S:=\{U\in \wp(X)\mid \exists n\in \NN_{\ge 1}, \exists(A_1,\dots,A_n)\in S^n, A_1\cap\cdots\cap A_n\subseteq U\}.$$
\end{propositionenv}
\begin{proofenv}
    Denote by $\mathcal{F}_S'$ the set on the right hand side of the equality. One has $\mathcal{F}_S'\subseteq \mathcal{F}_S$. It remains to check that $\mathcal{F}_S'$ is a filter containing $S$. By definition, $S\subseteq \mathcal{F}_S'$. If $(U,V)\in \mathcal{ F}_S'^2, \exists A_1,\dots,A_n,B_1,\dots,B_n\in S, A_1\cap\cdots\cap A_n\subseteq U, B_1\cap\cdots\cap B_n\subseteq V$, so $A_1\cap\cdots\cap A_n\cap B_1\cap\cdots\cap B_n\subseteq U\cap V$. If $W\supseteq U$, then $A_1\cap\cdots\cap A_n\subseteq W$, so $W\in \mathcal{F}_S'$.
\end{proofenv}
\begin{definitionenv}
    We say that a subset $S$ of $\wp(X)$ is a \textbf{filter basis} if, for any $(A,B)\in S\times S$, there exists $C\in S$, such that $C\subseteq A\cap B$.\footnote{If $n\in \NN_{\ge 1}$ and $(A_1,\dots,A_n)\in S^n,\exists C\in S$ such that $C\subseteq A_1\cap \dots\cap A_n$.}
\end{definitionenv}
\begin{remark}
    If $S$ is a filter basis, then 
    $$\mathcal{F}_S=\{U\in \wp(X)\mid \exists A\in S, A\subseteq U\}.$$
    If $S$ is a subset of $\wp(X)$, then 
    $$\mathcal{B}_S:=\{A_1\cap \cdots \cap A_n\mid n\in \NN,\ (A_1,\dots,A_n)\in S^n\}$$
     is a filter basis containing $S$. Moreover, $\mathcal{F}_S=\mathcal{F}_{\mathcal{B}_S}$.
\end{remark}
\begin{propositionenv}
    Let $X$ be a set. Then
    $$\mathcal{F}=\{U\in \wp(X)\mid X\backslash U \text{ is finite}\}$$
    is a filter on $X$. We call it the \textbf{Fréchet filter} of $X$.
\end{propositionenv}
\begin{proofenv}
    \quad
    \newline
    If $(U,V)\in \mathcal{F}^2, X\backslash (U\cap V)=(X\backslash U)\cup (X\backslash V)$, is finite.
    \newline
    If $U\in \mathcal{F}, W\in \wp(X), U\subseteq W$, then $(X\backslash W)\subseteq (X\backslash U)$ is finite.
\end{proofenv}
\begin{exampleenv}
    Let $I\subseteq \NN$ be an infinite set. Let $J\subseteq \NN$ be infinite, then $\{I_{\ge j}\mid j\in J \}$ is a filter basis that generates the Fréchet filter of $I$. $\{I_{\ge j}\mid j\in J\}$ is a totally ordered subset of $\wp(I)$, so it is a filter basis. For any $j\in J, I\backslash I_{\ge j}=I_{<j}$ is finite. Let $U\in $ Fréchet filter of $I$, $I\backslash U$ is finite. There exists $j\in J$ such that $\forall i\in I\backslash U,i<j$. So $I\backslash U\subseteq I_{<j}, U\supseteq I\backslash I_{j<}=I_{\ge j}$ .
\end{exampleenv}
\begin{exampleenv}
    Let $X$ be a set. We call \textbf{pseudometric} on $X$ any mapping
    $$\mathrm{d}: X\times X\to \RR_{\ge 0}.$$ 
    such that,
    \newline
    (1) $\forall x\in X, \mathrm{d}(x,x)=0.$
    \newline
    (2) $\forall (x,y)\in X^2, \mathrm{d}(x,y)=\mathrm{d}(y,x)$.
    \newline
    (3) (Triangle inequality) $\forall (x,y,z)\in X^3, \mathrm{d}(x,z)\le \mathrm{d}(x,y)+\mathrm{d}(y,z).$
    \newline
    $(X,d)$ is called the \textbf{pseudometric space}. If $$\forall (x,y)\in X^2,x\not=y\Rightarrow \mathrm{d}(x,y)>0,$$ then $(X,d)$ is called a \textbf{metric space}.

    Let $(X,d)$ be a pseudometric space. For any $x\in X$, and $\varepsilon\in \RR_{\ge 0}$, we denote by $B(x,\varepsilon)$ the set 
    $$\{y\in X\mid \mathrm{d}(x,y)<\varepsilon\},$$
    called the \textbf{open ball} \textit{center at $x$ of radius $\varepsilon$.} 
    
    Then  
    $$\mathcal{V}_x:=\{U\in \wp(X)\mid \exists \varepsilon\in \RR_{>0}, B(x,\varepsilon)\subseteq U\}$$
    is a filter, called the \textbf{filter of neighborhood} of $x$.
\end{exampleenv}
\begin{propositionenv}
    Let $J\subseteq \RR_{>0}$ be a non-empty subset such that $\inf J=0$. Then $\mathcal{B}_J=\{B(x,\varepsilon)\mid \varepsilon\in J\}$ is a filter basis such that $\mathcal{F}_{\mathcal{B}_J}=\mathcal{V}_x$.
\end{propositionenv}
\begin{proofenv}
    $\forall U\in \mathcal{V}_x,\exists \varepsilon\in J,\varepsilon<\delta$,
    $$B(x,\varepsilon)\subseteq B(x,\delta)\subseteq U.$$
\end{proofenv}
\section{Order Limit}
We fix a partially ordered set $(G,\le)$ assumed to be order complete.
\begin{exampleenv}
    \quad
    \newline
    (1) $\RR\cup\{-\infty,+\infty\}, \ \forall x\in \RR, -\infty<x<+\infty$.
    \newline
    (2) $\interval{0}{+\infty}$.
    \newline
    (3) $\left(\wp(\Omega),\subseteq\right)$.
\end{exampleenv}
\begin{definitionenv}
    Let $X$ bw a set and $f:X\longrightarrow G$ be a mapping. For any $U\in \wp(X)$, we define 
    $$f^s(U):=\sup_{x\in U}f(x)=\sup f(U).$$
    $$f^i(U):=\inf_{x\in U}f(x)=\inf f(U).$$
    If $U\not=\varnothing$, $f^s(U)\ge f^i(U)$. Let $\mathcal{F}$ be a filter on $X$. We define 
    $$\limsup_\mathcal{F} f:=\inf_{U\in \mathcal{F}} f^s(U).$$
    $$\liminf_\mathcal{F} f:=\sup_{U\in \mathcal{F}} f^i(U).$$
    They are called the \textbf{superior limit} and the \textbf{inferior limit} of $f$ along $\mathcal{F}$. If 
    $$\liminf_\mathcal{F} f=\limsup_\mathcal{F} f,$$
    we say that $f$ has a limit along $\mathcal{F}$, and we denote $\displaystyle \lim_{\mathcal{F}}f$ this value.
\end{definitionenv}
\begin{notationenv}
    Let $I\subseteq \NN$ be an infinite subset. We call sequence in $G$ parametrized by $I$ any element of $G^I=\{(a_n)_{n\in I}\mid \forall n\in I,a_n\in G\}$. If $\mathcal{F}$ is the Fréchet filter on $I$, then for any $f=(a_n)_{n\in I}\in G^I$, $\limsup_{\mathcal{F}}f$ is denote as $\displaystyle\limsup_{n\in I, n\rightarrow+\infty}a_n$ or as $\limsup_{n\rightarrow +\infty}a_n$. Resp. $\liminf$.
\end{notationenv}
\begin{propositionenv}
    Let $f:X\longrightarrow G$ be a mapping and $\mathcal{F}$ be a non-degenerate filter. Then 
    $$\forall (U,V)\in \mathcal{F}\times\mathcal{F}, f^{s}(U)\ge f^i(U).$$
    In particular 
    $$\limsup_\mathcal{F}f\ge \liminf_\mathcal{F}f.$$
\end{propositionenv}
\begin{proofenv}
    $$f^s(U)\ge f^s(U\cap V)\ge f^i(U\cap V)\ge f^i(V).$$
    Taking $\inf_{U\in \mathcal{F}}$, we get $\forall V\in \mathcal{F}$, $\limsup_{\mathcal{F}}f\ge f^i(V)$. Taking $\sup_{V\in \mathcal{F}}$, we get $\limsup_{\mathcal{F}}f\ge \liminf_{\mathcal{F}}f$.
\end{proofenv}
\begin{propositionenv}
    Let $f:X\longrightarrow G$ be a mapping, $\mathcal{B}$ be a filter basis on $X$ and $\mathcal{F}$ be the filter generated by $\mathcal{B}$. Then
    $$\limsup _{\mathcal{F}}f=\inf_{B\in \mathcal{B}}f^s(B),\ \liminf _{\mathcal{F}}f=\sup_{B\in \mathcal{B}}f^i(B).$$
\end{propositionenv}
\begin{proofenv}
    Since $\mathcal{B}\subseteq \mathcal{F}$, one has 
    $$\limsup_{\mathcal{F}}f=\inf_{U\in \mathcal{F}}f^s(U)\le \inf_{B\in \mathcal{B}}f^s(B).$$
    For any $U\in \mathcal{F},\exists A\in \mathcal{B}$ such that $U\supseteq A$. One has
    $$f^s(U)\ge f^s(A)\ge \inf_{B\in \mathcal{B}}f^s(B).$$
    Taking $\inf_{U\in \mathcal{F}}$, we get
    $$\limsup_{\mathcal{F}}f\ge \inf_{B\in \mathcal{B}}f^s(B).$$
\end{proofenv}
\begin{box2}
\textbf{Consequence:} \quad If $I\subseteq\NN$ is an infinite subset, $J\subseteq \NN$ is another infinite subset, $\forall (a_n)_{n\in I}\in G^I$,
$$\limsup_{n\in I,n\rightarrow+\infty}a_n=\inf_{j\in J,n\in I_{\ge j}}a_n,$$
$$\liminf_{n\in I,n\rightarrow+\infty}a_n=\sup_{j\in J,n\in I_{\ge j}}a_n.$$
\end{box2}
\begin{exampleenv}
     $a_n=(-1)^n, (a_n)_{n\in \NN}\in [-\infty,+\infty]^N$,
     $$\limsup_{n\rightarrow +\infty}(-1)^n=\inf_{j\in 2\NN}\sup_{n\ge j}(-1)^n=\inf_{j\in 2\NN}1=1.$$
     $$\liminf_{n\rightarrow+\infty}(-1)^n=-1.$$
\end{exampleenv}
\begin{exampleenv}
    $\left(\frac{1}{n}\right)_{n\in \NN_{\ge 1}}$,
    $$\limsup_{n\rightarrow+\infty}\frac{1}{n}=\inf_{j\in \NN_{\ge 1}}\sup_{n\ge j}\frac{1}{n}=\inf_{j\in \NN_{\ge 1}}\frac{1}{j}=0,$$
    $$\liminf_{n\rightarrow+\infty}\frac{1}{n}=\sup_{j\in \NN_{\ge 1}}\inf_{n\ge j}\frac{1}{n}=\sup_{j\in \NN_{\ge 1}}\frac{1}{j}=0.$$
\end{exampleenv}
\begin{propositionenv}
    Let $f,g:X\longrightarrow G$ be mappings and $\mathcal{F}$ be a filter on $X$. Suppose that there exists $A\in \mathcal{F}$ such that 
    $$\forall x\in A, f(x)\le g(x).$$
    Then,
    $$\limsup_{\mathcal{F}}f\le \limsup_{\mathcal{F}}g,\ \liminf_{\mathcal{F}}f\le \liminf_{\mathcal{F}}g.$$ 
\end{propositionenv}
\begin{proofenv}
    Let 
    $$\mathcal{B}=\{U\in \mathcal{F}\mid  U\subseteq A\}.$$
    $\mathcal{B}$ is a filter basis, and $\mathcal{B}\in \mathcal{F}$. For any $V\in \mathcal{ F}$, one has $V\cap A\in \mathcal{B}$ and $V\supseteq V\cap A$. So $\mathcal{F}$ is generated by $\mathcal{ B}$. For any $B\in \mathcal{B}$, one has $B\subseteq A$ and hence
    $$f^s(B)\le g^s(B),\ f^i(B)\le g^i(B).$$
    So 
    $$\inf_{B\in \mathcal{B}}f^s(B)\le \inf_{B\in \mathcal{B}}g^s(B),\ \sup_{B\in \mathcal{B}}f^i(B)\le \sup_{B\in \mathcal{B}}g^i(B).$$
\end{proofenv}
\begin{theoremenv}[Squeeze Theorem]
    Let $X$ be a set and $\mathcal{F}$ be a non-degenerate filter on $X$. Let $f,g,h$ be elements of $G^X$. Assume that there exists $A\in \mathcal{F}$ such that 
    $$\forall x\in A, f(x)\le g(x)\le h(x).$$
    If $f$ and $h$ have limits along $\mathcal{F}$, and 
    $$\lim_{\mathcal{F}}f=\lim_{\mathcal{F}}h,$$
    then, $g$ also has a limit along $\mathcal{F}$, and 
    $$\lim_{\mathcal{F}}f=\lim_{\mathcal{F}}g=\lim_{\mathcal{F}}h.$$
\end{theoremenv}
\begin{proofenv}
    $$\lim_{\mathcal{F}}f=\limsup_{\mathcal{F}}f\le \limsup_{\mathcal{F}}g\le \limsup_{\mathcal{F}}h=\lim_{\mathcal{F}}h.$$
    So 
    $$\limsup_{\mathcal{F}} g=\lim_{\mathcal{F}}f=\lim_{\mathcal{F}}h.$$
    $$\lim_{\mathcal{F}}f=\liminf_{\mathcal{F}}f\le \liminf_{\mathcal{F}}g\le \liminf_{\mathcal{F}}h=\lim_{\mathcal{F}}h.$$
    So 
    $$\liminf_{\mathcal{F}} g=\lim_{\mathcal{F}}f=\lim_{\mathcal{F}}h.$$
\end{proofenv}
\begin{exampleenv}
    Let $a>1$. Consider the sequence $\left(\frac{a^n}{n!}\right)_{n\in \NN}$. If $n\ge \NN\ge 2a $, $a\le \frac{N}{2}$, then 
    $$0\le \frac{a^n}{n!}\le \frac{a^N}{N!}\cdot\frac{a^{n-N}}{(N+1)\dots n}\le \frac{a^N}{N!}\frac{1}{2^{n-N}}.$$
    For any $n\ge N$, $0\le \frac{a^n}{n!}\le \frac{(2a)^N}{N!}\cdot \frac{1}{2^n}.$ So by squeeze theorem, $\displaystyle\lim_{n\rightarrow+\infty}\frac{a^n}{n!}=0$.
\end{exampleenv}
\begin{theoremenv}[Monotone Convergence Theorem]
    Let $I$ be an infinite subset of $\NN$ and $(a_n)_{n\in I}\in G^I$.
    \newline
    (1) If $(a_n)_{n\in I}$ is increasing, then $(a_n)_{n\in I}$ admits $\displaystyle \sup_{n\in i}a_n$ as its limit.
    \newline
    (2) If $(a_n)_{n\in I}$ is decreasing, then $(a_n)_{n\in I}$ admits $\displaystyle \inf_{n\in i}a_n$ as its limit.
    
\end{theoremenv}
\begin{proofenv}
    \quad\newline
    (1) Let $l=\sup_{n\in I} a_n$, $\forall n\in \NN,\ a_n\le l$. So 
    $$\limsup_{n\rightarrow +\infty}a_n\le\limsup_{n\rightarrow +\infty}l=l.$$
    $$\forall j\in I, \inf_{n\in I_{\ge j}}a_n=a_j,$$
    so $$\limsup_{n\rightarrow+\infty}a_n=\sup_{j\in I}\inf_{n\in I_{\ge j}}a_n=\sup_{j\in I}a_j=l.$$
    Hence,
    $$l=\liminf_{n\rightarrow+\infty}a_n\le \limsup_{n\rightarrow+\infty}a_n\le l .$$
    Which means 
    $$\lim_{n\rightarrow +\infty}a_n=l.$$
\end{proofenv}
\begin{propositionenv}
    Let $X$be a set and $Y\subseteq X$.\newline
    (1) If $\mathcal{F}$ is a filter on $X$, then 
    $$\left.\mathcal{F}\right|_Y:=\{U\cap Y\mid U\in \mathcal{F}\}$$
    is a filter on $Y$.
    \newline
    (2) If $\mathcal{B}$ is a filter basis on $X$, and $\mathcal{F}$ is the filter generated by $\mathcal{B}$, then 
    $$\left.\mathcal{B}\right|_Y:=\{B\cap Y\mid B\in \mathcal{B}\}$$
    is a filter basis generates $\left.\mathcal{F}\right|_Y$.
\end{propositionenv}
\begin{proofenv}
    \quad 
    \newline
    (1) Let $U$ and $V$ be elements of $\mathcal{ F}$, one has 
    $$(U\cap Y)\cap (V\cap Y)=(U\cap V)\cap Y\in \left.\mathcal{F}\right|_Y.$$
    Let $U\in \mathcal{F}, W\subseteq Y,U\cap Y\subseteq W$. Let $V=U\cup W\in \mathcal{F}$.
    $$Y\cap V=(U\cap Y)\cup(W\cap Y)=W.$$
    Hence $W\in \left.\mathcal{F}\right|_Y.$
    \newline
    (2) Let $B_1,B_2$ be elements of $\mathcal{B}$, then $\exists A\in B, A\subseteq B_1\cap B_2.$ Thus 
    $$A\cap Y\subseteq(B_1\cap Y)\cap (B_2\cap Y).$$
    So $\left.\mathcal{ B}\right|_Y$ is a filter basis. Moreover, $\left.\mathcal{B}\right|_Y\subseteq\left.\mathcal{F}\right|_Y$. Let $U\in J, \exists B\in \mathcal{B}$ such that $B\subseteq U$. Thus 
    $$B\cap Y\subseteq U\cap Y.$$
    So $U\cap Y$  contains an element of $\left.\mathcal{B}\right|_Y$.
\end{proofenv}
\begin{exampleenv}
    Let $I\subseteq \NN$ be an infinite subset, and $(a_n)_{n\in I}\in G^I$. If $J\subseteq I$ is an infinite subset, $\mathcal{F}$ be the filter on $I$, then $\mathcal{F}|_J$ is the Fréchet filter on $J$. $(a_n)_{n\in J}$ is called a subsequence of $(a_n)_{n\in I}$.
\end{exampleenv}