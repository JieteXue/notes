\chapter{Limit}
\section{Filters}
\begin{definitionenv}
    Let $X$ be a set. We call \textbf{filter} on $X$ any non-empty subset $\mathcal{F}$ of $\wp(X)$ this satisfies:
    \newline
    (1) $\forall(V_1,V_2)\in \mathcal{F}^2, V_1\cap V_2\in \mathcal{F}$.
    \newline
    (2) $ \forall V\in \mathcal{F}, \forall W\in \wp(X)$, if $V\subseteq W$, then $W\in \mathcal{F}$.
\end{definitionenv}
\begin{remark}
    \quad\newline
    If $\varnothing\in \mathcal{F}$, then $\mathcal{F}=\wp(X)$, we say that $\mathcal{F}$ is degenerate.

\begin{exampleenv}
    If $Y\subseteq X$, then 
    $$\mathcal{F}_Y:=\{V\in \wp(X)\mid Y\subseteq V\}$$
    is a filter, called the principal filter of $Y$.
\end{exampleenv}
If $\mathcal{F}$ is a non-degenerate filter such that, for any non-degenerate filter $\mathcal{G}$, one has $\mathcal{F}\nsubseteq \mathcal{G}$. We say that $\mathcal{F}$ is an \textbf{ultrafilter}.
\end{remark}
\begin{propositionenv}
    Let $I$ be a non-empty set and $(\mathcal{F}_i)_{i\in I}$ is a family of filters on $X$, then $\displaystyle \mathcal{F}:=\bigcap_{i\in I}\mathcal{F}_i$ is also a filter on $X$.
\end{propositionenv}
\begin{proofenv}
    \quad
    \newline
    (1) $\forall (V_1,V_2)\in \mathcal{F}^2$, one has 
    $$\forall i\in I, (V_1,V_2)\in \mathcal{F}_i^2,$$
    so $V_1\cap V_2\in \mathcal{F}_i$. This leads to $V_1\cap V_2\in \mathcal{F}$.
    \newline
    (2) $\forall V\in \mathcal{F}$, one has $\forall i\in I, V\in \mathcal{F}_i$. If $W\in \wp(X), W\supseteq V $, then $\forall i\in I, W\in \mathcal{F}_i$.
\end{proofenv}
\begin{definitionenv}
    Let $S$ be a subset of $\wp(X)$. We denote by $\mathcal{F}_S$ the intersection of all filters containing $S$. It is thus the least filter containing $S$. We call it the filter generated by $S$.
\end{definitionenv}
\begin{remark}
    If $Y\subseteq X$, then the principal filter $\mathcal{ F}_Y$ is generated by $\{Y\}$.
\end{remark}
\begin{propositionenv}
    Let $X$ be a set and $S$ be a non-empty subset of  $\wp(X)$, then 
    $$\mathcal{F}_S:=\{U\in \wp(X)\mid \exists n\in \NN_{\ge 1}, \exists(A_1,\dots,A_n)\in S^n, A_1\cap\cdots\cap A_n\subseteq U\}.$$
\end{propositionenv}
\begin{proofenv}
    Denote by $\mathcal{F}_S'$ the set on the right hand side of the equality. One has $\mathcal{F}_S'\subseteq \mathcal{F}_S$. It remains to check that $\mathcal{F}_S'$ is a filter containing $S$. By definition, $S\subseteq \mathcal{F}_S'$. If $(U,V)\in \mathcal{ F}_S'^2, \exists A_1,\dots,A_n,B_1,\dots,B_n\in S, A_1\cap\cdots\cap A_n\subseteq U, B_1\cap\cdots\cap B_n\subseteq V$, so $A_1\cap\cdots\cap A_n\cap B_1\cap\cdots\cap B_n\subseteq U\cap V$. If $W\supseteq U$, then $A_1\cap\cdots\cap A_n\subseteq W$, so $W\in \mathcal{F}_S'$.
\end{proofenv}
\begin{definitionenv}
    We say that a subset $S$ of $\wp(X)$ is a \textbf{filter basis} if, for any $(A,B)\in S\times S$, there exists $C\in S$, such that $C\subseteq A\cap B$.\footnote{If $n\in \NN_{\ge 1}$ and $(A_1,\dots,A_n)\in S^n,\exists C\in S$ such that $C\subseteq A_1\cap \dots\cap A_n$.}
\end{definitionenv}
\begin{remark}
    If $S$ is a filter basis, then 
    $$\mathcal{F}_S=\{U\in \wp(X)\mid \exists A\in S, A\subseteq U\}.$$
    If $S$ is a subset of $\wp(X)$, then 
    $$\mathcal{B}_S:=\{A_1\cap \cdots \cap A_n\mid n\in \NN,\ (A_1,\dots,A_n)\in S^n\}$$
     is a filter basis containing $S$. Moreover, $\mathcal{F}_S=\mathcal{F}_{\mathcal{B}_S}$.
\end{remark}
\begin{propositionenv}
    Let $X$ be a set. Then
    $$\mathcal{F}=\{U\in \wp(X)\mid X\backslash U \text{ is finite}\}$$
    is a filter on $X$. We call it the \textbf{Fréchet filter} of $X$.
\end{propositionenv}
\begin{proofenv}
    \quad
    \newline
    If $(U,V)\in \mathcal{F}^2, X\backslash (U\cap V)=(X\backslash U)\cup (X\backslash V)$, is finite.
    \newline
    If $U\in \mathcal{F}, W\in \wp(X), U\subseteq W$, then $(X\backslash W)\subseteq (X\backslash U)$ is finite.
\end{proofenv}
\begin{exampleenv}
    Let $I\subseteq \NN$ be an infinite set. Let $J\subseteq \NN$ be infinite, then $\{I_{\ge j}\mid j\in J \}$ is a filter basis that generates the Fréchet filter of $I$. $\{I_{\ge j}\mid j\in J\}$ is a totally ordered subset of $\wp(I)$, so it is a filter basis. For any $j\in J, I\backslash I_{\ge j}=I_{<j}$ is finite. Let $U\in $ Fréchet filter of $I$, $I\backslash U$ is finite. There exists $j\in J$ such that $\forall i\in I\backslash U,i<j$. So $I\backslash U\subseteq I_{<j}, U\supseteq I\backslash I_{j<}=I_{\ge j}$ .
\end{exampleenv}
\begin{exampleenv}
    Let $X$ be a set. We call \textbf{pseudometric} on $X$ any mapping
    $$\mathrm{d}: X\times X\to \RR_{\ge 0}.$$ 
    such that,
    \newline
    (1) $\forall x\in X, \mathrm{d}(x,x)=0.$
    \newline
    (2) $\forall (x,y)\in X^2, \mathrm{d}(x,y)=\mathrm{d}(y,x)$.
    \newline
    (3) (Triangle inequality) $\forall (x,y,z)\in X^3, \mathrm{d}(x,z)\le \mathrm{d}(x,y)+\mathrm{d}(y,z).$
    \newline
    $(X,d)$ is called the \textbf{pseudometric space}. If $$\forall (x,y)\in X^2,x\not=y\Rightarrow \mathrm{d}(x,y)>0,$$ then $(X,d)$ is called a \textbf{metric space}.

    Let $(X,d)$ be a pseudometric space. For any $x\in X$, and $\varepsilon\in \RR_{\ge 0}$, we denote by $B(x,\varepsilon)$ the set 
    $$\{y\in X\mid \mathrm{d}(x,y)<\varepsilon\},$$
    called the \textbf{open ball} \textit{center at $x$ of radius $\varepsilon$.} 
    
    Then  
    $$\mathcal{V}_x:=\{U\in \wp(X)\mid \exists \varepsilon\in \RR_{>0}, B(x,\varepsilon)\subseteq U\}$$
    is a filter, called the \textbf{filter of neighborhood} of $x$.
\end{exampleenv}
\begin{propositionenv}
    Let $J\subseteq \RR_{>0}$ be a non-empty subset such that $\inf J=0$. Then $\mathcal{B}_J=\{B(x,\varepsilon)\mid \varepsilon\in J\}$ is a filter basis such that $\mathcal{F}_{\mathcal{B}_J}=\mathcal{V}_x$.
\end{propositionenv}
\begin{proofenv}
    $\forall U\in \mathcal{V}_x,\exists \varepsilon\in J,\varepsilon<\delta$,
    $$B(x,\varepsilon)\subseteq B(x,\delta)\subseteq U.$$
\end{proofenv}
\section{Order Limit}
We fix a partially ordered set $(G,\le)$ assumed to be order complete.
\begin{exampleenv}
    \quad
    \newline
    (1) $\RR\cup\{-\infty,+\infty\}, \ \forall x\in \RR, -\infty<x<+\infty$.
    \newline
    (2) $\interval{0}{+\infty}$.
    \newline
    (3) $\left(\wp(\Omega),\subseteq\right)$.
\end{exampleenv}
\begin{definitionenv}
    Let $X$ be a set and $f:X\longrightarrow G$ be a mapping. For any $U\in \wp(X)$, we define 
    $$f^s(U):=\sup_{x\in U}f(x)=\sup f(U).$$
    $$f^i(U):=\inf_{x\in U}f(x)=\inf f(U).$$
    If $U\not=\varnothing$, $f^s(U)\ge f^i(U)$. Let $\mathcal{F}$ be a filter on $X$. We define 
    $$\limsup_\mathcal{F} f:=\inf_{U\in \mathcal{F}} f^s(U).$$
    $$\liminf_\mathcal{F} f:=\sup_{U\in \mathcal{F}} f^i(U).$$
    They are called the \textbf{superior limit} and the \textbf{inferior limit} of $f$ along $\mathcal{F}$. If 
    $$\liminf_\mathcal{F} f=\limsup_\mathcal{F} f,$$
    we say that $f$ has a limit along $\mathcal{F}$, and we denote $\displaystyle \lim_{\mathcal{F}}f$ this value.
\end{definitionenv}
\begin{notationenv}
    Let $I\subseteq \NN$ be an infinite subset. We call sequence in $G$ parametrized by $I$ any element of $G^I=\{(a_n)_{n\in I}\mid \forall n\in I,a_n\in G\}$. If $\mathcal{F}$ is the Fréchet filter on $I$, then for any $f=(a_n)_{n\in I}\in G^I$, $\limsup_{\mathcal{F}}f$ is denote as $\displaystyle\limsup_{n\in I, n\rightarrow+\infty}a_n$ or as $\displaystyle\limsup_{n\rightarrow +\infty}a_n$. Resp. $\liminf$.
\end{notationenv}
\begin{propositionenv}
    Let $f:X\longrightarrow G$ be a mapping and $\mathcal{F}$ be a non-degenerate filter. Then 
    $$\forall (U,V)\in \mathcal{F}\times\mathcal{F}, f^{s}(U)\ge f^i(V).$$
    In particular 
    $$\limsup_\mathcal{F}f\ge \liminf_\mathcal{F}f.$$
\end{propositionenv}
\begin{proofenv}
    $$f^s(U)\ge f^s(U\cap V)\ge f^i(U\cap V)\ge f^i(V).$$
    Taking $\dis\inf_{U\in \mathcal{F}}$, we get $\forall V\in \mathcal{F}$, $\limsup_{\mathcal{F}}f\ge f^i(V)$. Taking $\dis\sup_{V\in \mathcal{F}}$, we get $\limsup_{\mathcal{F}}f\ge \liminf_{\mathcal{F}}f$.
\end{proofenv}
\begin{propositionenv}
    Let $f:X\longrightarrow G$ be a mapping, $\mathcal{B}$ be a filter basis on $X$ and $\mathcal{F}$ be the filter generated by $\mathcal{B}$. Then
    $$\limsup _{\mathcal{F}}f=\inf_{B\in \mathcal{B}}f^s(B),\ \liminf _{\mathcal{F}}f=\sup_{B\in \mathcal{B}}f^i(B).$$
\end{propositionenv}
\begin{proofenv}
    Since $\mathcal{B}\subseteq \mathcal{F}$, one has 
    $$\limsup_{\mathcal{F}}f=\inf_{U\in \mathcal{F}}f^s(U)\le \inf_{B\in \mathcal{B}}f^s(B).$$
    For any $U\in \mathcal{F},\exists A\in \mathcal{B}$ such that $U\supseteq A$. One has
    $$f^s(U)\ge f^s(A)\ge \inf_{B\in \mathcal{B}}f^s(B).$$
    Taking $\inf_{U\in \mathcal{F}}$, we get
    $$\limsup_{\mathcal{F}}f\ge \inf_{B\in \mathcal{B}}f^s(B).$$
\end{proofenv}
\begin{box2}
\textbf{Consequence:} \quad If $I\subseteq\NN$ is an infinite subset, $J\subseteq \NN$ is another infinite subset, $\forall (a_n)_{n\in I}\in G^I$,
$$\limsup_{n\in I,n\rightarrow+\infty}a_n=\inf_{j\in J}\sup_{n\in I_{\ge j}}a_n,$$
$$\liminf_{n\in I,n\rightarrow+\infty}a_n=\sup_{j\in J}\inf_{n\in I_{\ge j}}a_n.$$
\end{box2}
\begin{exampleenv}
     $a_n=(-1)^n, (a_n)_{n\in \NN}\in [-\infty,+\infty]^N$,
     $$\limsup_{n\rightarrow +\infty}(-1)^n=\inf_{j\in 2\NN}\sup_{n\ge j}(-1)^n=\inf_{j\in 2\NN}1=1.$$
     $$\liminf_{n\rightarrow+\infty}(-1)^n=-1.$$
\end{exampleenv}
\begin{exampleenv}
    $\left(\frac{1}{n}\right)_{n\in \NN_{\ge 1}}$,
    $$\limsup_{n\rightarrow+\infty}\frac{1}{n}=\inf_{j\in \NN_{\ge 1}}\sup_{n\ge j}\frac{1}{n}=\inf_{j\in \NN_{\ge 1}}\frac{1}{j}=0,$$
    $$\liminf_{n\rightarrow+\infty}\frac{1}{n}=\sup_{j\in \NN_{\ge 1}}\inf_{n\ge j}\frac{1}{n}=\sup_{j\in \NN_{\ge 1}}0=0.$$
\end{exampleenv}
\begin{propositionenv}
    Let $f,g:X\longrightarrow G$ be mappings and $\mathcal{F}$ be a filter on $X$. Suppose that there exists $A\in \mathcal{F}$ such that 
    $$\forall x\in A, f(x)\le g(x).$$
    Then,
    $$\limsup_{\mathcal{F}}f\le \limsup_{\mathcal{F}}g,\ \liminf_{\mathcal{F}}f\le \liminf_{\mathcal{F}}g.$$ 
\end{propositionenv}
\begin{proofenv}
    Let 
    $$\mathcal{B}=\{U\in \mathcal{F}\mid  U\subseteq A\}.$$
    $\mathcal{B}$ is a filter basis, and $\mathcal{B}\in \mathcal{F}$. For any $V\in \mathcal{ F}$, one has $V\cap A\in \mathcal{B}$ and $V\supseteq V\cap A$. So $\mathcal{F}$ is generated by $\mathcal{ B}$. For any $B\in \mathcal{B}$, one has $B\subseteq A$ and hence
    $$f^s(B)\le g^s(B),\ f^i(B)\le g^i(B).$$
    So 
    $$\inf_{B\in \mathcal{B}}f^s(B)\le \inf_{B\in \mathcal{B}}g^s(B),\ \sup_{B\in \mathcal{B}}f^i(B)\le \sup_{B\in \mathcal{B}}g^i(B).$$
\end{proofenv}
\begin{theoremenv}[Squeeze Theorem]
    Let $X$ be a set and $\mathcal{F}$ be a non-degenerate filter on $X$. Let $f,g,h$ be elements of $G^X$. Assume that there exists $A\in \mathcal{F}$ such that 
    $$\forall x\in A, f(x)\le g(x)\le h(x).$$
    If $f$ and $h$ have limits along $\mathcal{F}$, and 
    $$\lim_{\mathcal{F}}f=\lim_{\mathcal{F}}h,$$
    then, $g$ also has a limit along $\mathcal{F}$, and 
    $$\lim_{\mathcal{F}}f=\lim_{\mathcal{F}}g=\lim_{\mathcal{F}}h.$$
\end{theoremenv}
\begin{proofenv}
    $$\lim_{\mathcal{F}}f=\limsup_{\mathcal{F}}f\le \limsup_{\mathcal{F}}g\le \limsup_{\mathcal{F}}h=\lim_{\mathcal{F}}h.$$
    So 
    $$\limsup_{\mathcal{F}} g=\lim_{\mathcal{F}}f=\lim_{\mathcal{F}}h.$$
    $$\lim_{\mathcal{F}}f=\liminf_{\mathcal{F}}f\le \liminf_{\mathcal{F}}g\le \liminf_{\mathcal{F}}h=\lim_{\mathcal{F}}h.$$
    So 
    $$\liminf_{\mathcal{F}} g=\lim_{\mathcal{F}}f=\lim_{\mathcal{F}}h.$$
\end{proofenv}
\begin{exampleenv}
    Let $a>1$. Consider the sequence $\left(\frac{a^n}{n!}\right)_{n\in \NN}$. If $n\ge N\ge 2a $, $a\le \frac{N}{2}$, then 
    $$0\le \frac{a^n}{n!}\le \frac{a^N}{N!}\cdot\frac{a^{n-N}}{(N+1)\dots n}\le \frac{a^N}{N!}\frac{1}{2^{n-N}}.$$
    For any $n\ge N$, $0\le \frac{a^n}{n!}\le \frac{(2a)^N}{N!}\cdot \frac{1}{2^n}.$ So by squeeze theorem, $\displaystyle\lim_{n\rightarrow+\infty}\frac{a^n}{n!}=0$.
\end{exampleenv}
\begin{theoremenv}[Monotone Convergence Theorem]
    Let $I$ be an infinite subset of $\NN$ and $(a_n)_{n\in I}\in G^I$.
    \newline
    (1) If $(a_n)_{n\in I}$ is increasing, then $(a_n)_{n\in I}$ admits $\displaystyle \sup_{n\in I}a_n$ as its limit.
    \newline
    (2) If $(a_n)_{n\in I}$ is decreasing, then $(a_n)_{n\in I}$ admits $\displaystyle \inf_{n\in I}a_n$ as its limit.
    
\end{theoremenv}
\begin{proofenv}
    \quad\newline
    (1) Let $l=\sup_{n\in I} a_n$, $\forall n\in \NN,\ a_n\le l$. So 
    $$\limsup_{n\rightarrow +\infty}a_n\le\limsup_{n\rightarrow +\infty}l=l.$$
    $$\forall j\in I, \inf_{n\in I_{\ge j}}a_n=a_j,$$
    so $$\liminf_{n\rightarrow+\infty}a_n=\sup_{j\in I}\inf_{n\in I_{\ge j}}a_n=\sup_{j\in I}a_j=l.$$
    Hence,
    $$l=\liminf_{n\rightarrow+\infty}a_n\le \limsup_{n\rightarrow+\infty}a_n\le l .$$
    Which means 
    $$\lim_{n\rightarrow +\infty}a_n=l.$$
\end{proofenv}
\begin{propositionenv}
    Let $X$ be a set and $Y\subseteq X$.\newline
    (1) If $\mathcal{F}$ is a filter on $X$, then 
    $$\left.\mathcal{F}\right|_Y:=\{U\cap Y\mid U\in \mathcal{F}\}$$
    is a filter on $Y$.
    \newline
    (2) If $\mathcal{B}$ is a filter basis on $X$, and $\mathcal{F}$ is the filter generated by $\mathcal{B}$, then 
    $$\left.\mathcal{B}\right|_Y:=\{B\cap Y\mid B\in \mathcal{B}\}$$
    is a filter basis generates $\left.\mathcal{F}\right|_Y$.
\end{propositionenv}
\begin{proofenv}
    \quad 
    \newline
    (1) Let $U$ and $V$ be elements of $\mathcal{ F}$, one has 
    $$(U\cap Y)\cap (V\cap Y)=(U\cap V)\cap Y\in \left.\mathcal{F}\right|_Y.$$
    Let $U\in \mathcal{F}, W\subseteq Y,U\cap Y\subseteq W$. Let $V=U\cup W\in \mathcal{F}$.
    $$Y\cap V=(U\cap Y)\cup(W\cap Y)=W.$$
    Hence $W\in \left.\mathcal{F}\right|_Y.$
    \newline
    (2) Let $B_1,B_2$ be elements of $\mathcal{B}$, then $\exists A\in B, A\subseteq B_1\cap B_2.$ Thus 
    $$A\cap Y\subseteq(B_1\cap Y)\cap (B_2\cap Y).$$
    So $\left.\mathcal{ B}\right|_Y$ is a filter basis. Moreover, $\left.\mathcal{B}\right|_Y\subseteq\left.\mathcal{F}\right|_Y$. Let $U\in \mathcal{F}, \exists B\in \mathcal{B}$ such that $B\subseteq U$. Thus 
    $$B\cap Y\subseteq U\cap Y.$$
    So $U\cap Y$  contains an element of $\left.\mathcal{B}\right|_Y$.
\end{proofenv}
\begin{exampleenv}
    Let $I\subseteq \NN$ be an infinite subset, and $(a_n)_{n\in I}\in G^I$. If $J\subseteq I$ is an infinite subset, $\mathcal{F}$ be the filter on $I$, then $\mathcal{F}|_J$ is the Fréchet filter on $J$. $(a_n)_{n\in J}$ is called a subsequence of $(a_n)_{n\in I}$.
\end{exampleenv}
\begin{propositionenv}
    Let $f:X\longrightarrow G$ be a mapping, $\mathcal{F}$ be a filter on $X$, $Y\subseteq X$. Then 
    \newline
    (1) 
    $$\limsup_{\mathcal{F}|_Y}f|_Y\le \limsup_{\mathcal{F}}f,$$
    $$\liminf_{\mathcal{F}|_Y}f|_Y\ge \liminf_{\mathcal{F}}f.$$
    (2) Suppose that $\mathcal{F}|_Y$ is non-degenerate and $f$ has a limit along $\mathcal{F}$, then $f|_Y$ has a limit along $\mathcal{F}|_Y$ and 
    $$\lim_{\mathcal{F}}f=\lim_{\mathcal{F}|_Y}f|_Y.$$
    (3) If $Y\in \mathcal{F}$, then 
    $$\limsup_{\mathcal{F}|_Y}=\limsup_{\mathcal{F}}f,$$
    $$\liminf_{\mathcal{F}|_Y}=\liminf_{\mathcal{F}}f.$$
\end{propositionenv}
\begin{proofenv}
    \quad 
    \newline
    $\forall U\in \mathcal{F}, f^s(U\cap Y)\le f^s(U)$. So 
    $$\limsup_{\mathcal{F}|_Y}f|_Y=\inf_{U\in \mathcal{F}}f^s(U\cap Y)\le\inf_{U\in \mathcal{F}}f^s(U)= \limsup_{\mathcal{F}}f.$$
    (2) $$\lim_{\mathcal{F}}f=\limsup_{\mathcal{F}}f\ge\limsup_{\mathcal{F}|_Y}f|_Y\ge \liminf_{\mathcal{F}|_Y}f|_Y\ge \liminf_{\mathcal{F}}f=\lim_{\mathcal{F}}f.$$
    (3) $\mathcal{F}|_Y$ is a filter basis that generates $\mathcal{F}$ if $Y\in \mathcal{F}$,
    $$\limsup_{\mathcal{F}|_Y}f|_Y=\inf_{V\in \mathcal{F}|_Y}f^s(U)=\inf_{U\in \mathcal{F}}f^s(U)=\limsup_{\mathcal{F}}f.$$
\end{proofenv}
\begin{theoremenv}[Bolzano-Weierstrass]
   Suppose that $G$ is totally ordered. Let $I\subseteq \NN$  be an infinite subset and $(a_n)_{n\in I}$ be a sequence in $G$.
   \newline
   (1) There exists an infinite subset $J_1$ such that $(a_n)_{n\in I}$ is monotone and admits $\dis\limsup_{n\in I,n\rightarrow +\infty}a_n$ as its limit.
   \newline
   (2) There exists an infinite subset $J_2$ such that $(a_n)_{n\in I}$ is monotone and admits $\dis\liminf_{n\in I,n\rightarrow +\infty}a_n$ as its limit.
    
\end{theoremenv}
\begin{proofenv}
    \quad\newline
    (1) Let 
    $$J=\{n\in I\mid \forall m\in I_{\ge n},a_m\le a_n\}.$$
    If $J$ is infinite, $(a_n)_{n\in J}$ is decreasing. Hence it admits
    $$\alpha:=\inf_{n\in J}a_n$$
    as its limit. For any $n\in J,\ \sup_{m\in I_{\ge n}}a_m=a_n$. So 
    $$\limsup_{n\in I,n\rightarrow+\infty}a_n=\inf_{n\in J}\sup_{m\in I_{\ge n}}a_m=\alpha.$$
    Suppose that $J$ is finite. Pick $n_0\in I $ such that $\forall j\in J,j<n_0$. We construct  in a recursive way a strictly increasing sequence $(n_k)_{k\in \NN}$ in $I$ as follows: Suppose $n_0<n_1<\dots<n_k$ have been chosen. Since $G$ is totally ordered, there exists $i\in I$ such that $n_0\le i\le n_k$ and 
    $$a_i=\max\{a_j\mid j\in I,n_0\le j\le n_k\}.$$
    Since $i\notin J$, there exists $n_{k+1}\in I, n_{k+1}>i$ such that 
    $$a_{n_{k+1}}>a_i.$$
    Note that $n_{k+1}>n_k$. Let 
    $$J_1=\{n_{k}\mid k\in \NN\},$$
    $(a_n)_{n\in J_1}$ is increasing, hence it admits 
    $$\beta:=\sup_{n\in J}a_n$$
    as its limit. For any $j\in I$ such that $j\ge n_0$, there exists $k\in \NN$ such that $j\le n_k$. Thus $a_j\le a_{n_{k+1}}\le \beta$. So $\dis\limsup_{n\in I,n\rightarrow+\infty}a_n\le \beta$. Moreover, since $J_1\subseteq I$, 
    $$\beta=\lim_{n\in J_1,n\rightarrow+\infty}a_n=\limsup_{n\in J_1,n\rightarrow +\infty}a_n\le\limsup_{n\in I,n\rightarrow +\infty}a_n.$$
    Therefore,
    $$\beta=\limsup_{n\in I,n\rightarrow+\infty}a_n.$$
\end{proofenv}




\section{Partially Ordered Groups}
\begin{definitionenv}
    Let $(G,*)$ be a group, and $\le $ be a partial order on $G$. If 
    $$\forall (a,b,c)\in G^3,a<b\Rightarrow a*c<b*c \text{ and } c*a<c*b,$$
    we say that $(G,*,\le)$ is a \textbf{partially ordered group}. If in addition $\le $ is a total order, we say that $(G,*,\le)$ is a \textbf{totally ordered group}. (Resp. semigroup, monoid.)
\end{definitionenv}
\begin{exampleenv}
    \quad
    (1) $(\RR,+,\le)$.
    \quad
    (2) $(\RR_{>0},\cdot,\le)$.
    \quad
    (3) $(\NN\backslash\{0\},\cdot,\mid)$.
\end{exampleenv}
\begin{remark}
    \quad
    \newline
    (1) If $(G,*)$ is a partially ordered group, then 
    $$\forall (a,b,c)\in G^3,a\le b\Rightarrow a*c\le b*c,\ c*a\le c*b.$$
    (2) $(G,\hat{*},\le )$ is a partially ordered group.
    \newline
    Resp. semigroup, monoid.
\end{remark}
\begin{propositionenv}
    \quad\newline
    Let $(G,*,\le)$ be a partially ordered semigroup.  Let $(a_1,a_2,b_1,b_2)\in G^4$. 
    \newline
    (1) If $a_1\le a_2,\ b_1\le b_2,$ then $a_1*b_1\le a_2*b_2.$
    \newline
    (2) If $a_1< a_2,\ b_1\le b_2,$ then $a_1*b_1< a_2*b_2.$
    \newline
    (3) If $a_1\le a_2,\ b_1<b_2,$ then $a_1*b_1< a_2*b_2.$
\end{propositionenv}
\begin{proofenv}
    \quad \newline
    (1) $a_1*b_1\le a_2*b_1\le a_2*b_2$
    \newline
    (2),(3) At least one of the above inequality is strict.
\end{proofenv}
\begin{propositionenv}
    Let $(G,*,\le)$ be a partially ordered semigroup, $(x,y,a)\in G^3$. Assume that, either $\le $ is a total order, or $(G,*)$ is a monoid and $a\in G^\times$. Then the following conditions are quivalent:
    \newline
    (1) $x\le y$.
    \newline
    (2) $x*a\le y*a$.
    \newline
    (3) $a*x\le a*y$.
\end{propositionenv}
\begin{proofenv}
    By definition, $(1)\Rightarrow (2)$, $(1)\Rightarrow (3)$. Assume that $x*a\le y*a$. If $(G,*)$ is a monoid and $a\in G^\times$, then 
    $$x=(x*a)*\iota(a)\le (y*a)*\iota(a)=y.$$
    Suppose that $\le $ is a total order. If $x\not\le y$, then $x>y$ and $x*a>y*a$, contradiction.
\end{proofenv}
\begin{corollaryenv}
    A totally ordered semigroup satisfies the left and right cancellation laws.
\end{corollaryenv}
\begin{proofenv}
    Let $(G,*,\le)$ be a totally ordered semigroup. Let $(x,y,a)\in G^3$ such that $x*a=y*a$. Then 
    $$x*a\le y*a, y*a\le x*a.$$
    Hence $x\le y$ and $y\le x$.
\end{proofenv}
\begin{propositionenv}
    Let $(G,*,\le)$ be a partially ordered monoid. Then,
    $$\iota:G^\times\longrightarrow G^\times $$
    is strictly decreasing.
\end{propositionenv}
\begin{proofenv}
    Let $(x,y)\in G^\times\times G^\times$ such that $x< y$. Then 
    $$e=\iota(x)*x<\iota(x)*y,$$
    where $e$ is the neutual element of $(G,*)$. Thus 
    $$e*\iota(y)<\iota(x)*y*\iota(y).$$
    That is $\iota(y)<\iota(x)$.
\end{proofenv}
\begin{propositionenv}
    Let $(G,*,\le)$ be a totally ordered group, and $e$ be the neutual element of $(G,*)$. If $G\not=\{e\}$, then $G$ has neither a greatest element nor a least element.
\end{propositionenv}
\begin{proofenv}
    Suppose that $(G,\le)$ has a greastest element $\beta$. We first show by contradiction that $\beta\not=e$. Suppose that $e=\max G$. Pick $x\in G,x\not= e$. Then $x<e$. Thus $\iota (x)>\iota(e)$. Contradiction. If $\beta>e, \beta*\beta>e*\beta=\beta$, contradiction, too.
\end{proofenv}



\section{Enhancement}
\begin{definitionenv}
    Let $(S,*,\le)$ be a partially ordered semigroup. Suppose that $(S,\le )$ has no greastest element and has no least element. Let $\bot$ and $\top$ be formal elements and let 
    $$\bar{S}=S\cup\{\bot,\top\}.$$
    We extend $\le $ to $\bar{S}$ by letting $\bot<x<\top,\forall x\in S$. We extend $*$ to a mapping
    $$\left(\bar{S}\times\bar{S}\right)\backslash \{(\bot,\top),(\top,\bot)\}\longrightarrow \bar{S},$$
    such that 
    $$\forall x\in S\cup\{\top\},x*\top=\top*x=\top.$$
    $$\forall x\in S\cup\{\bot\},x*\bot=\bot*x=\bot.$$
    $\top*\bot$ and $\bot*\top$ are NOT DEFINED. $(\bar{S},*,\le)$ is called the \textbf{enhancement} of $(S,*,\le)$. If $A$ and $B$ are subset of $\bar{S}$, we denote by $A*B$ the set 
    $$\{x*y\mid (x,y)\in A\times B\ ,\{x,y\}\not=\{\bot,\top\}\}.$$
\end{definitionenv}
\begin{exampleenv}
    \quad 
    \newline
    (1) $(\RR,+,\le ), \bar{\RR}=\RR\cup\{+\infty,-\infty\}$.
    \newline
    (2) $(\RR_{>0},\cdot,\le),\bar{\RR}_{>0}=\RR_{>0}\cup\{0,+\infty\}$.
\end{exampleenv}
\begin{remark}
    \quad 
    \newline
    (1) $\forall (a,b)\in \bar{S}\times \bar{S}, a*b$ is defined if and only if $b*a$ is defined. 
    \newline
    (2) If $*$ is commutative, and $a*b$ is defined, then $a*b=b*a$.
\end{remark}
\begin{definitionenv}
    Let $a_0,\dots,a_n$ be elements of $\bar{S}$, if $a_0*\dots*a_n$ is defined, and $(a_0,\dots,a_{n-1})*a_n$ is also defined, then we let $a_0*\dots*a_n=\left(a_0*\dots*a_{n-1}\right)*a_n$.
\end{definitionenv}
\begin{propositionenv}
    Let $a_0,\dots,a_n$ be elements of $\bar{S}$. For any $i\in \{0,\dots,n\}$, $a_0*\dots*a_{i-1}*(a_i*a_{i+1})*\dots*a_n$ is defined if and only if $a_0*\dots*a_n$ is defined. Moreover, $a_0*\dots*a_n=a_0*\dots (a_{i-1}*a_i)*\dots*a_n $.
\end{propositionenv}
\begin{proofenv}
    Both terms are defined is and only if 
    $$\{\top,\bot\}\nsubseteq\{a_0,\dots,a_n\}.$$
    If $\bot\in\{a_0,\dots,a_n\}$, then both terms are equal to $\bot$. If $\top\in\{a_0,\dots,a_n\}$, then both terms are equal to $\top$. 
\end{proofenv}
\begin{propositionenv}
    Let $(S,*,\le)$ be a partially ordered semigroup. Let $(a,b)\in \bar{S}\times \bar{S}$. If $a<b$, then for any $c\in S$, $a*c<b*c,c*a<c*b$.
\end{propositionenv}
\begin{proofenv}
    If $\{a,b\}\subseteq S$, this follows from the definition of a partially ordered semigroup. If $a=\bot$, then $b>\bot$. $a*c=c*a=\bot$. $b*c\not=\bot, c*b\not=\bot$. So $a*c<b*c,c*a<c*b$. If $\{a,b\}\subseteq S$ and $a\not=\bot$, then $b=\top$, $b*c=c*b=\top$. $a*c\not=\top,c*a\not=\top$. So $a*c<b*c$, $c*a<c*b$.
\end{proofenv}
\begin{propositionenv}
    \quad \newline
    Let $(S,*,\le)$  be a partially ordered semigroup and $(x,y,a,b)\in \bar{S}^4$. 
    \newline
    (1) If $x<a$ and $y<b$, then $x*y$ and $a*b$ are defined, and $x*y<a*b$.
    \newline
    (2) If $x\le a, y\le b$ and $x*y$  and $a*b$ are defined, $x*y\le a*b$.
\end{propositionenv}
\begin{proofenv}
    \quad\newline
    (1) Since $x<a,y<b, \top\notin\{x,y\},\bot\notin\{a,b\}$. So $x*y$ and $a*b$ are defined.
    If $\top\in \{a,b\}$, then $a*b=\top$. Since $\top\notin\{x,y\},x*y\not=\top$, so $x*y<a*b$.
    If $\bot\in\{x,y\}$, then $x*y=\bot$. Since $\bot\notin\{a,b\}$, $a*b\not=\bot$ . So $x*y<a*b$.
    If $\top\notin\{a,b\},\bot\notin\{x,y\},$, then $\{x,y,a,b\}\subseteq S$. So $x*y<x*b<a*b$.
    \newline
    (2) If $\top\in\{x,y\}$, then $\top\in \{a,b\}$, so $x*y=\top=a*b$.
    If $\bot\in\{x,y\}$, then $\bot\in\{a,b\}$, so $x*y=\bot=a*b$.
    If $\top\in \{x,y\}$, then $x*y=\bot\le a*b$.
    If $\bot\in\{x,y\}$, then $x*y\ge \top= a*b$.
    If $\{x,y,a,b\}\subseteq S$, then $x*y\le a*y \le a*b$.
\end{proofenv}
\begin{propositionenv}
    Let $(S,*,\le)$ be a partially ordered monoid and $e\in S$ be the neutual element. Let $(a,b)\in \bar{S}\times\bar{S}$, with $a\in S^\times\cup\{\bot,\top\}$. Then the following conditions are equivalent.
    \newline
    (1) $a<b$.
    \newline
    (2) $\iota(a)*b$ is defined and $e<\iota(a)*b$. (Where $\iota(\bot)=\top, \iota(\top)=\bot$.)
    \newline
    (3) $b*\iota(a)$ is defined and $e<b*\iota(a)$.
\end{propositionenv}
\begin{proofenv}
    Suppose that $a<b$. Then $a$ and $b$ cannot be both $\top$ or be both $\bot$. Hence $\{\iota(a),b\}\not=\{\bot,\top\}$. Therefore, $\iota(a)*b$ and $$b*\iota(a)$$ are defined. If $\{a,b\}\subseteq S$, then $e=\iota(a)*a<\iota(a)*b$, $e=a*\iota(a)<b*\iota(a)$. If $a=\bot$, then $\iota(a)=\top$ and $\iota(a)*b=b*\iota(a)=\top$. So $e<\iota(a)*b,e<b*\iota(a)$. If $b=\top$, then $\iota(a)*b=b*\iota(a)=\top>e$.
    \newline
    Assume (2). $\iota(a)*b$ is defined and $e<\iota(a)*b$. If $a\in S$,
    $$a=a*e<a*(\iota(a)*b)=(a*\iota(a))*b=e*b=b.$$
    If $a=\bot, \iota(a)=\top, b\not=\bot$, so $a<b$. If $a=\top, \iota (a)=\bot, \iota(a)*b=\bot<e$, contradiction.
\end{proofenv}
\begin{corollaryenv}
    Let $(S,*,\le)$ be a partially ordered monoid and $(a,b)\in \left(S^\times\cup\{\bot,\top\}\right)^2$. Then $a<b$ if and only if $\iota(a)>\iota(b)$.
\end{corollaryenv}
\begin{proofenv}
    If $a<b$,then $\iota(a)*b$ is defined and 
    $$e<\iota(a)*b=\iota(a)*\iota(\iota(b)).$$
    So $\iota(b)<\iota(a)$.
\end{proofenv}
\begin{lemmaenv}
    Let $(S,*,\le)$ be a partially ordered monoid. Let $A\subseteq \bar{S}, b\in S^\times\cup\{\bot,\top\}$. Then the following statements hold:
    \newline
    (1) If $\sup(A)*b$ is defined, then $A*\{b\}$ has a supremum in $\bar{S}$, and $$\sup(A*\{b\})=\sup(A)*b.$$
    (2) If $\inf(A)*b$ is defined, then $A*\{b\}$ has a infimum in $\bar{S}$, and $$\inf(A*\{b\})=\inf(A)*b.$$
    (3) If $b*\sup(A)$ is defined, then $A*\{b\}$ has a supremum in $\bar{S}$, and $$\sup(\{b\}*A)=b*\sup(A).$$
    (4) If $b*\inf(A)$ is defined, then $A*\{b\}$ has a infimum in $\bar{S}$, and $$\inf(\{b\}*A)=b*\inf(A).$$
\end{lemmaenv}
\begin{proofenv}
    \quad \newline
    (1) Suppose that $b=\bot$, then $\sup(A)\not=\top$, $\sup(A)*b=\bot$. $A*\{b\}\subseteq\{\bot\}$, so $\sup(A*\{b\})=\bot$.
    Suppose that $b=\top$, then $\sup(A)\not=\bot$, $A\not=\varnothing$, $\sup(A)*b=\top$. So $\sup(A*\{b\})=\top=\sup(A)*\top$.
    We suppose that $b\in S^\times$. $\forall a\in A,\ A\le \sup(A)$, so $a*b\le \sup(A)*b$. This means that $\sup(A)*b$ is an upper bound of $A*\{b\}$. Let $M$ be an upper bound of $A*\{b\}$. For any $a\in A$, $a\times b\in A*\{b\}$, so $a*b\le M$. Hence, 
    $$a=(a*b)*\iota(b)\le M*\iota(b).$$
    We then deduce $\sup(A)\le M*\iota(b)$. Hence $\sup(A)*b\le M*\iota(b)*b$. Therefore, $\sup(A)*b$ is the supremum of $A*\{b\}$.
\end{proofenv}
\begin{remark}
    Consider
    $$S=\{0\}\cup\interval[open right]{2}{3}\cup\interval[open right ]{4}{+\infty}\subseteq \RR.$$ 
    $A=\interval[open right]{2}{3}, \sup(A)=4, A+\{2\}=\interval[open right]{4}{5}, \sup(A+\{2\})=5, \sup(A)+2=6.$
\end{remark}
\begin{theoremenv}
    Let $(S,*,\le)$ be a partially ordered group. Let $A$ and $B$ be subsets of $\bar{S}$.
    \newline
    (1) If $\sup(A)*\sup(B)$ is defined, then $A*B$ has a supremum in $\bar{S}$ and
    $$\sup(A*B)=\sup(A)*\sup(B).$$
    (2) If $\inf(A)*\inf(B)$ is defined, then $A*B$ has a infimum in $\bar{S}$ and
    $$\inf(A*B)=\inf(A)*\inf(B).$$
\end{theoremenv}
\begin{proofenv}
    For any $(a,b)\in A\times B$, if $a*b$ id defined, then 
    $$a*b\le \sup(A)*\sup(B).$$
    So $\sup(A)*\sup(B)$ is an upper bound of $A*B$. If $\bot\in \{\sup(A),\sup(B)\}$, then $A*B$ has $\bot$ as an upper bound. So $\sup(A*B)=\bot=\sup(A)*\sup(B)$. We suppose that $\bot\notin \{\sup(A)*\sup(B)\}$. Thus $A\backslash\{\bot\}\not=\varnothing$, $B\backslash\{\bot\}\not=\varnothing$. Suppose that $\sup(A)=\top$. Take $b\in B\backslash\{\bot\}$.
    \[\sup(A*B)\ge\sup(A)*\{b\}=\sup(A)*b=\top.\]
    So $\sup(A*B)=\top=\sup(A)*\sup(B)$. Similarly, if $\sup(B)=\top$, then 
    $$\sup(A*B)=\top=\sup(A)*\sup(B).$$
    Suppose that $\top\notin\{\sup (A),\sup (B)\}$. For any $b\in B$, $\sup(A)*b$ is defined since $\sup(A)\in S$. Hence
    $$\sup(A)*\{b\}=\sup(A)*b.$$
    $$A*B=\bigcup_{b\in B}A*\{b\},$$
    $$\{\sup(A)*b\mid b\in B\}=\{\sup(A)\}*B.$$
    By the lemma, $\{\sup(A)\}*B$ has a supremum, which is $\sup(A)*\sup(B)$. So $\sup(A*B)$ exists, and is equal to 
    $$\{\sup(A*\{b\})\mid b\in B\}=\sup(A)*\sup(B).$$
\end{proofenv}
\begin{corollaryenv}
    Let $(S,*,\le)$ be a partially ordered group. Let $f,g: X\longrightarrow \bar{S}$ be two mappings. Let 
    $$Y=\{x\in X\mid f(x)*g(x) \text{ is defined }\}.$$
    Let $$f*g: Y\longrightarrow \bar{S},$$
    $$y\longmapsto f(y)*g(y).$$
    (1) If $(\sup f)*(\sup g)$ is defined, and $f*g$ has a supremum, then $$\sup(f*g)\le \sup(f)*\sup(g).$$
    (2) If $(\inf f)*(\inf g)$ is defined, and $f*g$ has a infimum, then $$\inf(f*g)\ge \inf(f)*\inf(g).$$
\end{corollaryenv}
\begin{proofenv}
    Let $A=f(X), B=g(X)$. By the theorem, $A*B$ has a supremum, and 
    $$\sup(A*B)=\sup(A)*\sup(B).$$
    Let $$C=(f*g)(Y)=\{f(y)*g(y)\mid y\in Y\}.$$
    One has 
    $$C\subseteq A*B=\{f(x)*g(y)\mid (x,y)\in X\times X,\  f(x)*g(y)\text{ is defined }\}.$$
    So $\sup(C)\le \sup(A*B).$
\end{proofenv}
\begin{theoremenv}
    Let $(S,*,\le)$ be a partially ordered group. We suppose that $\bar{S}$ is order complete. Let $X$ be a set and 
    $f,g:X\longrightarrow \bar{S}$ be mappings. Let $\mathcal{F}$ be a filter on $X$ that is non-degenerate. Suppose that $\forall x\in X, f(x)*g(x)$ is defined. Then 
    $$\limsup_{\mathcal{F}}(f*g)\le \limsup_{\mathcal{F}}f*\limsup_{\mathcal{F}}g,$$
    $$\limsup_{\mathcal{F}}(f*g)\ge \limsup_{\mathcal{F}}f*\liminf_{\mathcal{F}}g,$$
    $$\liminf_{\mathcal{F}}(f*g)\ge \liminf_{\mathcal{F}}f*\liminf_{\mathcal{F}}g,$$
    $$\liminf_{\mathcal{F}}(f*g)\le \liminf_{\mathcal{F}}f*\limsup_{\mathcal{F}}g.$$
    Provided that the term on the right hand side id defined.

\end{theoremenv}
\begin{proofenv}
    \quad\newline
    (1) $\forall U\in\mathcal{F},$
    $$(f*g)^s(U)\le f^s(U)*g^s(U).$$
    Provided that $f^s(U)* g^s(U)$ is defined.
    If $\dis \limsup_{\mathcal{F}}f*\limsup_{\mathcal{F}}g$ is defined, then 
    $$\limsup _{\mathcal{F}}(f*g)=\inf_{U\in \mathcal{F}}(f*g)^s(U)\le \inf_{U\in \mathcal{F}}\left[f^s(U)*g^s(U)\right].$$
\begin{align*}
\limsup_{\mathcal{F}}f*\limsup_{\mathcal{F}}g=&\left(\inf_{U\in \mathcal{F}}f^s(U)\right)*\left(\inf_{V\in\mathcal{F}}g^s(V)\right)\\
=&\inf_{(U,V)\in \mathcal{F}\times\mathcal{F},\text{ defined}}\left(f^s(U)*g^s(V)\right)
\end{align*}
If $(U,V)\in \mathcal{F}\times\mathcal{F}$ is such that $f^s(U)*g^s(V)$ is defined, then 
$$f^s(U)* g^s(V)\ge f^s(U\cap V)*g^s(U\cap V)\ge l$$
provided that $f^s(U\cap V)*g^s(U\cap V)$ is defined.
If $f^s(U\cap V)*g^s(U\cap V)$ is not defined, then $\top\in\{f^s(U),g^s(V)\}$, so that 
$$f^s(U)*g^s(V)=\top \ge l.$$
Therefore, $$\limsup_{\mathcal{F}}f*\limsup_{\mathcal{F}}g\ge l\ge \limsup_{\mathcal{F}}f*g.$$
(2) $$\limsup_{\mathcal{F}}f*g\ge \limsup_\mathcal{F}f *\liminf_\mathcal{F}g.$$
Let $U\in\mathcal{F}$. Suppose that $\left(f*g\right)^s(U)\not=\top$. $\forall V\in \mathcal{F}$, one has $\forall x\in U\cap V$,
$$\left(f*g\right)^s(U\cap V)\ge f(x)*g(x)\ge f(x)*g^i(U\cap V)\ge f(x)*g^i(V).$$
So $$\left(f*g\right)^s(U)\ge f^s(U\cap V)*g^i(V),$$
provided that $f^s(U\cap V)*g^i(V)$ is defined. Taking the infimum which respect to $U$, we obtain
$$\limsup_{\mathcal{F}}f*g\ge \limsup_{\mathcal{F}}f*g^i(V),$$
provided that $\dis \limsup_{\mathcal{F}}f*g^i(V)$ is defined. Taking the supremum with respect to $V$, we obtain
$$\limsup_{\mathcal{F}}f*g\ge \limsup_{\mathcal{F}}f*\liminf_{\mathcal{F}}g.$$
\end{proofenv}
\begin{corollaryenv}
    Let $(S,*,\le)$ be a partially ordered group, such that $\bar{S}$ is order complete. Let $f,g: X\longrightarrow \bar{S}$ be mappings such that $\forall x\in X, f(x)*g(x)$ is defined. Let $\mathcal{F}$ be a  non-degenerate filter on $X$. Assume that $g$ has a limit along $\mathcal{F}$.
    \newline
    (1) $$\limsup_{\mathcal{F}}f*g=\limsup_{\mathcal{F}}f*\lim_{\mathcal{F}}g.$$
    (2) $$\liminf_{\mathcal{F}}f*g=\liminf_{\mathcal{F}}f*\lim_{\mathcal{F}}g.$$
    Provided that the term on the right hand side is defined.
\end{corollaryenv}
\begin{proofenv}
 $$\limsup_{\mathcal{F}}f*g\le \limsup_{\mathcal{F}}f*\limsup_{\mathcal{F}}g=\limsup_{\mathcal{F}}f*\lim_{\mathcal{F}}g.$$
    $$\limsup_{\mathcal{F}}f*g\ge \limsup_{\mathcal{F}}f*\liminf_{\mathcal{F}}g=\limsup_{\mathcal{F}}f*\lim_{\mathcal{F}}g.$$
\end{proofenv}
\begin{exampleenv}
    \quad \newline
    (1)
    $$\limsup_{n\rightarrow+\infty}\left((-1)^n+\frac{1}{n}\right)=\limsup_{n\rightarrow+\infty}(-1)^n=1.$$
    (2) For $a>1$, let $a=1+b$. $a^n=(1+b)^n\ge nb$. so 
    $$0\le \frac{\sqrt{n}}{a^n}\le \frac{1}{b\sqrt{n}},$$
    $$\lim_{n\rightarrow +\infty}\frac{\sqrt{n}}{a^n}.$$
    $$\forall k\in \NN_{\ge 1}, \ \lim_{n\rightarrow+\infty}\left(\frac{\sqrt{n}}{a^n}\right)^k.$$
\end{exampleenv}
