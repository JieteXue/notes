\chapter{Topology}
\section{Topological spaces}
\begin{propositionenv}
    Let $X$ be a set and for any $x\in X$, let $\mathcal{G}_x$ be a filter contained in the principal filter of $\{x\}$ ($\forall U\in \mathcal{G}_x,\ x\in U$). Denote by $\mathscr{T}$ the set 
    $$\{U\in \wp(X) \mid \forall x\in U, U\in\mathcal{G}_x\}.$$
    Then the following conditions are satisfied.
    \newline
    (1) $\{\varnothing, X\} \subseteq \mathscr{T}$.
    \newline
    (2) If $(U_1,U_2)\in \mathscr{T}^2$, then $U_1\cap U_2\in \mathscr{T}$.
    \newline
    (3) If $I$ is a set and $\left(U_i\right)_{i\in I}\in \mathscr{T}^I$, then $\dis \bigcup_{i\in I}U_i\in \mathscr{T}$.
    \newline
    Moreover, $\forall x\in X,\ \mathcal{B}_x=\{U\in \mathscr{T}\mid x\in U\}$ is a filter basis contained in $\mathcal{G}_x$. It generates $\mathcal{G}_x$ if the following condition is satisfied:
    $$\forall U\in \mathcal{G}_x,\ \exists V\in \mathcal{G}_x,\ V\subseteq U \text{ and } \forall y\in V,\ V\in \mathcal{G}_y.$$
\end{propositionenv}
\begin{proofenv}
    \ \newline
    (1) $\dis \varnothing \in \mathscr{T}, X\in \bigcap_{x\in X}\mathcal{G}_x$.
    \newline
    (2) $\forall x\in U_1\cap U_2,\ U_1\in \mathcal{G}_x,\ U_2\in \mathcal{G}_x$, so $U_1\cap U_2\in \mathcal{G}_x$.
    \newline
    (3) Let $\dis U=\bigcup_{i\in I}U_i$. $\forall x\in U,\ \exists i\in I,\ x\in U_i$, so $U_i\in \mathcal{G}_x$. Since $U\supseteq U_i$, so $U\in\mathcal{G}_x$.
    $$\mathcal{B}_x:=\{U\in \mathscr{T}\mid x\in U\}.$$
    If $U\in \mathcal{B}_x$, then $x\in U$, so $U\in \mathcal{G}_x$. Hence $\mathcal{B}_x\subseteq\mathcal{G}_x$. If $(U,V)\in \mathcal{B}_x^2$, then $U\cap V\in \mathscr{T}$, and $x\in U\cap V$. So $U\cap V\in \mathcal{B}_x$. So $\mathcal{B}_x$ is a filter basis. Suppose the condition is satisfied. For any $U\in \mathcal{G}_x,\ \exists V\in \mathcal{G}_x\cap\mathscr{T}$, such that $V\subseteq U$. Note that $V\in \mathcal{B}_x$, so $\mathcal{G}_x$ is generated by $\mathcal{B}_x$.
\end{proofenv}
\begin{definitionenv}
    Let $X$ be a set. We call \textbf{topology} on $X$ any subset $\mathscr{T}$ of $\wp(X)$ that satisfies the following conditions:
     \newline
    (1) $\{\varnothing, X\} \subseteq \mathscr{T}$.
    \newline
    (2) If $(U_1,U_2)\in \mathscr{T}^2$, then $U_1\cap U_2\in \mathscr{T}$.
    \newline
    (3) For any set $I$,  $ \forall \left(U_i\right)_{i\in I}\in \mathscr{T}^I$, then $\dis \bigcup_{i\in I}U_i\in \mathscr{T}$.
    \newline
    $(X,\mathscr{T})$ is called a \textbf{topological space}.
    \newline
    If $\forall x\in X,\ \mathcal{B}_x$ is a filter basis of $X$ contained in the principal filter of $\{x\}$, then 
    $$\mathscr{T}=\{U\in \wp(X)\mid \forall x\in U,\ \exists V_x\in \mathcal{B}_x,\ V_x\subseteq U\}$$
    is a topology on $X$, called the \textbf{topology generated by} $(\mathcal{B}_x)_{x\in X}$. More generally, if $\forall x\in X,\ S_x $ is a subset of the principal filter of $\{x\}$ and $\mathcal{G}_x$ is the filter generated by $S_x$, then we say that 
    $$\mathscr{T}=\{U\in\wp(X)\mid \forall x\in U,\ U\in \mathcal{G}_x\}$$
    is the topology generated by $(S_x)_{x\in X}$.

\end{definitionenv}
\begin{exampleenv}
    \ \newline
    (1) Let $\mathcal{G}_x=\{X\}$. The topology by $\left(\mathcal{G}_x\right)_{x\in X}$ is $\{\varnothing,X\}$, called the \textbf{trivial topology} on $X$.  
    \newline
    (2) Let $\mathcal{G}_x=\mathcal{F}_{\{x\}}$ be the principal filter. The topology generated by $\left(\mathcal{G}_x\right)_{x\in X}$ is $\wp(X)$. This topology is called the \textbf{discrete topology} on $X$.
    \newline
    (3) Let $(X,\mathrm{d})$ be a semimetric space. 
    $$\mathrm{d}:X\times X\longrightarrow \mathbb{R}_{\ge 0},\  \mathrm{d}(x,y)=\mathrm{d}(y,x),\ \mathrm{d}(x,z)\le \mathrm{d}(x,y)+\mathrm{d}(y,z),\ \mathrm{d}(x,x)=0.$$
    $\forall \varepsilon>0,\ \forall x\in X$, let $B(x,\varepsilon)=\{y\in X\mid \mathrm{d}(x,y)<\varepsilon\}$, $\{B(x,\varepsilon)\mid \varepsilon\in \RR_{>0}\}=: \mathcal{B}_x$ is a filter basis on $X$, $\mathcal{B}_x$ is contained in the principal filter of $\{x\}$. The topology 
    $$\mathscr{T}=\{U\in \wp(X)\mid \forall x\in U,\ \exists \varepsilon\in \RR_{>0},\ B(x,\varepsilon)\subseteq U\}$$
    is called the \textbf{topology induced by the semimetric $\mathrm{d}$}.
    \newline
    (4) Let $(G,\le )$ be a totally ordered set $\forall x\in G$, let $S_x=\{G_{>a}\mid a<x\}\cup \{G_{<b}\mid x<b\}$
\end{exampleenv}
\begin{propositionenv}
    $\forall x\in X,\ \forall \varepsilon\in \RR_{>0}, B(x,\varepsilon)\in \mathscr{T}$.
\end{propositionenv}
\begin{proofenv}
    $\forall y\in B(x,\varepsilon),\ \mathrm{d}(x,y)<\varepsilon$. Let $r=\varepsilon-\mathrm{d}(x,y)>0$, we claim that $B(y,r)\subseteq B(x,\varepsilon)$. Let $z\in B(y,r), \ \mathrm{d}(y,z)<r$. Hence,
    $$\mathrm{d}(x,z)\le \mathrm{d}(x,y)+\mathrm{d}(y,z)<\mathrm{d}(x,y)+r=\mathrm{d}(x,y)+\varepsilon -\mathrm{d}(x,y)=\varepsilon.$$
\end{proofenv}
\begin{remark}
    On $\RR$, one has a mertic 
    $$\mathrm{d}: \RR\times \RR\longrightarrow \RR_{\ge 0},$$
    $$\left(a,b\right)\longmapsto |a-b|.$$
    $$B(x,\varepsilon)=\interval[open]{x-\varepsilon}{x+\varepsilon}.$$
    $$\mathcal{B}_x=\{\interval[open]{x-\varepsilon}{x+\varepsilon}\mid \varepsilon\in \RR_{>0}\}.$$
    Let $\mathscr{T}_\mathrm{d}$ be the topology generated by $(\mathcal{B}_x)_{x\in \RR}$. Let $\mathscr{T}$ be the order topology generated by $\left(S_x\right)_{x\in \RR}$, where 
    $$S_x:=\{\RR_{>a}\mid a<x\}\cup \{\RR_{<b}\mid x<b\}.$$
\end{remark}
\begin{propositionenv}
    For any $x\in \RR$, $\mathcal{F}(\mathcal{B}_x)=\mathcal{F}(S_x)$.
\end{propositionenv}
\begin{proofenv}
    $\forall \varepsilon>0,\ \interval[open]{x-\varepsilon}{x+\varepsilon}=\RR_{<x+\varepsilon}\cap \RR_{>x-\varepsilon}\in \mathcal{F}(S_x)$. So $\mathcal{F}(\mathcal{B}_x)\subseteq \mathcal{F}(S_x)$.
    $$\forall a\in \RR,\ a<x,\ \RR_{>a}\supseteq \interval[open]{a}{2x-a}=\interval[open]{x-(x-a)}{x+(x-a)},\ \RR_{>a}\in \mathcal{F}(\mathcal{B}_x).$$
    $$\forall b\in \RR, b>x,\RR_{<b}\supseteq \interval[open]{2x-b}{b}=\interval[open]{x+(b-x)}{x+(b-x)},$$
    So, $\RR_{<b}\subseteq\mathcal{F}(\mathcal{B}_x)$. Hence $S_x\subseteq \mathcal{F}(\mathcal{B}_x)$, which leads to $\mathcal{F}(S_x)\subseteq\mathcal{F}(\mathcal{B}_x)$.
\end{proofenv}
\begin{definitionenv}
    Let $(X,\mathscr{T})$ be a topological space. For any $x\in X$ and any $V\in \wp(X)$, if there exists $U\in \mathscr{T}$ such that $x\in U\subseteq V$, then we say that $V$ is a \textbf{neighborhood of $x$}. We call \textbf{open subset} of $X$ any subset of $X$ that belongs to $\mathscr{T}$. If $U\in \mathscr{T}$, such that $x\in U$, we say that $U$ is an \textbf{open neighborhood of $x$}. We denote by $\mathcal{V}_x(\mathscr{T})$ the set of all neighborhoods of $x$.
\end{definitionenv}
\begin{propositionenv}
    $\mathcal{V}_x(\mathscr{T})$ is a filter on $X$ contained in the principal filter of $\{x\}$. Moreover, the topology generated by $\left(\mathcal{V}_x(\mathscr{T})\right)_{x\in X}$ identifies with $\mathscr{T}$.
\end{propositionenv}
\begin{proofenv}
    \ \newline
    (1) If $(V_1,V_2)\in \mathcal{V}_x(\mathscr{T})^2,\ \exists (U_1,U_2)\in \mathscr{T}^2$, such that $x\in U_1\subseteq V_1,\ x\in U_2\subseteq V_2$. Hence, $x\in U_1\cap U_2\subseteq V_1\cap V_2$, so $V_1\cap V_2\in \mathcal{V}_{x}(\mathscr{T})$.
    \newline
    (2) If $V\in \mathcal{V}_x(\mathscr{T})$, $W\in \wp(X),\ V\subseteq W$. $\exists U\in \mathscr{T}$, $x\in U\subseteq V\subseteq W$, so $W\in \mathcal{V}_x(\mathscr{T})$. Let $\mathscr{T}'$ be the topology generated by $\left(\mathcal{V}_x(\mathscr{T})\right)_{x\in X}$. By definition,
    $$\mathscr{T}'=\{U\subseteq X\mid \forall x\in U, U\in \mathcal{V}_x(\mathscr{T})\}.$$
    For any $U\in \mathscr{T},\ \forall x\in U  $, $U$ is a open neighborhood of $x$, so $U\in \mathscr{T}'$. Let $U\in \mathscr{T}', \ \forall x\in U,\ \exists V_2\in \mathscr{T},\ x\in V_x\subseteq U$.
    $$U=\bigcup_{x\in U}\{x\}\subseteq \bigcup_{x\in U}V_x\subseteq U.$$
    $$U=\bigcup_{x\in U}V_x\in \mathscr{T}.$$
\end{proofenv}
\begin{propositionenv}
    Let $X$ be a set, $\left(\mathscr{T}_i\right)_{i\in I}$ be a family of topologies on $X$. Then 
    $$\mathscr{T}=\bigcap_{i\in I}\mathscr{T}_i$$
    is a topology on $X$.
\end{propositionenv}
\begin{proofenv}
    \ \newline
    (1) $\forall i\in I,\ \{\varnothing,X\}\subseteq \mathscr{T}_i$, so $\{\varnothing,X\}\subseteq \mathscr{T}$.
    \newline
    (2) If $(U_1,U_2)\in \mathscr{T}^2$, then for any $i\in I$, $\ U_1\cap U_2\in \mathscr{T}_i$, so $\dis U_1\cap U_2\in \bigcap_{i\in I}\mathscr{T}_i$.
    \newline
    (3) For any set $J$ and any $\left(U_j\right)_{j\in J}\in \mathscr{T}^J$, one has $\forall i\in I,\ \forall j\in J,\ U_j\in \mathscr{T}_i$, so
    $$\bigcup_{j\in J}U_j\in \mathscr{T}_i.$$
    Therefore, $$\bigcup_{j\in J}U_j\in \bigcap_{i\in I}\mathscr{T}_i.$$ 
\end{proofenv}
\begin{definitionenv}
    Let $S$ be a subset of $\wp(X)$, we denote by $\mathscr{T}_S$ the intersection of all topologies containing $S$, we call it the topology generated by $S$. 
\end{definitionenv}
\begin{definitionenv}
    \ \newline
    Let $\mathcal{B}$ be a subset of $\wp(X)$, we say that $\mathcal{B}$ is a topological basis if:
    \newline
    (1) $\dis X=\bigcup_{V\in \mathcal{B}}V$.
    \newline
    (2) $\forall (U,V)\in \mathcal{B}\times\mathcal{B},\ \forall x\in U\cap V,\ \exists W_x\in \mathcal{B},\ x\in W_x\subseteq U\cap V$.
\end{definitionenv}
\begin{definitionenv}
    Let $X$ be a set and $\mathscr{T}_1$ and $\mathscr{T}_2$ be two topologies on $X$. If $\mathscr{T}_1\subseteq\mathscr{T}_2$, we say that $\mathscr{T}_1$ is coarser than $\mathscr{T}_2$ and $\mathscr{T}_2$ is finer than $\mathscr{T}_1$. 
    
    If $S\subseteq\wp(X)$, we denote by $\mathscr{T}_S$ the intersection of all topology containing $S$. It is the coarsest topology containing $S$. 
    
    Let  $\mathcal{B}\subseteq \wp(X)$. If $\dis X=\bigcup_{V\in \mathcal{B}}V$ and $\forall (U,V)\in \mathcal{B}\times\mathcal{B},\ \forall x\in U\cap V,\ \exists W_x\in \mathcal{B},$ such that $x\in W_x\subseteq U\cap V$, we say that $\mathcal{B}$ is a \textbf{topological basis} on $X$.
\end{definitionenv}
\begin{propositionenv}
    Let $S$ be s subset of $\wp(X)$. Let 
    $$\mathcal{B}_S:=\{X\}\cup\left\{\bigcap_{i=1}^{n}A_i\mid n\in \NN_{\ge 1},\ (A_1,\dots,A_n)\in S^n\right\},$$
    then, $\mathcal{B}_S$ is a topological basis on $X$. Moreover, $\mathscr{T}_S=\mathscr{T}_{\mathcal{B}_S}$.
\end{propositionenv}
\begin{proofenv}
    Since $X\in \mathcal{B}_S$, $\dis\bigcup_{V\in \mathcal{B}_S}V=X$. Let $(U,V)\in \mathcal{B}_S\times\mathcal{B}_S$. If $U=X$, then $U\cap V=V\in \mathcal{B}_S$. Similarly, if $V=X$, then $U\cap V=U\in \mathcal{B}_S$. 

    If $U=A_1\cap\dots\cap A_n,\ V=B_1\cap\dots\cap B_m$, then $\{A_1,\dots, A_n,B_1,\dots,B_m\}\subseteq \mathcal{B}_S$. 
    $$U\cap V=A_1\cap\dots\cap A_n\cap B_1\cap\dots\cap B_m\in \mathcal{B}_S.$$
    Since $S\subseteq\mathcal{B}_S\subseteq\mathscr{T}_{\mathcal{B}_S}$, so $\mathscr{T}_{S}\subseteq\mathscr{T}_{\mathcal{B}_S}, \ X\in \mathscr{T}_S$. If $(A_1,\dots,A_n)\in S^n$, then $(A_1,\dots,A_n)\in\mathscr{T}_{\mathcal{B}_S}$. So $A_1\cap\dots\cap A_n\in \mathscr{T}_S$. Hence $\mathcal{B}_S\subseteq\mathscr{T}_{S}$. Therefore, $\mathscr{T}_{\mathcal{B}_S}\subseteq\mathscr{T}_S$, so $\mathscr{T}_{\mathcal{B}_S}=\mathscr{T}_S$.
\end{proofenv}
\begin{propositionenv}
    Let $\mathcal{B}$ be a topological basis on a set $X$. Then
    $$\mathscr{T}_{\mathcal{B}}=\left\{U\in\wp(X)\mid \exists \text{ a set }I \text{ and }(V_i)_{i\in I}\in \mathcal{B}^I, U=\bigcup_{i\in I}V_i\right\}.$$
\end{propositionenv}
\begin{proofenv}
    We denote by $\mathscr{T}$ te set 
    $$\{U\in \wp(X)\mid U\text{ can be written as as the union of a family sets in }\mathcal{B}\}.$$
    By definition, $\mathcal{B}\subseteq\mathscr{T}\subseteq\mathscr{T}_{\mathcal{B}}$. It remains to check that $\mathscr{T}$ is a topology.

    By definition, $X\in \mathscr{T},\ \varnothing\in \mathscr{T}$. Moreover, the union of a family of elements of $\mathscr{T}$ remains in $\mathscr{T}$. Let $\dis U=\bigcup_{i\in I}U_i$ and $\dis V=\bigcup_{j\in J}V_j$ be elements of $\mathscr{T}$, where, $U_i\in\mathcal{B}, V_{j}\in \mathcal{B}$. Then 
    $$ U\cap V=\bigcup_{(i,j)\in I\times J}\left(U_i\cap V_{j}\right).$$
    For any $x\in U_i\cap V_j$, $\exists W_x^{(i,j)}\in \mathcal{B},\ x\in W_x^{(i,j)}\subseteq U_i\cap V_j$. $\dis U_i\cap V_j=\bigcup_{x\in U_i\cap V_j}W_x^{(i,j)}$, so 
    $$U\cap V=\bigcup_{(i,j)\in I\times J}\bigcup_{x\in U_i\cap V_j}W_x^{(i,j)}.$$
\end{proofenv}


\section{Convergence}
We fix a topology space $(E,\mathscr{T})$, $l\in E$ and $S\subseteq \wp(E)$ that generates the filter $\mathcal{V}_l(\mathscr{T})$ of all neighborhood  of $l$.
\begin{definitionenv}
    Let $f:X\longrightarrow Y$ be a mapping. If $\mathcal{F}$ is a filter on $X$, we denote by $f_*(\mathcal{F})$ the set $\{B\subseteq Y\mid f^{-1}(B)\in\mathcal{F}\}$. 
\end{definitionenv}
\begin{propositionenv}
    $f_*(\mathcal{F})$ is a filter on $Y$.
\end{propositionenv}
\begin{proofenv}
    Let $(B_1,B_2)\in f_*(\mathcal{F})$,
    $$f^{-1}(B_1\cap B_2)=f^{-1}(B_1)\cap f^{-1}(B_2)\in \mathcal{F}.$$
    Let $B\in f_*(\mathcal{F}),\ C\supseteq B$. $f^{-1}(C)\supseteq f^{-1}(B)\in\mathcal{F}$, so $f^{-1}(C)\in \mathcal{F}$.
\end{proofenv}
\begin{propositionenv}
    Let $f:X\longrightarrow Y, \ g:Y\longrightarrow Z$, be mappings. $\mathcal{F}$ be a filter on $X$. Then
    $$\left(g\circ f\right)_{*}(\mathcal{F})=g_*(f_*(\mathcal{F})).$$
\end{propositionenv}
\begin{proofenv}
    \begin{align*}
        \left(g\circ f\right)_*(\mathcal{F})=&\{C\subseteq Z\mid \left(g\circ f\right)^{-1}(C)\in\mathcal{F}\}\\
        =&\{C\subseteq Z\mid f^{-1}(g^{-1}(C))\in \mathcal{F}\}\\
        =&\{C\subseteq Z\mid g^{-1}(C)\in f_*(\mathcal{F})\}\\
        =&g_*(f_*(\mathcal{F})).
    \end{align*}
\end{proofenv}
\begin{propositionenv}
    Let $B$ be a filter basis in $X$,  $f:X\longrightarrow Y$ be a mapping and $\mathcal{F}$ be the filter generated by $\mathcal{B}$. Then $f(\mathcal{B}):\{f(U)\mid U\in \mathcal{B}\}$  is a filter basis on $Y$ and $f_*(\mathcal{F})$ is the filter generated by $f(\mathcal{B})$.
\end{propositionenv}
\begin{proofenv}
    Let $U$ and $V$ be elements of $\mathcal{B}$. Then $\exists W\in \mathcal{B},\ W\subseteq U\cap V$. Hence $f(W)\subseteq f(U\cap V)\subseteq f(U)\cap f(V).$ Moreover, for any $U\in\mathcal{B}$, $U\subseteq f^{-1}(f(U))$. So $f^{-1}(f(U))\in \mathcal{F}$. Therefore, $f(\mathcal{B})\subseteq f_*(\mathcal{F}) $. Let $A\in f_*(\mathcal{F})$. Then, $f^{-1}(A)\in \mathcal{F}$. So $\exists V\in \mathcal{B},\ V\subseteq f^{-1}(A)$. Hence $f(V)\subseteq A$. Therefore, $f_*(\mathcal{F})$ is a filter basis generated by $f_*(\mathcal{B})$
\end{proofenv}
\begin{definitionenv}
    Let $f:X\longrightarrow E$ be a mapping, $\mathcal{F}$ be a non-degenerate filter on $X$. If $f_*(\mathcal{F})\supseteq \mathcal{V}_l(J)$, we say that $f$ \textbf{converges} to $l$ along $\mathcal{F}$.
    $$\lim_{\mathcal{F}}f=l $$
    denotes ``$f$ converges to $l$ along $\mathcal{F}$''.

    This condition is equivalent to 
    $$\forall V\in S_l,\ f^{-1}(V)\in \mathcal{F}.$$
    If $\mathcal{B}$ is a filter basis which generates $\mathcal{F}$. This condition is also 
    $$\forall V\in S_l,\ \exists U\in \mathcal{B},\ f(U)\subseteq V.$$
    
\end{definitionenv}
\begin{exampleenv}
    \ \newline
    (1) Let $I\subseteq \NN$ be an infinite subset and $x=(x_n)_{n\in I}\in \NN^I$. Let $\mathcal{F}$ be the Fréchet filter on $I$. If $x$ converges to $l$ along $\mathcal{F}$, or equivalently,
    $$\forall V\in S_l,\ \exists N\in \NN,\ \forall n\in I_{\ge \NN},\ x_n\in V.$$
    We say that the sequence $(x_n)_{n\in I}$ \textbf{converges to $l$} when $n$ tends to the infinity, denote as 
    $$\lim_{n\rightarrow+\infty}x_n=l.$$
    \newline
    (2) Let $(X,\mathscr{T}_X)$ be a topological space. Let $Y\subseteq X$ be a subset of $X$, $p\in X$. Let 
    $$\mathcal{F}=\mathcal{V}_p(\mathscr{T}_X)|_Y=\{V\cap Y\}\mid V\in \mathcal{V}_p(\mathscr{T}_X).$$
    Assume that $\mathcal{F}$ is non-degenerate. Let $f:X\longrightarrow E$ be a mapping. If $f$ converges to $l$ along $\mathcal{F}$, we day that $f(x)$ converges to $l$ when $x\in Y$ tends to $p$, denoted as 
    $$\lim_{x\in Y,\ x\rightarrow p}f(x)=l.$$
    This condition is equivalent to:
    $$\forall V\in S_l,\ \exists U \text{ a open neighborhood of }p, \ \forall x\in U\cap Y, \ f(x)\in V.$$
    In general, if $g:Y\longrightarrow E$ is a mapping such that
    $$\forall V\in S_l,\ \exists U \text{ a open neighborhood of }p, \ \forall x\in U\cap Y, \ g(x)\in V,$$
    then we say $g(x)$ converges to $l$ when $x\in Y$ tends to $p$, denoteed as 
    $$\lim_{x\in Y,\ x\rightarrow p}g(x)=l.$$
\end{exampleenv}
\begin{remark}
    If $(E,\mathrm{d})$ is a semimetric space, $\mathscr{T}$ is the semimetric topology. Then condition in $(1)$ becomes:
    $$\forall \varepsilon >0,\ \exists N\in \NN,\ \forall n\in I_{\ge N},\ \mathrm{d}(x_n,l)<\varepsilon.$$
    The conditions in $(2)$ becomes:
    $$\forall \varepsilon >0,\ \exists U \text{ a open neighborhood of }p, \ \forall x\in U\cap Y, \ \mathrm{d}(g(x),l)<\varepsilon.$$
    If furthermore, $(X,\mathscr{T}_X)$ is a semimetric space with semimetric $\mathrm{d}_X$. The condition becomes 
    $$\forall \varepsilon>0,\ \exists \delta>0,\ \forall x\in Y,\ \mathrm{d}_X(x,p)<\delta\Rightarrow \mathrm{d}(g(x),l)<\varepsilon.$$
\end{remark}
\begin{exampleenv}
    \ \newline
    (3) Consider $X=\RR$. Let $Y\subseteq X$. Consider the filter $\mathcal{F}$ generated by $\{\RR_{>M},\ M\in \RR\}$. Suppose that $\mathcal{F}|Y$ is non-degenerate. Let $g:Y\longrightarrow E$. If $\dis \lim_{\mathcal{F}|_Y}g=l$, we say that $g(x)$ converges to $l$ when $x$ tends to $+\infty$, denoted as $$\lim_{x\in Y,\ x\rightarrow+\infty}g(x)=l.$$
     This condition is
    $$\forall V\in S_l,\ \exists M\in \RR_{>0},\ \forall x\in Y,\ x>M\Rightarrow g(x)\in V.$$
    If $(E,d)$ is a metric space, it becomes:
    $$\forall \varepsilon >0,\ \exists M\in \RR_{>0},\ \forall x\in Y,\ x>M\Rightarrow \mathrm{d}(g(x),l)<\varepsilon.$$
\end{exampleenv}
\begin{exampleenv}
    Let $(G,\le)$ be a totally ordered set, $\mathcal{F}$ be the ordered topology on $G$. It is generated by $\{G_{>a}\mid a\in g\}\cup\{G_{<b}\mid b\in G\}$. If $l\in G$, then $\mathcal{V}_x(\mathscr{T})$ is generated by 
    $$S_l:=\{G_{>a}\mid a<l\}\cup\{G_{<b}\mid l<b\}.$$
    Assume that $(G,\le)$ is order complete. Let $f:X\longrightarrow G$ be a mapping and $\mathcal{F}$  be  a non-degenerate filter on $X$.
    \newline
    (1) Assume that $f$ converges to $l$ along $\mathcal{F}$. $\forall a< l, \ U_a:=f^{-1}\left(G_{>a}\right)\in \mathcal{F}$. $\forall x\in U_a, f(x)>a$. So $\liminf_{\mathcal{F}}f\ge a$. If $\sup\left(G_<l\right)=l$, then $\liminf_{\mathcal{F}}f\ge l.$ If $\sup\left(G_{<l}\right)<l$, we denote $a=\sup\left(G_{<l}\right)$. $\forall a\in U_a, f(x)\ge l$. So $\liminf_{\mathcal{F}}f\ge l$. Similarly, $\liminf_{\mathcal{F}}f\le l$. So $f$ admits $l$ as its limit.
    \newline
    (2) Assume that $\limsup_{\mathcal{F}}f=\liminf_{\mathcal{F}}f=l$. 
    $$\liminf_{\mathcal{F}}f=l\Rightarrow \sup_{U\in \mathcal{F}}f^i(U)=l, \forall a<l, \exists U\in \mathcal{F}, f^i(U)>a, f^{-1}(G_{>a})\in \mathcal{F}.$$
    $$\limsup_{\mathcal{F}}f=l\Rightarrow \forall b>l, f^{-1}(G_{<b})\in \mathcal{F}.$$
    Therefore, $f$ converges to $l$ along $\mathcal{F}$.
\end{exampleenv}

