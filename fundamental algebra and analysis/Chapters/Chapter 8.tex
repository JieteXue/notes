\chapter{Topology}
\section{Topological spaces}
\begin{propositionenv}
    Let $X$ be a set and. For any $x\in X$, let $\mathcal{G}_x$ be a filter contained in the principal filter of $\{x\}$ ($\forall U\in \mathcal{G}_x,\ x\in U$). Denote by $\mathscr{T}$ the set 
    $$\{U\in \wp(X) \mid \forall x\in U, U\in\mathcal{G}_x\}.$$
    Then the following conditions are satisfied.
    \newline
    (1) $\{\varnothing, X\} \subseteq \mathscr{T}$.
    \newline
    (2) If $(U_1,U_2)\in \mathscr{T}^2$, then $U_1\cap U_2\in \mathscr{T}$.
    \newline
    (3) If $I$ is a set and $\left(U_i\right)_{i\in I}\in \mathscr{T}^I$, then $\dis \bigcup_{i\in I}U_i\in \mathscr{T}$.
    \newline
    Moreover, $\forall x\in X,\ \mathcal{B}_x=\{U\in \mathscr{T}\mid x\in U\}$ is a filter basis contained in $\mathcal{G}_x$. It generates $\mathcal{G}_x$ if the following condition is satisfied:
    $$\forall U\in \mathcal{G}_x,\ \exists V\in \mathcal{G}_x,\ V\subseteq U \text{ and } \forall y\in V,\ V\in \mathcal{G}_y.$$
\end{propositionenv}
\begin{proofenv}
    \ \newline
    (1) $\dis \varnothing \in \mathscr{T}, X\in \bigcap_{x\in X}\mathcal{G}_x$.
    \newline
    (2) $\forall x\in U_1\cap U_2,\ U_1\in \mathcal{G}_x,\ U_2\in \mathcal{G}_x$, so $U_1\cap U_2\in \mathcal{G}_x$.
    \newline
    (3) Let $\dis U=\bigcup_{i\in I}U_i$. $\forall x\in U,\ \exists i\in I,\ x\in U_i$, so $U_i\in \mathcal{G}_x$. Since $U\supseteq U_i$, so $U\in\mathcal{G}_x$.
    $$\mathcal{B}_x:=\{U\in \mathscr{T}\mid x\in U\}.$$
    If $U\in \mathcal{B}_x$, then $x\in U$, so $U\in \mathcal{G}_x$. Hence $\mathcal{B}_x\subseteq\mathcal{G}_x$. If $(U,V)\in \mathcal{B}_x^2$, then $U\cap V\in \mathscr{T}$, and $x\in U\cap V$. So $U\cap V\in \mathcal{B}_x$. So $\mathcal{B}_x$ is a filter basis. Suppose the condition is satisfied. For any $U\in \mathcal{G}_x,\ \exists V\in \mathcal{G}_x\cap\mathscr{T}$, such that $V\subseteq U$. Note that $V\in \mathcal{B}_x$, so $\mathcal{G}_x$ is generated by $\mathcal{B}_x$.
\end{proofenv}
\begin{definitionenv}
    Let $X$ be a set. We call \textbf{topology} on $X$ any subset $\mathscr{T}$ of $\wp(X)$ that satisfies the following conditions:
     \newline
    (1) $\{\varnothing, X\} \subseteq \mathscr{T}$.
    \newline
    (2) If $(U_1,U_2)\in \mathscr{T}^2$, then $U_1\cap U_2\in \mathscr{T}$.
    \newline
    (3) For any set $I$,  $ \forall \left(U_i\right)_{i\in I}\in \mathscr{T}^I$, then $\dis \bigcup_{i\in I}U_i\in \mathscr{T}$.
    \newline
    $(X,\mathscr{T})$ is called a \textbf{topological space}.
    \newline
    If $\forall x\in X,\ \mathcal{B}_x$ is a filter basis of $X$ contained in the principal filter of $\{x\}$, then 
    $$\mathscr{T}=\{U\in \wp(X)\mid \forall x\in U,\ \exists V_x\in \mathcal{B}_x,\ V_x\subseteq U\}$$
    is a topology on $X$, called the \textbf{topology generated by} $(\mathcal{B}_x)_{x\in X}$. More generally, if $\forall x\in X,\ S_x $ is a subset of the principal filter of $\{x\}$ and $\mathcal{G}_x$ is the filter generated by $S_x$, then we say that 
    $$\mathscr{T}=\{U\in\wp(X)\mid \forall x\in U,\ U\in \mathcal{G}_x\}$$
    is the topology generated by $(S_x)_{x\in X}$.

\end{definitionenv}
\begin{exampleenv}
    \ \newline
    (1) Let $\mathcal{G}_x=\{X\}$. The topology by $\left(\mathcal{G}_x\right)_{x\in X}$ is $\{\varnothing,X\}$, called the \textbf{trivial topology} on $X$.  
    \newline
    (2) Let $\mathcal{G}_x=\mathcal{F}_{\{x\}}$ be the principal filter. The topology generated by $\left(\mathcal{G}_x\right)_{x\in X}$ is $\wp(X)$. This topology is called the \textbf{discrete topology} on $X$.
    \newline
    (3) Let $(X,\mathrm{d})$ be a semimetric space. 
    $$\mathrm{d}:X\times X\longrightarrow \mathbb{R}_{\ge 0},\  \mathrm{d}(x,y)=\mathrm{d}(y,x),\ \mathrm{d}(x,z)\le \mathrm{d}(x,y)+\mathrm{d}(y,z),\ \mathrm{d}(x,x)=0.$$
    $\forall \varepsilon>0,\ \forall x\in X$, let $B(x,\varepsilon)=\{y\in X\mid \mathrm{d}(x,y)<\varepsilon\}$, $\{B(x,\varepsilon)\mid \varepsilon\in \RR_{>0}\}=: \mathcal{B}_x$ is a filter basis on $X$, $\mathcal{B}_x$ is contained in the principal filter of $\{x\}$. The topology 
    $$\mathscr{T}=\{U\in \wp(X)\mid \forall x\in U,\ \exists \varepsilon\in \RR_{>0},\ B(x,\varepsilon)\subseteq U\}$$
    is called the \textbf{topology induced by the semimetric $\mathrm{d}$}.
    \newline
    (4) Let $(G,\le )$ be a totally ordered set $\forall x\in G$, let $S_x=\{G_{>a}\mid a<x\}\cup \{G_{<b}\mid x<b\}$
\end{exampleenv}
\begin{propositionenv}
    $\forall x\in X,\ \forall \varepsilon\in \RR_{>0}, B(x,\varepsilon)\in \mathscr{T}$.
\end{propositionenv}
\begin{proofenv}
    $\forall y\in B(x,\varepsilon),\ \mathrm{d}(x,y)<\varepsilon$. Let $r=\varepsilon-\mathrm{d}(x,y)>0$, we claim that $B(y,r)\subseteq B(x,\varepsilon)$. Let $z\in B(y,r), \ \mathrm{d}(y,z)<r$. Hence,
    $$\mathrm{d}(x,z)\le \mathrm{d}(x,y)+\mathrm{d}(y,z)<\mathrm{d}(x,y)+r=\mathrm{d}(x,y)+\varepsilon -\mathrm{d}(x,y)=\varepsilon.$$
\end{proofenv}
\begin{remark}
    On $\RR$, one has a mertic 
    $$\mathrm{d}: \RR\times \RR\longrightarrow \RR_{\ge 0},$$
    $$\left(a,b\right)\longmapsto |a-b|.$$
    $$B(x,\varepsilon)=\interval[open]{x-\varepsilon}{x+\varepsilon}.$$
    $$\mathcal{B}_x=\{\interval[open]{x-\varepsilon}{x+\varepsilon}\mid \varepsilon\in \RR_{>0}\}.$$
    Let $\mathscr{T}_\mathrm{d}$ be the topology generated by $(\mathcal{B}_x)_{x\in \RR}$. Let $\mathscr{J}$ be the order topology generated by $\left(S_x\right)_{x\in \RR}$, where 
    $$S_x:=\{\RR_{>a}\mid a<x\}\cup \{\RR_{<b}\mid x<b\}.$$
\end{remark}
\begin{propositionenv}
    For any $x\in \RR$, $\mathcal{F}(\mathcal{B}_x)=\mathcal{F}(S_x)$.
\end{propositionenv}
\begin{proofenv}
    $\forall \varepsilon>0,\ \interval[open]{x-\varepsilon}{x+\varepsilon}=\RR_{<x+\varepsilon}\cap \RR_{>x-\varepsilon}\in \mathcal{F}(S_x)$. So $\mathcal{F}(\mathcal{B}_x)\subseteq \mathcal{F}(S_x)$.
    $$\forall a\in \RR,\ a<x,\ \RR_{>a}\supseteq \interval[open]{a}{2x-a}=\interval[open]{x-(x-a)}{x+(x-a)},\ \RR_{>a}\in \mathcal{F}(\mathcal{B}_x).$$
    $$\forall b\in \RR, b>x,\RR_{<b}\supseteq \interval[open]{2x-b}{b}=\interval[open]{x+(b-x)}{x+(b-x)},$$
    So, $\RR_{<b}\subseteq\mathcal{F}(\mathcal{B}_x)$. Hence $S_x\subseteq \mathcal{F}(\mathcal{B}_x)$, which leads to $\mathcal{F}(S_x)\subseteq\mathcal{F}(\mathcal{B}_x)$.
\end{proofenv}
\begin{definitionenv}
    Let $(X,\mathscr{T})$ be a topological space. For any $x\in X$ and any $V\in \wp(X)$, if there exists $U\in \mathscr{T}$ such that $x\in U\subseteq V$, then we say that $V$ is a \textbf{neighborhood of $x$}. We call \textbf{open subset} of $X$ any subset of $X$ that belongs to $\mathscr{T}$. If $U\in \mathscr{T}$, such that $x\in U$, we say that $U$ is an \textbf{open neighborhood of $x$}. We denote by $\mathcal{V}_x(\mathscr{T})$ the set of all neighborhoods of $x$.
\end{definitionenv}
\begin{propositionenv}
    $\mathcal{V}_x(\mathscr{T})$ is a filter on $X$ contained in the principal filter of $\{x\}$. Moreover, the topology generated by $\left(\mathcal{V}_x(\mathscr{T})\right)_{x\in X}$ identifies with $\mathscr{T}$.
\end{propositionenv}
\begin{proofenv}
    \ \newline
    (1) If $(V_1,V_2)\in \mathcal{V}_x(\mathscr{T})^2,\ \exists (U_1,U_2)\in \mathscr{T}^2$, such that $x\in U_1\subseteq V_1,\ x\in U_2\subseteq V_2$. Hence, $x\in U_1\cap U_2\subseteq V_1\cap V_2$, so $V_1\cap V_2\in \mathcal{V}_{x}(\mathscr{T})$.
    \newline
    (2) If $V\in \mathcal{V}_x(\mathscr{T})$, $W\in \wp(X),\ V\subseteq W$. $\exists U\in \mathscr{T}$, $x\in U\subseteq V\subseteq W$, so $W\in \mathcal{V}_x(\mathscr{T})$. Let $\mathscr{T}'$ be the topology generated by $\left(\mathcal{V}_x(\mathscr{T})\right)_{x\in X}$. By definition,
    $$\mathscr{T}'=\{U\subseteq X\mid \forall x\in U, U\in \mathcal{V}_x(\mathscr{T})\}.$$
    For any $U\in \mathscr{T},\ \forall x\in U  $, $U$ is a open neighborhood of $x$, so $U\in \mathscr{T}'$. Let $U\in \mathscr{T}', \ \forall x\in U,\ \exists V_2\in \mathscr{T},\ x\in V_x\subseteq U$.
    $$U=\bigcup_{x\in U}\{x\}\subseteq \bigcup_{x\in U}V_x\subseteq U.$$
    $$U=\bigcup_{x\in U}V_x\in \mathscr{T}.$$
\end{proofenv}
\begin{propositionenv}
    Let $X$ be a set, $\left(\mathscr{T}_i\right)_{i\in I}$ be a family of topologies on $X$. Then 
    $$\mathscr{T}=\bigcap_{i\in I}\mathscr{T}_i$$
    is a topology on $X$.
\end{propositionenv}
\begin{proofenv}
    \ \newline
    (1) $\forall i\in I,\ \{\varnothing,X\}\subseteq \mathscr{T}_i$, so $\{\varnothing,X\}\subseteq \mathscr{T}$.
    \newline
    (2) If $(U_1,U_2)\in \mathscr{T}^2$, then for any $i\in I$, $\ U_1\cap U_2\in \mathscr{T}_i$, so $\dis U_1\cap U_2\in \bigcap_{i\in I}\mathscr{T}_i$.
    \newline
    (3) For any set $J$ and any $\left(U_j\right)_{j\in J}\in J^I$, one has $\forall i\in I,\ \forall j\in J,\ U_j\in \mathscr{T}_i$, so
    $$\bigcup_{j\in J}U_j\in \mathscr{T}_i.$$
    Therefore, $$\bigcup_{j\in J}U_j\in \bigcap_{i\in I}\mathscr{T}_i.$$ 
\end{proofenv}
\begin{definitionenv}
    Let $S$ be a subset of $\wp(X)$, we denote by $\mathscr{T}_S$ the intersection of all topologies containing $S$, we call it the topology generated by $S$. 
\end{definitionenv}
\begin{definitionenv}
    \ \newline
    Let $\mathcal{B}$ be a subset of $\wp(X)$, we say that $\mathcal{B}$ is a topological basis if:
    \newline
    (1) $\dis X=\bigcup_{V\in \mathcal{B}}V$.
    \newline
    (2) $\forall (U,V)\in \mathcal{B}\times\mathcal{B},\ \forall x\in U\cap V,\ \exists W_x\in \mathcal{B},\ x\in W_x\subseteq U\cap V$.
\end{definitionenv}


