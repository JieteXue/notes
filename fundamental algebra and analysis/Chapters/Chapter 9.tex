\chapter{Differential Calculus}
\section{Landau symbol}
\quad In this section, we fix a complete valued field \defm{9.1}{$(K,\left|\ \cdot\ \right|)$} and a normed vector space $(V,\pl\cdot\pl)$ over $K$.
\begin{definitionenv}
    Let $X$ be a set, $f:X\longrightarrow V$, $g:X\longrightarrow \RR_{\ge 0}$ be mappings. 
    \newline
    \quad Let $Y\subseteq X$ be a subset. We use the expression 
    $$f(x)=\mathcal{O}(g(x))$$ to denote the statement:
    $$\exists C>0, \forall x\in Y, \pl f(x)\pl \le C\cdot g(x).$$
    \quad Let $\mathcal{F}$ be a filter on $X$, we use the expression 
    $$f(x)=\mathcal{O}\left(g(x)\right) \text{ along } \mathcal{F}$$ to denote the statement:
    $$\exists C>0, \exists A\in \mathcal{F}, \pl f(x)\pl\le C\cdot g(x), \forall x\in A.$$
    We use the expression 
    $$f(x)=o\left(g(x)\right)\text{ along }\mathcal{F}$$ to denote the statement: 
    $$\exists \varepsilon:X\longrightarrow \RR_{\ge 0}, \exists A\in \mathcal{F}, \lim_{\mathcal{F}}\varepsilon=0 \text{ and }\forall x\in A, \pl f(x)\pl\le \varepsilon(x)g(x).$$
\end{definitionenv}
\begin{propositionenv}
    Let $X$ be a set and $\mathcal{F}$ be a filter on $X$.
    \newline
    (1) Let $f:X\longrightarrow V$, $g:X\longrightarrow \RR_{\ge 0}$ be mappings. If $f(x)=o(g(x))$ along $\mathcal{F}$, then $f(x)=\mathcal{O}(g(x))$ along $\mathcal{F}$.
    \newline
    (2) \begin{enumerate}
        \item Let $f_1:X\longrightarrow V$, $f_2:X\longrightarrow V$ and $g:X\longrightarrow \RR_{\ge 0}$ be mappings. If $f_1(x)=\mathcal{O}(g(x))$ and $f_2(x)=\mathcal{O}(g(x))$ along $\mathcal{F}$, then $f_1(x)+f_2(x)=\mathcal{O}(g(x))$ along $\mathcal{F}$.
        \item Let $f_1:X\longrightarrow V$, $f_2:X\longrightarrow V$ and $g:X\longrightarrow \RR_{\ge 0}$ be mappings. If $f_1(x)=o(g(x))$ and $f_2(x)=o(g(x))$ along $\mathcal{F}$, then $f_1(x)+f_2(x)=o(g(x))$ along $\mathcal{F}$.
    \end{enumerate}
    (3) Let $\lambda:X\longrightarrow K$, $f:X\longrightarrow V$, $g:X\longrightarrow \RR_{\ge 0}$, $h:X\longrightarrow \RR_{\ge 0}$ be mappings. 
    \begin{enumerate}
        \item If $\lambda(x)=\mathcal{O}\left(g(x)\right)$ along $\mathcal{F}$, $f(x)=\mathcal{O}\left(h(x)\right)$ along $\mathcal{F}$, then
            $$(\lambda f)(x)=\lambda(x)f (x)=\mathcal{O}\left(g(x)h(x)\right).$$
        \item If $\lambda(x)=\mathcal{O}\left(g(x)\right)$ along $\mathcal{F}$, $f(x)=o\left(h(x)\right)$ along $\mathcal{F}$, or if $\lambda(x)=o\left(g(x)\right)$ along $\mathcal{F}$, $f(x)=\mathcal{O}\left(h(x)\right)$ along $\mathcal{F}$, then
            $$\lambda(x)f(x)=o\left(g(x)h(x)\right).$$
    \end{enumerate}
\end{propositionenv}
\begin{proofenv}
    \ \newline
    (1) We have $\varepsilon: X\longrightarrow \RR_{\ge 0}$, $A\in \mathcal{F}$ such that $\lim_{\mathcal{F}}\varepsilon=0$ and $\forall x\in A$, $\pl f(x)\pl\le \varepsilon (x)g(x)$. Since $\lim_{\mathcal{F}}\varepsilon=0$, there exists $B\in \mathscr{T}$ such that $\forall x\in B$, $\left|\varepsilon(x)\right|<1$, hence $\forall x\in A\cap B$, $\pl f(x)\pl\ge (x)$.
    \newline
    (2)
    \begin{enumerate}
        \item $A_1,A_2\in \mathcal{F}, C_1,C_2>0$, $\forall x\in A_1, \pl f_1(x)\pl\le C_1g(x), \forall x\in A_2, \pl f_2(x)\pl\le C_2g(x)$. So $f_1(x)+f_2(x)=\mathcal{O}(g(x))$
        \item Let $\varepsilon_1:X\longrightarrow \RR_{\ge 0}$, $\varepsilon_2:X\longrightarrow \RR_{\ge 0}$, $A\in \mathcal{F}$, $\lim_{\mathcal{F}}\varepsilon_1=\lim_{\mathcal{F}}\varepsilon_2=0$. $\forall x\in A_1, \pl f_1(x)\pl\le \varepsilon_1(x)\cdot g(x), \forall x\in A_2, \pl f_2(x)\pl\le \varepsilon_2(x)g(x)$. So $\lim_{\mathcal{F}}\varepsilon_1+\varepsilon_2=0$.
            $$\forall x\in A_1\cap A_2,\ \pl f_1(x)+f_2(x)\pl \le \pl f_1(x)\pl+\pl f_2(x)\pl\le \left(\varepsilon_1(x)+\varepsilon_2(x)\right)g(x).$$
    \end{enumerate}
    (3) 
    \begin{enumerate}
        \item There exists $(C_1,C_2)\in \RR_{>0}^2$ and $(A_1,A_2)\in \mathcal{F}^2$ such that
            $$\forall x\in A_1,\ \left|\lambda(x)\right|\le C_1 g(x),\ \forall x\in A_2,\ \pl f(x)\pl\le C_2 h(x).$$
            Hence, 
            $$\forall x\in A_1\cap A_2,\ \pl (\lambda(x)f(x))\pl \le \left|\lambda(x)\right|\cdot \pl f(x)\pl \le C_1 C_2 g(x) h(x).$$
        \item We assume that 
            $$\lambda(x)=\mathcal{O}\left(g(x)\right)\text{ along }\mathcal{F},\ f(x)=o\left(h(x)\right) \text{ along }\mathcal{F}.$$
            There exists $(A_1,A_2)\in \mathcal{F}\times \mathcal{F}, C\in \RR_{\ge 0}$ and a mapping $\varepsilon:X\longrightarrow \RR_{\ge 0}$ such that 
            $$\forall x\in A_1,\  \left|\lambda(x)\right|\le C\cdot g(x),\ \forall x\in A_2, \pl f(x)\pl\le \varepsilon(x)h(x).$$
            Then one has 
            $$\lim_{\mathcal{F}}C\varepsilon(x)=0$$
            and 
            $$\forall x\in A_1\cap A_2,\ \pl (\lambda(x)f(x))\pl\le \left|\lambda(x)\right|\cdot \pl f(x)\pl\le C\cdot g(x)\cdot \varepsilon(x)h(x)$$
            As required. 
    \end{enumerate}
\end{proofenv}
\begin{exampleenv}
    \ \newline
    (1) Let $I\subseteq\NN$ infinite. Let $(V,\pl\cdot\pl)$ be a normed vector space over complete valued field $(K,\left|\ \cdot\ \right|)$. Let $\mathcal{F}$ be the filter on $I$. Let $(x_n)_{n\in I}\in V^I, (b_n)_{n\in I}\in \RR_{\ge 0}^I$. We denote by 
        $$x_n=\mathcal{O}\left(b_n\right),\ n\in I,\ n\rightarrow+\infty$$
        the statement $x_n=\mathcal{O}\left(b_n\right)$ along $\mathcal{F}$. Namely,
        $$\exists N\in \NN,\ \exists C>0,\ \forall n\in I_{\ge N},\ \pl x_n\pl \le C\cdot b_n.$$
        $$x_n=o\left(b_n\right),\ n\in I,\ n\rightarrow +\infty$$
        denotes the statement $x_n=o\left(b_n\right)$ along $\mathcal{F}$. Namely, 
        $$\exists (\varepsilon_n)_{n\in I}\text{ such that } \lim_{n\rightarrow +\infty}\varepsilon_n=0,\ \exists N\in \NN,\ \forall n\in I_{\ge N},\ \pl x_n\pl \le \varepsilon_{n}\cdot b_n.$$
    \newline
    (2) Let $(X,\mathscr{T})$ be a topological space, $Y\subseteq X$, $y_0\in \overline{Y}$. Let $f:Y\longrightarrow V$ and $g:Y\longrightarrow \RR_{\ge 0}$ be mappings. 
    $$\mathcal{F}=\mathcal{V}_{y_0}\left(\mathscr{T}\right)|_Y\coloneq\{U\cap Y\mid U \text{ is a neighborhood of } y_0 \}$$
    $f(y)\mathcal{O}\left(g(y)\right)$, $y\in Y$, $y\rightarrow y_0$ denotes $f(y)=\mathcal{O}\left(g(y)\right)$ along $\mathcal{F}$. Namely, 
    $$\exists C>0,\ \exists U\in \mathcal{V}_{y_0}\left(\mathscr{T}\right),\ \forall y\in U\cap Y,\ \pl f(y)\pl \le C\cdot g(y).$$
    $$f(y)=o\left(g(y)\right),\ y\in Y,\ y\rightarrow y_0$$
    denotes $f(y)=o\left(g(y)\right)$ along $\mathcal{F}$. Namely, 
    $$\exists \varepsilon:Y\longrightarrow \RR_{\ge 0},\ \lim_{y\in Y,y\rightarrow y_0}\varepsilon(y)=0,\ \exists U\in \mathcal{V}_{y_0}\left(\mathscr{T}\right),$$
    $$ \forall y\in U\cap Y,\ \pl f(y)\pl \le \varepsilon(y)g(y).$$
    (3) Let $\mathcal{F}$ be a filter on $\RR$ generated by subsets of the form $\interval[open right]{a}{+\infty}$. Let $Y\subseteq \RR$ not bounded from above. Let $f:Y\longrightarrow V$ and $g:Y\longrightarrow \RR_{\ge 0}$ be mappings. Then
    $$f(y)=\mathcal{O}\left(g(y)\right),\ y\in Y,\ y\rightarrow +\infty$$
    denotes $f(y)=\mathcal{O}\left(g(y)\right)$ along $\mathcal{F}|_Y$. Namely, 
    $$\exists C>0,\ \exists a\in \RR,\ \forall y\in Y_{\ge a},\ \pl f(y)\pl \le C\cdot g(y).$$
    $$f(y)=o\left(g(y)\right),\ y\in Y,\ y\rightarrow +\infty$$
    denotes $f(y)=o\left(g(y)\right)$ along $\mathcal{F}|_Y$. Namely, 
    $$\exists \varepsilon:Y\longrightarrow \RR_{\ge 0},\ \lim_{y\rightarrow +\infty}\varepsilon(y)=0,\ \exists a\in \RR,\ \forall y\in Y_{\ge a},\ \pl f(y)\pl \le \varepsilon(y)g(y).$$
\end{exampleenv}
\section{Differentiability}
We fix a complete valued field \defm{9.2.1}{$(K,\left|\ \cdot\ \right|)$}. We suppose that there exists $a\in K^\times$, such that $\left|a\right|<1$. Let $(E, \pl \cdot\pl_E)$ and $(F,\pl \cdot\pl_F)$ be normed vector spaces over $K$.
$$\mathscr{L}(E,F)\coloneq\{\varphi\in\mathrm{Hom}_{K}(E,F)\mid \pl\varphi\pl<+\infty\}.$$
$(\mathscr{L}(E,F),\pl \cdot\pl) $ is a normed vector space over $K$.
\begin{definitionenv}
    Let $U\subseteq E$ be subset and $p\in U^\circ$. We say that a mapping $f:U\longrightarrow F$ is \textbf{differentiable} at $p$ if there exists $\varphi\in\mathscr{L}(E,F)$ such that 
    $$f(p+h)-f(p)-\varphi(h)=o\left(\pl h\pl_E\right),\ h\rightarrow0_E.$$
    If $U=U^\circ$ and $f$ is differentiable at every point of $U$, we say that $f$ is \textbf{differentiable} on $U$.
\end{definitionenv}
\begin{propositionenv}
    Assume that $f:U\longrightarrow F$ is differentiable at $p\in U^\circ$. There exists a unique $\varphi\in \mathscr{L}(E,F)$ such that
    $$f(p+h)-f(p)-\varphi(h)=o\left(\pl h\pl_E\right),\ h\rightarrow 0_E.$$
\end{propositionenv}
\begin{lemmaenv}
    $\forall \eta\in \mathscr{L}(E,F)$, $\forall r>0$. 
    $$\pl\eta\pl=\sup_{x\in E,0<\pl x\pl_E\le r}\frac{\pl\eta(x)\pl_F}{\pl x\pl_E}=\sup_{\substack{x\in E\\0<\pl x\pl_E< r}}\frac{\pl\eta(x)\pl_F}{\pl x\pl_E}.$$
\end{lemmaenv}
\begin{proofenv}[of Lemma]
    $\pl\eta\pl\ge \sup_{x\in E,0<\pl x\pl_E< r}\frac{\pl\eta(x)\pl_F}{\pl x\pl_E}$. $\forall y\in E\backslash\{0\}$, $\pl a^N y\pl_{E}=\left|a\right|^N\pl y\pl_E<r$.
    $$\frac{\pl \eta(a^N y)\pl_F}{\pl a^N y\pl_E}=\frac{\left|a\right|^N\cdot\pl \eta(y)\pl_F}{\left|a\right|^N\cdot\pl y\pl_E}=\frac{\pl \eta(y)\pl_F}{\pl y\pl_E}\le \sup_{\substack{x\in E\\0<\pl x\pl_E< r}}\frac{\pl\eta(x)\pl_F}{\pl x\pl_E}.$$
\end{proofenv}
\begin{proofenv}[of Proposition]
    Suppose $\varphi,\psi\in\mathscr{L}(E,F)$ are such that
    $$f(p+h)-f(p)-\varphi(h)=o\left(\pl h\pl_E\right),\ h\rightarrow 0_E,$$
    $$f(p+h)-f(p)-\psi(h)=o\left(\pl h\pl_E\right),\ h\rightarrow 0_E.$$
    Then
    $$\varphi(h)-\psi(h)=o\left(\pl h\pl_E\right),\ h\rightarrow 0_E.$$
    $$\exists r>0, \exists\varepsilon:\overline{B}(0_E,r)\longrightarrow \RR_{\ge0}\text{ such that }\lim_{h\rightarrow 0_E}\varepsilon(h)=0.$$
    $$\forall h\in \overline{B}(0_E,r),\ \pl (\varphi-\psi)(h)\pl_F = \varepsilon(h)\pl h\pl_E.$$
    $$\pl \varphi-\psi\pl=\sup_{\substack{x\in E\\0<\pl h\pl_E< r'}}\frac{\pl\varphi(h)-\psi(h)\pl_F}{\pl h\pl_E}\le\sup_{0<\pl h\pl_E<r'}\varepsilon(h).$$
    Taking the limit when $r'\rightarrow 0$, by $\limsup_{h\rightarrow 0_E}\varepsilon(h)=0$. We get $\pl \varphi-\psi\pl=0$, hence $\varphi=\psi$.
\end{proofenv}
\begin{definitionenv}
    Let $U\subseteq E$ and $f:U\longrightarrow F$ be a mapping that is differentiable at $p\in U^\circ$. The unique $\varphi\in \mathscr{L}(E,F)$ such that
    $$f(p+h)-f(p)-\varphi(h)=o\left(\pl h\pl_E\right),\ h\rightarrow 0_E$$
    is called the \textbf{differential} of $f$ at $p$ and is denoted as
    $$\DD(f(p)).$$
\end{definitionenv}
\begin{exampleenv}
    \ \newline
    (1) $f:U\longrightarrow F$, $f(x)\equiv c$, $c\in F$. 
    $$f(x+h)-f(x)=0_E=o\left(\pl h\pl_E\right).$$
    So $f$ is differentiable at every point of $U$ and $\DD(f(x))=0_F$.
    \newline
    (2) $\varphi\in \mathscr{L}(E,F)$. 
    $$\varphi(p+h)-\varphi(p)-\varphi(h)=0_F=o\left(\pl h\pl_E\right).$$
    So $\varphi$ is differentiable at every point of $E$ and $\DD(\varphi(p))=\varphi$.
    \newline
    (3) Let $\left(F_i,\pl\cdot\pl_i\right)$ be normed vector spaces over $K$, $i\in\{1,\ldots,n\}$, $n\in\NN$. Suppose that $F=F_1\oplus\dots\oplus F_n$ and 
    $$\pl\left(s_1,\dots,s_n\right)\pl_F=\max\{\pl s_1\pl_1, \ldots,\pl s_n\pl_n\}.$$
    Let $U\subseteq E$ be an open subset, $f_i:U\longrightarrow F_i$ be a mapping. 
    $$f:U\longrightarrow F,\ f(x)=\left(f_1(x),\ldots,f_n(x)\right).$$
    $$f(p+h)-f(p)=\left(f_1(p+h)-f_1(p),\ldots,f_n(p+h)-f_n(p)\right).$$
    Suppose that each $f_i$ is differentiable
    \begin{align*}
        &f(p+h)-f(p)-\left(\DD f_1(p)(h),\ldots,\DD f_n(p)(h)\right)|_F\\
    =&\max_{i\in\{1,\ldots,n\}}\pl f_i(p+h)-f_i(p)-\DD f_i(p)(h)\pl_{F_i}\\
    =&o\left(\pl h\pl_E\right).
    \end{align*}
    So $f$ is differentiable at $p$ and
    $$\DD f(p)(h)=\left(\DD f_1(p)(h),\ldots,\DD f_n(p)(h)\right).$$
    (4) Suppose that $E=K$. If $U\subseteq K$ is open and $f:U\longrightarrow F$ is differentiable at $p\in U$. We denote by $f'(p)$ the element $\DD f(p)(1)\in F$.
    $$f(p+h)-f(p)-\DD f(p)(h)=o\left(\pl h\pl_E\right).$$ 
    So
    $$f(p+h)-f(p)-hf'(p)=o\left(\pl h\pl_E\right),$$
    $$\frac{f(p+h)-f(p)}{h}-f'(p)=o(1).$$
    That is,
    $$\lim_{h\rightarrow 0}\frac{f(p+h)-f(p)}{h}=f'(p).$$
\end{exampleenv}
\begin{theoremenv}
    Let $(E,\pl\cdot\pl_E)$, $(F,\pl\cdot\pl_F)$, $(G,\pl\cdot\pl_G)$ be normed vector spaces over a complete valued field $(K,\left|\ \cdot\ \right|)$. Let $U\subseteq E$ and $V\subseteq F$ be open subsets, $f:U\longrightarrow F$ and $g:V\longrightarrow G$ be mappings such that $f(U)\subseteq V$. Let $p\in U$. If $f$ is differentiable at $p$ and $g$ is differentiable at $f(p)$, then $g\circ f:U\longrightarrow G$ is differentiable at $p$ and
    $$\DD(g\circ f)(p)(h)=\DD g(f(p))\left( \DD f(p)(h)\right).$$
\end{theoremenv}
\begin{proofenv}
    $$f(p+h)-f(p)-\DD f(p)(h)=o\left(\pl h\pl_E\right),$$
    so, 
    $$f(p+h)-f(p)=\mathcal{O}\left(\pl h\pl_E\right).$$
    \begin{align*}
        &g(f(p+h))-g(f(p))-\DD g(f(p))\left(f(p+h)-f(p)\right)\\
        =&o\left(\pl f(p+h)-f(p)\pl_F\right)=o\left(\mathcal{O}\pl h\pl_E\right)=o\left(\pl h\pl_E\right).
    \end{align*}
    \begin{align*}
        &\DD g(f(p))\left(f(p+h)-f(p)\right)-\DD g(f(p))\left(\DD f(p)(h)\right)\\
        =& \DD g(f(p))\left(f(p+h)-f(p)-\DD f(p)(h)\right)\\
        =& \mathcal{O}\left(o\left(\pl h\pl_E\right)\right)=o\left(\pl h\pl_E\right).
    \end{align*}
    So, 
    $$g(f(p+h))-g(f(p))-\DD g(f(p))\left(\DD f(p)(h)\right)=o\left(\pl h\pl_E\right).$$
\end{proofenv}
\begin{remark}
    If $(E,\pl\cdot\pl_E)=(K,\left|\ \cdot\ \right|)$, 
    $$(g\circ f)'(p)=\DD g(f(p))(f'(p)).$$
    If $E=F=K$, $\pl\cdot\pl_E=\pl\cdot\pl_F=\left|\ \cdot\ \right|$.
    $$(g\circ f)'(p)=g'(f(p))\cdot f'(p).$$ 
\end{remark}
\begin{remark}
    Let $U\subseteq E$ be open. $f:U\longrightarrow F_1\times \dots\times F_n$. If $f$ is differentiable at $p\in U$, for any $i\in\{1,\ldots,n\}$, the mapping
    $$f_i\coloneq\pi_i\circ f:U\longrightarrow F_i$$
    is differentiable at $p$ and
    $$\DD(f_i)(p)(h)=\DD \pi_i(f(p))\left(\DD f(p)(h)\right)=\pi_{i}\left(\DD f(p)(h)\right).$$
\end{remark}


\section{Multilineal Mappings}
\begin{definitionenv}
    Let $K$ be a commutative unitary ring. Let $E_1,\ldots,E_n;F$ be $K$-modules. We say that 
    $$\varphi:E_1\times\ldots\times E_n\longrightarrow F$$
    is $n$-linear if for any $i\in\{1,\ldots,n\}$ and any $(x_1,\ldots,x_{i-1},x_{i+1},\ldots,x_n)\in E_1\times\ldots\times E_{i-1}\times E_{i+1}\times\ldots\times E_n$, the mapping
    $$E_i\longrightarrow F,\ x_i\mapsto \varphi(x_1,\ldots,x_{i-1},x_i,x_{i+1},\ldots,x_n)$$
    is a homomorphism of $K$-modules. ($K$-linear mapping)
    \newline
    If $n=1$, $1$-linear is also called linear.
    \newline
    If $n=2$, $2$-linear is also called bilinear.
\end{definitionenv}
\begin{exampleenv}
    \ \newline
    (1) $K\times K\longrightarrow K$ $(a,b)\longmapsto ab$ is bilinear.
    \newline
    (2) $K^n\times K^n\longrightarrow K$ $(x,y)\longmapsto x\cdot y=\sum_{i=1}^n x_i y_i$ is bilinear.
    \newline
    (3) $K\times \ldots \times K\longrightarrow K$ $(x_1,\ldots,x_n)\longmapsto x_1\cdots x_n$ is $n$-linear.
\end{exampleenv}
\begin{definitionenv}
    We denote by $\mathrm{Hom}_{K}^{(n)} (E_1\times\ldots\times E_n,F)$ the set of $n$-linear mappings from $E_1\times\ldots\times E_n$ to $F$.
\end{definitionenv}
\begin{definitionenv}
    Let $(K,\left|\ \cdot\ \right|)$ be a complete valued field. 
    \newline
    Let $(E_1,\pl\cdot\pl_{E_1}),\ldots,(E_n,\pl\cdot\pl_{E_n}),(F,\pl\cdot\pl_F)$ be normed vector spaces over $K$. For any $\varphi\in \mathrm{Hom}_{K}^{(n)}(E_1\times\ldots\times E_n,F)$, we define
    $$\pl \varphi\pl \coloneq \sup_{\substack{x_i\in E_i\backslash\{0\}\\i\in\{1,\ldots,n\}}}\frac{\pl \varphi(x_1,\ldots,x_n)\pl_F}{\pl x_1\pl_{E_1}\cdots \pl x_n\pl_{E_n}}.$$
    We denote by $\mathscr{L}(E_1\times\dots\times E_n,F)$ the set
    $$\{\varphi\in \mathrm{Hom}_{K}^{(n)}(E_1\times\dots\times E_n,F)\mid \pl \varphi\pl<+\infty\}.$$
    $\mathscr{L}^{(n)}(E_1\times\ldots\times E_n,F)$ is a normed vector space of $\mathrm{Hom}_{K}^{(n)}(E_1\times\ldots\times E_n,F)$, and the norm is $\pl\cdot\pl$. 
\end{definitionenv}
\begin{theoremenv}
    Let $(E_1,\pl\cdot\pl_{E_1}),\ldots,(E_n,\pl\cdot\pl_{E_n}),(F,\pl\cdot\pl_F)$ be normed vector spaces over $K$.
    Let $\varphi\in \mathscr{L}^{(n)}(E_1\times\ldots\times E_n,F)$. For any $p=(p_1,\ldots,p_n)\in E_1\times \ldots\times E_n$, $\varphi$ is differentiable at $p$ and
    $$\DD \varphi(p)(h_1,\ldots,h_n)=\sum_{i=1}^n \varphi(p_1,\ldots,p_{i-1},h_i,p_{i+1},\ldots,p_n).$$
\end{theoremenv}
\begin{proofenv}
    \begin{align*}
        \varphi(p+h)-\varphi(p)=\sum_{i=1}^n & \varphi(p_1+h_1,\ldots,p_{i-1}+h_{i-1},p_{i}+h_i,p_{i+1},\ldots,p_n)\\
        -&\varphi(p_1+h_1,\ldots,p_{i-1}+h_{i-1},p_{i},p_{i+1},\ldots,p_n)\\
    \end{align*}
    \begin{align*}
        & \varphi(p_1+h_1,\ldots,p_{i-1}+h_{i-1},h_i,p_{i+1},\ldots,p_n)\\
        -& \varphi(p_1,\ldots,p_{i-1},h_i,p_{i+1},\ldots,p_n)\\
        =&\sum_{j=1}^{i-1}\varphi (p_1+h_1,\ldots,p_{j-1}+h_{j-1},h_j,p_{j+1},\ldots,h_i,\ldots,p_n).
    \end{align*}
    \begin{align*}
        &\pl \varphi(p_1+h_1,\ldots,p_{j-1}+h_{j-1},h_j,p_{j+1},\ldots,h_i,\ldots,p_n)\pl_F\\
        \le &\pl \varphi\pl \cdot \prod_{\substack{k=1}}^{j-1} \pl p_k+h_k\pl_{E_k}\cdot\pl h_j\pl_{E_j}\cdot \prod_{k=j+1}^{i-1}\pl p_k\pl_{E_k}\cdot \pl h_i\pl_{E_i} \cdot \prod_{\substack{k=i+1}}^{n}\pl p_k\pl_{E_k}\\
        =&\mathcal{O}\left(\pl h\pl^2\right)=o(h),\ h\rightarrow 0.
    \end{align*}
\end{proofenv}
\begin{definitionenv}
    Let $K$ be a commutative unitary ring. $n\in\NN_{\ge1}$, $E$ and $F$ be $K$-modules. We say that 
    $$\varphi\in \mathrm{Hom}_{K}^{(n)}(E_1\times\ldots\times E_n,F)$$
    is \textbf{symmetric} if
    $$\forall \sigma\in \mathfrak{S}_{\{1,\ldots,n\}},\ \forall (x_1,\ldots,x_n)\in E^n,\ \varphi(x_{\sigma(1)},\ldots,x_{\sigma(n)})=\varphi(x_1,\ldots,x_n).$$
    Let $P:E\longrightarrow F$ be a mapping. If there exists a symmetric $\varphi\in \mathrm{Hom}_{K}^{(n)}(E\times\ldots\times E,F)$ such that
    $$\forall x\in E,\ P(x)=\varphi(x,\ldots,x),$$
    we say that $P$ is a \textbf{homogeneous polynomial mapping of degree $\mathbf{n}$}.

    If $F=K$, $P$ is called a \textbf{homogeneous polynomial} on $E$. The symmetric polynomial mapping $\varphi$ is called the \textbf{polarization} of $P$.
\end{definitionenv}
\begin{propositionenv}
    Let $(K,\left|\ \cdot\ \right|)$ be a complete valued field that is non-trivial. Let $(E,\pl\cdot\pl_E)$ and $(F,\pl\cdot\pl_F)$ be normed vector spaces over $K$. Assume that $P:E\longrightarrow F$ is a homogeneous polynomial mapping of degree $n$. Which admits a bounded polarization $\varphi$. Then $P$ is differentiable on $E$ and,
    $$\forall (x,h)\in E\times E,\ \DD P(x)(h)=n\varphi(x,\ldots,x,h).$$
\end{propositionenv}
\begin{proofenv}
    Let $$ \begin{array}{rrcl}
            \Delta:&E&\longrightarrow& E^n,\\
             &x&\longmapsto& (x,\ldots,x).
        \end{array}$$
        Then $P=\varphi\circ \Delta$. Since $\varphi$ and $\Delta$ are differentiable, so it is $P$. 

    Moreover,
    \begin{align*}
        \DD P(x)(h)&=\DD \varphi(\Delta(x))\left(\DD \Delta(x)(h)\right)\\
        &=\DD \varphi(x ,\ldots,x)(h,\ldots,h)\\
        &=\sum_{i=1}^n \varphi(x,\ldots,x,h,x,\ldots,x)\\
        &=n\varphi(x,\ldots,x,h).
    \end{align*}
\end{proofenv}
\begin{remark}
    Assume that $E=K$. Let $P:K\longrightarrow F$ be a homogeneous polynomial mapping of degree $n$ of form $P(x)=x^n s$, where $s\in F$. Its polarization is of the form 
    $$\varphi(a_1,\ldots,a_n)=a_1\cdots a_n s.$$
    $$P'(x)=\DD P(x)(1)=n\varphi(x,\ldots,x,1)=n x^{n-1} s.$$
\end{remark}
\begin{propositionenv}
    Let $n$ be a positive integer $n\ge 2$. Let $(E_1,\pl\cdot\pl_1),\ldots,(E_n,\pl\cdot\pl_n), (F,\pl\cdot\pl_F)$ be normed vector spaces. For any $i\in\{1,\ldots,n_1\}$, the mapping
    $$\mathscr{L}^{(n)}(E_1,\ldots,E_n;F)\overset{f}{\longrightarrow}\mathscr{L}^{(i)}(E_1,\ldots,E_i,\mathscr{L}^{(n-i)}(E_{i+1},\ldots,E_n;F))$$
    $$\varphi\longmapsto \left(\substack{E_1\times \ldots e_n\longrightarrow\mathscr{L}^{(i)}(E_{i+1},\ldots,E_n;F)\\ (x_1,\ldots,x_i)\longmapsto\left(\substack{(x_{i+1},\ldots,x_n)\longmapsto \varphi(x_1,\ldots,x_n)\\E_{i+1}\times\ldots\times E_n\in F}\right)}\right)$$
    is an isomorphism of vector spaces over $K$, and in the same time an isometry, ($\pl f(\varphi)\pl=\pl\varphi\pl$).
\end{propositionenv}
\begin{remark}
    \ \newline
    $\forall (x_1,\ldots,x_n)\in E_1\times\ldots\times E_n$, $f(\varphi)(x_1,\ldots,x_i)(x_{i+1},\ldots,x_n)=\varphi(x_1,\ldots,x_n)$
\end{remark}
\begin{proofenv}
    $\forall (x_1,\ldots,x_n)\in E_1\times\ldots\times E_n$, 
    \begin{align*}
        \varphi(x_1,\ldots,x_n):&E_{i+1}\times\ldots\times E_n\longrightarrow F \text{ is bounded }\\
        &(x_{i+1},\ldots,x_n)\longmapsto \varphi(x_1,\ldots,x_n)
    \end{align*}
    Since 
    $$\pl\varphi(x_1,\ldots,x_n)\pl_F\le \left(\pl\varphi\pl\cdot\pl x_1\pl\ldots\pl x_i\pl\right)\pl x_{i+1}\pl\ldots\pl x_n\pl.$$
    \begin{align*}
        \pl f(\varphi)\pl&=\sup_{x_j\in E_j\backslash\{0\},j=1,\ldots,i}\frac{\pl \varphi(x_1,\ldots,x_i,\cdot)\pl}{\pl x_1\pl_{E_1}\cdots \pl x_n\pl_{E_i}}\\
        &=\sup_{x_j\in E_j\backslash\{0\},j=1,\ldots,i}\sup_{x_k\in E_k\backslash\{0\},k=i+1,\ldots,n}\frac{\pl \varphi(x_1,\ldots,x_n)\pl_F}{\pl x_1\pl_{E_1}\cdots \pl x_n\pl_{E_n}}\\
        &=\pl \varphi\pl.
    \end{align*}
    Hence $f$ is injective. ($\ker(f)=\{0\}$)

    For any $\psi\in \mathscr{L}^{(i)}(E_1,\ldots,E_i,\mathscr{L}^{(n-i)}(E_{i+1},\ldots,E_n))$,
    $$\begin{array}{rrcl}
        \varphi:&E_1\times\ldots\times E_n&\longrightarrow &F\\
        &(x_1,\ldots,x_n)&\longmapsto &\psi(x_1,\ldots,x_i)(x_{i+1},\ldots,x_n)
    \end{array}$$
    belongs to $\mathscr{L}^{(n)}(E_1,\ldots,E_n;F)$ and $f(\varphi)=\psi$. So $f$ is surjective.
\end{proofenv}
\begin{corollaryenv}
    If $E_1,\ldots,E_n$ are all finite dimensional, then 
    $$\mathscr{L}^{(n)}(E_1,\ldots,E_n;F)=\mathrm{Hom}_{K}^{(n)}(E_1\times\ldots\times E_n,F).$$ 
\end{corollaryenv}
\begin{proofenv}
    If $n=1$, $\mathscr{L}(E_1,F)=\mathrm{Hom}_{K}(E_1,F)$.
    \begin{align*}
        \mathscr{L}^{(n)}(E_1,\ldots,E_n;F)
        \cong &\mathscr{L}(E_1,\mathscr{L}^{(n-1)}(E_2,\ldots,E_n;F))\\
        = &\mathrm{Hom}_{K}(E_1,\mathrm{Hom}_{K}^{(n-1)}(E_2\times\ldots\times E_n,F))\\
        \cong &\mathrm{Hom}_{K}^{(n)}(E_1\times\ldots\times E_n,F).
    \end{align*}
\end{proofenv}

Let $(K,\left|\ \cdot\ \right|)$ be a complete nontrivial valued field. Let $(E,\pl\cdot\pl_E)$ and $(F,\pl\cdot\pl_F)$ be normed vector spaces over $K$.
\begin{definitionenv}
    Let $U\subseteq E$ be an open subset of $E$, $f:U\longrightarrow F$ be a mapping. 

    If $f$ is continuous on $U$, we say that $f$ is \textbf{of class $\mathcal{C}^0$} and we denote by 
    $$\DD^0f$$
    the mapping $f:U\longrightarrow F$. Denote by 
    $$\mathcal{C}^0(U,F)$$
    the set of mappings from $U$ to $F$.
    $$U\overset{(f,g)}{\longrightarrow}K\times K\overset{\times}{\longrightarrow}K$$
    $$p\longmapsto (f(p),g(p))\longmapsto f(p)\times g(p)$$
    Let $p\in U$. If $f$ is differentiable on an open neighborhood $V$ of $p$ such that $V\subseteq U$. Then
    \begin{align*}
        \DD f:&V\longrightarrow \mathscr{L}(E,F)\\
        &x\longmapsto \DD f(x)
    \end{align*} 
    is a mapping. If $\DD f$ is $(n-1)$-times differentiable at $p$, we say that $f$ is \textbf{of class $\mathcal{C}^n$} at $p$. If $f$ is of class $\mathcal{C}^n$ at every point of $U$, we say that $f$ is \textbf{$\mathbf{n}$-times differentiable} at $p$. We denote by
    $$\DD^n f(p)\in\mathscr{L}^{(n)}(E,\ldots,E,F)$$
    the $n$-linear mapping that sends $(h_1,\ldots,h_n)\in E^n$ to
    $$\DD^{n-1} (\DD f)(p)(h_1,\ldots,h_{n-1})(h_n)\in F.$$
\end{definitionenv}
\begin{remark}
    $$\DD^n f(p)(h_1,\ldots,h_n)=\DD^{i} (\DD^{n-i} f)(p)(h_1,\ldots,h_{i})(h_{i+1},\ldots,h_n).$$
\end{remark}
\newpage
\section{Convexity}
\begin{definitionenv}
    Let $E$ be a vector space over a field $K$. $S\subseteq E$ be a non-empty subset. 

    We call affine combination of elements of $S$ any element of $E$ of the form
    $$a_1s_1+a_2s_2+\cdots+a_ns_n,$$
    where $n\in\NN_{\ge1}$, $s_1,\ldots,s_n\in S$, $a_1,\ldots,a_n\in K$ such that
    $$a_1+a_2+\cdots+a_n=1.$$
    We denote by $\mathrm{Aff}(S)$ the set of all affine combinations of elements of $S$. One has $S\subseteq\mathrm{Aff}(S)$. $\mathrm{Aff}(S)$ is called the affine hull of $S$.

    If $S=\Aff(S)$, we say that $S$ is an affine subspace of $E$.
\end{definitionenv}
\begin{propositionenv}
    \ \newline
    (1) If $F$ is a vector subspace of $E$, $\forall p\in E$, $$p+F=\{p+x\mid x\in F\}$$ is an affine subspace of $E$.
    \newline
    (2) If $A\subseteq E$ is an affine subspace of $E$. For any $p\in A$, 
    $$A-p\coloneq \{x-p\mid x\in A\}$$
    is a vector subspace of $E$, which is not dependent on the choice of $p$. We call it the vector space \textbf{associated} with $A$.
\end{propositionenv}
\begin{proofenv}
    \ \newline
    (1) Let $(x_1,\ldots,x_n)\in F^n$, $(a_1,\ldots,a_n)\in K^n$, such that $\sum_{i=1}^n a_i=1$. Then
    \begin{align*}
        \sum_{i=1}^n a_i (p+x_i)&=p\cdot \sum_{i=1}^n a_i + \sum_{i=1}^n a_i x_i\\
        &=p+\sum_{i=1}^n a_i x_i\in p+F.
    \end{align*}
    (2) Let $(x_1,\ldots,x_n)\in A^n$, $(b_1,\ldots,b_n)\in K^n$.
    \begin{align*}
        \sum_{i=1}^n b_i (x_i - p)&=\sum_{i=1}^n b_i x_i - \left(\sum_{i=1}^n b_i\right) p\\
        &=\left(\sum_{i=1}^n b_i x_i+ \left(1-\sum_{i=1}^n b_i\right)p\right)   - p\\
        &\in A - p.
    \end{align*}
    Let $q\in A$, $\forall x\in A$, $x-p=(x-q)+(q-p)\in A-q$. So $A-p\subseteq A-q$. By symmetry, $A-q\subseteq A-p$. Hence $A-p=A-q$.
\end{proofenv}
\begin{exampleenv}
    Let $A$ be an $m$ by $p$ matrix with coefficients in $\RR$. Let $(b_1,\ldots,b_n)\in E^m$. Consider the linear equation
    $$ A\begin{pmatrix}
x_1\\
\vdots\\
x_p
\end{pmatrix}=\begin{pmatrix}
b_1\\
\vdots\\
b_m
\end{pmatrix}.$$
    The solution set is 
    $$S\coloneq \{(x_1,\dots,x_p)\in E^p\mid A \begin{pmatrix}
x_1\\
\vdots\\
x_p
\end{pmatrix}=\begin{pmatrix}
b_1\\
\vdots\\
b_m
\end{pmatrix}\}.$$
Claim: $S$ is an affine subspace of $E^p$.
\begin{proofenv}
    Let $\underline{x}^{(1)},\dots, \underline{x}^{(n)}$ be elements of $S$, where
    $\underline{x}^{(i)}=(x_1^{(i)},\dots,x_p^{(i)}).$
    Let $(a_1,\ldots,a_n)\in \RR^n$, $\underline{x}=a_1x^{(1)}+\dots+a_nx^{(n)}$.
    $$A\begin{pmatrix}
        x_1\\
        \vdots\\
        x_p
    \end{pmatrix}=A\left(a_1\begin{pmatrix}
        x_1^{(1)}\\
        \vdots\\
        x_p^{(1)}
    \end{pmatrix}+\dots+a_n\begin{pmatrix}
        x_1^{(n)}\\
        \vdots\\
        x_p^{(n)}
    \end{pmatrix}\right).$$
    $$a_1 A\begin{pmatrix}
        x_1^{(1)}\\
        \vdots\\
        x_p^{(1)}
    \end{pmatrix}+\dots a_n A\begin{pmatrix}
        x_1^{(n)}\\
        \vdots\\
        x_p^{(n)}
    \end{pmatrix}=(a_1+\dots+a_n)\begin{pmatrix}
        b_1\\
        \vdots\\
        b_n
    \end{pmatrix}=\begin{pmatrix}
        b_1\\
        \vdots\\
        b_n
    \end{pmatrix}.$$
    $$x_j=a_1x_j^{(1)}+\dots+a_n x_j^{(n)}.$$
\end{proofenv}
\end{exampleenv}
\begin{propositionenv}
    Let $S\subseteq E$. Then $\Aff(S)$ is the smallest affine subspace of $E$ containing $S$.
\end{propositionenv}
\begin{proofenv}
    \ \newline
    Let $A\subseteq E$ be an affine subspace containing $S$. $\forall n\in\NN_{\ge1}$, $\forall (x_1,\dots,x_n)\in S^n\subseteq A^n$, $(a_1,\dots,a_n)\in \RR$, $a_1+\dots+a_n=1$, one has
    $$\sum_{i=1}^n a_i x_i\in A.$$
    So $\Aff(S)\subseteq A$.

    \quad To show that $\Aff(S)$ is an affine subspace containing $S$, it is sufficient to check that $\Aff(S)$ is an affine subspace.

    \quad If $S=\varnothing$, then $\Aff(S)=\varnothing$. It is an affine subspace.

    Suppose that $S\neq \varnothing$, $p\in S$. We prove that $\Aff(S)-p$ is equal to $\mathrm{Span}_\RR (S-p)$. 
    
    Let $y=a_1x_1+\dots+a_nx_n\in \Aff(S)$.
    $$y-p=a_1(x_1-p)+\dots+a_n(x_n-p)\in \mathrm{Span} _\RR (S-p).$$
    Let $(x_1,\dots,x_n)\in S^n$, $(b_1,\dots,b_n)\in \RR^n$.
    $$\sum_{i=1}^n b_i (x_i - p)=\left(\sum_{i=1}^n b_i x_i+ \left(1-\sum_{i=1}^n b_i\right)p\right)   - p\in \Aff(S)-p.$$
\end{proofenv}
\begin{definitionenv}
    Let $S\subseteq E$. We call \textbf{convex combination} of elements of $S$ any element of $E$ of the form
    $$a_1s_1+a_2s_2+\cdots+a_ns_n,$$
    where $n\in\NN_{\ge1}$, $s_1,\ldots,s_n\in S$, $a_1,\ldots,a_n\in \RR_{\ge 0}$ such that
    $$a_1+a_2+\cdots+a_n=1.$$
    We denote by $\mathrm{Conv}(S)$ the set of all convex combinations of elements of $S$. $\mathrm{Conv}(S)$ is called the \textbf{convex hull} of $S$. One has $S\subseteq \mathrm{Conv}(S)\subseteq \mathrm{Aff}(S)$.
\end{definitionenv}
\begin{propositionenv}
    Let $E$ be a vector space over $\RR$ and $C\subseteq E$. Then $C$ is convex if and only if
    $$\forall (x,y)\in C^2,\  \forall \lambda\in [0,1],\  \lambda x+(1-\lambda)y\in C.$$
\end{propositionenv}
\begin{proofenv}
    It is sufficient to check ``$\Leftarrow$''. We prove by induction on $n$ that 
    $$\forall n\in \NN_{\ge1},\ \forall (x_1,\ldots,x_n)\in C^n,\ \forall (a_1,\ldots,a_n)\in \RR^n_{\ge 0},\ \sum_{i=1}^n a_i=1,\ \sum_{i=1}^n a_i x_i\in C.$$
    The case where $n=1$ is trivial. The case where $n=2$  comes from the hypothesis. 

    Suppose $n\ge 3$ in assuming that the statement holds for any integer less than $n$. If $a_n=1$, then $a_1=\dots=a_{n-1}=0$, so $\sum_{i=1}^n a_i x_i=x_n\in C$. If $a_n< 1$, we have $a_1+\dots+a_{n-1}=1-a_n>0$. By the induction hypothesis,
    $$x\coloneq \sum_{i=0}^{n-1} \frac{a_i}{1-a_n} x_i\in C.$$
    Taking $y=x_n$, $t=1-a_n$,
    $$C\ni tx+(1-t)y=\sum_{i=1}^{n} a_ix_i.$$
\end{proofenv}

\section{Mean Value Theorems}
\begin{theoremenv}[Mean Value Inequality]
    Let $(F,\pl\cdot\pl_F)$ be normed vector spaces over $\RR$. Let $(a,b)\in \RR^2$ such that $a<b$. Let $f:[a,b]\longrightarrow F$ be a continuous mapping that is differentiable on $\interval[open]{a}{b}$. Then 
    $$\pl f(b)-f(a)\pl_F\le  (b-a)\cdot \sup_{t\in \interval[open]{a}{b}}\pl f'(t)\pl_F.$$
    
\end{theoremenv}
\begin{proofenv}
    We may suppose that $\dis \sup_{t\in \interval[open]{a}{b}}\pl f'(t)\pl_F<+\infty$. Take
    $$M>\sup_{t\in \interval[open]{a}{b}}\pl f'(t)\pl_F.$$
    Let $m=\frac{a+b}{2}$. Let
    $$J=\{x\in[m,b]\mid \forall t\in [m,x], \pl f(t)-f(m)\pl_{F} \le M(t-m)\}.$$
    It is an interval containing $m$. So it is of the form 
    \[\interval[open right]{m}{c} \text{ or } [m,c]
    \]
    $$\forall t\in \interval[open right]{m}{c},\ \pl f(t)-f(m)\pl_{F}\le M(t-m).$$
    Taking the limit $t<c$, $t\rightarrow c$, we get $c\in J$. So $J=[m,c]$. We then check $c=b$.

    \quad If $c\neq b$, then $c\in \interval[open]{a}{b}$, so $f$ is differentiable at $c$. That is 
    $$\pl f(c+h)-f(c)\pl_{F}=\pl f'(c)h+o(\pl h \pl )\pl_{F}\le \pl f'(c)\pl_{F}h+o(\pl h\pl ),\ h\rightarrow 0.$$ 
    Since $M>\pl f'(c)\pl_F$, $\exists h_{0}>0$ such that 
    $$\forall h\in \interval[open left]{0}{h_0},\ \pl f(c+h)-f(c)\pl_{F}\le Mh.$$
    \begin{align*}
        \pl f(c+h)-f(m)\pl&\le \pl f(c+h)-f(c)\pl +\pl f(c)-f(m)\pl \\
        &\le M h +M(c-m)=M(c+h-m).
    \end{align*}
    So $[m,c+h_0]\subseteq J$, contradiction. Thus $b=c$. $\pl f(b)-f(m)\pl_F\le M(b-m).$

    \quad By the same reason, $\pl f(m)-f(a)\pl_F\le M(m-a)$. So
    $$\pl f(b)-f(a)\pl_F\le \pl f(b)-f(m)\pl_F+\pl f(m)-f(a)\pl_F\le  M(b-a).$$
    Taking the limit when $\dis M\rightarrow \sup_{t\in \interval[open]{a}{b}}\pl f'(t)\pl_F$, we get the announced result.
\end{proofenv}
\begin{corollaryenv}
    Let $(E,\pl \cdot\pl_{E})$ and $(F,\pl\cdot\pl_F)$ be normed vector spaces over $\RR$. $U\subseteq E$ be an open subset, and $(x,y)\in U^2$ such that 
    $$[x,y]=\{tx+(1-t)y\mid t\in [0,1]\}\subseteq U.$$
    Let $f:U\longrightarrow F$ be a differentiable mapping. Then
    $$\pl f(x)-f(y)\pl_F \le \left(\sup_{z\in \interval[open]{x}{y}}\pl \DD f(z)\pl\right)\cdot \pl x-y\pl_E .$$ 
\end{corollaryenv}
\begin{proofenv}
    Let 
    $$\begin{array}{rrcl}
        g:&[0,1]&\longrightarrow& U\\
        &t&\longmapsto& tx+(1-t)y.
    \end{array}$$
    $$g(0)=x,\ g(1)=y,\ g'(t)=x-y.$$
    Then, 
    $$\left(f\circ g\right)'(t)=\DD f\left(g(t)\right)\left(x-y\right),$$
    $$\DD \left(f\circ g\right)(t)(1)=\DD f\left(g(t)\right)\left(\DD g(t)(1)\right).$$
    By the theorem, 
    \begin{align*}
        \pl f(x)-f(y)\pl_F&= \pl f(g(1))-f(g(0))\pl_F\\
        &\le \sup_{t\in \interval[open]{0}{1}}\pl \DD f\left(g(t)\right)(x-y)\pl_F\\
        &\le \sup_{t\in \interval[open]{0}{1}}\left|\DD f\left(g(t)\right)\right|\cdot \pl x-y\pl_E\\
        &=\sup_{z\in \interval[open]{x}{y}}\pl \DD f(z)\pl\cdot \pl x-y\pl_E.
    \end{align*}

\end{proofenv}
\begin{definitionenv}
    Let $(X,\mathscr{T})$  be a topological space, $p\in X$. Let $U$ be a neighborhood of $p$ and $f: U\longrightarrow \RR$ be a mapping. If there exists a neighborhood $V$ of $p$ such that $p\in V\subseteq U$ and 
    $$\forall x\in V,\ f(p)\ge f(x),$$
    we say that $p$ is a \textbf{local maximum point} of $f$ on $U$.

    \quad If $p$ is a local maximum point or a local minimum point, we say that $p$ is a \textbf{local extremum} of $f$ on $U$.

    \quad If $(E,\pl\cdot\pl_E)$ and $(F,\pl \cdot\pl_F)$ are normed vector spaces. $U\subseteq E$ open, $f:U\rightarrow F$ is differentiable. If $p\in U$ is such that
    $$\DD f(p)=0\in \mathscr{L}(E,F),$$ 
    we say that $p$ is a \textbf{critical point} of $f$.
\end{definitionenv}
\begin{theoremenv}
    Let $(E,\pl\cdot\pl)$ be a normed vector space over $\RR$. $U\subseteq E$ be an open subset, $f:U\longrightarrow \RR$ be a differentiable mapping. If $p\in U$ is a local extremum point of $f$, then  it is a critical point $(\DD f(p)=0)$.
\end{theoremenv}
\begin{proofenv}
    There exists $r>0$ such that $p+B(0,r)\subseteq U$ and 
    $$\left(h\in B(0,r)\right)\longmapsto f(p+h)- f(p)\in \RR$$
    does not change the sign.

    $\forall h\in B(0,r),\ \forall \in [0,1],$
    $$\left(f(p+th)-f(p)\right)\left(f(p-th)-f(p)\right)\ge 0.$$
    Taking the limit when $t\rightarrow 0$, $-\DD f(p)(h)^2\ge 0$. So $\DD f(p)(h)=0.$
\end{proofenv}
\begin{theoremenv}[Rolle]
    Let $(a,b)\in \RR^2$, $a<b$. Let $f:[a,b]\longrightarrow \RR$ be a continuous mapping that is differentiable on $\interval[open]{a}{b}$. If $f(a)=f(b)$, then
    $$\exists t\in \interval[open]{a}{b},\ f'(t)=0.$$
    
\end{theoremenv}
\begin{proofenv}
    If there exists $t$ which is in $\interval[open]{a}{b}$ and is an extremum point of $f$, then $f'(t)=0$.
    Since $[a,b]$ is compact and $f$ is continuous, so $f$ attains its maximum and minimum.
    
    \quad If the extremum points of $f$ are in $\{a,b\}$. Since $f(a)=f(b)$, $f$ is compact, so $f'(t)=0$ on $\interval[open]{a}{b}$.
\end{proofenv}
\begin{theoremenv}[Gronwall inequality]
    Let $(F,\pl\cdot\pl)$ be a normed vector space over $\RR$, $(a,b)\in \RR^2$, $a<b$. Let $f:[a,b] \longrightarrow F$ and $g:[a,b]\longrightarrow \RR$ be differentiable mappings on $\interval[open]{a}{b}$.
    If $\forall t\in \interval[open]{a}{b}$, $\pl f'(t)\pl \le g'(t)$, then 
    $$\pl f(b)-f(a)\pl_F\le g(b)-g(a).$$ 
\end{theoremenv}
\begin{proofenv}
    Let $m\in \interval[open]{a}{b}$. Let $\varepsilon>0$, 
    $$J\coloneq\{t\in[m,b]\mid \forall s\in [m,t],\ \pl f(s)-f(m)\pl_F\le g(s)-g(m)+\varepsilon(s-m)\}.$$
    Since $f$ and $g$ are continuous, $J$ is a closed interval of the form $[m,c]$.
    
    If $c<b$, 
    $$\begin{array}{rl}
        f(c+h)=f(c)+hf'(c)+o(h),&\\
        g(c+h)=g(c)+hg'(c)+o(h),&\ h>0,\ h\rightarrow 0.
    \end{array}$$
    $\exists \delta>0$, such that $[c,c+\delta]\subseteq [c,b]$ and $\forall h\in [0,\delta]$,
    $$\pl f(c+h)-f(c)\pl \le h\pl f'(c)\pl+ \frac{\varepsilon}{2}h.$$
    $$g(c+h)-g(c)\ge hg'(c)-\frac{\varepsilon}{2}h.$$
    So, 
    $$\pl f(c+h)-f(c)\pl\le g(c+h)-g(c)+\varepsilon h.$$
    By the triangle inequality,
    $$\pl f(c+h)-f(m)\pl\le g(c+h)-g(m)+\varepsilon(c+h-m).$$
    So $J\supseteq [m,c+\delta]$, contradiction.

    Therefore $c=b$.
    $$\pl f(b)-f(m)\pl\le g(b)-g(m)+\varepsilon(b-m).$$
    A similar argument shows that
    $$\pl f(m)-f(a)\pl\le g(m)-g(a)+\varepsilon(m-a).$$
    Hence,
    $$\pl f(b)-f(a)\pl\le g(b)-g(a)+\varepsilon(b-a).$$
    $$\pl f(c+h) -f(c) + h f'(c)\pl \le \varphi(h)h,\ \lim_{h\rightarrow 0}\varphi(h)=0.$$
    $$\exists \delta>0,\ \forall h>0,\ 0\le h< \delta\Rightarrow \left| \varphi(h)\right|\le \frac{\varepsilon}{2}.$$
\end{proofenv}
\begin{theoremenv}[Mean value theorem of Lagrange]
    Let $(a,b)\in \RR^2$, $a<b$. Let $f:[a,b]\longrightarrow \RR$ be a continuous mapping that is differentiable on $\interval[open]{a}{b}$. Then 
    $$\exists \xi\in \interval[open]{a}{b},\ f(b)-f(a)=f'(\xi)(b-a).$$
    
\end{theoremenv}
\begin{proofenv}
    Let $g:[a,b]\longrightarrow \RR$.
    $$ g(t)\coloneq f(b)-f(t)+C(b-t), \text{ where } C=-\frac{f(b)-f(a)}{b-a}.$$
    Then $g(a)=g(b)=0$, $g'(t)=-f'(t)-C$. 
    $$\exists \xi\in \interval[open]{a}{b},\ g'(\xi)=0,\ f'(\xi)=-C=\frac{f(b)-f(a)}{b-a}.$$
\end{proofenv}

\begin{theoremenv}[Darboux]
    Let $I$ be an open interval in $\RR$ and $f:I\longrightarrow \RR$ be a differentiable mapping. Then $f'(I)$  is an interval.
\end{theoremenv}
\begin{proofenv}
    Let $a$, $b$ be two elements in $I$ such that $a<b$. Let
    $$\begin{array}{rrcl}
        g:&[a,b]&\longrightarrow& \RR\\
        &t&\longmapsto& \begin{cases}
            \frac{f(t)-f(a)}{t-a},\ t\neq a\\
            f'(a),\ t=a
        \end{cases}
        
    \end{array}$$
    $g$ is continuous, and $g([a,b])$ is an interval. By the mean value theorem of Lagrange, $g([a,b])\subseteq f'(I)$.
    
    Let
    $$\begin{array}{rrcl}
        h:&[a,b]&\longrightarrow& \RR\\
        &t&\longmapsto& \begin{cases}
            \frac{f(t)-f(b)}{t-b},\ t\neq b\\
            f'(b),\ t=b
        \end{cases}
        
    \end{array}$$
    $h([a,b])$ is an interval contained in $f'(I)$. 

    \quad $h([a,b])\cup g([a,b])$ is an interval since 
    $$ \frac{f(b)-f(a)}{b-a}\in h([a,b])\cap g([a,b]),$$
    $$\{ f'(a),f'(b)\}\subseteq h([a,b])\cup g([a,b]).$$
    So the interval liking $f'(a)$, $f'(b)$ is contained in $f'(I)$. Hence, $f'(I)$ is an interval.
\end{proofenv}



\section{Higher Differential}

We fix a complete non-trivially valued field \defm{9.6.1}{$(K,\left|\ \cdot\ \right|)$}. Let $(E,\pl\cdot\pl_E)$ and $(F,\pl\cdot\pl_F)$ be normed vector spaces over $K$.

\begin{definitionenv}
    Let $U\subseteq E$ be an open subset, $f:U\longrightarrow F$ be a mapping, $p\in U$.
    \newline
    (1) If $f$ is continuous at $p$, we say that $f$ is $0$-time differentiable at $p$, and we let 
    $$\DD^0 f(p)\coloneq f(p).$$
    (2) If $f$ is differentiable at $p$, we say that $f$ is $1$-time differentiable at $p$, and we let 
    $$\DD^1 f(p)\coloneq \DD f(p).$$
    (3) Let $n\ge 2$. If exists open neighborhood $V$ of $p$ such that $V\subseteq U$ and $f$ is differentiable on $V$ and $\DD f$ is $n-1$-time differentiable on $V$, we say that $f$ is $n$-time differentiable at $p$, and we let
    $$\DD^n f(p)\in \mathscr{L}(E,\ldots,E,F)$$
    be the multilineal mapping sending $(h_1,\ldots,h_n)\in E^n$ to
    $$\DD^{n-1} (\DD f)(p)(h_1,\ldots,h_{n-1})(h_n).$$
    If $E=K$, $\DD^n f(p)(1,\ldots,1)$ is denoted as $ f^{(n)}(p)\in F$. $f^{(0)}(p)$ is often denoted as $f(p)$. 
\end{definitionenv}
\begin{remark}
    $\forall i\in \{1,\ldots,n\}$,
    $$ \DD^{n} f(p)(h_1,\ldots,h_n) = \DD^{i} (\DD^{n-i} f)(p)(h_1,\ldots,h_{i})(h_{i+1},\ldots,h_n).$$
    If $E=K$, 
    $$f^{(n)}(p)(h_1,\ldots,h_n)=\DD^{n-1} (\DD f)(p)(h_1,\ldots,h_{n-1})(h_n).$$
\end{remark}
\begin{definitionenv}
    Let $X$ be a set, we denote by $\mathfrak{S}_X$ the element of all bijection from $X$ to $X$. $(\mathfrak{S}_X,\circ)$ forms a group.
    The identity mapping $\mathrm{Id}_X$ is the neutral element of $(\mathfrak{S}_X,\circ)$. $(\mathfrak{S}_X,\circ)$ is called the symmetric group of $X$. The elements of $(\mathfrak{S}_X,\circ)$ are called permutations of $X$.

    \quad Let $n\in \NN_{\ge 2}$, $x_1,\ldots,x_n$ be distinct elements of $X$. We denote by $(x_1\ x_2\cdots x_n)$ the element of $\mathfrak{S}_X$ that sends $x_i$ to $x_{i+1}$, $(i\in \{1,\ldots,n-1\})$, $x_n$ to $x_1$, $y\in X\backslash\{x_1,\ldots,x_n\}$ to $y$ itself. This element is called an $n$-cycle.
    A $2$-cycle is also called a transposition.
\end{definitionenv}
\begin{remark}
    $\mathfrak{S}_X$ acts on $X$.
    $$\begin{array}{rcl}
        \mathfrak{S}_X\times X&\longrightarrow &X\\
        (\sigma,x)&\longmapsto& \sigma(x).
    \end{array}$$
    If $\sigma\in\mathfrak{S}_X$, $x\in X$, we denote by $\mathrm{orb}_\sigma(x)$ the set $\{\sigma^n(x)\mid n\in \ZZ\}$.
    $$\left \langle \sigma\right \rangle \coloneq \{\sigma^n\mid n\in \ZZ\}\subseteq \mathfrak{S}_X$$
    is a group. $\mathrm{orb}_\sigma(x)$ is the orbit of $x$ under the action of $\left \langle \sigma\right \rangle$.
\end{remark}
\begin{propositionenv}
    If $\orb_\sigma(x)$ is finite of $d$ elements, then $\sigma^d(x)=x$, and $\orb_\sigma (x)=\{x,\sigma(x),\ldots,\sigma^{d-1}(x)\}$. Moreover, the restriction of $\sigma$ to $\orb_\sigma(x)$ identifies to the restriction of the cycle $(x,\sigma(x),\ldots,\sigma^{d-1}(x))$.
\end{propositionenv}
\begin{proofenv}
    Since $\orb_\sigma(x)$ is finite,
    $$\{(n,m)\in \ZZ^2\mid n<m,\ \sigma^n(x)=\sigma^m(x)\}$$
    Let $$l=\min\{m-n\mid (n,m)\in \ZZ^2\,\ n<m,\ \sigma^n(x)=\sigma^m(x)\}.$$
    Then $x,\sigma(x),\ldots,\sigma^{l-1}(x)$ are distinct, and $\sigma^l(x)=x$.
    $\forall n\in \ZZ$, then $n$ can be written as $n=lp+r$, where $p\in \ZZ$, $r\in \{0,\ldots,l-1\}$.
    $$ \sigma^n(x)=\sigma^r(\sigma^{lp}(x))=\sigma^r((\sigma^{l}\circ\dots\sigma^l)(x))=\sigma^r(x).$$
    So, $\orb_\sigma(x)=\{x,\sigma(x),\ldots,\sigma^{l-1}(x)\}$, ($l=d$).
\end{proofenv}
\begin{remark}
    If $X$ is finite, then $X$ can be written as a distinct union of orbits (under the action of $\left\langle \sigma\right\rangle$). Let $d_i=\#(\mathrm{orb}_\sigma(x_i))$, $i=1,\ldots,n$, then
    $$\sigma|_{\orb_\sigma (x^{(i)})}=\left(x^{(i)},\sigma(x^{(i)}),\ldots,\sigma^{d_i-1}(x^{(i)})\right)|_{\orb_\sigma(x^{(i)})}.$$
    So $\sigma=\tau_1\circ\dots\circ\tau_n$, where $\tau_i\coloneq (x^{(i)},\sigma(x^{(i)}),\ldots,\sigma^{d_i-1}(x^{(i)}))$.
\end{remark}
\begin{corollaryenv}
    Suppose that $X$ is finite. Any $\sigma\in \mathfrak{S}_X$ can be written as a composition of transpositions.
\end{corollaryenv}
\begin{proofenv}
    $$(x_1 \ \dots \ x_n)=(x_1\ x_2)\circ (x_2 \dots \ x_n),$$
    So, $$(x_1 \ \dots \ x_n)=(x_1\ x_2)\circ (x_2\ x_3)\circ \cdots \circ (x_{n-1}\ x_n).$$
\end{proofenv}
\begin{definitionenv}
    Denote by $\mathfrak{S}_n$ the symmetric group $\mathfrak{S}_{\{1,\ldots,n\}}$. A composition of the form $(i \ i+1)$, $i\in \{1,\ldots,n-1\}$  is called an adjacent transposition.
\end{definitionenv}
\begin{corollaryenv}
    Any $\sigma\in \mathfrak{S}_n$ can be written as a composition of adjacent transpositions.
\end{corollaryenv}
\begin{proofenv}
    Let $(j,k)\in\{1,\ldots,n\}^2$, $j<k$,
    $$(j-1\ j)\circ(j\ k)\circ (j-1\ j)=(j-1\ k).$$
    $$(j\ k)=(j \ j+1)\circ (j+1\ j+2)\circ \cdots \circ (k-1\ k)\circ\dots (j\ j+1).$$
\end{proofenv}
\begin{theoremenv}[Schwarz]
    Let $U\subseteq E$ be an open subset, $f: U\longrightarrow F$ be a mapping. $n\in \NN_{\ge 1}$, $p\in U$. Assume that $f$ is $n$-times differentiable at $p$. Then $\forall \sigma\in \mathfrak{S}_{n}$, $\forall (h_1,\ldots h_n)\in E^n$,
    $$\DD^n f(p)(h_1,\ldots,h_n)=\DD^n f(p)(h_{\sigma(1)},\ldots,h_{\sigma(n)}).$$
\end{theoremenv}
\begin{proofenv}[By induction] 
    The case where $n=1$ is trivial.
    Case $n = 2$: Exists $V$ open, $p\in V\subseteq U$. $f$ is differentiable on $V$ and $\DD f$ is differentiable at $p$.
    $$ \DD f(p+h)(\cdot)-\DD f(p)(\cdot) -\DD^2 f(p)(h,\cdot)=o\left(\pl h\pl_E\right).$$
    Let $\varepsilon>0$, $\exists \delta>0$, $\forall h\in E$, $\pl h\pl_E\le 2\delta$ $\Rightarrow p+h\in V$ and
    $$ \pl \DD f(p+h)(\cdot)- \DD f(p)(\cdot) -\DD^2 f(p)(h,\cdot)\pl \le \varepsilon\pl h\pl_E.$$
    Let $h\in E$ such that $\pl h\pl_E\le \delta$. Define $g_h:B(0,\delta)\longrightarrow F$ as
    $$g_h(k)=f(p+h+k)-f(p+h)-f(p+k)+f(p)-\DD^2 f(p)(h,k).$$
    Then,
    \begin{align*}
        \DD g_h(k)(\cdot)=&\DD f(p+h+k)(\cdot)-\DD f(p+k)(\cdot)-\DD^2 f(p)(h,\cdot)\\
        =&\DD f(p+h+k)(\cdot)-\DD f(p)(\cdot)-\DD^2 f(p)(h+k,\cdot)\\
        -&(\DD f(p+k)(\cdot)-\DD f(p)(\cdot)-\DD^2 f(p)(k,\cdot))
    \end{align*}
    $$\pl \DD g_h(k)(\cdot)\pl \le \varepsilon\pl h+ k\pl_E+ \varepsilon\pl k\pl_E\le 3\varepsilon\max\{\pl k\pl_E,\pl h\pl_E\}.$$
    $g_h(0)=0$. Therefore, $\pl g_{h}(k)\pl \le 3\varepsilon\max{\pl k\pl_E,\pl h\pl_E}^2$ (mean value inequality).
    $$\pl g_h(k)- g_h(0)\pl\le \left(\sup_{t\in \interval[open]{0}{1}}\pl \DD g_h(tk)\pl\right)\cdot\pl k\pl .$$
    Therefore, 
    $$ f(p+h+k)-f(p+h)-f(p+k)+f(p)-\DD^2 f(p)(h,k)=o\left(\max\{\pl k\pl_E,\pl h\pl_E\}^2\right).$$
    By symmetry, 
    $$ f(p+h+k)-f(p+h)-f(p+k)+f(p)-\DD^2 f(p)(k,h)=o\left(\max\{\pl k\pl_E,\pl h\pl_E\}^2\right).$$
    $$\DD^2 f(p)(h,k)-\DD^2 f(p)(k,h)=o\left(\max\{\pl h\pl_E,\pl k\pl_h\}^2\right).$$
    $$ \DD^2 f(p)(th,tk)-\DD^2 f(p)(tk,th)=o(\left|t\right|^2),\ t\rightarrow 0.$$
    $$\DD^2 f(p)(h,k)-\DD^2 f(p)(k,h)=o(1),\ t\rightarrow 0.$$

    \quad Suppose $n\ge 3$.
    $$ \DD^n f(p)(h_1,\ldots,h_n)=\DD^{n-1}(\DD f)(p)(h_1,\ldots,h_{n-1})(h_n).$$
    If $\sigma=(j\ j+1)$, $j\le 2$,
    $$\DD^{n-1}(\DD f)(p)(h_1,\ldots,h_{n-1})(h_n)=\DD^{n-1}(\DD f)(p)(h_{\sigma(1)},\ldots,h_{\sigma(n-1)})(h_n)$$
    by the induction hypothesis, if $\sigma=(n-1\ n)$,
    $$\DD^n f(p)(h_1,\ldots,h_n)=\DD^2\left((\DD^{n-2} f)(h_1,\ldots,h_{n-2})(h_{n-1}, h_n)\right)$$
    \begin{align*}
        \DD^n f(p)(h_{\sigma(1)},\ldots,h_{\sigma(n)})&=\DD^n f(p)(h_1,\ldots,h_{n-1})\\
        &=\DD^2\left((\DD^{n-2} f)(h_1,\ldots,h_{n-2})(h_{n}, h_{n-1})\right)\\
        &=\DD^n f(p)(h_1,\ldots,h_{n}).
    \end{align*}
\end{proofenv}


\newpage
\section{Taylor's Formula}
\begin{theoremenv}[Toylor-Young]
    Let $(E,\pl \cdot\pl_E)$ and $(F,\pl \cdot\pl_F)$ be normed vector spaces over $\RR$, $U\subseteq E$ open, $n\in \NN$, $f:U\longrightarrow F$ be a mapping, $p\in U$.
    Suppose that $f$ is $n$-times differentiable at $p$. Then
    $$ f(x)=f(p)+\sum_{k=1}^{n}\frac{1}{k!}\DD^{k} f(p)(x-p,\ldots,x-p)+o(\pl x-p\pl^n),\ x\rightarrow p.$$
\end{theoremenv}
\begin{proofenv}[By induction on $n$]
    \ \newline
    $n=0$, $f(x)=f(p)+o(1)$  follows by continuity of $f$; $n=1$ follows by the differentiability of $f$.

    \quad From $n-1$ to $n$. Let $g:U\longrightarrow F$
    $$g(x)=f(x)-f(p)-\sum_{k=1}^{n}\frac{1}{k!}\DD^k f(p)(x-p,\ldots,x-p).$$
    $g$ is differentiable on an open neighborhood of $p$,
    {$$\DD g(x)(h) =\DD f(x)(h)-\sum_{k=1}^{n}\frac{1}{k!}k \DD^{k}f(p)(x-p,\ldots,x-p,h)$$}
    $$\DD g(x)=\DD f(x)-\sum_{l=0}^{n-1}\frac{1}{l!}\DD^l(\DD f)(x-p,\ldots,x-p)\overset{\text{hyp.}}{=\! =}o(\pl x-p\pl^{n-1}), x\rightarrow p.$$
    So $g(x)=o(\pl x-p\pl^n)$.
    $$ \forall \varepsilon>0,\ \exists\delta>0,\forall x\in B(p,\delta),\ \pl \DD g(x)\pl\le \varepsilon\pl x-p\pl^{n-1}.$$
    $g(p)=0$, so
    $$\pl g(x)-g(p)\pl\le \varepsilon\pl x-p\pl ^{n-1}\cdot\pl x-p\pl = \varepsilon\pl x-p\pl ^n.$$
\end{proofenv}

\begin{theoremenv}[Taylor-Lagrange]
    Let $(a,b)\in \RR^2$, $a<b$. $f:[a,b]\longrightarrow \RR$ be a continuous mapping. Suppose that $f$ is $(n+1)$-times differentiable on $\interval[open]{a}{b}$ and $\forall k\in \{3,\ldots,n\}$, $f^{(k)}:\interval[open]{a}{b}\longrightarrow \RR$   tends to a continuous mapping $[a,b]\longrightarrow \RR$. Then
    $$\exists \xi\in \interval[open]{a}{b}, f(b)-\sum_{k=0}^{n}\frac{(b-a)^k}{k!}f^{(k)}(a)=\frac{f^{(n+1)}(\xi)(b-a)^{n+1}}{(n+1)!}.$$
\end{theoremenv}
\begin{proofenv}
    Let $g:[a,b]\longrightarrow \RR$.
    $$ g(t)\coloneq \sum_{k=0}^{n}\frac{(b-t)^k}{k!}f^{(k)}(t)-C\frac{(b-t)^{n+1}}{(n+1)!}.$$
    Then $g(b)=f(b)$, $\dis g(a)=\sum_{k=0}^{n}\frac{(b-a)^k}{k!}f^{(k)}(a)-C\frac{(b-a)^{n+1}}{(n+1)!}$.
    \begin{align*}
        g'(t)=&\sum_{k=0}^{n}\frac{(b-t)^k}{k!}f^{(k+1)}(t)-\sum_{k=1}^{n}\frac{(b-t)^{k-1}}{(k-1)!}f^{(k)}(t)+C\frac{(b-t)^n}{n!}\\
        =& \frac{(b-t)^n}{n!}f^{(n+1)}(t)+C\frac{(b-t)^n}{n!}.
    \end{align*}
    Take $C$ such that $g(a)=g(b)$. Then by Rolle's theorem,
    $\exists \xi\in \interval[open]{a}{b}, g'(\xi)=0$, $C=-f^{(n+1)}(\xi)$. Then,
    $$g(a)=\sum_{k=0}^{n}\frac{(b-a)^k}{k!}f^{(k)}(a)+\frac{f^{(n+1)}(\xi)}{(n+1)!}+\frac{f^{(n+1)}(\xi)}{(n+1)!}(b-a)^{n+1}=f(b)=g(b).$$
\end{proofenv}

\begin{theoremenv}
    Let $(E,\pl \cdot\pl_E)$ and $(F,\pl \cdot\pl_F)$ be two normed vector spaces over $\RR$, $U\subseteq E$ be an open subset, and $f: U\longrightarrow F$ be a mapping that is $(n+1)$-times differentiable, where $n\in \NN$.
    Let $p\in U$, $h\in E$ such that $\forall t\in [0,1]$, $p+th\in U$. Let 
    $$M=\sup_{t\in [0,1]}\pl \DD^{n+1} f(p+th)\pl.$$
    Then,
    $$\pl f(p+h)-\sum_{k=0}^{n}\frac{1}{k!}\DD^k f(p)(h,\ldots,h)\pl_F \le \frac{M}{(n+1)!}\pl h\pl_E^{n+1}.$$
\end{theoremenv}
\begin{proofenv}
    We define $\phi:[0,1]\longrightarrow F$
    $$ \phi(t)=f(p+th)+\sum_{k=1}^{n}\frac{(1-t)^k}{k!}\DD^k f(p+th)(h,\ldots,h).$$
    $$ \phi(0)=\sum_{k=0}^{n}\frac{1}{k!}\DD^k f(p)(h,\ldots,h),\ \phi(1)=f(p+h).$$
    \begin{align*}
        \phi'(t)=& \DD f(p+h)(h)+ \sum_{k=1}^{n}\frac{(1-t)^k}{k!}\DD^{k+1} f(p+th)(h,\ldots,h)\\
        -& \sum_{k=1}^{n}\frac{(1-t)^{k-1}}{(k-1)!}\DD^k f(p+th)(h,\ldots,h)\\
        =& \sum_{k=0}^{n}\frac{(1-t)^k}{k!}\DD^{k+1} f(p+th)(h,\ldots,h)\\
        -&\sum_{l=0}^{n-1}\frac{(1-t)^{l}}{(l)!}\DD^{l+1} f(p+th)(h,\ldots,h)\\
        =& \frac{(1-t)^n}{n!}\DD^{n+1} f(p+th)(h,\ldots,h).
    \end{align*}
    So, 
    $$ \pl \phi'(t)\pl\le M\pl h\pl_E^{n+1} \frac{(1-t)^n}{n!},\ t\in [0,1].$$
    By Gronwall's inequality,
    $$ \pl \phi(1)-\phi(0)\pl_F \le M\cdot \pl h\pl ^{n+1}\frac{1}{(n+1)!}.$$
\end{proofenv}


\section{Banach Space}
\begin{propositionenv}
    Let $(X,\dd)$ be a metric space and $(x_n)_{n\in \NN}$ be a sequence in $X$. If 
    $$\sum_{n\in \NN}\dd(x_n,x_{n+1})<+\infty,$$
    then $(x_n)_{n\in \NN}$ is a Cauchy sequence.
\end{propositionenv}
\begin{proofenv}
    Let $N\in \NN$. If $(n,m)\in \NN_{\ge N}^2$, $n>m$, by the triangle inequality, 
    $$ \dd(x_m,x_n)\le \sum_{k=m}^{n-1}\dd(x_k,x_{k+1})\le \sum_{k\ge N}\dd(x_k,x_{k+1}).$$
    So, 
    $$ 0\le \sup_{(n,m)\in \NN_{\ge N}} \dd(x_n,x_m)\le \sum_{k\ge N}\dd(x_k,x_{k+1}).$$
    Taking the limit when $N\rightarrow +\infty$, we get 
    $$ \lim_{N\rightarrow +\infty}\sup_{(n,m)\in \NN_{\ge N}^2} \dd(x_n,x_m)=0.$$
    Let $(a_n)_{n\in\NN}\in \RR^\NN$. If $ \left(\sum_{k=0}^{n}a_k\right)_{n\in\NN}$ converges to some $l$ in $\RR$. Then, $l-\sum_{k=0}^{N-1}a_k$ converges to $0$. If $a_k\le 0$ for any $k\in\NN$, $\dis l-\sum_{k=0}^{N-1}a_k=\sum_{k=N}^{+\infty}a_k$.
    $$ l-\sum_{k=0}^{N-1}a_k=\lim_{n\rightarrow +\infty}\left(\sum_{k=0}^{n}a_k-\sum_{k=0}^{N-1}a_k\right)=\lim_{n\rightarrow +\infty}\sum_{k=N}^{n}a_k.$$

\end{proofenv}

\begin{definitionenv}
    Let $(K,\left| \ \cdot\ \right|)$ be a complete valued field and $(E,\pl\cdot\pl)$ be a normed vector space over $K$. If $E$ equipped with the metric
    $$\begin{array}{rcl}
        E\times E &\longrightarrow & \RR_{\ge 0}\\
        (x,y) &\longmapsto & \pl x-y\pl_E.
    \end{array}$$
    is complete, we say that $(E,\pl \cdot\pl )$ is a \textbf{Banach space}.

    Let $(E,\pl\cdot\pl)$ be a Banach space. If $(x_n)_{n\in\NN}$ is a sequence in $E$ such that $\sum_{n\in\NN}\pl x_n\pl<+\infty$, we say that $\sum_{n\in\NN}x_n$ \textbf{converges absolutely}.
\end{definitionenv}
\begin{remark}
    Suppose that $\dis \sum_{n\in \NN}x_n$ converges absolutely. Then $\dis \left(\sum_{k=0}^{n}x_k\right)_{n\in\NN}$ is a Cauchy sequence, since 
    $$ \pl x_n \pl=\pl \sum_{k=0}^{n}x_k- \sum_{k=0}^n x_{k-1}\pl.$$
    So, $\dis \sum_{n\in\NN}x_n$ converges.
\end{remark}
\begin{theoremenv}[Root test of Cauchy]
    Let $(E,\pl \cdot\pl)$ be a Banach space and $(x_n)_{n\in \NN}\in E^\NN$. Let
    $$ r=\limsup_{n\rightarrow +\infty}\pl x_n\pl^{\frac{1}{n}}\in [0,+\infty]$$
    If $r<1$, then $\dis \sum_{n\in \NN}x_n$ converges absolutely.
    \newline
    If $r>1$, then $\dis \sum_{n\in \NN}x_n$ diverges.
\end{theoremenv}
\begin{lemmaenv}
    If a series $\dis \sum_{n\in \NN}x_n$ converges, then $\dis \lim_{n\rightarrow+\infty}\pl x_n\pl =0$.
\end{lemmaenv}
\begin{proofenv}[of lemma]
    $$ \pl x_n\pl =\pl \sum_{k=0}^{n}x_k-\sum_{k=0}^{n-1}x_k\pl .$$
    Since $\dis \sum_{k}^n x_k$ converges to some $l\in E$.
    $$ \lim_{n\rightarrow+\infty}\pl x_n\pl =\lim_{n\rightarrow+\infty}\pl \sum_{k=0}^{n}x_k- \sum_{k=0}^{n-1}x_k\pl =\pl l-l\pl=0.$$
\end{proofenv}
\begin{proofenv}[of theorem]
    If $r>1$, $\exists \beta>1$ such that $r>\beta$. Since $r=\limsup_{n\rightarrow +\infty}\pl x_n\pl ^{\frac{1}{n}}$, $\exists I\subseteq \NN$ infinite such that $\lim_{n\in I,n\rightarrow+\infty}\pl x_n \pl^{\frac{1}{n}}=r$ (Bolzano-Weierstrass).
    $$\exists N\in \NN, \forall n\in I_{\ge N}, \pl x_n\pl^{\frac{1}{n}}\ge \beta.$$
    So, $\pl x_n\pl \ge \beta^n\ge 1$. So $\sum_{n\in \NN}x_n$ diverges.

    If $r<1$, $\exists\alpha\in \interval[open]{0}{1}$,$r<\alpha$. Since $r=\limsup_{n\rightarrow +\infty}\pl x_n\pl ^{\frac{1}{n}}$,
    $$\exists N\in \NN,\ \forall n\ge N,\ \pl x_n\pl^{\frac{1}{n}}\le \alpha, \pl x_n\pl \le \alpha^n.$$
    So,
    $$ \sum_{n\ge N}\pl x_n\pl \le \sum_{n\ge N}\alpha^n=\frac{\alpha^N}{1-\alpha}<+\infty.$$
    Therefore, $\sum_{n\in \NN}x_n$ converges absolutely.
\end{proofenv}
\begin{theoremenv}[Ratio test of D'Alembert]
    Let $(E,\pl \cdot\pl )$ be a Banach space and $(x_n)_{n\in \NN}\in E^\NN$.
    \newline
    (1) If 
    $$ \limsup_{n\rightarrow+\infty}\frac{\pl x_{n+1}\pl}{\pl x_n\pl}<1,$$
    then $\sum_{n\in\NN}x_n$ converges absolutely.
    \newline
    (2) If 
    $$ \limsup_{n\rightarrow+\infty}\frac{\pl x_{n+1}\pl}{\pl x_n\pl}>1,$$
    then $\sum_{n\in\NN}x_n$ diverges.
\end{theoremenv}
\begin{proofenv}
\ \newline
(1) Let $0<\alpha<1$ such that 
$$ \limsup_{n\rightarrow +\infty}\frac{\pl x_{n+1}\pl}{\pl x_n\pl}<\alpha.$$
$$ \exists N\in \NN, \forall x\in \NN_{\ge N}, \pl x_{n+1}\pl \le  \alpha \pl x_n\pl\le \alpha^{n+1-N}\pl x_N\pl.$$
Thus,
$$ \sum_{n\ge \NN}\pl x_n\pl \le \sum_{n\ge \NN}\pl x_N\pl \alpha^{n-N}=\pl x_N\pl \frac{1}{1-\alpha}<+\infty.$$
(2) Let $\beta>1$ such that 
$$ \liminf_{n\rightarrow +\infty}\frac{\pl x_{n+1}\pl }{\pl x_n\pl}>\beta.$$
$$ \exists N\in \NN, x_N\neq 0, \text{ and } \forall n\in \NN_{\ge N},\pl x_{n+1}\pl\ge \beta\pl x_n\pl$$
$$ \forall n\ge N,\pl x_n\pl\ge \beta ^{n-N} \pl x_N\pl \rightarrow +\infty\  (n\rightarrow+\infty)$$
So $\sum_{n\in \NN}x_n$ diverges.
\end{proofenv}
\begin{box2}
    Let $z\in \CC$. The series $\sum_{n\in \NN}\frac{z^n}{n!}$ converges absolutely since
    $$\left|\frac{z^{n+1}/(n+1)!}{z^n/n!}\right|=\frac{ |z| }{n+1}\rightarrow 0\  (n\rightarrow +\infty).$$
    We denote by $\mathrm{e}^z$ this limit.
\end{box2}


\section{Local inversion}
\begin{definitionenv}
     Let $X$ be a topological space and $Y\subseteq X$. If $\overline{Y}=X$, we say that $Y$ is dense.
\end{definitionenv}
\begin{theoremenv}[Baire]
    Let $(X,\dd)$ be a complete metric space. Let $(\Omega_n)_{n\in \NN}$ be a sequence of dense open subset of $X$.
    Let $\dis \Omega=\bigcap_{n\in\NN}\Omega_n$, then $\Omega$ is dense in $X$.
\end{theoremenv}
\begin{proofenv}
    Suppose that $\Omega$ is not dense. Let $x_0\in X\backslash \overline{\Omega}$, exists $\varepsilon>0$ such that $B(x_0,\varepsilon)\subseteq X\backslash \overline{\Omega}$.

    Let $r_0=\varepsilon$. We construct in a recursive way sequence $(x_n)_{n\in \NN}\in X^\NN$ and $(r_n)_{n\in \NN}\in \RR_{\ge 0}^n$ as follows.

    \quad Suppose that $(x_n,r_n)$ is chosen. $B(x_n,r_n)\cap \Omega_{n}\neq \varnothing$. We pick $x_{n+1}\in X$ and $r_{n+1}\le \frac{x_n}{2}$ such that $B(x_{n+1},r_{n+1})\subseteq B(x_n,r_n)\cap \Omega_n$, $\dd(x_{n+1},x_n)<r_n$. $\sum_{n\in \NN}r_n<+\infty$ (ratio test).

    \quad Then the sequence converges to some $l$. For any $n\in\NN$, $x_n\in B(x_0,\varepsilon)$. So $l\in \overline{B}(x_0,\varepsilon)$.

    \quad Moreover, $\forall n\in \NN$, $l\in \overline{B}(x_{n+1},r_{n+1})\subseteq B_{x_n,r_n}\cap \Omega_{n}$. Thus $l\in \cap_{n\in \NN}\Omega_{n}=\Omega$. Contradiction.
\end{proofenv}
\begin{corollaryenv}
    Let $(X,\dd)$ be a non-empty complete metric space and $(Y_n)_{n\in \NN}$ be a family of closed subsets of $X$ such that $X=\bigcup_{n\in \NN}Y_n$. Then exists $n\in \NN$ such that $Y^\circ\neq\varnothing$.
\end{corollaryenv}
\begin{proofenv}
    Let $\Omega_n=X\backslash Y_n$. Suppose that $\forall n\in \NN$, $Y_n^\circ=\varnothing$.
    Then $\overline{\Omega}_n=X\backslash Y_n^\circ=X$. Thus $\Omega\coloneq \bigcap_{n\in \NN}\Omega_n$ is dense in $X$. Namely, $X=\Omega$. So
    $$\varnothing=X\backslash \overline{\Omega}=\left(X\backslash \Omega\right)^\circ=\left(X\backslash \bigcap_{n\in \NN}\Omega_n\right)^\circ=\left(\bigcup_{n\in \NN}Y_n\right)^\circ=X^\circ=X.$$
    Contradiction.
\end{proofenv}

\begin{theoremenv}[Banach]
    Let $(K,\left| \ \cdot\ \right|)$ be a complete non-trivially valued filed, and $E$ be a vector space over $K$. Let $\pl \cdot\pl_1$, $\pl \cdot\pl_2$ be two norms on $E$ such that $(E,\pl\cdot\pl_1)$ and $(E,\pl\cdot\pl_2)$ are both Banach spaces. 

    \quad If $\exists C>0$ such that $\pl \cdot\pl_2\le C\pl\cdot\pl_1$. Then $\pl\cdot\pl_1$ and $\pl \cdot\pl_2$ are equivalent. ($\exists C'>0$, $\pl\cdot\pl_1\le C'\pl\cdot\pl_2$) 
\end{theoremenv}
\begin{proofenv}
    For $x\in E$ and $r>0$. Let
    $$ B_i(x,r)\coloneq \{y\in E\mid \pl y-x\pl_i<r\},\ i=1,2 $$
    $\forall y\subseteq E$, let $\overline{Y}^{\pl\cdot\pl_2}$ be the closure of $Y$ in $(E,\pl\cdot\pl_2)$.
    $$ E=\bigcup_{n\ge 1}B_1(0,n)=\bigcup_{n\ge 1}\overline{B_1(0,n)}^{\pl\cdot\pl_2}.$$
    Hence, $\exists n_0\ge 1$, $p\in E$, $r_0>0$ such that 
    $$ B_2(p,r_0)\subseteq \overline{B_1(0,n_0)}^{\pl\cdot\pl_2}$$
    or equivalently,
    $$ B_2(0,r_0)\subseteq \overline{B_1(-p,n_0)}^{\pl\cdot\pl_2}\subseteq \overline{B_1(0,n_0+\pl p\pl_1)}^{\pl\cdot\pl_2}$$
    since $\forall x\in B_1(-p,n_0)$
    $$ \pl x_\pl_1=\pl x-p+p\pl_1\le \pl x -p\pl+\pl p\pl_1<n_0+\pl p\pl_1.$$
    Let $r_1=n_0+\pl p\pl_1$,
    $$ B_{2}(0,r_0)\subseteq \overline{B_1(0,r_1)}^{\pl\cdot\pl_2}\subseteq B_1(0,r_1)+B_2(0,r_0|a|)$$
    where $a\in K$, $0<|a|<\frac{1}{2}$. 

    In fact, $\forall x\in \overline{B_1(0,r_0)}^{\pl\cdot\pl_2}$, exists sequence $(x_n)_{n\in \NN}\in B_1(0,r_1)^{\NN}$, such that $x_n\rightarrow x$ in $(E,\pl\cdot\pl_2)$, $\exists n\in \NN$, $\pl x_n-x\pl_2<r_0|a|$
    $$ B_2(0,r_0|a|^n)\subseteq  B_1(0,r_1|a|^n)+B_2(0,r_0|a|^{n+1})$$
    Let $y\in B_2(0,r_0)$, we choose $(x_0,y_0)\in B_1(0,r_1)\times B_2(0,r_0|a|)$ such that $y=x_0+y_0$. When $(x_n,y_n)$ si chosen, let $(x_{n+1},y_[n+1])\in B_1(0,r_0|a|^{n+1})\times B_2(0,r_0|a|^{n+2})$, $y_n=x_{n+1}+y_{n+1}$, $y=y_n+\sum_{k=0}^{n} x_k$. So $\sum_{n\in \NN}x_n$ converges to $y$.

    Moreover, $\sum_{n\in\NN}\pl x_n\pl<+\infty$, so it converges in $(E,\pl\cdot\pl_1)$ to some $x$.
    Therefore, $x=y$ since $\pl\cdot\pl_2\le C\pl\cdot\pl_1$. So $\pl y\pl _1\pl x\pl_1\le \sum_{n\in \NN}\pl x_n\pl_1\le \frac{r_1}{1-|a|}$.

    \quad Therefore $B_2(0,r_0)\subseteq B_1(0,\frac{r}{1-|a|})$. So $\pl\cdot\pl_1$ is bounded by a constant times $\pl\cdot\pl_2$.
\end{proofenv}
\begin{propositionenv}
    Let $(E,\pl\cdot\pl_E)$ and $(F,\pl\cdot\pl_F)$ be Banach spaces over a complete non-trivially valued filed $(K,\left|\ \cdot\ \right|)$, and $f:E\longrightarrow F$ be a bounded mapping.
    \newline
    (1) If $f$ is invertible, then $f^{-1}$ is bounded.
    \newline
    (2) If $f$ is surjective, for any $U\subseteq E$ open, $f(U)$ is open in $F$.
\end{propositionenv}
\begin{proofenv}
    \ \newline
    (1) We define a mapping 
    $$\begin{array}{rrcl}
        \pl\cdot\pl_E':& E&\longrightarrow& \RR_{\ge 0}\\
        & x&\longmapsto& \pl f(x)\pl_{F}.
    \end{array}$$
    This is a norm on $E$. In fact, if $\pl x\pl_E'=\pl f(x)\pl_F=0$, then $f(x)=0_F$. So $x=0_E$. Moreover, 
    $$\forall x\in E,\ \pl x\pl'_E=\pl f(x)\pl_F\le \pl f\pl \pl x\pl_E.$$
    So there exists $C>0$ such that $\pl\cdot\pl_E\le C\pl \cdot\pl'_E$. That is, 
    $$ \forall y\in F,\ \pl y\pl_F=\pl f(f^{-1}(y))\pl_F=\pl f^{-1}(y)\pl'_E\ge C^{-1}\pl f^{-1}(y)\pl_E.$$
    So, $\pl f^{-1}\pl \le C$.
    \newline
    (2) Let 
    $$ E_0=\ker(f)=\{x\in E\mid f(x)=0_F\}.$$
    This is a closed vector subspace of $E$. 
    $\pl\cdot\pl_E$ induces by passing to quotient a norm $\pl\cdot\pl_Q$ on $Q\coloneq E/E_0$.
    Let 
    $$\begin{array}{rrcl}
        g:& Q&\longrightarrow& F\\
        &[x]&\longmapsto & f(x).
    \end{array} $$
    This is a $K$-linear bijection.

    \quad If $\alpha \in Q$, 
    $$\forall x\in \alpha,\ \pl g(\alpha)\pl_F=\pl f(x)\pl_F\le \pl f\pl \cdot\pl x\pl_E. $$
    Since $\pl \alpha\pl_Q\coloneq \inf_{x\in \alpha}\pl x\pl_E$, $\pl g(\alpha)\pl_F\le \pl f\pl\cdot\pl \alpha\pl_Q$.
    So $\pl g \pl\le \pl f\pl$. By (1), $g^{-1}$ is bounded (hence is continuous).

    \quad If $V\subseteq Q$ is open, then $g(V)\subseteq F$ is open. 
    Let $U\subseteq E$ be an open subset. Let 
    $$\begin{array}{rrcl}
        \pi:& E &\longrightarrow& Q\\
        & x&\longmapsto &[x].
    \end{array}$$
    Let $x\in U$, $r>0$ such that $B(x,r)\subseteq U$. For any $\alpha\in Q$, if 
    $$ \pl \alpha-[x]\pl_Q=\inf_{y\in \alpha} \pl y-x\pl_E<r,$$
    then, exists $y\in \alpha$ such that $\pl y-x\pl_E<r$.

    Therefore, 
    $$ B([x],r)\subseteq\pi(B(x,r))\subseteq \pi (U).$$
    This means that $\pi (U)$ is open. So $f(U)=g(\pi(U))$ is open.
\end{proofenv}

\begin{definitionenv}
    Let $(E,\pl\cdot\pl_E)$ and $(F,\pl\cdot\pl_F)$ be normed vector space over a complete non-trivially valued field $(K,\left| \ \cdot\ \right|)$, $U\subseteq E$ open, $f:U\longrightarrow F$. If $\forall p\in U$, $f$ is $n$-times differentiable at $p$, and $\DD^n f: U\longrightarrow \mathscr{L}^{(n)}(E,\ldots,E,F)$ is continuous, we say that $f$ is of class $\mathcal{C}^n$.

    \quad If $\forall n\in \NN$, $f$ is $n$-times differentiable on $U$, we say that $f$ is smooth, or of class $\mathcal{C}^\infty$. ($\forall n\in \NN$, $f$ is of class $\mathcal{C}^n$.)
\end{definitionenv}
\begin{propositionenv}
   Let $(E,\pl\cdot\pl_E)$, $(F,\pl\cdot\pl_F)$ and $(G,\pl\cdot\pl_G)$ be normed vector space over a complete non-trivially valued field $(K,\left| \ \cdot\ \right|)$. $U\subseteq E$, $V\subseteq F$ be open subsets, $f: U\longrightarrow V$, $g:V\longrightarrow G$ be mappings. $n\in\NN$.
   \newline
   (1) Let $p\in U$. If $f$ is $n$-times differentiable at $p$ and $g$ is $n$-times differentiable at $f(p)$, then $g\circ f$ is $n$-times differentiable at $p$.
   \newline
   (2) If $f$ is of class $\mathcal{C}^{n}$ on $U$ and $g$ is of class $\mathcal{C}^n$ on $V$, then $g\circ f$ is of class $\mathcal{C}^n$ on $U$.
\end{propositionenv}
\begin{proofenv}[induction on $n$]
    \ \newline
    $n=0$, continuity composition.
    \newline
    $n=1$, differentiability of composition.
    \newline
    $n\ge 2$, 
    $$ \DD (g\circ f)(p)(\cdot)=\DD g(f(p))(\DD f(p)(\cdot))$$
    Let 
    $$\begin{array}{rrcl}
        \Phi:&\mathscr{L}(F,G)\times \mathscr{L}(E,F)&\longrightarrow & \mathscr{L}(E,G)\\
        &(\alpha, \beta)&\longmapsto & \alpha\circ\beta.
    \end{array}$$
    This is a bounded bilinear mapping. $\pl \alpha\circ\beta\pl \le \pl \alpha\pl\cdot\pl\beta\pl$.
    $$(\pl \alpha\circ\beta(h)\pl_G=\pl \alpha(\beta(h))\pl_G\le \pl \alpha\pl\cdot\pl \beta(h)\pl_F\le \pl\alpha\pl\pl \beta\pl\pl h\pl_E)$$
    $\Phi$ is of class $\mathcal{C}^{\infty}$.
    $$ \DD (g\circ f)=\Phi(\DD g\circ f, \DD f).$$
    (2) Since $\DD g$ and $\DD f$ is of class $\mathcal{C}^{n-1}$, we obtain that $\DD (g\circ f)$ is of class $\mathcal{C}^{n-1}$, so $g\circ f$ is of class $\mathcal{C}^{n}$.
    \newline
    (1) If $g$ is $n$-times differentiable at $f(p)$, $\DD g$ is $(n-1)$-times differentiable at $f(p)$. So $\DD g\circ f$ is $(n-1)$-times differentiable at $p$.
    $\DD f$ is $(n-1)$-times differentiable at $p$. So $\DD (g\circ f)$ is $(n-1)$-times differentiable at $p$.
\end{proofenv}
\begin{theoremenv}
    Let $(E,\pl\cdot\pl)$ be a Banach space over a complete non-trivially valued filed $(K,\left| \ \cdot\ \right|)$. Let 
    $$ \GL (E)\coloneq \{\varphi\in \mathscr{L}(E,E)\mid \varphi \text{ is invertible}\}.$$
    This set forms a group under $\circ$.
    \newline
    (1) $\forall \varphi\in \mathscr{L}(E,E)$, if $\pl \varphi\pl<1$, then $\mathrm{Id}_E+\varphi\in \GL(E)$.
    \newline
    (2) $\GL(E)\subseteq \mathscr{L}(E,E)$ is open.
    \newline
    (3) $$\begin{array}{rrcl}
        \iota:& \GL(E)&\longrightarrow &\GL(E)\\
        &\varphi&\longmapsto &\varphi^{-1}
    \end{array}$$ is of class $\mathcal{C}^{\infty}$.
\end{theoremenv}
\begin{proofenv}
    \ \newline
    (1) The series $\sum_{n\in\NN}(-1)^{n}\varphi^{n}$ converges absolutely since $\pl \varphi^n\pl\le \pl \varphi\pl ^n$. 

    Let $\eta$ be the limit of $\sum_{n\in\NN} (-1)^n\varphi^n$. 
    $$ \left(\mathrm{Id}+\varphi\right)\circ \sum_{k=0}^{n}(-1)^k\varphi^k=\mathrm{Id}+(-1)^n\varphi ^{n+1}.$$
    Taking the limit when $n\rightarrow +\infty$, we get $\left(\mathrm{Id}_E+\varphi\right)\circ \eta=\mathrm{Id_E}$. For the same reason, $\eta \circ\left(\mathrm{Id}_E+\varphi\right)=\mathrm{Id}_E$.
    \newline
    (2) If $f\in \GL (E)$, $\forall \varphi\in \mathscr{L}(E,E)$ such that 
    $$\pl \varphi\pl<\frac{1}{\pl f^{-1}\pl},\ f+\varphi=f\circ\left(\mathrm{Id}_E+f^{-1}\circ \varphi\right),\ \pl f^{-1}\circ \varphi\pl \le \pl f^{-1}\pl \cdot\pl \varphi\pl <1.$$
    So $\mathrm{Id}_E+f^{-1}\circ \varphi\in \GL(E)$. Hence $f+\varphi\in \GL(E)$.
    \newline
    (3) Let $f\in \GL(E)$, $\varphi\in \mathscr{L}(E,E)$. $\pl \varphi\pl\le \frac{1}{\pl f^{-1}\pl }$.
    \begin{align*}
        \iota\left(f+\varphi\right)-\iota\left(f\right)=&\left(f+\varphi\right)^{-1}-f^{-1}\\
        =& \left(f\circ\left(\mathrm{Id}_E+f^{-1}\circ\varphi\right)\right)^{-1}-f^{-1}\\
        =& \left(\mathrm{Id}_E+f^{-1}\circ\varphi\right)^{-1}\circ f^{-1}-f^{-1}\\
        =&\sum_{n\in\NN} (-1)^n \left(f^{-1}\circ \varphi\right)^n\circ f^{-1}-f^{-1}\\
        =& -f^{-1}\circ\varphi\circ f^{-1}+o(\pl \varphi\pl)
    \end{align*}
    since 
    \begin{align*}
        &\sum_{n\ge 2}\pl (-1)^n\left(f^{-1}\circ\varphi\right)^n \circ f^{-1}\pl\\
        \le & \sum_{n\ge 2}\pl f^{-1}\pl \cdot \left(\pl f^{-1}\pl\cdot\pl \varphi\pl\right) ^n\\
        =& \pl \varphi\pl ^2 \left( \pl f\pl^3 \cdot\sum_{n\ge 2} \left(\pl f^{-1}\pl \cdot\pl \varphi\pl\right)^{n-2}\right)\\
        =&o(\pl \varphi\pl ).
    \end{align*}
    Let
    $$\begin{array}{rrcl}
        \Phi:& \mathscr{L}(E,E)^3&\longrightarrow& \mathscr{L}(E,E)\\
        & (\alpha,\beta, \gamma)&\longmapsto& \alpha\circ\beta\circ\gamma.
    \end{array}$$
    bounded $3$-linear mapping. 
    $$ \DD \iota(f)(\cdot)=-\Phi \left(\iota(f), \cdot, \iota(f)\right).$$
    By induction, we obtain that $\iota$ is of class $\mathcal{C}^n$ for any $n\in \NN$.
\end{proofenv}

\begin{definitionenv}
    Let $(X,\dd)$ be a metric space, $f: X\longrightarrow X$ be a mapping. If exists $\alpha\in \interval[open]{0}{1}$, such that $f$ is $\alpha$-Lipschitzian, we say that $f$ is a \textbf{contraction}.
\end{definitionenv}
\begin{definitionenv}
    Let $f:X\longrightarrow X$ be a mapping. If $x\in X$ is such that $f(x)=x$, we say that $x$ is a \textbf{fixed point} of $f$.
\end{definitionenv}
\begin{theoremenv}[Banach fixed point theorem]
    Let $(X,\dd)$ be a non-empty complete metric space and $f: X\longrightarrow X$ be a contraction. Then $f$ admits a unique fixed point.
\end{theoremenv}
\begin{proofenv}
    \ \newline
    `` Uniqueness'': Let $\alpha\in \interval[open]{0}{1}$, such that $f$ is $\alpha$-Lipschitzian. If $a$ and $b$ are fixed point of $f$, then $\dd(a,b)=\dd(f(a),f(b))\le \alpha \dd (a,b)$. 
    So $\dd(a,b)=0$, $a=b$.
    \newline
    `` Existence'': Let $x_0\in X$. For any $n\in \NN$, let $x_n=f^n(x_0)$. Then 
    $$\dd(x_n,x_{n+1})=\dd(f(x_{n-1}),f(x_n))\le \alpha \dd(x_{n-1},x_n)\le \dots\le \alpha^n \dd(x_0,x_1).$$
    So 
    $$ \sum_{n\in \NN} \dd (x_n,x_{n+1})\le \sum_{n\in\NN}\alpha^n \dd(x_0,x_1)=\frac{1}{1-\alpha}\dd (x_0,x_1)<+\infty.$$
    Hence $(x_n)_{n\in \NN}$ converges to some $a\in X$.
    $$ \dd(a,f(a))=\lim_{n\rightarrow +\infty}\dd(x_n,f(x_n))=\lim_{n\rightarrow+\infty}\dd (x_n,x_{n+1})=0.$$
    So $a=f(a)$.
\end{proofenv}
\begin{definitionenv}
    Let $(E,\pl\cdot\pl_E)$ and $(F,\pl\cdot\pl_F)$ be normed vector spaces over a complete value filed $(K,\left|\ \cdot\ \right|)$, $U\subseteq E$, $V\subseteq F$ be open subsets, $f:U\longrightarrow V$ be a bijection, $n\in \NN\cup\{\infty\}$.
    If $f$ and $f^{-1}$ are both of class $\mathcal{C}^{n}$, we say that $f$ is a $\mathcal{C}^{n}$-diffeomorphism.
\end{definitionenv}
\begin{theoremenv}
    Let $(E,\pl\cdot\pl_E)$ and $(F,\pl\cdot\pl_F)$ be Banach spaces over $\RR$, $U\subseteq E$ open and $f:U\longrightarrow F$ be a mapping of class $\mathcal{C}^n$ ($n\in \NN\cup \{\infty\}$).
    Let $p\in U$. Suppose that $\DD f(p)\in \mathscr{L}(E,F)$ is invertible. Then there exists a open neighborhood $V$ of $p$ contained in $U$, such that $f|_{V}:V\longrightarrow f(V)$ is a $\mathcal{C}^n$-homeomorphism. Moreover
    $$ \DD f^{-1}(y)=\DD f(f^{-1}(y))^{-1}.$$
\end{theoremenv}
\begin{proofenv}
    By replacing $f$ by 
    $$\tilde{ f}:x\longmapsto \DD f(p)^{-1}\left(f(p+x)-f(p)\right).$$
    We may assume that $E=F$, $p=f(p)=0$, $\DD f(p)=\mathrm{Id}_E$.
    $$ \DD \tilde{f}(0)(h)=\DD f(p)^{-1}(\DD f(p)(h))=h,\DD \tilde{f}(0)=\mathrm{Id}_E.$$
    Let $\mu: U\longrightarrow E$, $\mu(x)=f(x)-x$, $\DD \mu(0)=0$. Since $\DD f$ is continuous, so is $\DD \mu$.
    $$ \exists r>0,\ \forall x \in \overline{B}(0_E,r),\ \pl \DD \mu(x)\pl\le \frac{1}{2}.$$
    So $\mu$ is $\frac{1}{2}$-Lipschitzian on $\overline{B}(0_E,r)$ (mean value inequality).
    $$ \forall (x,y)\in \overline{B}(0_E,r)^2,\ \pl f(x)- f(y)\pl \ge \pl x-y\pl \pl \mu(x)-\mu(y)\pl \ge \frac{1}{2}\pl x-y\pl.$$
    So $f$ is injective on $\overline{B}(0_E,r)$. Let $a\in \overline{B}(0_E,\frac{r}{2})$.
    $$ \forall x\in \overline{B}(0_E,r),\ \pl a-\mu(x)\pl\le \pl a\pl+\pl \mu(x)\pl \le \frac{1}{2}r+\frac{1}{2}r=r.$$
    Let
    $$\begin{array}{rrcl}
        \nu :&\overline{B}(0,r)&\longrightarrow &\overline{B}(0,r)\\
        &x&\longmapsto &a-\mu(x)
    \end{array}$$
    $\nu$ is a contraction. By Banach's fixed point theorem, 
    $$\exists ! g(a)\in \overline{B}(0,r),\ \nu(g(a))=a-\mu(g(a))=a-f(g(a)).$$
    That is $f(g(a))=a$. Let $W=B(0,\frac{r}{2})$, $V=f^{-1}(W)\cap B(0,r)$, $f|_V:V\longrightarrow W$ is a bijection.

    \quad $\forall z\in B(0,r)$, $\DD f(z)=\mathrm{Id}_E+\DD \mu(z)\in \GL(E)$. 
    $$\forall (x,x_0)\in V\times V,y=f(x), y_0=f(x_0), y-y_0=\DD f(x_0)(x-x_0)+o\left(\pl x-x_0\pl\right).$$
    $$ \pl x-x_0 \pl= \pl y-y_0- (\mu(x)-\mu(x_0))\pl \le \pl y-y_0\pl +\frac{1}{2}\pl x-x_0\pl ,$$
    $$ \frac{1}{2} \pl f^{-1}(y)-f^{-1}(y_0)\pl =\frac{1}{2} \pl x-x_0 \pl \le \pl y-y_0\pl. $$
    So,
    $$ \DD f(x_0)(x-x_0)=y-y_0+o(\pl y-y_0\pl),$$
    \begin{align*}
        f^{-1}(y)-f^{-1}(y_0)=&x-x_0=\DD f(x_0)^{-1}(y-y_0)+o(\pl y-y_0\pl )\\
        =&\DD f(f^{-1}(y_0))^{-1}(y-y_0)+o(\pl y-y_0\pl) \\
    \end{align*}
    Thus,
    $$ \DD f^{-1}=\iota\circ \DD f \circ f^{-1}.$$
\end{proofenv}
\begin{propositionenv}
    Let $n\in \NN_{\ge 1}$. Let $(K,\left|\ \cdot\ \right|)$ be a complete valued field, $(E_i,\pl\cdot\pl_i)$, $i\in \{1,\ldots,n\}$ be normed vector spaces over $K$, $(F,\pl\cdot\pl_F)$ be a Banach space over $K$. 
    Then, $(\mathscr{L}^{(n)}(E_1,\ldots,E_n, F),\pl \cdot\pl)$ is a Banach space.
\end{propositionenv}
\begin{proofenv}
    Let $(\varphi_i)_{i\in \NN}$ be a Cauchy sequence in $\mathscr{L}^{(n)}(E_1,\ldots,E_n, F)$. 
    For $N\in \NN$, let
    $$ \varepsilon_N\coloneq \sup_{(i,j)\in \NN_{\ge N}} \pl \varphi_i-\varphi_j \pl,\ \lim_{N\rightarrow +\infty}\varepsilon_N=0.$$
    For any $(x_1,\ldots,x_n)\in E_1\times\dots\times E_n$, and any $(i,j)\in\NN_{\ge N}^2$,
    $$ \pl \varphi_i(x_1,\ldots,x_n)-\varphi_j(x_1,\ldots,x_n)\pl \le \pl \varphi_i-\varphi_j\pl\cdot\prod_{l=1}^{n}\pl x_l\pl_l\le \varepsilon_N\prod_{l=1}^{n}\pl x_l\pl_l.$$
    So $(\varphi_i(x_1,\ldots,x_n))_{i\in \NN}$ is a Cauchy sequence in $F$, hence it converges to some element of $F$, denoted as $\varphi(x_1,\ldots,x_n)$.

    \quad Note that $\varphi$ is a point-wise limit of an $n$-linear mapping, so it is also $n$-linear. 
    \begin{align*}
        \pl \varphi(x_1,\ldots,x_n)\pl_F&=\lim_{i\rightarrow +\infty}\pl \varphi_i(x_1,\ldots,x_n)\pl_F\\
        &\le \limsup_{i\rightarrow +\infty} \pl \varphi_i\pl \cdot\pl x_1\pl_{E_1}\cdots \pl x_n\pl_{E_n}\\
        \le \left(\sup_{i\in \NN}\pl \varphi_i\pl\right)\cdot\pl x_1\pl_{E_1}\cdots \pl x_n \pl_{E_n}
    \end{align*}
    So $\varphi\in \mathscr{L}(E_1,\ldots,E_n, F)$.

    \quad For fixed $N\in \NN$, $\forall (x_1,\ldots,x_n)\in E_1\times \dots E_n$,
    \begin{align*}
        &\pl \varphi(x_1,\ldots,x_n)-\varphi_N(x_1,\ldots,x_n)\pl_F\\
        =& \lim_{n\rightarrow +\infty}\pl \varphi_n(x_1,\ldots,x_n)-\varphi_N(x_1,\ldots,x_n)\pl_F\\
        \le & \varepsilon_N \pl x_1 \pl \cdot\pl x_n\pl.
    \end{align*}
    So $0\le \pl \varphi-\varphi_N\pl\le \varepsilon_N$. By squeeze theorem
    $$ \lim_{N\rightarrow +\infty}\pl \varphi-\varphi_N\pl=0.$$
\end{proofenv}


\newpage
\section{Uniform Convergence}
\begin{definitionenv}
    Let $X$ be a set, $(Y,\mathscr{T})$ be a topological space, $(f_n)_{n\in\NN}$ be a sequence of mappings from $X$ to $Y$. We say that the sequence $(f_n)_{n\in\NN}$ converges \textbf{point-wise} to a mapping $f:X\longrightarrow Y$ if for every $x\in X$ the sequence $(f_n(x))_{n\in\NN}$ converges to $f(x)$.

    \quad Suppose that $(Y,\dd)$ is a metric space. We say that the sequence $(f_n)_{n\in\NN}$ converges \textbf{uniformly} to a mapping $f:X\longrightarrow Y$ if
    $$ \lim_{n\rightarrow +\infty} \sup_{x\in X} \dd(f_n(x),f(x))=0.$$
\end{definitionenv}
\begin{remark}
    Let $f,g: X\longrightarrow Y$ be mappings. 
    $$ \dd(f,g)=\sup_{x\in X}\dd(f(x),g(x))$$
    is a metric. Uniform convergence can be seen as convergence of $(f_n)_{n\in\NN}$ with respect to this metric.
\end{remark}

\begin{theoremenv}
    Let $(X,\mathscr{T}_X)$ be a topological space, $(Y,\dd_{Y})$ be a metric space, $(f_n)_{n\in\NN}$ be a sequence of mappings from $X$ to $Y$ that converges uniformly to a mapping $f:X\longrightarrow Y$. If $\forall n\in \NN$, $f_n$ is continuous at $p\in X$, then $f$ is continuous at $p$.
\end{theoremenv}
\begin{proofenv}
    We will prove that for any $\varepsilon>0$, $f^{-1}(B(f(p),\varepsilon))$ is a neighborhood of $p$.

    Let $n\in\NN$ such that
    $$ \sup_{x\in X}\dd(f_n(x),f(x))<\frac{\varepsilon}{3}.$$
    We claim that 
    $$ f_n^{-1}(B(f_n(x),\frac{\varepsilon}{3}))\subseteq f^{-1}(B(f(x),\varepsilon)).$$
    Let $x$ be a element of $X$ such that 
    $$ \dd (f_n(x),f_n(p))<\frac{\varepsilon}{3}.$$
    One has
    $$ \dd(f(x),f(p))\le \dd(f(x),f_n(x))+\dd(f_n(x),f_n(p))+\dd(f_n(p),f(p))<\varepsilon.$$
\end{proofenv}
\begin{theoremenv}
    Let $(X,\dd_X)$ and $(Y,\dd_Y)$ be two metric spaces, $(f_n)_{n\in\NN}$ be sequence of uniformly continuous mappings from $X$ to $Y$. Suppose that $(f_n)_{n\in\NN}$ converges uniformly to $f:X\longrightarrow Y$. Then $f$ is uniformly continuous.
\end{theoremenv}
\begin{proofenv}
    Let $\varepsilon>0$. There exists $n\in \NN$ such that 
    $$ \sup_{x\in X}\dd(f_n(x),f(x))<\frac{\varepsilon}{3}.$$
    $f_n$ is uniformly continuous, so the exists $\delta>0$
    $$ \forall (x,y)\in X\times X,\ \dd(x,y)<\delta\Rightarrow \dd(f_n(x),f_n(y))<\frac{\varepsilon}{3}.$$
    Therefore, for any $(x,y)\in X\times X$ such that $\dd(x,y)<\delta$,
    $$ \dd(f(x),f(y))\le \dd(f(x),f_n(x))+\dd(f_n(x),f_n(y))+\dd(f_n(y),f(y))<\varepsilon.$$
    So $f$ is uniformly continuous.
\end{proofenv}
\begin{theoremenv}
    Let $(E,\pl\cdot\pl_E)$, $(F,\pl\cdot\pl_F)$ be normed vector spaces over a complete non-trivially valued field $(K,\left| \ \cdot\ \right|)$. Let $U\subseteq E$ open, $(f_n)_{n\in\NN}$ a sequence of differentiable mappings from $U$ to $F$. Let $f: U\longrightarrow F$, $g:U\longrightarrow \mathscr{L}(E,F)$ be mappings, $p\in U$. Suppose that 

    (1) The sequence $(\DD f_n)_{n\in \NN}$ converges uniformly to $g$.

    (2) $(f_n)_{n\in\NN}$ converges uniformly to $f$.

    (3) There exists $N\in \NN$ and mapping $\delta: U\longrightarrow \RR_{\ge 0}$ such that $\dis \lim_{x\rightarrow p}\delta(x)=0$ and for any $n\in\NN_{\ge N}$, any $x\in U$,
    $$ \pl f_n(x)-f_n(p)-\DD f_n(p)(x-p)\pl_{F}\le \delta(x) \pl x-p\pl_E.$$
    Then $f$ is differentiable and $\DD f=g$.
\end{theoremenv}
\begin{proofenv}
    For any $n\in \NN$, define
    $$ \varepsilon_n\coloneq \sup_{x\in U}\pl \DD f_n(x)-g(x)\pl,\ \dd_n\coloneq \sup_{x\in U}\pl f_n(x)-f(x)\pl.$$
    One has 
    \begin{align*}
        \pl f(x)-f(p)-g(p)(x-p)\pl \le& \pl (f(x)-f_n(x))-(f(p)-f_n(p))\pl\\
        &+\pl  f_n(x)-f_n(p) -\DD f_n(p)(x-p)\pl\\
        &+\pl \DD f_n(p)(x-p)-g(p)(x-p)\pl\\
        \le & 2 \dd_n+\delta(x)\pl x-p\pl_E+\varepsilon_n\pl x-p\pl_E.
    \end{align*}
    for sufficiently large $n$.

    Therefore, 
    $$ \limsup_{x\rightarrow p} \frac{\pl f(x)-f(p)-g(p)(x-p)\pl_F}{\pl x-p\pl_E}\le 2\varepsilon_n.$$
    Taking the limit when $n\rightarrow \infty$, we obtain
    $$ \limsup_{x\rightarrow p} \frac{\pl f(x)-f(p)-g(p)(x-p)\pl_F}{\pl x-p\pl_E}=0.$$
    Namely, 
    $$ f(x)-f(p)-g(p)(x-p)=o(\pl x-p\pl_E),\ x\rightarrow p.$$
\end{proofenv}
\begin{theoremenv}\label{9.10.7}
    Let $(E,\pl\cdot\pl_E)$ and $(F,\pl\cdot\pl_F)$ be normed vector spaces over $\RR$, $U\subseteq E$ open, $(f_n)_{n\in\NN}$ be sequence of differentiable mappings from $U$ to $F$, $g:U\longrightarrow \mathscr{L}(E,F)$. We suppose that

    (1) $(\DD f_n)_{n\in\NN}$ converges uniformly to $g$.

    (2) $(f_n)_{n\in\NN}$ converges point-wise to $f: U\longrightarrow F$.

    Then $f$ is differentiable and $\DD f=g$.
\end{theoremenv}
\begin{proofenv}
    Let $p\in U$, for any $(n,m)\in \NN\times\NN$, for any $n\in\NN$,
    $$ c_{m,n}\coloneq \sup_{x\in U}\pl \DD f_m(x)-\DD f_n(x)\pl, \ \varepsilon_n\coloneq \sup_{x\in U}\pl \DD f_n(x)-g(x)\pl.$$
    For $r>0$, $B(p,r)\subseteq U$ by the mean value inequality,
    $$ \pl (f_n(x)-f_m(x))-(f_n(p)-f_m(p))\pl \le c_{m,n}\pl x-p\pl, \  x\in B(p,r) .$$
    Passing to the limit when $m\rightarrow +\infty$, we obtain
    $$ \pl (f_n(x)-f(x))-(f_n(p)-f(p))\pl \le \varepsilon_n \pl x-p\pl.$$
    we have 
    \begin{align*}
        \pl f(x)-f(p)-g(p)(x-p)\pl \le & \pl (f(x)-f_n(x))-(f(p)-f_n(p))\pl \\
        &+\pl f_n(x)-f_n(p) -\DD f_n(p)(x-p)\pl\\
        &+\pl \DD f_n(p)(x-p)-g(p)(x-p)\pl.
    \end{align*}
    $$ \limsup_{x\rightarrow p} \frac{\pl f(x)-f(p)-g(p)(x-p)\pl_F}{\pl x-p\pl_E}\le 3\varepsilon_n.$$
    Taking the limit $n\rightarrow+\infty$
    $$ \lim_{x\rightarrow p} \frac{\pl f(x)-f(p)-g(p)(x-p)\pl_F}{\pl x-p\pl_E}=0.$$
    Namely, 
    $$ f(x)-f(p)-g(p)(x-p)=o(\pl x-p\pl_E),\ x\rightarrow p.$$
\end{proofenv}
\begin{propositionenv}
    Let $(E,\pl\cdot\pl_E)$, $(F,\pl\cdot\pl_F)$ be  normed vector spaces over $\RR$. Assume that $(F,\pl\cdot\pl_F)$ is a Banach space, $U\subseteq E$ be a path connected open, $(f_n)_{n\in\NN}$ be a sequence of differentiable mappings from $U$ to $F$. Suppose that 

    (1) $(\DD f_n)_{n\in\NN}$ converges uniformly to $g: U\longrightarrow \mathscr{L}(E,F)$.

    (2) There exists $p\in U$ such that $(f_n(p))_{n\in \NN}$ converges.

    Then the sequence $(f_n)_{n\in\NN}$ converges point-wise on $U$ to a differentiable mapping $f: U\longrightarrow F$ such that $\DD f=g$.
\end{propositionenv}
\begin{proofenv}
    We first treat the case where $U$ is convex.

    For any $(n,m)\in \NN\times \NN$, let 
    $$c_{m,n}\coloneq \sup_{x\in U}\pl \DD f_m(x)-\DD f_n(x)\pl.$$
    Let $x\in U$. By the mean value inequality,
    $$ \pl (f_n(x)-f_m(x))-(f_n(p)-f_m(p))\pl \le c_{m,n}\pl x-p\pl,$$
    which leads to
    $$ \pl f_n(x)- f_m(x)\pl_F\le \pl f_n(p)-f_m(p)\pl_F+c_{m,n}\pl x-p\pl_E.$$
    Therefore $(f_n(x))_{n\in \NN}$ is a Cauchy sequence in $F$ (Banach space), so $f_n(x)$ converges in $F$ to some $f(x)$. Now it suffices to use the theorem \ref{9.10.7}.
    
    We will now treat the general case. Let $x\in U$. There exists $\gamma:[0,1]\longrightarrow U$ continuous such that $\gamma(0)=p,\gamma(1)=x$. Let $I$  be the set of $t\in [0,1]$ such that $f_n(\gamma(s))$ converges for all $s\in [0,t]$. By definition, $I$ is an interval in $[0,1]$ and $0\in I$.
    Therefore, it is of the form $[0,c]$ or $\interval[open right]{0}{c}$.

    \quad Let $B(\gamma(c),r)\subseteq U$. Since $\gamma$ is continuous, $\gamma^{-1}(B(\gamma(c),r))$ is open in $[0,1]$ and $c\in \gamma^{-1}(B(\gamma(c),r))$. Assume by contraction that $I=\interval[open right]{0}{c}$, then $I\cap \gamma^{-1}(B(\gamma(c),r))\neq \varnothing$.
    There exists $q\in \gamma^{-1}(B(\gamma(c),r))\cap I$ such that $f_n(q)$ converges. So from the ``convex $U$ version'' $f$ converges point-wise on $B(\gamma(c),r)$.
    So $f_n(\gamma(c))$ converges. Contradiction. We deduce that $I=[0,c]$.

    \quad If $c\neq 1$, then $c$ is an adherent point of $\interval[open left]{c}{1}$.
    $\gamma^{-1}(B(\gamma(c),r))$ open, so there exists $r'>0$ such that $B(c,r')\subseteq \gamma^{-1}(B(\gamma(c),r))$. In particular, $B(c,r')\cap \interval[open left]{c}{1}$ is an open interval in $[0,1]$ that continuous.
    So $I\supseteq \interval[open left]{0}{c+r'}$. Contradiction. Therefore $c=1$.
\end{proofenv}

\begin{definitionenv}
    Let $U$ be a set and $(F,\pl\cdot\pl)$ be a Banach space over complete valued field $(K,\left| \ \cdot\ \right|)$. $(f_n)_{n\in \NN}\in \left(F^U\right)^\NN$ be a sequence of mappings from $U$ to $F$. If 
    $$ \sum_{n\in \NN}\sup_{p\in U}\pl f_n(p)\pl_F<+\infty,$$
    then we say that $\dis \sum_{n\in \NN}f_n$ \textbf{converges normally}.
\end{definitionenv}
\begin{propositionenv}
    If $\sum f_n$ converges normally, then it converges uniformly.
\end{propositionenv}
\begin{proofenv}
    For any $n\in\NN$, let $g_n=\sum_{k=0}^{n} f_k$. We need to check that the sequence $(g_n)_{n\in\NN}$ converges uniformly. For any $x\in U$, $\sum_{n\in\NN}\pl f_n(x)\pl<+\infty$.
    So $\sum_{n\in\NN}f_n(x)$ converges absolutely. In particular, $(g_n(x))_{n\in\NN}$ converges to some $g(x)$.
    \begin{align*}
        \pl g_n(x)-g(x) \pl_F=&\lim_{m\rightarrow +\infty}\pl g_n(x)-g_m(x)\pl_F\\
        \le & \lim_{m\rightarrow +\infty}\pl f_{n+1}(x)+\dots +f_m(x)\pl_F\\
        \le & \limsup_{m\rightarrow +\infty}\sum_{k\ge n+1}\pl f_k(x)\pl_F\\
        \le & \varepsilon_n.
    \end{align*}
    Let 
    $$\varepsilon_n=\sum_{k\ge n+1}\sup_{p\in U}\pl f_k(p)\pl_F, \ \lim_{n\rightarrow +\infty}\varepsilon_n=0.$$
    So, 
    $$ \limsup_{n\rightarrow +\infty}\left(\sup_{x\in U} \pl g_n(x)-g(x)\pl_F\right)=0,$$
    namely, $(g_n)_{n\in\NN}$ converges to $g$.
\end{proofenv}
\begin{propositionenv}\label{9.10.10}
    Let $(K,\left| \ \cdot\ \right|)$ be a complete valued field which is non-trivially valued, $(E,\pl\cdot\pl)$ be a normed vector space and $(F,\pl\cdot\pl_F)$ be a  Banach space over $K$. $U\subseteq E$ be an open subset, $(f_n)_{n\in\NN}$ be a sequence of differentiable mappings $U\longrightarrow F$ and $p\in U$. Assume that 
    \newline
    (1) $\dis \sum_{n\in\NN}f_n$ converges normally (uniformly suffices).
    \newline
    (2) $\dis \sum_{n\in\NN}\DD f_n$ converges normally (uniformly suffices).
    \newline
    (3) $\exists N\in \NN$ and mappings $(\delta_n: U\longrightarrow \RR_{\ge 0})_{n\in\NN_{\ge N}}$ such that
    \begin{enumerate}
        \item $\forall n\in \NN_{\ge N}$, $\lim_{x\rightarrow p}\delta_n(x)=\delta_n(p)=0$.
        \item $\sum_{n\in \NN}\delta_n$ converges normally (uniformly suffices).
        \item $\forall n\in \NN_{\ge N}$, $\forall x\in U$,
            $$ \pl f_n(x)-f_n(p)-\DD f_n(p)(x-p)\pl_F\le \delta_n(x)\pl x-p\pl_E.$$
    \end{enumerate}
    Let $f$ and $g$ be limits of $\dis \sum_{n\in\NN}f_n$ and $\dis \sum_{n\in \NN}\DD f_n$ respectively. Then $f$ is differentiable at $p$ and $\DD f=g$.
\end{propositionenv}
\begin{propositionenv}
    Let $(E,\pl\cdot\pl_E)$ be a normed vector space and $(F,\pl\cdot\pl_F)$ be a Banach space over $\RR$. Let $U\subseteq E$ open, and $(f_n:U\longrightarrow F)_{n\in\NN}$ be a sequence of mappings $U\longrightarrow F$. Suppose that
    \newline
    (1) $\sum_{n\in\NN} \DD f_n$ converges normally (uniformly suffices) to some $g:U\longrightarrow \mathscr{L}(E,F)$.
    \newline
    (2) $\sum_{n\in\NN}f_n$ converges point-wise to some $f:U\longrightarrow F$.
    \newline
    Then $f$ is differentiable on $U$ and $\DD f=g$.
\end{propositionenv}
\begin{remark}
    If $U$ is path connected, one can replace (2) by (2'): $\exists p\in U$, $\sum_{n\in\NN}f_n(p)$ converges.
\end{remark}

\section{Power Series}
We fix a complete non-trivially valued field $(K,\left| \ \cdot\ \right|)$, and let $(E,\pl\cdot\pl_E)$ be a Banach space over $K$. 
\begin{definitionenv}
    Let $(S_n)_{n\in\NN}\in E^\NN$ and $b\in K$. We call power series \textbf{centered at} $b$ with  coefficients $(s_n)_{n\in\NN}$ the sequence of polynomial mappings.
    $$ \left((z\in K)\longmapsto \sum_{l=0}^{n}(z-b)^l s_l\right)_{n\in\NN}$$
    denoted as 
    $$\sum_{l=0}^{n}(z-b)^l s_l.$$
    If  $S=\sum_{n\in\NN}(z-b)^ns_n$, we denote by $R(S)$ the element 
    $$ \left(\limsup_{n\rightarrow +\infty}\pl s_n\pl ^{\frac{1}{n}}\right)^{-1}\in [0,+\infty]$$
    called the \textbf{convergence radius} of $S$. ($0^+\coloneq+\infty$, $(+\infty)^{-1}\coloneq 0$)
\end{definitionenv}
\begin{propositionenv}
    Let $ S=\sum_{n\in\NN}(z-b)^ns_n$. 
    \newline
    (1) $\forall a\in K$, if $|a-b|<R(S)$, then $S(a)\coloneq \sum_{n\in\NN}(a-b)^n s_n$ converges absolutely.
    \newline
    (2) If $r>0$ such that $(r^n\pl s_n\pl)_{n\in\NN}$ is bounded, then $R(S)\ge r$.
    \newline
    (3) If $a\in K$ is such that $|a-b|> R(S)$, then $ \sum_{n\in \NN}(a-b)^n s_n$ diverges.
\end{propositionenv}
\begin{proofenv}
    \ \newline
    (1) 
    $$ \pl (a-b)^n s\pl ^\frac{1}{n}=\left(|(a-b)^n|\cdot\pl s_n\pl\right)^{\frac{1}{n}}=|a-b|\cdot\pl s_n\pl^{\frac{1}{n}}.$$
    $$ \limsup_{n\rightarrow +\infty}\pl (a-b)^ns_n\pl^{\frac{1}{n}}= |a-b|\cdot\limsup_{n\rightarrow +\infty}\pl s_n\pl^{\frac{1}{n}}.$$
    If $|a-b|<R(S)$, then $|a-b|\cdot\limsup_{n\rightarrow +\infty}\pl s_n\pl^{\frac{1}{n}}<1$. By the root test of Cauchy, $\sum_{n\in \NN}(a-b)^ns_n$ converges absolutely. 
    \newline
    (2) $$ \pl s_n\pl^{\frac{1}{n}}=\frac{1}{r}\left(r^n\pl s_n\pl\right)^{\frac{1}{n}}.$$
    Since $(r^n\pl s_n\pl)_{n\in\NN}$ is bounded, 
    $$ \limsup_{n\rightarrow +\infty}\left(r^n \pl s_n\pl\right)^{\frac{1}{n}}\le 1.$$
    So $\limsup_{n\rightarrow +\infty}\pl s_n\pl^{\frac{1}{n}}\le \frac{1}{r}.$ So $R(S)\ge r$.
    \newline
    (3) If $|a-b|>R(S)$, then 
    $$\limsup_{n\rightarrow+\infty}\pl (a-b)^ns_n\pl^{\frac{1}{n}}=|a-b|\cdot\limsup_{n\rightarrow+\infty}\pl s_n\pl^{\frac{1}{n}}>1.$$
    So $\sum_{n\in\NN}(a-b)^ns_n$ diverges.
\end{proofenv}
\begin{propositionenv}
    Let $\dis S=\sum_{n\in\NN} (z-b)^ns_n$ be a power series.
    \newline
    (1) $\forall r\in \RR_{\ge 0}$ such that $r<R(S)$. the series $S$ converges normally on $\overline{B}(b,r)$. 
    \newline
    (2) $\dis \left(a\in B(b,R(S))\right)\longmapsto S(a)\coloneq \sum_{n\in\NN}(a-b)^nS_n$ is continuous.
\end{propositionenv}
\begin{proofenv}
    \ \newline
    (1) $\forall a\in \overline{B}(b,r)$,
    $$ \sum_{n\in\NN}\pl (a-b)^n s_n \pl \le \sum_{n\in\NN} r^n \pl s_n\pl <+\infty$$
    since $\dis \limsup_{n\rightarrow +\infty} r\cdot \pl s_n\pl^{\frac{1}{n}}<1$.
    \newline
    (2) $a\longmapsto S(a)$ is continuous on any $B(b,r)$, $r<R(S)$. Since 
    $$B(b,R(S))=\bigcup_{r<R(S)}B(b,r),$$
    $S$ is continuous on $B(b,R(S))$.
\end{proofenv}
\begin{definitionenv}
    Let $\dis S=\sum_{n\in\NN}(z-b)^ns_n$. We define the formal derivative of $S$ as 
    $$ \sum_{n\in \NN_{\ge 1}}(z-b)^{n-1}(ns_{n}).$$
\end{definitionenv}
\begin{propositionenv}
    Let $S=\sum_{n\in\NN}(z-b)^ns_n$ be a formal power series. Let $P\in K[T]$. For any $n\in\NN$, let $P(n)\coloneq P(n 1_K)\in K$.
    Let 
    $$ S_p\coloneq \sum_{n\in\NN}(z-b)^n(P(n)s_n).$$
    Then $R(S_p)\ge R(S)$.
\end{propositionenv}
\begin{proofenv}
    We assume that $P\neq 0$, $P(T)$ is of the form
    $$ C_dT^d+C_{d-1}T^{d-1}+\cdots+C_1T+C_0, \ C_d\neq 0.$$
    $$|P(n)|=\mathcal{O}(n^d)=o(r^n), \text{ for any }r>1.$$
    Hence, $\exists N\in\NN$ such that $|P(n)|\le r^n$, $\forall n\in \NN_{\ge N}$.
    $$\limsup_{n\rightarrow+\infty}\pl P(n)s_n\pl^{\frac{1}{n}}\le r\cdot\limsup_{n\rightarrow+\infty}\pl s_n\pl^{\frac{1}{n}}.$$
    Taking the limit when $r\rightarrow 1$, get
    $$ \limsup_{n\rightarrow+\infty}\pl P(n)s_n\pl^{\frac{1}{n}}\le \limsup_{n\rightarrow+\infty}\pl s_n\pl^{\frac{1}{n}}.$$
    $$ R(S_p)\ge R(S).$$
\end{proofenv}
\begin{lemmaenv}
    Let $(z_0,z)\in K^2$, $n\in\NN_{\ge 1}$.
    $$ z^n -z_0^n-n z_0^{n-1}(z-z_0)=(z-z_0)^2 \sum_{j=0}^{n-2}(n-j-1)z^j z_0^{n-2-j}.$$
\end{lemmaenv}
\begin{proofenv}
    \begin{align*}
        z^n-z_0^n=&(z-z_0)\sum_{i=0}^{n-1}z^iz^{n-1-i}.\\
        z^n-z_0^n-nz_0^{n-1}(z-z_0)=&(z-z_0)\sum_{i=0}^{n-1}(z^i z_0^{n-1-i}-z_0^{n-1})\\
        =&(z-z_0)\sum_{i=0}^{n-1}z_0^{n-i-1} (z^i-z_0^i)\\
        =& (z-z_0)^2\sum_{i=1}^{n-1}z_0^{n-1-i}\sum_{j=0}^{i-1}z^jz_0^{i-j-1}\\
        =&(z-z_0)^2\sum_{j=0}^{n-2}\sum_{i=j+1}^{n-1}z^j z_0^{n-2-j}\\
        =&(z-z_0)^2 \sum_{j=0}^{n-2}(n-j-1)z^j z_0^{n-2-j}.
    \end{align*}
\end{proofenv}

\begin{theoremenv}
    Let $\sum_{n\in\NN}(z-b)^n s_n$ be a power series and $R$ be its convergence radius. For any $z\in B(b,R)$, let $S(z)$ be the limit of the series.
    Then the mapping $S: B(b,R)\longrightarrow E$ is differentiable, and its derivative is given by the limit of the power series
    $$ \sum_{n\in \NN_{\ge 1}}(z-b)^{n-1}(ns_{n}).$$
\end{theoremenv}
\begin{proofenv}
    Let $r<R$, $(z,z_0)\in B(b,r)^2$.
    \begin{align*}
        &\pl (z-b)^n s_n-(z_0-b)^n s_n- (z-z_0)(z_0-b)^{n-1} n s_n\pl \\
        =& |z-z_0|^2 \cdot\pl \sum_{j=0}^{n-2}(n-1-j)(z-b)^j(z_0-b)^{n-2-j}s_n\pl .\\
        &\pl \sum_{j=0}^{n-2}(n-1-j)(z-b)^j(z_0-b)^{n-2-j}s_n\pl \\
        \le & \sum_{j=0}^{n-2}(n-1-j)r^{n-2}\pl s_n\pl =\frac{n(n-1)}{2} r^{n-2}\pl s_n\pl.
    \end{align*}
    We know that 
    $$ \sum_{n\in\NN}\frac{n(n-1)}{2} r^{n-2} \pl s_n\pl<+\infty.$$
    Therefore, the result follows from the proposition \ref{9.10.10}.
\end{proofenv}
\begin{definitionenv}
    Let $(a_n)_{n\in\NN}\in K^{\NN}$ and $(s_n)_{n\in\NN}\in E^{\NN}$. We call \textbf{Cauchy product} of the series $\sum_{n\in\NN}a_n$ and $\sum_{n\in\NN} s_n$ as the series:
    $$ \sum_{n\in\NN}\left(\sum_{k=0}^{n} a_k s_{n-k}\right).$$
\end{definitionenv}
\begin{theoremenv}[Merterns]
    Let $(a_n)_{n\in\NN}\in K^{\NN}$ and $(s_n)_{n\in\NN}\in E^{\NN}$. Suppose that $\sum_{n\in\NN}a_n$ and $\sum_{n\in \NN}s_n$ converges to $b\in K$ and $t\in E$ respectively.
    \newline
    (1) If at least one of $\sum_{n\in\NN}a_n$, $\sum_{n\in\NN}s_n$ converges absolutely, then the their Cauchy product converges to $bt$.
    \newline
    (2) If both $\sum_{n\in\NN}a_n$ and $\sum_{n\in\NN}s_n$ converge absolutely, then the Cauchy product also converges absolutely.
\end{theoremenv}
\begin{proofenv}
    \ \newline
    (1) Suppose that $\sum_{n\in\NN}a_n$ converges absolutely. For any $n\in\NN$, let
    $$ A_n\coloneq \sum_{k=0}^{n}a_k,\ S_n\coloneq\sum_{k=0}^{n}s_k.$$
    For any $N\in\NN$, let
    $$ t_N = \sum_{n=0}^{N}\left(\sum_{k=0}^{n} a_k s_{n-k}\right)= \sum_{\substack{(k,l)\in\NN^2\\k+l\le N}}a_k s_l=\sum_{k=0}^{N}a_k S_{N-k}=A_n t+\sum_{k=0}^{N}a_k \left(S_{N-k}-t\right).$$
    Then,
    $$ t_N-bt=\left(A_N-b\right) t+\sum_{k=0}^{N}a_k \left(S_{N-k}-t\right).$$
    Let $\alpha\coloneq\sum_{n\in \NN}|a_n|$ and for any $n\in\NN$, let
    $$ \varepsilon_n\coloneq \sup_{m\in \NN,\ m\le n} \pl S_m-t \pl.$$
    For any $l\in \{0,\ldots,N\}$, one has
    \begin{align*}
        \pl \sum_{k=0}^{N}a_k \left(S_{N-k}-t\right)\pl &\le \sum_{k=0}^{N-l} |a_k|\cdot\pl S_{N-k} -t\pl +\sum_{k=N-l+1}^{N}|a_k|\cdot\pl S_{N-k}-t \pl \\
        &\le \varepsilon_{l}\cdot\alpha +\max_{i\in \{0,\ldots,l-1\}}\pl S_i-t\pl \cdot\sum_{k=N-l+1}^{N}|a_k|.
    \end{align*}
    We get 
    $$ \forall l\in \NN,\ \limsup_{N\rightarrow +\infty} \pl \sum_{k=0}^{N}a_k \left(S_{N-k}-t\right)\pl \le \varepsilon_{l} \alpha.$$
    Taking the infimum with respect to $l$, we get 
    $$ \limsup_{N\rightarrow+\infty}\pl \sum_{k=0}^{N} a_k \left(S_{N-k}-t\right)\pl=0.$$
    We deduce therefore that 
    $$ \lim_{N\rightarrow +\infty}t_N=bt.$$
    \newline
    (2) Let 
    $$ \alpha=\sum_{n\in\NN}|a_n|,\ \beta=\sum_{n\in\NN}\pl s_n\pl.$$
    For any $N\in\NN$, one has
    $$ \sum_{n=0}^{N} \pl \sum_{k=0}^{n}a_k s_{n-k}\pl \le \sum_{n=0}^{N}\sum_{k=0}^{n}|a_k|\cdot\pl s_n\pl\le \sum_{\substack{(k,l)\in \NN^2\\ k\le N,\ l\le N}} |a_k|\pl s_l\pl \le \alpha\cdot\beta.$$
    So the Cauchy product of $\sum_{n\in\NN}a_n$ and $\sum_{n\in\NN}s_n$ converges absolutely.
\end{proofenv}
\begin{exampleenv}
    Consider
    $$ \mathrm{e}^z=\exp(z)\coloneq\sum_{n\in\NN}\frac{z^n}{n!},\  z\in \mathbb{C}.$$
    By thw ratio test of D'Alembert, for any $r>0$, $\sum_{n\in\NN}\frac{r^n}{n!}<+\infty$. $\mathrm{e}^z$ is well defined.

    Let $\alpha\in \mathbb{C}$,
    $$ \exp'(\alpha z)= \alpha \exp(\alpha z).$$
    We define 
    $$ \cos (z)\coloneq \frac{\mathrm{e}^{\ii z}+\mathrm{e}^{-\ii z}}{2},\ \sin(z)\coloneq \frac{\mathrm{e}^{\ii z}-\mathrm{e}^{-\ii z}}{2\ii},$$
    $$ \cosh (z)\coloneq \frac{\mathrm{e}^{ z}+\mathrm{e}^{- z}}{2},\ \sinh(z)\coloneq \frac{\mathrm{e}^{ z}-\mathrm{e}^{- z}}{2}.$$

\end{exampleenv}
\begin{propositionenv}
    Let $(a,b,z)\in \CC^3$, then
    $$ \exp((a+b)z)=\exp(az)\exp(bz).$$
\end{propositionenv}
\begin{proofenv}
    The Cauchy product of $\dis \sum_{n\in\NN}\frac{(az)^n}{n!}$ and $\dis \sum_{n\in\NN}\frac{(bz)^n}{n!}$ is $\dis \sum_{n\in\NN}\frac{(a+b)^nz^n}{n!}$. Use the theorem of Merterns.
\end{proofenv}

\section{Directional Differential}
\begin{definitionenv}
    Let $(K,\left|\ \cdot\ \right|)$ be a complete non-trivially valued field, and $(E,\pl\cdot\pl_E)$ and $(F,\pl\cdot\pl_F)$ be normed vector spaces over $K$. Let $U\subseteq K$ open, $f:U\longrightarrow F$ be a mapping, $p\in U$, $h\in E$.
    If the limit 
    $$ \lim_{t\rightarrow 0} \frac{f(p+th)-f(p)}{t}$$
    exists, we say that $f$ admits the \textbf{directional derivative} at $p$ along $h$.
\end{definitionenv}
\begin{notationenv}
    $$ \pa_h f(p)= \lim_{t\rightarrow 0} \frac{f(p+th)-f(p)}{t}.$$
\end{notationenv}
\begin{definitionenv}
    Let $(E_1,\pl\cdot\pl_1),\ldots,(E_n,\pl\cdot\pl_n)$ be normed vector spaces, $E\coloneq E_1\times\dots\times E_n$,
    $$ \pl (s_1,\ldots,s_n)\pl=\max_{i\in \{1,\ldots,n\}}\pl s_i\pl.$$
    If $f: U\longrightarrow F$. We say that $f$ has the \textbf{$\mathbf{i}$-th partial differential} at $p=(p_1,\ldots,p_n)\in U$, if the mapping
    $$ x_i\longmapsto f(p_1,\ldots,p_{i-1},x_i,p_{i+1},\ldots,p_n)$$
    is differentiable at $p_i$. We denote the existing differential at $p_i$ by
    $$ \DD_i f(p)\in \mathscr{L}(E_i,F).$$
    In the case when $E_i=K$,
    $$ \DD_i f(p)(1)\coloneq \pa_i f(p) \text{ or }\frac{\pa f}{\pa x_i}(p).$$
    Note that
    $$\pa_i f(p)= \pa_{\substack{(0,\cdots,1,\cdots,0)\\ i\text{-th}}}f(p).$$
\end{definitionenv}
\begin{remark}
    Let $(K,\left| \ \cdot\ \right|)$ be a complete non-trivially valued filed, $(E_i,\pl\cdot\pl_i)$, $i\in\{1,\cdots,n\}$, $(F,\pl\cdot\pl_F)$ be normed vector spaces. $E=E_1\times\dots\times E_n$, equipped with the norm $\pl\cdot\pl$ defined as
    $$ \pl (x_1,\cdots,x_n)\pl=\max_{i\in \{1,\cdots,n\}}\pl x_i\pl_i.$$
    Let $U\subseteq E$ be an open subset, $p\in U$, $f:U\longrightarrow F$ be a mapping. If $f$ is differentiable at $p$, then $f$ has the $i$-th partial differential at $p$ for $i\in\{1,\cdots,n\}$.
    In fact,
    $$ f(p_1,\cdots,p_i+h_i,\cdots,p_n)=f(p)+\DD f(p)(0,\cdots,h_i,\cdots,0)+o(\pl h_i\pl_i).$$
    $$ \DD_i f(p)(h_i)=\DD f(p)(0,\cdots,h_i,\cdots,0).$$
    $$ \DD f(p)(h)=\sum_{i=1}^n \DD f(p)(0,\cdots,h_i,\cdots,0)=\sum_{i=1}^n \DD_i f(p)(h_i).$$
\end{remark}
\begin{propositionenv}
    Let $(E_i,\pl\cdot\pl_i)$, $i\in\{1,\cdots,n\}$, $(F,\pl\cdot\pl_F)$ be normed vector spaces over $\RR$, with $\dim _\RR(F)<+\infty$. Let $E=E_1\times\dots\times E_n$, equipped with the norm $\pl\cdot\pl$ defined as
    $$ \pl (x_1,\cdots,x_n)\pl=\max_{i\in \{1,\cdots,n\}}\pl x_i\pl_i.$$
    Let $U\subseteq E$ be an open subset, $f: U\longrightarrow F$ be a mapping.
    Suppose that, for any $i\in\{1,\cdots,n\}$, $f$ has $i^{\text{th}}$ partial differential on $U$, and $\DD_i f:U\longrightarrow \mathscr{L}(E_i,F) $ is continuous. Then $f$ is differentiable on $U$, and
    $$ \forall p\in U,\ \DD f(p)(h_1,\cdots,h_n)=\sum_{i=1}^{n} \DD_i f(p)(h_i).$$
\end{propositionenv}
\begin{proofenv}
    We first treat the case where $F=\RR$. Let $p\in U$, and $r>0$ such that $B(p,r)\subseteq U$. Let $h=(h_1,\cdots,h_n)\in B(0,r)$.
    \begin{align*}
        f(p+h)-f(p)=\sum_{i=1}^{n}&(f(p_1+h_1,\cdots,p_i+h_i+\cdots,p_{i+1},\cdots,p_n)\\
        &-f(p_1+h_1,\cdots,p_i,\cdots,p_{i+1},\cdots,p_n)).
    \end{align*}
    By the mean value theorem of Lagrange, 
    $$ \exists (t_1(h),\cdots,t_n(h))\in \interval[open]{0}{1}^n$$
    such that 
    $$ f(p+h)-f(p)=\sum_{i=1}^{n}\DD_i f(p_1+h_1,\cdots,p_i+t_i(h) h_i,\cdots,p_{i+1},\cdots,p_n)(h_i).$$
    \begin{align*}
        &f(p+h)-f(p)-\sum_{i=1}^{n}\DD_i f(p)(h)\\
        =&\sum_{i=1}^{n}\DD_i f(p_1+h_1,\cdots,p_i+t_i(h) h_i,\cdots,p_{i+1},\cdots,p_n)(h_i)\\
        -&\sum_{i=1}^{n}\DD_i f(p_1,\cdots,p_i,\cdots,p_{i+1},\cdots,p_n)(h_i)\\
        =& o(\pl h\pl).
    \end{align*}
\end{proofenv}
