\chapter{Integral Calculus}

\section{Differential 1-form}
\begin{definitionenv}
    Let $(K,\left|\ \cdot\ \right|)$ be a complete non-trivially valued field.
    
    Let $(E,\pl\cdot\pl_E)$ and $(F,\pl\cdot\pl_F)$ be normed vector spaces over $K$.
    Let $U\subseteq E$ be an open subset. We call \textbf{1-form} \textit{on $U$ with coefficients in} $F$ any mapping
    $$ \alpha: U\longrightarrow \mathscr{L}(E,F).$$
    If there exists $f:U\longrightarrow F$ differentiable such that $\DD f=\alpha$, we say that $\alpha$ is an \textbf{exact} $1$-form. (Sometimes $\DD f$ is also written as $\dd f$.)
\end{definitionenv}
\begin{definitionenv}
    We call a complete valued filed \textbf{extension} of $(K,\left|\ \cdot\ \right|)$ any complete valued field $(K',\left|\ \cdot\ \right|')$ such that $K\subseteq K'$ and $\left| \ \cdot\ \right|=\left| \ \cdot\ \right|'|_K$.
    \newline
    Let $(F,\pl\cdot\pl)$ be a normed vector space over $K$. If $\alpha: U\longrightarrow \mathscr{L}(E,K')$ and $s: U\longrightarrow F$ be mappings, we denote by 
    $$ \alpha\otimes s: U\longrightarrow \mathscr{L}(E,F)$$
    be the mapping sending $p\in U$ to
    $$ (h\in E)\longmapsto \alpha(p)(h)s(p).$$
    Note that 
    $$\pl \alpha(p)(h)s(p)\pl_F\le | \alpha(p)(h)|_{K'}\cdot\pl s(p)\pl_F\le \pl \alpha(p)\pl \cdot\pl s(p)\pl_F\cdot\pl h\pl_E.$$
    If $(F,\pl\cdot\pl_F)=(K',\left|\ \cdot\ \right|')$, $\alpha\otimes s$ is also written as $\alpha s$.
\end{definitionenv}
\begin{exampleenv}
    $(K,\left|\ \cdot\ \right|)=(\RR,\left|\ \cdot\ \right|)$, $K'=\CC$, $|x+\ii y|'\coloneq\sqrt{x^2+y^2}$.
\end{exampleenv}
\begin{exampleenv}
    Let $\varphi\in \mathscr{L}(E,F)$, 
    $$\begin{array}{rrcl}
        \DD \varphi:& E&\longrightarrow &\mathscr{L}(E,F)\\
        & p&\longmapsto& \varphi.
    \end{array}$$
    is a constant mapping.

    As a $1$-form, it is often written as $\dd \varphi$.
    \begin{exampleenv}
        $E=K^n$, $x_i: K^n\longrightarrow K$, $(p_1,\cdots,p_n)\longmapsto p_i$. $U\subseteq E$ open, $f: U\longrightarrow K$ differentiable.
        $$ \dd f(p)=\sum_{i=1}^{n}\frac{\pa f}{\pa x_i}(p)\dd x_i.$$
    \end{exampleenv}
    \begin{exampleenv}
        Let $w\in \CC$, $f: \RR\longrightarrow \CC$, $t\longmapsto \exp(wt)$.
        $$ \dd f(t)=f'(t)\dd t=w\exp(wt) \dd t.$$
    \end{exampleenv}
\end{exampleenv}
\begin{propositionenv}
    Let $(K',\left|\ \cdot\ \right|)$ be a complete valued extension of $(K,\left|\ \cdot\ \right|)$, and $(F,\pl\cdot\pl_F)$ be a normed vector space over $K'$. Let $(E,\pl\cdot\pl_E)$ be a normed vector space over $K$, $U\subseteq E$ be ann open subset.
    Let $f: U\longrightarrow K'$ and $g: U\longrightarrow F$ be two mappings that are differentiable, then 
    $$ \dd(f g)=f \dd g+\dd f \otimes g.$$
\end{propositionenv}
\begin{propositionenv}
    Let $(K',\left| \ \cdot\ \right|')$ be a complete valued extension of $(K,\left|\ \cdot\ \right|)$. $(E,\pl\cdot\pl_E)$ be a normed vector space over $K$, $(F,\pl\cdot\pl_F)$ be a normed vector space over $K'$. Let $U\subseteq E$ be an open subset, and $V\subseteq K'$ be an open subset. $f:U\longrightarrow V$, $g:V\longrightarrow F$ be differentiable mappings, then 
    $$ \dd(g\circ f)= \dd f \otimes (g'\circ f).$$
\end{propositionenv}
\begin{proofenv}
    For $p\in U$ and $h\in E$,
    \begin{align*}
        \DD(g\circ f)(p)(h)&=\DD g(f(p))(\DD f(p)(h))\\
        &=\DD f(p)(h)\cdot \DD g(f(p))(1)\\
        &=\DD f(p)(h)\cdot g'(f(p))\\
    \end{align*}
\end{proofenv}
\section{Primitive Functions}
\begin{propositionenv}
    Let $(E,\pl\cdot\pl)$ and $(F,\pl\cdot\pl)$ be normed vector spaces over $\RR$ and $U\subseteq E$ be a path connected open subset.
    If $f: U\longrightarrow F$ is a mapping such that $\dd f=0$, then $f$ is a constant mapping.
\end{propositionenv}
\begin{proofenv}
    Let $p$ and $q$ be elements of $U$. There exists $\gamma:[0,1]\longrightarrow U$ continuous and differentiable on $\interval[open]{0}{1}$, such that $\gamma(0)=p$, $\gamma(1)=q$.
    $$ \pl f(p)-f(q)\pl_F=\pl f(\gamma(0))- f(\gamma(1))\pl_F\le \sup_{t\in \interval[open]{0}{1}}\pl \DD f(\gamma(t))(\gamma'(t))\pl_F=0.$$
    So $f(p)=f(q)$.
\end{proofenv}
\begin{definitionenv}
    Let $I\subseteq \RR$ be an open interval and $\varphi: I\longrightarrow F$ be a mapping. If there exists $\varPhi: I\longrightarrow F$ such that $\varPhi'=\varphi$, we say that $\varPhi$ is a primitive function of $\varphi$. We denote by 
    $$ \int\varphi(t)\dd t$$
    an arbitrary primitive function of $\varphi$.
    By the previous proposition,
    $$ \int\varphi(t)\dd t=\varPhi(t)+C.$$
    where $C$ is a constant mapping.
\end{definitionenv}
\begin{exampleenv}
    Let $w\in \CC$,
    $$ \int \exp(wt)\dd t=\left\{\begin{array}{cl}
         \frac{\exp(wt)}{w}+C&,w\neq 0\\
         t+C&,w=0.
    \end{array}\right.$$
\end{exampleenv}
\begin{propositionenv}
    Let $I\subseteq \RR$ be an open interval. Let $g: I\longrightarrow \RR$ and $\varphi:I\longrightarrow F$  be mappings having $G: I\longrightarrow \RR$ and $\varPhi: I\longrightarrow F$ as primitive functions.
    Then
    $$ \int G(t) \dd \varPhi(t)+\int \dd G(t)\otimes \varPhi(t)=G(t)\varPhi(t) + C.$$
    or equivalently,
    $$ \int G(t) \dd t \otimes \varphi (t)+\int g(t)\dd t\otimes \varPhi(t)=G(t)\varPhi(t) + C.$$
    If $F=\RR$ or $F=\CC$, the formula can be written as 
    $$ \int G(t) \dd \varPhi(t)+\int \Phi(t) \dd G(t) =G(t)\varPhi(t) +C.$$
    or 
    $$ \int G(t)  \varphi (t) \dd t +\int  \varPhi(t) g(t) \dd t =G(t)\varPhi(t) + C.$$

\end{propositionenv}
\begin{exampleenv}
    $$ \int t\ee^{t} \dd t=\int  t\dd(\ee^t)= t\ee^t-\int \ee^t \dd t=t \ee^t-\ee^t+C.$$
\end{exampleenv}
\begin{propositionenv}
    Let $U\subseteq\RR$ be an open subset, $V\subseteq\RR$ be an open subset, $f: U\longrightarrow V$ and $g: V\longrightarrow F$ differentiable mappings. One has 
    $$ \int \dd f(t) \otimes g'(f(t)) =g(f(t))+C.$$
\end{propositionenv}
\begin{exampleenv}
    $$ \int \sin (t)\cos (t)\dd t= \int \sin(t)\dd(\sin(t))=\frac{1}{2}\sin(t)^2+C.$$
\end{exampleenv}

\section{Riesz Space}
We fix a set $\Omega$. We equipped $\RR^\Omega$ with the partial order $\le $ as follows:
$$\forall (f,g)\in \RR^\Omega\times \RR^\Omega,\ f\le g \Leftrightarrow \forall \omega\in \Omega,\ f(\omega)\le g(\omega).$$
If $(f_1,\cdots,f_n)\in \left(\RR^\Omega\right)^n$, $\inf\{f_1,\cdots,f_n\}$ and $\sup\{f_1,\cdots,f_n\}$ exists.
$$ \forall \omega\in \Omega,\ \inf\{f_1,\cdots,f_n\}(\omega)=\min\{f_1(\omega),\cdots,f_n(\omega)\}$$
$$ \forall \omega\in \Omega,\ \sup\{f_1,\cdots,f_n\}(\omega)=\max\{f_1(\omega),\cdots,f_n(\omega)\}$$

\begin{definitionenv}
    We call Riesz space on $\Omega$ any vector space $S$ of $\RR^\Omega$, such that 
    $$ \forall (f,g)\in S\times S,\ \inf\{f,g\}\in S.$$
\end{definitionenv}
\begin{remark}
    $\forall (f,g)\in S\times S$, 
    $$ \sup \{f,g\}=f+g-\inf\{f,g\}\in S.$$
    $$ |f|=\sup \{f,0\} -\inf\{f,0\} \in S.$$
    By induction, $\forall n\in \NN_{\ge 1}$, $\forall (f_1,\cdots,f_n)\in S^n$, 
    $$ \inf\{f_1,\cdots,f_n\}, \sup\{f_1,\cdots,f_n\}\subseteq S.$$
    $$ \forall \omega \in \Omega,\ \sup\{f,g\}(\omega)=\max\{f(\omega),g(\omega)\}=f(\omega)+g(\omega)-\min\{f(\omega),g(\omega)\}.$$
\end{remark}
\begin{definitionenv}
    Let $S$ be a Riesz space on $\Omega$. We call \textbf{integral operator} on $S$ any $\RR$-linear mapping $I:S\longrightarrow \RR$ such that
    \newline
    (1) $\forall (f,g)\in S\times S$, if $f\le g$, then $I(f)\le I(g)$.
    \newline
    (2) If $(f_n)_{n\in\NN}$ is a decreasing sequence in $S$, that converges point-wise to constant zero mapping $0$, one has 
    $$ \lim_{n\rightarrow +\infty}I(f_n)=0.$$
\end{definitionenv}
\begin{exampleenv}
    Let $\Omega=\RR$, $\forall A\subseteq \RR$, let 
    $$ \begin{array}{rrcl}
        \mathbbm{1}_A:& \RR &\longrightarrow& \{0,1\}\\
        &x&\longmapsto& \begin{cases}
            1,\ x\in A\\
            0,\ x\notin A
        \end{cases}
    \end{array}$$
    Let $S$ be the vector space of $\RR^\RR$ generated by mappings of the form $\mathbbm{1}_{\interval[open left]{a}{b}}$, ($a\le b$)
    

    Let $\varphi:\RR\longrightarrow \RR$ be a right continuous mapping,
    $$\forall t\in \RR,\  \varphi(t)=\lim_{\varepsilon>0,\varepsilon\rightarrow 0}\varphi(t+\varepsilon).$$
    which is increasing.
    Then $I_{\varphi}: S\longrightarrow \RR$, 
    $$I_\varphi\left(\sum_{i=1}^{n}\lambda_i \mathbbm{1}_{\interval[open left]{a_i}{b_i}}\right)\coloneq \sum_{i=1}^{n}\lambda_i\left(\varphi(b_i)-\varphi(a_i)\right)$$
    is an integral operator.
\end{exampleenv}
\begin{propositionenv}
    Let $\Omega$ be a set and $S$ be a Riesz space on $\Omega$. An $\RR$-linear mapping $I:S\longrightarrow \RR$ that satisfies $(f\le g\Rightarrow I(f)\le I(g))$ is an integral operator if and only if, for any increasing sequence $(f_n)_{n\in \NN}$ in $S$ that converges point-wise to some $f\in S$, one has
    $$ \lim_{n\rightarrow +\infty}I(f_n)=I(f).$$
\end{propositionenv}
\begin{proofenv}
    \ \newline
    ``$\Rightarrow$'': $(f-f_n)_{n\in\NN}$ is decreasing and converges to $0$. So
    $$ \lim_{n\rightarrow +\infty}I(f-f_n)=0.$$
    So $\dis \lim_{n\rightarrow +\infty}I(f_n)=I(f).$
    \newline
    ``$\Leftarrow$'': Let $(f_n)_{n\in\NN}$ be a decreasing sequence in $S$ that converges point-wise to $0$. Then $(-f_n)_{n\in\NN}$ is increasing and converges point-wise to $0$. So
    $$ \lim_{n\rightarrow +\infty}I(-f_n)=0.$$
    So, $\dis \lim_{n\rightarrow +\infty}I(f_n)=0.$
\end{proofenv}
\begin{propositionenv}
    Let $\Omega$ be a set and $S$ be a Riesz space on $\Omega$ and $I: S\longrightarrow \RR$ be an integral operator. Let $g\in S$ and $(f_n)_{n\in\NN}$ be an increasing sequence in $S$. If
    $$ \forall \omega\in \Omega,\ g(\omega)\le \lim_{n\rightarrow +\infty}f_n(\omega),$$
    then
    $$ I(g) \le \lim_{n\rightarrow +\infty}I(f_n).$$
\end{propositionenv}
\begin{proofenv}
    $(\inf\{g,f_n\})_{n\in\NN}$ is an increasing sequence in $S$.
    It converges to $g$. Hence, 
    $$ I(g)=\lim_{n\rightarrow +\infty}I(\inf\{g,f_n\})\le \lim_{n\rightarrow +\infty}I(f_n).$$
\end{proofenv}
\begin{corollaryenv}
    Let $(f_n)_{n\in\NN}$ and $(g_n)_{n\in\NN}$ be increasing sequences in $S$. Suppose that 
    $$ \forall \omega\in \Omega,\ \lim_{n\rightarrow +\infty}f_n(\omega)\le \lim_{n\rightarrow +\infty}g_n(\omega).$$
    Then,
    $$ \lim_{n\rightarrow +\infty}I(f_n)\le \lim_{n\rightarrow +\infty}I(g_n).$$
\end{corollaryenv}
\begin{proofenv}
    $\forall k\in \NN$, $\forall \omega \in\Omega$,
    $$ f_k (\omega)\le \lim_{n\rightarrow +\infty}f_n(\omega)\le \lim_{n\rightarrow +\infty}g_n(\omega).$$
    So $\dis I(f_k)\le \lim_{n\rightarrow +\infty}I(g_n)$.
    Taking the limit when $k\rightarrow +\infty$, we get
    $$ \lim_{k\rightarrow +\infty}I(f_k)\le \lim_{n\rightarrow +\infty}I(g_n).$$
\end{proofenv}
\begin{definitionenv}
    Let $S^{\uparrow}$ be the set of all mappings $f: \Omega \longrightarrow \interval[open left]{-\infty}{+\infty}$ that can be written as the point-wise limit of an increasing sequence in $S$.
\end{definitionenv}
\begin{remark}
    \ \newline
    (1) If $f\in S^{\uparrow}$, $\lambda>0$, then $\lambda f \in S^{\uparrow}$.
    \newline
    (2) If $(f,g)\in S^{\uparrow}\times S^{\uparrow}$, then $f+g\in S^{\uparrow}$, $\inf\{f,g\}\in S^{\uparrow}$, $\sup\{f,g\}\in S^{\uparrow}$.
    \newline
    (3) If $I:S\longrightarrow \RR$ is an integral operator, then for any $f\in S^{\uparrow}$ that is written as the point-wise limit of two increasing sequences $(f_n)_{n\in\NN}$ and $(g_n)_{n\in\NN}$ in $S$, then 
    $$ \lim_{n\rightarrow +\infty}I(f_n)=\lim_{n\rightarrow +\infty} I(g_n).$$
    We denote by $I(f)$ this limit.
\end{remark}
\begin{propositionenv}
    Let $(f_n)_{n\in\NN}\in (S^{\uparrow})^{\NN}$ be an increasing sequence, and $f$ be its point-wise limit. Then $f\in S^{\uparrow}$, and $\dis I(f)=\lim_{n\rightarrow +\infty}I(f_n)$ for any operator $I$.
\end{propositionenv}
\begin{proofenv}
    For any $k\in \NN$, let $(g_{k,m})_{m\in\NN}\in S^{\NN}$ be an increasing sequence in $S$ that converges point-wise to $f_k$. For any $n\in\NN$, let 
    $$h_n=\sup \{g_{0,n},g_{1,n},\cdots,g_{n,n}\}\in S.$$
    $(h_n)_{n\in\NN}$ is an increasing sequence in $S$. 

    \quad $\forall n\in \NN$, $\forall k\in \NN$, $k\le n$, one has
    $$ f_n\ge f_k \ge g_{k,n},\  f_n\ge h_n.$$
    So,
    $$ f=\lim_{n\rightarrow +\infty} f_n\ge \lim_{n\rightarrow +\infty}h_n \ge \lim_{n\rightarrow +\infty} g_{k,n}=f_k.$$
    This leads to 
    $$ f=\lim_{n\rightarrow +\infty} h_n,\   f\in S^{\uparrow}.$$
    One has
    $$ I(f)=\lim_{n\rightarrow +\infty} I(h_n) \le \lim_{n\rightarrow +\infty} I(f_n).$$
    Moreover, $\forall n\in\NN$, $f\ge f_n$, so $I(f)\ge I(f_n)$. Thus leads to 
    $$I(f)\ge \lim_{n\rightarrow +\infty}I(f_n).$$
\end{proofenv}
\begin{definitionenv}
    Let $\Omega$ be a set and $S$ be a Riesz space on $\Omega$. We denote by $S^{\downarrow}$ the set of all mappings $f: \Omega \longrightarrow \interval[open right]{-\infty}{+\infty}$ that can be written as the point-wise limit of a decreasing sequence in $S$.
\end{definitionenv}
\begin{remark}
    \ \newline
    (1) $f\in S^{\downarrow}\Leftrightarrow -f\in S^{\uparrow}$.
    \newline
    (2) If $f\in S^{\downarrow}$, $\lambda>0$, then $\lambda f \in S^{\downarrow}$.
    \newline
    (3) If $(f,g)\in S^{\downarrow}\times S^{\downarrow}$, then $f+g\in S^{\downarrow}$, $-\inf\{f,g\}\in S^{\downarrow}$, $-\sup\{f,g\}\in S^{\downarrow}$.
    \newline
    (4) If $(f_n)_{n\in\NN}\in (S^{\downarrow})^{\NN}$ is a decreasing sequence, then $$ \lim_{n\rightarrow +\infty}f_n\in S^{\downarrow}.$$
    \newline
    (5) If $I:S\longrightarrow \RR$ is an integral operator. For any $f\in S^{\downarrow}$, let
    $$ I(f)\coloneq - I(-f).$$
    \begin{enumerate}
        \item If $(f,g)\in S^{\downarrow}\times S^{\downarrow}$ or $(f,g)\in S^{\uparrow}\times S^{\uparrow}$,
        $$ f\le g\Rightarrow I(f)\le  I(g), I(f+g)=I(f)+I(g),$$
        $$I(\lambda f)=\lambda I(f), \forall \lambda\in \RR\backslash\{0\}.$$
        \item If $(f_n)_{n\in\NN}\in (S^{\downarrow})^{\NN}$ is a decreasing sequence, then $$ I(\lim_{n\rightarrow +\infty}f_n)=\lim_{n\rightarrow +\infty}I(f_n).$$
    \end{enumerate}
\end{remark}
\begin{propositionenv}
    Let $\Omega$ be a set, $S$ be a Riesz space on $\Omega$ and $I:S\longrightarrow\Omega$ be an integral operator. For any $(f,g)\in (S^{\uparrow}\cup S^{\downarrow})^2$, if $f\le g$, then $I(f)\le I(g)$.
\end{propositionenv}
\begin{proofenv}
    It is suffices to treat the case where $(f,g)\in S^{\uparrow}\times S^{\downarrow}$ or $(f,g)\in S^{\downarrow}\times S^{\uparrow}.$

    \quad If $(f,g)\in S^{\uparrow}\times S^{\downarrow}$, then $(-f,g)\in S^{\downarrow}\times S^{\downarrow}$, so $g-f\in S^{\downarrow}$. $I(g-f)=I(g)-I(f)\ge 0$. So $I(f)\le I(g)$.

    \quad If $(f,g)\in S^{\downarrow}\times S^{\uparrow}$, then $(-f,g)\in S^{\uparrow}\times S^{\uparrow}$, so $g-f\in S^{\uparrow}$. $I(g-f)=I(g)-I(f)\le 0$. So $I(f)\le I(g)$.
\end{proofenv}
\begin{definitionenv}
    Let $\Omega$ be a set, $S$ be a Riesz space on $\Omega$, and $I:S\longrightarrow \RR$ be an integral operator. Let $f: \Omega\longrightarrow \RR$ be a mapping. If 
    $$ \sup_{\substack{l\in S\\ l \le f}}I(l)=\inf _{\substack{\mu\in S\\ \mu \ge f}}I(\mu).$$
    We say that $f$ is \textbf{Riemann integrable}.
    
    \quad Let 
    $$\underline{I}(f)\coloneq \sup_{\substack{l\in S^{\downarrow}\\ l\le f}}I(l),$$
    $$ \overline{I}(f)\coloneq \inf_{\substack{\mu\in S^{\uparrow}\\ \mu \ge f}}I(\mu),$$
    then,
    $$ \underline{I}(f)\le I(f)\le \overline{I}(f).$$
    If $ \underline{I}(f)=\overline{I}(f)\in\RR$, we say that $f$ is \textbf{Daniell integrable}, and we denote by $I(f)$ the real number $\underline{I}(f)=\overline{I}(f)$.

    \quad We denote by $\mathcal{L}^{1}(I)$ the set of all Daniell integrable mappings from $\Omega$ to $\RR$. We got a mapping
    $$ I:\mathcal{L}^1 (I)\longrightarrow \RR.$$
\end{definitionenv}
\begin{lemmaenv}
    Let $\Omega$ be a set, $S$ be a Riesz space on $\Omega$, and $I:S\longrightarrow \RR$ be an integral operator.
    \newline
    (1) For any mapping $f:\Omega \longrightarrow \RR$,
    $$ \underline{I}(-f)=-\overline{I}(f),\ \overline{I}(-f)=-\underline{I}(f).$$
    In particular,
    $$ f\in\mathcal{L}^1(I)\Leftrightarrow -f\in\mathcal{L}^1(I).$$
    And in this case, 
    $$ -I(f)=I(-f).$$
    (2) For any $(f,g)\in \RR^\Omega\times \RR^{\Omega}$,
    $$ \underline{I}(f+g)\ge \underline{I}(f)+\underline{I}(g),\ \overline{I}(f+g)\le \overline{I}(f)+\overline{I}(g).$$
    In particular, if $(f,g)\in \mathcal{L}^1(I)\times \mathcal{L}^1(I)$, then $f+g\in \mathcal{L}^1(I)$, and $I(f+g)=I(f)+I(g).$
    \newline
    (3) For any $f\in \RR^{\Omega}$ and any $\lambda\in \RR_{>0}$,
    $$ \underline{I}(\lambda f)=\lambda \underline{I}(f),\ \overline{I}(\lambda f)=\lambda \overline{I}(f).$$
    In particular, if $f\in \mathcal{L}^1(I)$, then $\lambda f \in \mathcal{L}^1(I)$, and $I(\lambda f)=\lambda I(f).$
    \newline
    (4) If $(f,g)\in \RR^{\Omega}\times \RR^{\Omega}$ such that $f\le g$, then 
    $$ \underline{I}(f)\le \underline{I}(g),\ \overline{I}(f)\le \overline{I}(g).$$
    (5) If $(f:\Omega\longrightarrow \RR)\in S^\uparrow\cup S^{\downarrow}$ such that $I(f)\in \RR$, then $f\in \mathcal{L}^1(I)$.

\end{lemmaenv}
\begin{proofenv}
    \ \newline
    (1) If $\mu\in S^{\uparrow}$, $\mu \ge f$, then $-\mu\in S^{\downarrow}$, $-\mu\le -f$. So
    $$-I(\mu)=I(-\mu)\le \underline{I}(-f).$$
    $$ I(\mu)\ge -\underline{I}(-f).$$
    Taking $\dis \inf_{\substack{\mu\in S^{\uparrow}\\\mu\ge f}}$, we get 
    $$ \overline{I}(f)\ge -\underline{I}(-f).$$
    $\forall l\in S^{\downarrow}$, $l\le f$, one has $-l\in S^{\uparrow}$, $-l\ge -f$. So 
    $$ I(-l)\ge \overline{I}(-f),\  I(l)\le -\overline{I}(-f).$$
    Taking $\dis \sup_{\substack{l\in S^{\downarrow}\\l \le f}}$, we get 
    $$ \underline{I}(f)\le -\overline{I}(-f).$$
    Replacing $f$ by $-f$, we get
    $$ \underline{I}(-f)\ge -\overline{I}(f),\ -\overline{I}(-f)\ge\underline{I}(f).$$
    So $-\underline{I}(-f)=\overline{I}(f),\ -\overline{I}(-f)=\underline{I}(f)$.
    \newline
    (2) For any $(l_1,l_2)\in S^{\downarrow}\times S^{\downarrow}$, $l_1\le f$, $l_2\le g$. One has $l_1+l_2\le f+g$, so
    $$ \sup_{\substack{(l_1,l_2)\in S^{\downarrow}\times S^{\downarrow}\\ l_1\le f,l_2\le g}} I(l_1+l_2)\le \underline{I}(f+g).$$
    $$ \overline{I} (f+g)=-\underline{I}(-f-g)\ge -(\underline{I}(-f)+\underline{I}(-g))=\overline{I}(f)+\overline{I}(g).$$
    If $\overline{I}(f)=\underline{I}(f)$, $\overline{I}(g)=\underline{I}(g)$, one has
    $$\overline{I}(f)+\overline{I}(g)=\underline{I}(f)+\underline{I}(g)\le \underline{I}(f+g)\le \overline{I}(f+g).$$
    $$ \overline{I}(f+g)\le \overline{I}(f)+\overline{I}(g)=I(f)+I(g).$$
    (3) $$ \underline{ I}(\lambda f)=\sup_{\substack{l\in S\\ l\le \lambda f}}I(l)=\sup_{\substack{l\in S^{\downarrow}\\ l\le f}}I(\lambda l)=\lambda \underline{I}(f).$$
    $$ \overline{I}(\lambda f)=-\underline{I}(\lambda(-f))=-\lambda \underline{I}(-f)=\lambda \overline{I}(f).$$
    (5) Let $f\in S^{\uparrow}$. By definition, $\overline{I}(f)=I(f)$. Moreover, there exists an increasing sequence $(f_n)_{n\in\NN}\in S^{\NN}\subseteq (S^{\uparrow})^{\NN}$ such that
    $$ I(f)=\lim_{n\rightarrow \infty} I(f_n)\le \underline{I}(f).$$
    So, 
    $$ \underline{I}(f)= I(f)= \overline{I}(f).$$
\end{proofenv}
\begin{theoremenv}[Beppo Levi]
    Let $(f_n)_{n\in\NN}$ be a monotone sequence in $\mathcal{L}^1(I)$ such that converges point-wise to a mapping $f:\Omega\longrightarrow \RR$. If $\dis \lim_{n\rightarrow +\infty} I(f_n)\in\RR$, then
    $$ f\in \mathcal{L}^1(I),\  I(f)=\lim_{n\rightarrow +\infty} I(f_n).$$
\end{theoremenv}
\begin{proofenv}
    Suppose that $(f_n)_{n\in\NN}$ is increasing. By replacing $f_n$ by $f_n-f_0$ and $f$ by $f-f_0$, we may assume $f_0=0$.

    \quad Let $\varepsilon>0$. For any $n\in\NN_{\ge 1}$, let $\mu_n\in S^\uparrow$ such that $f_n-f_{n-1}\le \mu_n$ and
    $$ I(f_n-f_{n-1})\ge I(\mu_n)-\frac{\varepsilon}{2^n}.$$
    $$ f_n=\sum_{k=1}^{n}(f_k-f_{k-1})\le \mu_{1}+\cdots+\mu_n,$$
    and
    $$ I(f_n)=\sum_{k=1}^{n}I(f_k-f_{k-1})\ge \sum_{k=1}^{n}\left(I(\mu_k)-\frac{\varepsilon}{2^n}\right)\ge I(\mu_1)+\cdots+I(\mu_n)-\varepsilon.$$
    Let 
    $$\mu= \lim_{N\rightarrow +\infty} \sum_{k=1}^{N}\mu_k\in S^\uparrow.$$
    One has $\dis I(\mu)=\sum_{n\in\NN}I(\mu_n)$, $\dis \mu\ge \lim_{n\rightarrow+\infty}f_n=f$. Let $\dis \alpha=\lim_{n\rightarrow +\infty}I(f_n)$, one has
    $$ \alpha\ge I(\mu)-\varepsilon \ge \overline{I}(f)-\varepsilon.$$
    For any $n\in\NN$, let $l_n\in S^{\downarrow}$ such that $l_n\le f_n\le f$ and $I(l_n)\ge I(f_n)-\varepsilon$. Then 
    $$\alpha-\varepsilon\liminf_{n\rightarrow +\infty} I(l_n)\le \underline{ I}(f)
     .$$
    Thus, 
    $$ \alpha-\varepsilon\underline{I}(f)\le \overline{I}(f)\le \alpha+\varepsilon.$$
    Since $\varepsilon$ is arbitrary, we have 
    $$ \underline{I}(f)=\overline{I}(f)=\alpha=\lim_{n\rightarrow +\infty}I(f_n).$$
\end{proofenv}
\begin{theoremenv}[Daniell]
    $\mathcal{L}^1(I)$ forms a Riesz space on $\Omega$, and $I:\mathcal{L}^1(I)\longrightarrow \RR$ is an integral operator extending $I: S\longrightarrow \RR$.
\end{theoremenv}
\begin{proofenv}
    By the property of $\overline{I}$ and $\underline{I}$, $\mathcal{L}^1(I)$ is a vector subspace of $\RR^\Omega$ and $I:\mathcal{L}^1(I)\longrightarrow \RR$ is an $\RR$-linear mapping. Moreover, if $f\le g$, then $I(f)\le I(g)$.
    Let $(f_1,f_2)\in \mathcal{L}^1(I)^2$, $\forall \varepsilon>0$, $\exists(l_1,l_2)\in \left(S^\downarrow\right)^2$, $\exists (\mu_1,\mu_2)\in \left(S^\uparrow\right)^2$, 
    $$l_i\le f_i \le \mu_i,\ i\in \{1,2\}\text{ and } I(\mu_i-l_i)\le \frac{\varepsilon}{2}.$$
    Then,
    $$ \inf\{l_1,l_2\}\le \inf\{f_1,f_2\} \le \inf\{\mu_1,\mu_2\},$$
    and
    $$ \inf\{\mu_1,\mu_2\}-\inf\{l_1,l_2\}\le (\mu_1-l_1)+(\mu_2-l_2).$$
    Suppose that $\mu_1(\omega)\le \mu_2(\omega)$, $l_2(\omega)\le l_1(\omega)$. LFS$=\mu_1(\omega)-l_2(\omega)$, $\mu_2(\omega)\ge \mu_1(\omega)\ge l_1(\omega)$,
    $$ I\left(\inf\{\mu_1,\mu_2\}-\inf\{l_1,l_2\}\right)\le I(\mu_1-l_1)+I(\mu_2-l_2)\le \varepsilon.$$
    By Beppo Levi's theorem, if $(f_n)_{n\in\NN}$ is an increasing sequence in $\mathcal{L}^1(I)$ that converges to some $f\in \mathcal{L}^1(I)$. One has $I(f)=\lim_{n\rightarrow +\infty}I(f_n)$.
\end{proofenv}
\begin{remark}
    If $f\in\mathcal{L}^1(I)$, then $|f|\in \mathcal{L}^1(I)$.
\end{remark}
\begin{theoremenv}[Fatou's lemma]
    Let $(f_n)_{n\in\NN}$ be a sequence in $\mathcal{L}^1(I)$. Assume that there exists $g\in \mathcal{L}^1(I)$ such that $\forall n\in \NN$, $f_n\ge g$. Then
    $$ \liminf_{n\rightarrow +\infty}f_n\in \mathcal{L}^1(I),$$
    and 
    $$ I\left(\liminf_{n\rightarrow +\infty}f_n\right)\le \liminf_{n\rightarrow +\infty}I(f_n).$$
    Moreover, when $\dis \liminf_{n\rightarrow +\infty}I(f_n)<+\infty$ and $\dis \liminf_{n\rightarrow +\infty}f_n$ takes finite values, then
    $$ \liminf_{n\rightarrow +\infty}f_n\in \mathcal{L}^1(I).$$
\end{theoremenv}
\begin{proofenv}
    For any $n\in\NN$, let $g_n$ be 
    $$ \inf_{k\in \NN}f_{n+k}=\lim_{k\rightarrow +\infty} \inf\{f_n,f_{n+1},\cdots,f_{n+k}\}\ge g.$$
    $$ I(f_n)\ge \lim_{k\rightarrow +\infty}I\left(\inf\{f_n,\cdot,f_{n+k}\}\right)\ge I(g).$$
    By Beppo Levi's theorem, $g_n\in \mathcal{L}^1(I)$, and $I(g_n)\le I(f_n).$
    The sequence $(g_n)_{n\in\NN}$ is increasing and converges point-wise to $\dis \liminf_{n\rightarrow +\infty}f_n$.
    So $\dis \liminf_{n\rightarrow +\infty} f_n\in \mathcal{L}^1(I)^\uparrow$, and
    $$ I\left(\liminf_{n\rightarrow +\infty}f_n\right)=\lim_{n\rightarrow +\infty}I(g_n)\le \liminf_{n\rightarrow +\infty}I(f_n).$$
    If $\dis \liminf_{n\rightarrow +\infty}I(f_n)<+\infty$, then $\dis I\left(\liminf_{n\rightarrow +\infty} f_n\right)<+\infty$. By Beppo Levi's theorem, 
    $$\liminf_{n\rightarrow +\infty}f_n\in \mathcal{L}^1(I).$$
\end{proofenv}
\begin{theoremenv}[Dominated convergence theorem, Lebesgue]
    Let $(f_n)_{n\in\NN}$ be a sequence in $\mathcal{L}^1(I)$ that convergence pointwise to a mapping $f:\Omega\longrightarrow \RR$. Assume that there exists $g\in \mathcal{L}^1(I)$ such that
    $$ \forall n\in\NN, |f_n|\le g,$$
    then, 
    $$f\in \mathcal{L}^1(I)\text{ and }\dis I(f)=\lim_{n\rightarrow +\infty}I(f_n).$$
\end{theoremenv}
\begin{proofenv}
    $$ f_n\ge g_n,\ -f_n\ge -g_n, \forall n\in\NN.$$
    By Fatou's lemma,
    $$ I(\lim_{n\rightarrow +\infty}f_n)\le \lim_{n\rightarrow +\infty}I(f_n),$$
    $$I(\lim_{n\rightarrow +\infty}(-f_n))\le \liminf_{n\rightarrow +\infty}I(-f_n)=-\limsup_{n\rightarrow +\infty}I(f_n),$$
    $$ -I(g)\le \limsup_{n\rightarrow +\infty}I(f_n)\le I(\lim_{n\rightarrow +\infty}f_n)\le \liminf_{n\rightarrow +\infty}I(f_n)\le I(g).$$
    So $\left(I(f_n)\right)_{n\in\NN}$ converges to $\dis I\left(\lim_{n\rightarrow +\infty}f_n\right)\in \RR$. Hence,
    $$ \lim_{n\rightarrow +\infty} f_n\in \mathcal{L}^1(I).$$
\end{proofenv}
\section{Convexity$^*$}
\begin{definitionenv}
    Let $E$ be a vector space over $\RR$, $U\subseteq E$ convex. We say that the mapping $f:U\longrightarrow \RR$ is \textbf{convex} if the \textbf{epigraph} 
    $$ \Gamma_+(f) \coloneq \{(x,a)\in U\times \RR\mid f(x)\le a\}$$
    is convex in $E\times \RR$.

    We say that $f:U\longrightarrow \RR$ is \textbf{concave} if its \textbf{hypergraph}
    $$ \Gamma_-(f) \coloneq \{(x,a)\in U\times \RR\mid  f(x)\ge a\}$$
    is convex in $E\times \RR$.
\end{definitionenv}
\begin{propositionenv}
    Let $E$ be a vector space over $\RR$, $U\subseteq E$ convex, and $f:U\longrightarrow \RR$ a mapping. Then the following conditions are equivalent:
    \newline
    (1) $f$ is convex.
    \newline
    (2) For any $(x,y)\in U\times U$, and $t\in [0,1]$,
    $$ f(tx+y(1-t))\le tf(x)+y(1-t)f(y).$$
\end{propositionenv}
\begin{proofenv}
    \ \newline
    (1)$\Rightarrow$(2): Note that $((x,f(x)), (y,f(y)))\in \Gamma_+^2(f)$, $(x,y)\in U^2$.
    $$ t(x,f(x))+(1-t)(y,f(y))=(tx+y(1-t),tf(x)+(1-t)f(y))\in \Gamma_+(f).$$
    Hence, 
    $$f(tx+y(1-t))\le tf(x)+(1-t)f(y).$$
    (2)$\Rightarrow$(1): Let $((x,a),(y,b))\in \Gamma_+^2(f)$, then $a\ge f(x)$, $b\ge f(y)$. Let $t\in [0,1]$, then
    $$ t a + (1-t)b\ge tf(x)+(1-t)f(y)\ge f(tx+(1-t)y).$$
    Hence, 
    $$ (tx+(1-t)y, ta+(1-t)b)\in \Gamma_+(f).$$
\end{proofenv}
\begin{propositionenv}
    Let $E$ be a vector space over $\RR$, $U\subseteq E$ convex, and $f:U\longrightarrow \RR$ a mapping. Then the following conditions are equivalent:
    \newline
    (1) $f$ is concave.
    \newline
    (2) For any $(x,y)\in U\times U$, and $t\in [0,1]$,
    $$ f(tx+y(1-t))\ge tf(x)+y(1-t)f(y).$$
\end{propositionenv}
\begin{proofenv}
    \ \newline
    (1)$\Rightarrow$(2): Note that $((x,f(x)), (y,f(y)))\in \Gamma_-^2(f)$, $(x,y)\in U^2$.
    $$ t(x,f(x))+(1-t)(y,f(y))=(tx+y(1-t),tf(x)+(1-t)f(y))\in \Gamma_-(f).$$
    Hence, 
    $$f(tx+y(1-t))\ge tf(x)+(1-t)f(y).$$
    (2)$\Rightarrow$(1): Let $((x,a),(y,b))\in \Gamma_-^2(f)$, then $a\le f(x)$, $b\le f(y)$. Let $t\in [0,1]$, then
    $$ t a + (1-t)b\le tf(x)+(1-t)f(y)\le f(tx+(1-t)y).$$
    Hence, 
    $$ (tx+(1-t)y, ta+(1-t)b)\in \Gamma_-(f).$$
\end{proofenv}
\begin{propositionenv}\label{10.4.4}
    Let $E$ be a vector space over $\RR$, $U\subseteq E$ convex, and $f:U\longrightarrow \RR$ a mapping. $(f_i)_{i\in I}$ is a family of linear forms on $U$. ($f_i: E\longrightarrow \RR$ linear.) $(c_i)_{i\in I}$ is a family of real numbers.
    If
    $$ \forall p\in U, f(p)=\sup_{i\in I} (f_i(p)+c_i),$$
    then, $f$ is convex.
\end{propositionenv}
\begin{proofenv}
    Let $(x,y)\in U^2$, $t\in [0,1]$, then for any $i\in I$,
    $$f_{i}(tx+(1-t)y)+c_i= t(f_i(x)+c_i)+(1-t)(f_i(y)+c_i)\le tf(x) +(1-t)f(y).$$
    Taking the supremum with respect to $i$, we obtain
    $$ f(tx+y(1-t))\le tf(x)+(1-t)f(y).$$
\end{proofenv}
\begin{propositionenv}
    Let $(E,\pl \cdot\pl)$ be a normed vector space over $\RR$, $U\subseteq E$ be a convex open subset, $f:U\longrightarrow\RR$ be a differentiable mapping.
    Then $f$ is convex if and only if
    $$ \forall (p,x)\in U^2, \ f(x)\ge f(p)+ \DD f(p)(x-p).$$
    Moreover, when $f$ is convex, then
    $$ \forall x\in U, \ f(x)=\sup_{p\in U}\left(f(p)+\DD f(p)(x-p)\right).$$
\end{propositionenv}
\begin{proofenv}
    For any $p\in U$, we define
    $$\begin{array}{rrcl}
        g_p: &U&\longrightarrow &\RR\\
        &x&\longmapsto &f(p)+\DD f(p)(x-p). 
    \end{array}$$
    We have that $f(p)=g_p(p)$.
    $$ \forall (p,x)\in U^2,\ f(x)\ge g_p(x) \Rightarrow f=\sup_{p\in U} g_p.$$
    By proposition \ref{10.4.4},  $f$ is convex. 

    \quad Conversely, assume that $f$ is convex, $(p,x)\in U^2$, $t\in[0,1]$,
    $$ f(tx+(1-t)p)= f(p+t(x-p))\le t f(x) +(1-t) f(p)=f(p)+t(f(x)-f(p)).$$
    $f$ is differentiable at $p$,
    $$ f(p+t(x-p))=f(p)+t \DD f(p)(x-p)+o(|t|).$$
    Taking the limit when $t\rightarrow 0$, we get
    $$ f(x)-f(p)\ge \DD f(p)(x-p).$$
\end{proofenv}
\begin{definitionenv}
    Let $E$ be a vector space over $\RR$. \textbf{Bilinear form} on $E$ is a bilinear mapping from $E\times E$ to $\RR$. Let $\varphi: E\times E\longrightarrow \RR $ be a symmetric bilinear form.
    
    If 
    $$ \forall x\in E, \ \varphi(x,x)\ge 0,$$
    we say that $\varphi$ is \textbf{semipositive}.

    If
    $$ \forall x\in E\backslash\{0\}, \ \varphi(x,x)>0,$$
    we say that $\varphi$ is \textbf{positive define}.
\end{definitionenv}
\begin{exampleenv}
    Let $(x_1,\cdots,x_n)$ and $(y_1,\cdot,y_n)$ be elements of $\RR^n$,
    $$ \left((x_1,\cdots,x_n),(y_1,\cdots,y_n)\right) \longmapsto \sum_{i=1}^{n} x_i y_i$$
    is a linear bilinear positive define form on $\RR^n$.
\end{exampleenv}
\begin{definitionenv}
    Let $E$ be a vector space over $\RR$, $\varphi: E\times E\longrightarrow \RR$ be a symmetric bilinear form.
    $$ \ker (\varphi)\coloneq \{x\in E\mid \forall y\in E, \ \varphi(x,y)=0\}$$
    is the intersection of $\ker\left(\varphi(\cdot,y)\right)$ over all $y\in E$.

    \quad The \textbf{isotropic cone} of $\varphi$ is the set of $x\in E$ such that $\varphi(x,x)=0$. $\ker(\varphi)$ is contained  in the isotropic cone of $\varphi$.
\end{definitionenv}
\begin{propositionenv}
    Let $E$ be a vector space over $\RR$, $\varphi: E\times E\longrightarrow \RR$ be a symmetric bilinear form.
    If $\varphi$ is semipositive, then $\ker(\varphi)$ is equal to the isotropic cone of $\varphi$.
\end{propositionenv}
\begin{proofenv}
    It is suffices to show that any element $y$ of the isotropic cone of $\varphi $ is in $\ker(\varphi)$.

    \quad Let $x\in E$, $t\in \RR$,
    $$ \varphi(x+ty,x+ty)=\varphi(x,x)+ 2t\varphi(x,y)+t^2\varphi(y,y)\ge 0.$$
    Since $\varphi(y,y)=0$, we obtain 
    $$ \forall t\in \RR,\ \varphi(x,x) + 2t\varphi(x,y)\ge 0,$$
    $$ \forall -t\in \RR,\ \varphi(x,x) - 2t\varphi(x,y)\ge 0.$$
    Thus, for any $t\in\RR$,
    $$\left(\varphi(x,x) + 2t\varphi(x,y)\right)\left(\varphi(x,x) - 2t\varphi(x,y)\right)= \varphi(x,x)^2-4t^2\varphi(x,y)^2\ge 0. $$
    Take the limit $|t|\rightarrow +\infty$, we obtain, $\varphi(x,y)=0$.
\end{proofenv}
\begin{theoremenv}[Cauchy-Schwartz]
    Let $E$ be a vector space over $\RR$, $\varphi: E\times E\longrightarrow \RR$ be a semipositive, bilinear form. For any $(x,y)\in E\times E$,
    $$ \varphi(x,y)^2\le \varphi (x,x)\varphi(y,y).$$
    The equality holds if and only if $\varphi(y-x,h)=0$ for any $h\in E$.
\end{theoremenv}
\begin{proofenv}
    First, we show that if $[x]=h[y]$ in  $E/\ker(\varphi)$ then $\varphi(x,y)^2=\varphi(x,x)\varphi(y,y)$.

    We have 
    $$\{x-ah,y-bh\}\subseteq \ker \varphi.$$
    $$ \varphi(x,y)=\varphi((x-ah)+ah,(y-bh)+bh)=\varphi(ah,bh)=ab\varphi(h,h).$$
    $$ \varphi(x,x)=a^2\varphi(h,h),\ \varphi(y,y)=b^2\varphi(h,h).$$
    Hence, 
    $$ \varphi(x,y)^2\varphi(x,x)\varphi(y,y).$$
    We know if $\varphi(y,y)=0$, then $y\in \ker\varphi$. In this case, $[y]=0$. So $[x], [y]$ are colinear in $E/\ker \varphi$.

    Assume that $\varphi(y,y)\neq 0$, $t\in\RR$, 
    $$\varphi(x+ty,x+ty)=t^2 \varphi(y,y)+\varphi(x,x)+2t\varphi(x,y)\ge 0.$$
    Take $t=-\frac{\varphi(x,y)}{\varphi(y,y)}$, we obtain
    $$ \varphi(x,y)^2\le \varphi(x,x)\varphi(y,y).$$
    If the equality holds, then $\varphi(x+ty,x+ty)=0$, for $t=-\frac{\varphi(x,y)}{\varphi(y,y)}$ and hence $x+ty\in \ker\varphi$.
\end{proofenv}
\begin{theoremenv}
    Let $(E,\pl\cdot\pl)$ be a normed vector space over $\RR$, $U\subseteq E$ be an open convex subset, $f: U\longrightarrow \RR$ be a second-order differentiable mapping. If $\DD^2 f(p)$ is semipositive for any $p$, then $f$ is convex.
\end{theoremenv}
\begin{proofenv}
    Let $(p,x)\in U^2$, we define
    $$\begin{array}{rrcl}
        g: & [0,1]&\longrightarrow & \RR\\
        & t&\longmapsto & f(tx+(1-t)p).
    \end{array}$$
    Then, 
    $$g'(t)=\DD f(p+t(x-p))(x-p),\ g''(t)=\DD^2 f(p+t(x-p))(x-p,x-p)\ge 0.$$
    By Taylor-Lagrange, there exists $\xi\in [0,1]$,
    $$ g(1)-g(0)=g'(0)+\xi g''(\xi)\le g'(0)= \DD f(p)(x-p).$$
    So $f(x)-f(p)\ge \DD f(p)(x-p)$. So $f$ is convex.
\end{proofenv}
\section{Semirings}
\begin{definitionenv}
    Let $\Omega$ be a set. We call semiring on $\Omega$ any $\mathcal{C}\subseteq \mathscr{P}(\Omega)$ that satisfies
    \newline
    (1) $\varnothing \in \mathcal{C}$.
    \newline
    (2) $\forall (A,B)\in \mathcal{C}\times \mathcal{C}$, $A\cap B\in \mathcal{C}$.
    \newline
    (3) $\forall (A,B)\in \mathcal{C}\times \mathcal{C}$, there exists a family $C_1,\cdots,C_n$ of pairwise disjoint sets in $\mathcal{C}$ such that 
    $$B\backslash A=\bigcup_{i=1}^{n}C_i.$$
\end{definitionenv}
\begin{exampleenv}
    $\Omega=\RR$, $\mathcal{C}=\{\interval[open left]{a}{b}\mid (a,b)\in \RR^2,a\le b\}$.
    \newline
    (1) $\varnothing=\interval[open left]{0}{0}\in \mathcal{C}$.
    \newline
    (2) $\interval[open left]{a}{b}\cap \interval[open left]{c}{d}\neq \varnothing\Leftrightarrow c<b\ (a\le c)$. When $c<b$, $\interval[open left]{a}{b}\cap \interval[open left]{c}{d}=\interval[open left]{c}{b}$.
    \newline
    (3) $B\backslash A=B\backslash\left(A\cap B\right)$. We may assume $A\subseteq B$. If $A=\interval[open left]{a}{b}$, $B=\interval[open left]{c}{d}$, $A\subseteq B$ implies $c\le a$, $b\le d$.
\end{exampleenv}
\begin{propositionenv}
    Let $\Omega$ be a set and $\mathcal{C}$ be a semiring on $\Omega$.
    \newline
    (1) Let $B\in \mathcal{C}$. Let $A_1,\cdots,A_n$ be sets in $\mathcal{C}$. Then $B\backslash \left(A_1\cup \cdots\cup A_n\right)$ can be written as the union of  a finite family of pairwise disjoint sets in $\mathcal{C}$.
    \newline
    (2) Let $\Theta$ be a finite subset of $\mathcal{C}$. There exists a finite family $\Phi$ of pairwise disjoint sets in $\mathcal{C}$ such that each element of $\Theta$ can be written as the union of some elements of $\Phi$.
    \newline
    (3) Let $\mathcal{A}$ be the set
    $$ \{A\in \mathscr{P}(\Omega)\mid \exists n\in\NN, \exists (A_1,\cdots,A_n)\in \mathcal{C}^n, A=A_1\cup \cdots\cup A_n\}.$$
    Then any $A\in\mathcal{A}$ can be written as the union of a finite family of pairwise disjoint sets in $\mathcal{C}$.
    In particular, $\forall (A,A')\in \mathcal{A}^2$, $\{A\cup A',A\cap A', A\backslash A'\}\subseteq \mathcal{A}$.
\end{propositionenv}
\begin{proofenv}
    \ \newline
    (1) We reason by induction on $n$. The case where $n=0$ is trivial.
    Suppose that $\dis B\backslash(A_1,\cup\cdots,\cup A_{n-1})=\bigcup_{i=1}^{n}C_i$, where $C_1,\cdots,C_m$ are pairwise disjoint sets in $\mathcal{C}$. Then 
    $$B\backslash(A_1,\cup\cdots,\cup A_{n})=\bigcup_{i=1}^{n}C_i\backslash A_n.$$
    Each $C_i\backslash A_n$ is of the form $\dis \bigcup_{j=1}^{d_i}D_{ij}$ with $D_{i,1},\cdots,D_{i,d_i}$  in $\mathcal{C}$, pairwise disjoint. So
    $$ B\backslash(A_1,\cup\cdots,\cup A_{n})=\bigcup_{i=1}^{n}\bigcup_{j=1}^{d_i}D_{ij}.$$
    \newline
    (2) Suppose that $\Theta=\{B_1,\cdots,B_n\}$. For any $i\in \{1,\cdots,n\}$, one has
    $$ B_i=\bigcup_{i\in J\subseteq\{1,\cdots,n\}}\left(\bigcap_{j\in J}B_j\right)\backslash\left(\bigcup_{k\in \{1,\cdots,n\}\backslash J} B_k\right).$$
    For any $J\subseteq\{1,\cdots,n\}$, $J\neq \varnothing$, we let
    $$ B_j\coloneq \left(\bigcap_{j\in J}B_j\right)\backslash \left(\bigcup_{k\in\{1,\cdots,n\}\backslash J}B_k\right)= \left(\bigcap_{j\in J}B_j\right)\bigcap \left(\bigcap_{k\in \{1,\cdots,n\}\backslash J} \complement_\Omega B_k\right).$$
    $ \left(B_J\right)_{\substack{J\subseteq \{1,\cdots,n\}\\ J\neq \varnothing}}$ are pairwise disjoint. By (1), each $B_J$ is the union of fa finite family of pairwise disjoint elements $C_{J,1},\cdots,C_{J,d_J}$ in $\mathcal{C}$. Let 
    $$\Phi=\{C_{J,l}\mid  l\in \{1,\cdots,d_J\}, J\subseteq \{1,\cdots,n\}, J\neq\varnothing\}.$$
    (3) By (2), there exists a finite subset $\Phi$ of pairwise disjoint elements of $\mathcal{C}$ such that each $A_i$ is the union of some sets in $\Phi$. Then 
    $$A=\bigcup_{\substack{C\in \Phi\\C\subseteq A}}C.$$
\end{proofenv}
\begin{propositionenv}
    Let $\Omega$ be a set and $\mathcal{C}$ be a semiring on $\Omega$. Let $S$ be the vector subspace of $\RR^\Omega$ generated by mappings of the form $\mathbbm{1}_A$, $A\in \mathcal{C}$.
    \newline
    (1) Any pair $(f,g)\in S^2$ can be written as 
    $$ f=\sum_{i=1}^{n}a_i\mathbbm{1}_{C_i},\ g=\sum_{i=1}^{n}b_i\mathbbm{1}_{C_i},$$
    where $n\in\NN$, $(a_1,\cdots,a_n)\in \RR^\NN$, $(b_1,\cdots,b_n)\in \RR^\NN$, $(C_1,\cdots,C_n)\in \mathcal{C}^n$, pairwise disjoint.
    \newline
    (2) $S$ is a Riesz space.
\end{propositionenv}
\begin{proofenv}
    \ \newline
    (1) By definition, $f$ and $g$ are of the form
    $$ f=\sum_{A\in \Theta_f}\lambda_A\mathbbm{1}_{A},\ g=\sum_{B\in \Theta_g} \mu_B\mathbbm{1}_{B},$$
    where $\Theta_f$ and $\Theta_g$ are finite subsets of $\mathcal{C}$, $\lambda_A$ and $\mu_B$ are real numbers. Let $\Theta=\Theta_f\cup \Theta_g$. There is a subset $\Phi\subseteq \mathcal{C}$ consisting of pairwise disjoint sets, such that element of $\Theta$ can be written as the union of some sets in $\Phi$.

    \quad Suppose that $\Phi=\{C_1,\cdots,C_n\}$. Then
    $$ f=\sum_{i=1}^{n}\left(\sum_{\substack{A\in \Theta_f\\ A\cap C_i\neq \varnothing}}\lambda_A\right)\mathbbm{1}_{C_i},\ g=\sum_{i=1}^{n}\left(\sum_{\substack{B\in \Theta_g\\ B\cap C_i\neq \varnothing}}\mu_B\right)\mathbbm{1}_{C_i}.$$
    (2) If
    $$ f=\sum_{i=1}^{n}a_i\mathbbm{1}_{C_i},\ g=\sum_{i=1}^{n}b_i\mathbbm{1}_{C_i},$$
    with $C_1,\cdots,C_n$ in $\mathcal{C}$ pairwise disjoint, then 
    $$\inf\{f,g\}=\sum_{i=1}^{n}\min\{a_i,b_i\}\mathbbm{1}_{C_i}\in S.$$
\end{proofenv}
\section{$\sigma$-additive Functions}
\begin{definitionenv}
    Let $\Omega$ be a set, $\mathcal{C}\subseteq \mathscr{P}(\Omega)$, $\mu:\mathcal{C}\longrightarrow \RR_{\ge 0}$ be a mapping. We say that $\mu$ is \textbf{additive} if for any finite family $(A_n)^{n}_{i=1}$ of pairwise disjoint sets in $\mathcal{C}$ such that $A_1\cup\cdots\cup A_n\in \mathcal{C}$. One has
    $$\mu(A_1\cup\cdots\cup A_n)=\sum_{i=1}^{n}\mu(A_i).$$
\end{definitionenv}
\begin{remark}
    If $\varnothing \in \mathcal{C}$, then $\varnothing=\varnothing\cup\varnothing$. So $\mu(\varnothing)=2\mu(\varnothing)$, that means $\mu(\varnothing)=0$.
\end{remark}
\begin{exampleenv}
    Let $\varphi:\RR\longrightarrow \RR$ right continuous and increasing.
    $$ \mathcal{C}=\{\interval[open left]{a}{b}\mid (a,b)\in \RR^2,a\le b\}.$$
    We define 
    $$\mu_{\varphi}:\mathcal{C}\longrightarrow \RR_{\ge 0},\ \mu_{\varphi}(\interval[open left]{a}{b})=\varphi(b)-\varphi(a).$$
    If $a_0\le \cdots\le a_n$ are real numbers.
    $$ \mu_\varphi\left(\interval[open left]{a_0}{a_n}\right)=\varphi(a_n)-\varphi(a_0)=\sum_{i=1}^{n}\varphi(a_i)-\varphi(a_{i-1})=\sum_{i=1}^{n}\mu_{\varphi}\left(\interval[open left]{a_{i-1}}{a_i}\right).$$
    Therefore, $\mu_\varphi$ is additive.
\end{exampleenv}
\begin{propositionenv}
    Let $\Omega$ be a set, $\mathcal{C}$ be a semiring on $\Omega$, $\mu:\mathcal{C}\longrightarrow \RR_{\ge 0}$ be a additive mapping, $S$ be the vector subspace of $\RR^\Omega$ generated by $\mathbbm{1}_A$, where $A\in \mathcal{C}$.
    \newline
    (1) There exists a unique $\RR$-linear mapping $I:S\longrightarrow \RR$ such that $I(\mathbbm{1}_A)=\mu(A)$ for any $A\in \mathcal{C}$.
    \newline
    (2) Let 
    $$ \mathcal{A}=\{A\in \mathscr{P}(\Omega)\mid \exists n\in \NN, \exists (A_1,\cdots,A_n)\in \mathcal{C}^n, A=A_1\cup\cdots\cup A_n\}.$$
    Then $\mu$ extends in a unique way to an additive mapping from $\mathcal{A}$ to $\RR_{\ge 0}$.
\end{propositionenv}
\begin{proofenv}
    \ \newline
    (1) If $I:S\longrightarrow \RR$ exists, then it is unique since $S$ is generated by $\mathbbm{1}_A$, $A\in \mathcal{C}$. ($\forall f\in  S$, $f$ is of form $\sum_{i=1}^{n}a_i\mathbbm{1}_{A_i}$, $A_i\in \mathcal{C}$, $a_i\in \RR$. $I(f)$ should be $\sum_{i=1}^{n}a_i\mu(A_i)$.)
    It remains to check that such $I$ is well defined. Suppose that $f\in S$ can be written as 
    $$ f=\sum_{A\in \Theta}\lambda_A\mathbbm{1}_{A}=\sum_{B\in \Theta'}\lambda_B'\mathbbm{1}_{B'}.$$
    We aim to check that
    $$ \sum_{A\in \Theta}\lambda_A\mu(A)=\sum_{B\in \Theta'}\lambda_B'\mu(B).$$
    Take $C_1,\cdots,C_n\in \mathcal{C}$, pairwise disjoint, such that exch $A\in \Theta \cup\Theta'$ can be written as the union of some sets among $\{C_1,\cdots,C_n\}$.
    $$ f=\sum_{i=1}^{n}\left(\sum_{\substack{A\in \Theta\\ C_i\cap A\neq\varnothing}}\lambda_A\right)\mathbbm{1}_{C_i}=\sum_{i=1}^{n}\left(\sum_{\substack{B\in \Theta'\\ C_i\cap B\neq\varnothing}}\lambda_B'\right)\mathbbm{1}_{C_i}.$$
    $$\forall i\in \{1,\cdots,n\},\ \sum_{\substack{A\in \Theta\\ C_i\cap A\neq \varnothing}}\lambda_A=\sum_{\substack{B\in \Theta'\\C_i\cap B\neq \varnothing}}\lambda_B'.$$
    \begin{align*}
        \sum_{A\in \Theta}\lambda_A \mu(A)&=\sum_{A\in \Theta}\lambda_A\sum_{\substack{i\in\{1,\cdots,n\}\\ A\cap C_i\neq \varnothing}}\mu(C_i)\\
        &=\sum_{i=1}^{n}\mu(C_i)\sum_{\substack{A\in \Theta\\ A\cap C_i\neq \varnothing}}\lambda_A\\
        &=\sum_{i=1}^{n}\mu(C_i)\sum_{\substack{B\in \Theta'\\ B\cap C_i\neq \varnothing}}\lambda_B'\\
        &=\sum_{B\in \Theta'}\lambda_B'\sum_{\substack{i\in\{1,\cdots,n\}\\ B\cap C_i\neq \varnothing}}\mu(C_i)\\
        &=\sum_{B\in \Theta'}\lambda_B'\mu(B).
    \end{align*}
    (2) We take, for any $A\in \mathcal{A}$, $\mu(A)$ as $I(\mathbbm{1}_A)$. If $A$ is write as a disjoint union $B_1\cup\cdots\cup B_m$, with $B_i\in \mathcal{A}$.
    $$ I(\mathbbm{1}_A)=I(\sum_{j=1}^{m}\mathbbm{1}_{B_j})=\sum_{j=1}^{m}I(\mathbbm{1}_{B_j}).$$
    
\end{proofenv}
\begin{definitionenv}
    Let $\Omega$ be a set and $\mathcal{C}\subseteq \mathscr{P}(\Omega)$. We say that a mapping $\mu:\mathcal{C}\longrightarrow \RR_{\ge 0}$ is $\sigma$-additive if, for any countable set $\Theta$ and any family $(C_i)_{i\in \Theta}$ of pairwise disjoint sets in $\mathcal{C}$, one has
    $$ \bigcup_{i\in \Theta}C_i\in \mathcal{C} \Rightarrow \mu\left(\bigcup_{i\in \Theta}C_i\right)=\sum_{i\in \Theta}\mu(C_i).$$
    $$\left(\sum_{i\in \Theta}\mu(C_i)\right)\coloneq \sup_{\substack{\Theta'\subseteq\Theta \\ \Theta' \text{ finite}}}\sum_{i\in \Theta'}\mu(C_i).$$
\end{definitionenv}

\begin{propositionenv}
    Let $\Omega$ be a set and $\mathcal{C}$ be a semiring on $\Omega$.
    $$ \mathcal{A}\coloneq \{ A\in \mathscr{P}(\Omega)\mid \exists n\in \NN, \exists (A_1,\cdots,A_n)\in \mathcal{C}^n, A=A_1\cup\cdots\cup A_n\}.$$
    Let $\mu:\mathcal{C}\longrightarrow \RR_{\ge 0}$ be an additive mapping that extends in a unique way to an additive mapping. $\mu: \mathcal{A}\longrightarrow \RR_{\ge 0}$, $S\subseteq \RR^{\Omega}$ be the vector subspace generated by $\mathbbm{1}_A$, $A\in \mathcal{C}$.
    Then the following conditions are equivalent:
    \newline
    (1) $\mu: \mathcal{C}\longrightarrow \RR_{\ge 0}$ is $\sigma$-additive.
    \newline
    (2) $\mu: \mathcal{A}\longrightarrow \RR_{\ge 0}$ is $\sigma$-additive.
    \newline
    (3) For any decreasing sequence $(A_n)_{n\in\NN}\in \mathcal{A}^{\NN}$ such that $\dis \bigcap_{n\in\NN}A_n=\varnothing$,
    $$ \lim_{n\rightarrow +\infty} \mu(A_n)=0.$$
    (4) $I:S\longrightarrow \RR$, ($\RR$-linear mapping) that tends $\mathbbm{1}_A$, $A\in \mathcal{C}$ to $\mu(A)$ is an integral operator.
\end{propositionenv}
\begin{proofenv}
    \ \newline
    (1)$\Rightarrow$(2) Let $(A_n)_{n\in\NN}\in \mathcal{A}^\NN$ be pairwise disjoint sequence. 
    $$ \forall A_n=\bigcup_{j=0}^{d_n}C_{n,j},\ C_{n,j}\in \mathcal{C},\ (C_{n,j})_{j=0}^{d_n}\text{ pairwise disjoint.}$$
    $$ A\coloneq \bigcup_{n\in\NN} A_n,\ A=B_1\cup \cdots\cup B_n \text{ disjoint union } B_{i}\in \mathcal{C}.$$
    $$ B_i =B_i \cap A =\bigcup_{n\in\NN} \bigcup_{j=0}^{d_n} \left( B_i \cap C_{n,j}\right).$$
    Since $\mu$ is $\sigma$-additive on $\mathcal{C}$,
    $$ \mu(B_i)=\sum_{n\in\NN}\sum_{j=0}^{d_n} \mu \left(B_i\cap C_{n,j}\right)=\lim_{N\rightarrow +\infty} \sum_{n=0}^{N}\sum_{j=0}^{d_n}\mu\left(B_i\cap C_{n,j}\right).$$
    \begin{align*}
        \mu(A)=\sum_{i=1}^{m}\mu(B_i) &=\lim_{N\rightarrow +\infty} \sum_{i=1}^{N}\sum_{n=0}^{N}\sum_{j=0}^{d_n} \mu\left(B_i\cap C_{n,j}\right)\\
        &=\lim_{N\rightarrow +\infty} \sum_{n=0}^{N}\sum_{i=0}^{m}\sum_{j=0}^{d_n}\mu\left(B_i\cap C_{n,j}\right)\\
        &= \lim_{N\rightarrow +\infty} \mu(A_n)=\sum_{n\in\NN}\mu(A_n).
    \end{align*}
    (2)$\Rightarrow$(3) Let $n\in\NN$, $B_n=A_n\backslash A_{n+1}\in \mathcal{A}$. Then $\mu(A_n)=\mu(B_n)+\mu(A_{n+1})$.
    $$ \mu(A_0)=\mu(B_0)+\mu(B_1)+\cdots+\mu(B_{N-1})+\mu(A_N),\ \forall N\in \NN.$$
    $A_0$ is a disjoint union of $(B_n)_{n\in\NN}$,
    $$\mu(A_0)=\lim_{N\rightarrow +\infty} \sum_{n=0}^{N-1}\mu(B_n)=\lim_{N\rightarrow +\infty}\left(\mu(A_0)-\mu(A_N)\right).$$
    So 
    $$ \lim_{N\rightarrow +\infty} \mu(A_N)=0.$$
    (3)$\Rightarrow$(4)Let  $(f_n)_{n\in\NN}$ be a decreasing in $S$ converging pointwise to $0$ (as a mapping).
    $$ B\coloneq \{\omega\in \Omega \mid f_0 (\omega )>0\}\in \mathcal{A},\ M=\max\{f_0(\omega)\mid \omega\in \Omega\}.$$
    For any $\varepsilon>0$, $n\in\NN$, $A_n^\varepsilon\coloneq\{\omega\in \Omega\mid f_n(\omega)\ge\varepsilon\}\in\mathcal{A}$, $(A_n^\varepsilon)_{n\in\NN}$ decreasing in $\mathcal{A}$. Since $\dis \lim_{n\rightarrow +\infty}f_n=0$, $\dis \bigcap_{n\in\NN}A_n^\varepsilon=\varnothing$. 
    $$ \forall n\in\NN, \ 0\le f_n\le \varepsilon \mathbbm{1}_B +M \mathbbm{1}_{A_n^\varepsilon},$$
    $$ \forall n\in\NN, 0\le I(f_n)\le \varepsilon I(\mathbbm{1}_B)+M\mu(A_n^\varepsilon).$$
    (4)$\Rightarrow$(1) Let $(C_n)_{n\in\NN}$ pairwise disjoint in $\mathcal{C}$, $A\coloneq \bigcup_{n\in\NN}C_n\in \mathcal{C}$.
    $\forall n\in \NN, f_n\coloneq \sum_{k=0}^{n}\mathbbm{1}_{C_k}$ (increasing) converging pointwise to $\mathbbm{1}_A$. $(\mathbbm{1}_A-f_n)_{n\in\NN}\rightarrow 0$ pointwise.
    $$ \mu(A)\coloneq I(\mathbbm{1}_A)=\lim_{n\rightarrow +\infty} I(f_n)=\lim_{n\rightarrow +\infty} \mu(C_k)=\sum_{k\in \NN}\mu(C_k).$$
\end{proofenv}
\begin{propositionenv}
    Let $\varphi:\RR\longrightarrow \RR$ be an increasing right continuous mapping. Let $\mathcal{C}\subseteq \mathscr{P}(\RR)$ consisting of $\interval[open left]{a}{b}$, $a,b\in \RR$, $a<b$, $S\subseteq \RR^\RR$ vector subspace generated by $\mathbbm{1}_A$, $A\in \mathcal{C}$. $I_\varphi: S\longrightarrow \RR$ be $\RR$-linear mapping, $I_\varphi(\mathbbm{1}_{\interval[open left]{a}{b}})\coloneq \varphi(b)-\varphi(a).$ $\mathcal{A}\subseteq \mathscr{P}(\RR)$. 
    $$ \mathcal{A}\coloneq \{A\mid A=C_1\cup \cdots\cup C_n,\ C_1,\cdots,C_n\in \mathcal{C} \}.$$
    (1) Let $\varepsilon>0$, $A\in \mathcal{A}$, $A\neq\varnothing$. There exists $B\in \mathcal{A}$, $\varnothing\neq \overline{B}\subseteq A$ and $I_\varphi(\mathbbm{1}_A)-I_\varphi(\mathbbm{1}_B)\le \varepsilon$.
    \newline
    (2) $I_\varphi$ is an integral operator.
    \newline
    $I_\varphi$ is typically called Steltjes integral.
\end{propositionenv}
\begin{proofenv}
    \ \newline
    (1) \newline
    (A) $A\in \mathcal{C}$, $A=\interval[open left]{a}{b}$. There exists $a'\in \interval[open]{a}{b}$ such that $\varphi(a')-\varphi(a)\le \varepsilon$ (by right continuity of $\varphi$).
    $B\coloneq \interval[open left]{a'}{b}$, $\overline{B}=[a',b]\subseteq \interval[open left]{a}{b}=A$. $I_\varphi(\mathbbm{1}_B)=\varphi(b)-\varphi(a)$. So $I_\varphi(\mathbbm{1}_A)-I_\varphi(\mathbbm{1}_B)\le \varepsilon$.
    \newline
    (B)
    $A\in \mathcal{A}$, $A=A_1\cup\cdots\cup A_n$, $\{A_1,\cdots,A_n\}\subseteq\mathcal{C}$, for any $i\in\{1,\cdots,n\}$, $B_i\in \mathcal{C}$, $\varnothing\neq \overline{B}_i\subseteq A_i$, $I_\varphi(\mathbbm{1}_{A_i})-I_\varphi(\mathbbm{1}_{B_i})\le \frac{\varepsilon}{n}.$ $B\coloneq \bigcup B_i$.
    $$ I_\varphi (\mathbbm{1}_A)-I_\varphi(\mathbbm{1}_B)=\sum_{i=1}^{n}\left(I_\varphi(\mathbbm{1}_{A_i})-I_\varphi(\mathbbm{1}_{B_i})\right)\le \varepsilon.$$
    (2) Let $(A_n)_{n\in\NN}$ be a decreasing sequence in $\mathcal{A}$ such that $\bigcap_{n\in\NN}A_n=\varnothing$, fix $\varepsilon>0$, $n\in\NN$, $B_n\in\mathcal{A}$,
    $\overline{B}_n\subseteq A_n$ connected and non-empty, $I_\varphi(\mathbbm{1}_{A_n})-I_\varphi(\mathbbm{1}_{B_n})\le \frac{\varepsilon}{2^n}.$ $\forall n\in\NN$, $C_n\coloneq B_0 \cap \cdots \cap B_n\subseteq\overline{B_0}\cap \cdots \cap \overline{B}_n$ ($\varnothing=\bigcap_{n\in\NN} A_n\supseteq \bigcap_{n\in\NN}\overline{B}_n=\varnothing$). 
    So 
    $$ \bigcap_{n=0}^{N}=\overline{B}_n=\varnothing.$$
    $ B_n\backslash C_n=B_n\backslash \left(B_n\cap C_{n-1}\right)=B_n\backslash C_{n-1}\subseteq A_n\backslash C_{n-1}\subseteq A_{n-1}\backslash C_{n-1}$.
    $$ I_\varphi(\mathbbm{1}_{A_n\backslash C_n})=I_\varphi\left(\mathbbm{1}_{B_n\backslash C_n}\right)+I_\varphi\left(\mathbbm{1}_{A_n\backslash B_n}\right)\le I_\varphi\left(\mathbbm{1}_{A_{n-1}\backslash C_{n-1}}\right)+\frac{\varepsilon}{2^n}.$$
    $\forall n\in \NN$, $I_\varphi(\mathbbm{1}_{A_n})\le \frac{\varepsilon}{2^n}+\frac{\varepsilon}{2^{n-1}}+\cdots+\frac{\varepsilon}{2}\le \varepsilon.$
    So $\dis \lim_{n\rightarrow +\infty}I_\varphi(\mathbbm{1}_{A_n})=0.$
\end{proofenv}

\section{Measurable Space}
\begin{definitionenv}
    Let $\Omega$ be a set. $\sigma$-\textbf{algebra} on $\Omega$ is any $\mathcal{A}\subseteq \mathscr{P}(\Omega)$ such that
    \newline
    (1) For any $I$ countable, $(A_i)_{i\in I}\in \mathcal{A}^{I}$, $\dis \bigcup_{i\in I}A_i\in \mathcal{A}$.
    \newline
    (2) $A\in \mathcal{A}$, then $\Omega\backslash A\in \mathcal{A}$.
    \newline
    Note that (1) implies that $\varnothing\in \mathcal{A}$, in addition of (2), $\Omega\in \mathcal{A}$.
    $$ \bigcap_{i\in I}A_i=\Omega \backslash \left(\bigcup_{i\in I}\left(\Omega\backslash A_i\right)\right)\in \mathcal{A}.$$
    If $\mathcal{A}$ is a $\sigma$-algebra on set $\Omega$, then we call $(\Omega,\mathcal{A})$ a \textbf{measurable space}.
\end{definitionenv}
\begin{propositionenv}
    Let $I$ be a countable set, $\Omega$ be a set, $J\neq \varnothing$, $(\mathcal{A}_{j})_{j\in J}$, $\mathcal{A}_j$ is a $\sigma$-algebra on $\Omega$, then $\dis \mathcal{A}\coloneq \bigcap_{j\in J}\mathcal{A}_j$  is a $\sigma$-algebra on $\Omega$.
\end{propositionenv}
\begin{proofenv}
    Let $(A_i)_{i\in I}\in \mathcal{A}^I$. $\forall j\in J$, $(A_i)_{i\in I}\in \mathcal{A}_j^I$, so $\bigcup_{i\in I} A_i\in \mathcal{A}_j$ and hence
    $$ \bigcup_{i\in I} A_i\in \bigcap_{j\in J}\mathcal{A}_j=\mathcal{A}.$$
    $A\in \mathcal{A}$, so $\forall j\in J$, $A\in \mathcal{A}_j$, so $\Omega\backslash A\in A_j$ and hence
    $$ \Omega\backslash A\in \bigcap_{j\in J} \mathcal{A}_j =\mathcal{A}.$$
\end{proofenv}
\begin{exampleenv}
    Let \(\Omega\) be a set. Then \(\mathscr{P}(\Omega)\) is a \(\sigma\)-algebra on \(\Omega\). Moreover, if \(\mathcal{C}\) is a subset of \(\mathscr{P}(\Omega)\), we denote by \(\sigma(\mathcal{C})\) the intersection of all \(\sigma\)-algebras containing \(\mathcal{C}\). It is a \(\sigma\)-algebra on \(\Omega\), which is the smallest \(\sigma\)-algebra containing \(\mathcal{C}\). We call it the \(\sigma\)-algebra generated by \(\mathcal{C}\).
\end{exampleenv}
\begin{exampleenv}
    Let $(X,\mathscr{T})$ be a topological space. The $\sigma$-algebra generated by $\mathscr{T}$ is called the \textbf{Borel $\sigma$-algebra} on $(X,\mathscr{T})$.
\end{exampleenv}
\begin{propositionenv}
    Let $A\subseteq [-\infty,+\infty]$. We define a binary relation on $A$: $(x,y)\in A\times A$, $x\sim y$ if and only if there exists an interval $J$ contained in $A$ such that $\{x,y\}\subseteq J$. Then $\sim $ is an equivalent relation on $A$. Each equivalent class is an interval.
\end{propositionenv}
\begin{proofenv}
    By definition, if $x\sim y$ and $y\sim z$, $(J_1,J_2)\in A^2$, $\{x,y\}\subseteq J_1$, $\{y,z\}\subseteq J_2$, so $\varnothing\neq J_1\cap J_2\subseteq A$, $\{x,z\}\subseteq J_1\cup J_2$, so $x\sim z$.

    Let $\alpha$ be an equivalent class and $(x,y)\in \alpha^2$, $x<y$. $\{x,y\}\subseteq J\subseteq A$, hence $[x,y]\subseteq J$. Hence $\forall z\in [x,y]$, $z\in \alpha$. So $\alpha$ is an interval.
\end{proofenv}
\begin{remark}\label{10.7.6}
    Equip $[-\infty,+\infty]$ with order topology, $U\subseteq [-\infty,+\infty]$ open. Consider $\sim$ for $U$, $\alpha$ an equivalent class. For any $x\in \alpha$, there exists a neighborhood of $x$ contained in $U$, which is an interval.
    Hence $\alpha$ is a neighborhood of $x$. In particular, $\alpha$ is open. $U$ is a disjoint union of open intervals. Any open interval in $[-\infty,+\infty]$ contains a rational number. Hence any open $U\subseteq[-\infty,+\infty]$ is a disjoint union of countable open intervals.
\end{remark}
\begin{exampleenv}
    We equip the extended real line $\interval[open left]{-\infty}{+\infty}$ with the order topology $\mathscr{T}$. Then its Borel $\sigma$-algebra $\sigma(\mathscr{T})$ is generated by intervals of the form $\interval[open left]{-\infty}{b}$ for $b \in \mathbb{Q}$. In fact, if we denote by $\mathcal{A}$ the $\sigma$-algebra
\[
\sigma(\{\interval[open left]{-\infty}{b} \mid b \in \mathbb{Q}\}),
\]
then by definition one has $\mathcal{A} \subseteq \sigma(\mathscr{T})$.

Conversely, for any $x \in \mathbb{R} \cup \{+\infty\}$, one has
\[
\interval[open left]{-\infty}{x} = \bigcup_{b \in \mathbb{Q},\, b < x} \interval[open left]{-\infty}{b} \in \mathcal{A}.
\]

We then deduce that, for any $x \in \interval[open left]{-\infty}{+\infty}$, one has
\[
\interval[open right]{x}{+\infty} = \interval[open left]{-\infty}{+\infty} \setminus \interval[open left]{-\infty}{x} \in \mathcal{A}.
\]

Finally, for any $x \in \mathbb{R} \cup \{-\infty\}$, one has
\[
\interval{x}{+\infty} = \bigcup_{a \in \mathbb{Q},\, a > x} \interval[open right]{a}{+\infty} \in \mathcal{A}.
\]

Moreover, for any $(a, b) \in \interval[open left]{-\infty}{+\infty}^2$ such that $a < b$, one has
\[
\interval{a}{b} = \interval[open left]{-\infty}{b} \cap \interval{a}{+\infty}.
\]

Therefore, all intervals that are open in $\interval[open left]{-\infty}{+\infty}$ belongs to $\mathcal{A}$. Finally, by Remark \ref{10.7.6}, we obtain that any open subset of $\interval[open left]{-\infty}{+\infty}$ belongs to $\mathcal{A}$.
\end{exampleenv}
\begin{definitionenv}
    Let $(\Omega_1,\mathcal{A}_1)$, $(\Omega_2,\mathcal{A}_2)$ be measurable spaces, $f:\Omega_1\longrightarrow \Omega_2$ be a mapping. We say that $f$ is $\mathcal{A}_1$ measurable, if 
    $$ \forall A\in \mathcal{A}_2,\ f^{-1}(A)\in \mathcal{A}_1.$$
\end{definitionenv}
\begin{remark}
    Let $(\Omega,\mathcal{A})$ be a measurable space, $A\in \mathcal{A}$, then $\mathbbm{1}_A$ is measurable.
\end{remark}
\begin{propositionenv}
    Let $(\Omega_1,\mathcal{A}_1)$, $(\Omega_2,\mathcal{A}_2)$, $(\Omega_2,\mathcal{A}_3)$ be measurable spaces. $f:\Omega_1\longrightarrow \Omega_2$, $g:\Omega_2\longrightarrow \Omega_3$ be mappings. If $f$ and $g$ are measurable, then $f\circ g$ is measurable.
\end{propositionenv}
\begin{proofenv}
    Let $A\in \mathcal{A}_3$, $g^{-1}(A)\in \mathcal{A}_2$, $(g\circ f)^{-1}(A)=f^{-1}(g^{-1}(A))\in \mathcal{A}_1$.
\end{proofenv}
\begin{propositionenv}
    Let $(\Omega,\mathcal{A})$ be a measurable space, $X$ be a set, $\mathcal{C}\subseteq \mathscr{P}(X)$, $f:\Omega\longrightarrow X$ be a mapping. If $\forall A\in \mathcal{C}$, $f^{-1}(A)\in \mathcal{A}$, then $f$ is measurable for $X$ considered $\sigma(\mathcal{C})$. In particular, continuous mappings are measurable (Borel $\sigma$-algebras)
\end{propositionenv}
\begin{proofenv}
    Let 
    $$\mathcal{A}'\coloneq \{A\in \mathscr{P}(X)\mid f^{-1}(A)\in \mathcal{A}\}, \mathcal{C}\subseteq \mathcal{A}'.$$
    We will show that $\mathcal{A}'$ is a $\sigma$-algebra. 

    \quad Let $I$ be a countable set, $(A_i)_{i\in I}\in \mathcal{A}'^{I}$,
    $$ f^{-1}\left(\bigcup_{i\in I}A_i\right)=\bigcup_{i\in I}f^{-1}(A_i)\in \mathcal{A}.$$
    So, $\dis \bigcup_{i\in I}A_i\in \mathcal{A}'$. For $A\in \mathcal{A}'$,
    $$ f^{-1}\left(X\backslash A\right)=\Omega\backslash f^{-1}(A)\in \mathcal{A}.$$
    So, $X\backslash A\in \mathcal{A}'$. Thus $\mathcal{A}'$ is a $\sigma$-algebra, $\sigma(\mathcal{C})\subseteq \mathcal{A}'$.
\end{proofenv}
\begin{definitionenv}
    Let $\Omega$ be a set, $(E_i,\mathcal{E}_i)_{i\in \Theta}$ be measurable spaces. 
    
    \quad For any $i\in \Theta$, fix $f_i:\Omega\longrightarrow E_i$, $f=(f_i)_{i\in \Theta}$.
    $$ \sigma(f)\coloneq \sigma\left(\bigcup_{i\in \Theta}\{f_i^{-1}(A_i),\ A_i\in \mathcal{E}_i\}\right).$$
    It is the smallest $\sigma$-algebra make all $f_i$ measurable. If $\dis \Omega=\prod_{i\in \Theta}E_i$, $f_i=\pi_i$ be the projection mapping, then $\sigma(f)$ is called the product $\sigma$-algebra of $(\mathcal{E}_i)_{i\in \Theta}$, denoted as 
    $$ \bigotimes_{i\in \Theta} \mathcal{E}_i\coloneq \sigma(f),\ f=(f_i)_{i\in \Theta}=(\pi_i)_{i\in \Theta}.$$
\end{definitionenv}
\begin{propositionenv}
    Let $X$ be a set, $(E_i,\mathcal{E}_i)_{i\in \Theta}$ be measurable spaces, $f_i:X\longrightarrow E_i$ be mappings. $(X,\sigma(f))$, $(\Omega, \mathcal{A})$ measurable spaces, $g:\Omega\longrightarrow X$ a mapping. $g$ is measurable if and only if $f_i\circ g$ is measurable for any $i\in \Theta$.
\end{propositionenv}
\begin{proofenv}
    $f_i$ is measurable by definition. If $g$ is measurable, then $f_i\circ g$ is measurable. Conversely, if $f_i\circ g$ is measurable, $A_i\in \mathcal{E}_i$,
    $$ g^{-1}\left(f_i^{-1}\left(A_i\right)\right)=\left(f_i\circ g\right)^{-1}(A_i)\in \mathcal{A}.$$
    So $g$ is measurable.
\end{proofenv}
\begin{remark}
    Let $(X_1,\mathscr{T}_1),\cdots,(X_n, \mathscr{T}_n)$ be topological spaces, $X=X_1\times \cdots\times X_n$. The product topology on $X$ was generated by 
    $U_1\times \cdots \times U_n \text{ with } U_i\in \mathscr{T}_i, X\rightarrow X_i \text{ continuous and measurable, }i\in\{1,\cdots,n\}.$
    The Borel $\sigma$-algebra contains the product of $\sigma$-algebra $\dis \bigotimes_{i\in \Theta}(\mathscr{T}_i)$. They are equal if for all $i\in \Theta$, $\mathscr{T}_i$ admits a countable basis $\mathcal{B}_i$.
    In this case, $\{U_1\times \cdots\times U_n\mid (U_1,\cdots, U_n)\in \mathcal{B}_1\times\cdots\times \mathcal{B}_n\}$ generates the product topology, so any open set of $X$ belongs to $\dis \bigotimes_{i\in \Theta}\sigma(\mathscr{T}_i)$ or $\RR$, $\mathcal{B}=\{\interval[open]{p}{q}\mid (p,q)\in \QQ^2, p<q\}$. The product $\sigma$-algebra on $\RR^n$ equals the Borel $\sigma$-algebra on $\RR^n$.
    If $(\Omega,\mathcal{A})$ is a measurable space, $f,g:\Omega\longrightarrow \RR^n$ measurable, then $f+g$ and $f\cdot g$ are measurable.
\end{remark}
\begin{notationenv}
    Let $\Omega$ be a set, $f:\Omega\longrightarrow [-\infty,+\infty]$ be a mapping, $P$ be a condition on $[-\infty,+\infty]$, $\{P(f)\}$ denotes the set
    $$ f^{-1}\left(\{t\in [-\infty,+\infty]\mid P(t)\}\right)= \{\omega\in \Omega\mid P(f(\omega))\}.$$
    For example, $\{f>0\}=\{\omega\in \Omega\mid  f(\omega)>0\}$.
\end{notationenv}
\begin{remark}
    Let $\Omega$ be a set, $f:\Omega\longrightarrow \RR_{\ge 0}$ be a mapping, $\forall n\in \NN$,
    $$f_n\coloneq \sum_{k=1}^{n\cdot2^n-1} \frac{k}{2^n}\mathbbm{1}_{\{\frac{k}{2^n}\le f\le \frac{k+1}{2^n}\}}+\mathbbm{1}_{\{f\ge n\}}.$$
\end{remark}
\begin{theoremenv}
    Let $(\Omega,\mathcal{A})$ be a measurable space, $(X,\mathscr{T})$ be a topological space such that $\mathscr{T}$ is given by metric $\dd$. $(f_n)_{n\in\NN}$ is measurable. If $(f_n)_{n\in\NN}$ converges pointwise to some $f:\Omega\longrightarrow X$, then $f$ is measurable.
\end{theoremenv}
\begin{proofenv}
    Let $Y\subseteq X$ be a closed subset, $\dd(\cdot,Y):X\longrightarrow \RR_{\ge 0}$ continuous (Lipschitzian).
    \begin{align*}
        f^{-1}(Y)\coloneq &\{\omega\in \Omega \mid \lim_{n\rightarrow +\infty} \dd(f_n(\omega),Y) = 0\}\\
        =&\{\omega\in \Omega\mid \limsup_{n\rightarrow +\infty} \dd(f_n(\omega),Y) = 0\}\\
        =& \bigcap_{m\in\NN_{\ge 1}}\bigcup_{N\in \NN}\bigcap_{n\in\NN_{\ge N}}\{\omega\in \Omega\mid \dd(f_n(\omega), Y)<m^{-1}\}.
    \end{align*}
    Since $f_n$ is measurable and $\dd(\cdot,Y)$ is continuous (so measurable).
    $$ \{\omega\in \Omega\mid \limsup_{n\rightarrow+\infty} \dd(f_n(\omega,Y)) = 0\}\in \mathcal{A}.$$
\end{proofenv}
\newpage
\section{Monotone Class Theorem}
\begin{lemmaenv}
    Let $n\in\NN_{\ge 1}$, and
    $$\begin{array}{rrcl}
        \varphi_n:&\RR_{\ge 0}&\longrightarrow& \RR\\
        & x&\longmapsto &x^n
    \end{array}$$
    be a mapping.
    \newline
    (1) $\varphi_n$ is convex.
    \newline
    (2) $\forall x\in \RR_{\ge 0}$,
    $$ \varphi_n(x)=x^n=\sup_{a\in \QQ_{\ge 0}} \max\{na^{n-1}x-(n-1)a^n,0\}.$$
\end{lemmaenv}
\begin{proofenv}
    \ \newline
    (1) $\varphi''(x)=n(n-1)x^{n-2}\ge 0$.
    \newline
    (2) $\forall a\ge 0$, 
    $$ \varphi_n(x)\ge \varphi_n(a)+\varphi'_n(a)(x-a)=a^n+na^{n-1}(x-a).$$
    Since $\QQ$ is dense in $\RR$ and $\varphi_n$ is continuous, so (2) holds.
\end{proofenv}
\begin{definitionenv}
    Let $\Omega$ be a set and $\mathcal{H}$ be a family of bounded mappings from $\Omega$ to $\RR_{\ge 0}$. We say that $\mathcal{H}$ is a \textbf{$\lambda$-family}\footnote{The following monotone class theorem is named by weaker definition: monotone class (of functions) that do not need the condition of non-negative.} if the following conditions are satisfied:
    \newline
    (1) $\mathbbm{1}_{\Omega}\in \mathcal{H}$.
    \newline
    (2) If $(f,g)\in \mathcal{H}\times \mathcal{H}$, $(a,b)\in \RR_{\ge 0}\times \RR_{\ge 0}$, then $af+bg\in \mathcal{H}$.
    \newline
    (3) If $(f,g)\in \mathcal{H}\times \mathcal{H}$, $f\le g$, then $g-f\in \mathcal{H}$.
    \newline
    (4) If $(f_n)_{n\in \NN}$ is an increasing and uniformly bounded sequence in $\mathcal{H}$, then
    $$ \lim_{n\rightarrow +\infty} f_n \in \mathcal{H}.$$
\end{definitionenv}
\begin{remark}
    Let $(f_i)_{i\in \Theta}$ is a countable and uniformly bounded family in $\mathcal{H}$. Assume that $\forall(f,g)\in \mathcal{H}$, $\sup\{f,g\}\in \mathcal{H}$, then 
    $$ \sup_{i\in \Theta} f_i \in \mathcal{H}.$$
    In fact, we assume that $\Theta=\NN$. For any $n\in \NN$, let $g_n=\sup\{f_0,\cdots,f_n\}$, then $(g_n)_{n\in \NN}\in \mathcal{H}^\NN$ is increasing, uniformly bounded, and converges to $\sup_{i\in \Theta}f_i\in \mathcal{H}$.
\end{remark}
\begin{theoremenv}[Monotone class theorem]
    Let $\Omega$ be a set. Let $\mathcal{H}$ be a $\lambda$-family on $\Omega$. If $\forall (f,g)\in \mathcal{H}\times \mathcal{H}$, $\inf\{f,g\}\in \mathcal{H}$. Then any bounded $\sigma(\mathcal{H})$-measurable mapping from $\Omega$ to $\RR_{\ge 0}$ belongs to $\mathcal{H}$.
\end{theoremenv}
\begin{proofenv}
    $\forall (f,g)\in \mathcal{H}\times \mathcal{H}$, $\sup\{f,g\}=f+g-\inf\{f,g\}\in \mathcal{H}$.
    Moreover, for any $a\in \RR_{\ge 0}$, $\forall f\in \mathcal{H}$,
    $$ \sup\{ f-a\mathbbm{1}_{\Omega},0\}=\sup\{f,a\mathbbm{1}_\Omega\}-a\mathbbm{1}_{\Omega}\in \mathcal{H}.$$
    We then deduce by the lemma that $\forall n\in\NN_{\ge 1}$, $f^n\in \mathcal{H}$.
    In fact, by the lemma
    $$f^n=\sup_{a\in \QQ_{>0}}\left(\sup\{na^{n-1}f-(n-1)a^n\mathbbm{1}_\Omega,0\}\right)\in \mathcal{H}.$$

    Let $\mathcal{A}=\{A\in\mathscr{P}(\Omega)\mid \mathbbm{1}_{A}\in \mathcal{H}\}$.
    \newline
    (1) If $A\in \mathcal{A}$, $\mathbbm{1}_{\Omega\backslash A}=\mathbbm{1}_{\Omega}-\mathbbm{1}_{A}\in \mathcal{H}$.
    \newline
    (2) If $(A_i)_{i\in \Theta}\in \mathcal{A}^\Theta$, with $\Theta$ countable, $\dis A=\bigcup_{i\in \Theta}A_i$. $\mathbbm{1}_{A}=\sup_{i\in \Theta}\mathbbm{1}_{A_i}\in\mathcal{H}$.
    Therefore $\mathcal{A}$ is a $\sigma$-algebra contained in $\sigma(\mathcal{H})$.

    \quad Let $f\in \mathcal{H}$, $t>0$. One has $\inf\{\mathbbm{1}_{\Omega}, t^{-1}f\}\in \mathcal{H}$.
    So $\forall n\in \NN_{\ge 1}$, $\inf\{\mathbbm{1}_{\Omega}, t^{-1}f\}^n\in \mathcal{H}$.
    $$ \left(\mathbbm{1}_{\Omega}-\inf\{\mathbbm{1}_{\Omega},t^{-1} f\}^n\right)_{n\in \NN} $$
    is increasing and converges to $\mathbbm{1}_{\{f<t\}}$. 
    So $\mathbbm{1}_{\{f<t\}}\in \mathcal{H}$, $\{f<t\}\in \mathcal{A}$, so $f$ is $\mathcal{A}$-measurable.
    (In fact, $\sigma(\mathcal{H})=\mathcal{A}$. By definition, $\sigma(\mathcal{H})$ is the smallest set making all the mappings in $\mathcal{H}$ measurable, so $\sigma(\mathcal{H})$ is contained in $\mathcal{A}$.)
\end{proofenv}
\begin{theoremenv}
    Let $\Omega$ be a set and $\mathcal{H}$ be a $\lambda$-family on $\Omega$. Let $\mathcal{H}_0\subseteq \mathcal{H}$. Suppose that $\forall(f,g)\in \mathcal{H}_0\times \mathcal{H}_0$, $fg\in \mathcal{H}_0$. Then any bounded $\sigma(\mathcal{H}_0)$-measurable mapping $\Omega\longrightarrow \RR_{\ge 0}$ belongs to $\mathcal{H}$.
\end{theoremenv}
\begin{proofenv}
    We may assume that $\mathcal{H}$ is the smallest $\lambda$-family containing $\mathcal{H}_0$ (by taking the intersection of all $\lambda$-families containing $\mathcal{H}_0$).
    Let $\mathcal{H}_1=\{f\in \mathcal{H}\mid \forall g\in \mathcal{H}_0, fg\in \mathcal{H}\}\supseteq \mathcal{H}_0.$
    Moreover, $\mathcal{H}_1$ is a $\lambda$-family, so $\mathcal{H}_1=H$. Let $\mathcal{H}_2=\{f\in \mathcal{H}\mid \forall g\in \mathcal{H},fg\in \mathcal{H}\}\supseteq\mathcal{H}_0$. $\mathcal{H}_2$ is a $\lambda$-family, so $\mathcal{H}_2=\mathcal{H}$.
    Therefore, $\mathcal{H}$ is stable by multiplication.

    Let $(f,g)\in \mathcal{H}\times\mathcal{H}$, $|f-g|\le 1$.
    $$ (f-g)^2=f^2+g^2-2fg\in \mathcal{H}.$$
    $(z\longmapsto z-\frac{1}{2}z^2)$ is increasing on $[0,1]$. Let $(f_n)_{n\in \NN}$ be the sequence of mappings defined as $f_0=0$, $f_{n+1}=f_n+\frac{1}{2}\left((f-g)^2-f_n^2\right)$.
    We prove by induction that $f_n\in \mathcal{H}$ and $f_n\le |f-g|$.
    \newline
    $n=0$, $f_0=0\in \mathcal{H}$ and $f_0\le |f-g|$. Suppose $f_n\in \mathcal{H}$, $f_n\le |f-g|$.
    Then $f_{n+1}\in \mathcal{H}$, $f_{n+1}\ge f_n$.
    $$f_{n+1}=\varphi(f_n)+\frac{1}{2}(f-g)^2\le \varphi(|f-g|)+\frac{1}{2}(f-g)^2 = |f-g|.$$
    $(f_n)_{n\in \NN}$ converges to $|f-g|$. So $|f-g|\in \mathcal{H}$. $\inf\{f,g\}=\frac{1}{2}\left(f+g - |f-g|\right)\in \mathcal{H}$. So any bounded $\sigma(\mathcal{H})$-measurable mappings $\Omega\longrightarrow \RR_{\ge0}$ belongs to $\mathcal{H}$.
\end{proofenv}
\begin{theoremenv}
    Let $\Omega$ be a set, $\mathcal{L}$ be a Riesz space on $\Omega$. We assume that $\forall (f,g)\in \mathcal{L}\times \mathcal{L}^\uparrow$, $\inf\{f,g\}\in \mathcal{L}$.
    Then the following statements hold.
    \newline
    (1) $\forall (f,g)\in \mathcal{L}^\uparrow\times \mathcal{L}^\uparrow$, $f\le g$ and $g-f$ is well defined, then $g-f\in \mathcal{L}^\uparrow$. 
    \newline
    (2) Let $\mathcal{A}=\{A\in \mathscr{P}(\Omega)\mid \mathbbm{1}_{A}\in \mathcal{L}^\uparrow\}$.
    \begin{enumerate}
        \item If $(A,B)\in\mathcal{A}^2$, $B\backslash A\in \mathcal{A}$.
        \item If $\Theta$ is countable and $(A_i)_{i\in \Theta}\in \mathcal{A}^{\Theta}$, then $\dis \bigcup_{i\in \Theta}A_i\in \mathcal{A}$.
    \end{enumerate}
    (3) Let $f\in \mathcal{L}^\uparrow$, $f\ge 0$. Suppose that
    $$ B\coloneq\{\omega\in \Omega\mid f(\omega)>0\}\in \mathcal{A}.$$
    Then, $\forall t>0$, $\{\omega\in \Omega\mid 0<f(\omega)<t\}\in \mathcal{A}.$
    \newline
    (4) Let $f:\Omega\longrightarrow [0,+\infty]$ such that
    $$ B\coloneq \{\omega\in \Omega\mid f(\omega)>0\}\in \mathcal{A}.$$
    If $\forall t>0$, $A_t\coloneq\{\omega\in \Omega\mid 0<f(\omega)<t\}\in\mathcal{A}$, then $f\in \mathcal{L}^\uparrow$.
\end{theoremenv}
\begin{proofenv}
    \ \newline
    (1) Let $(g_n)_{n\in\NN}\in \mathcal{L}^\NN$ be an increasing sequence such that $\dis g=\lim_{n\rightarrow +\infty}g_n$.
    $\forall n\in \NN$, $\inf\{g_n,f\}\in \mathcal{L}$ and $\dis \lim_{n\rightarrow+\infty}\inf\{g_n,f\}=f$,
    $$ \left(g_n-\inf\{g_n,f\}\right)_{n\in \NN}=\left(\sup\{g_n-f,0\}\right)_{n\in \NN}$$
    is increasing and converges to $g-f$. So $(g-f)\in \mathcal{L}^\uparrow$.
    \newline
    (2)
    \begin{enumerate}
        \item $$ \mathbbm{1}_{A\cap B}=\inf\{\mathbbm{1}_A,\mathbbm{1}_B\}\in \mathcal{L}^\uparrow.$$
            $$ \mathbbm{1}_{B\backslash A} =\mathbbm{1}_{B}-\mathbbm{1}_{A\cap B}\in \mathcal{L}^\uparrow (\text{by (1)}).$$
        \item If $(A,B)\in \mathcal{A}^2$, then
        $$ \mathbbm{1}_{A\cup B}=\sup\{\mathbbm{1}_{A}, \mathbbm{1}_{B}\}\in \mathcal{L}^\uparrow.$$
        So $A\cup B\in \mathcal{A}$. The case where $\Theta$ is finite is true.
        We assume that $\Theta=\NN$. 
        $$ \mathbbm{1}_{\bigcup_{n\in \NN}A_n} =\lim_{n\rightarrow +\infty} \left(\sup\{\mathbbm{1}_{A_0},\cdots,\mathbbm{1}_{A_n}\}\right)\in\mathcal{L}^\uparrow.$$
    \end{enumerate}
    (3) $$\sup\{f-a\mathbbm{1}_{B},0\}=\sup\{f,a\mathbbm{1}_B\}-a\mathbbm{1}_{B}\in \mathcal{L}^\uparrow.$$
        $$ f^n=\sup_{a\in \QQ_{>0}} \sup\{na^{n-1}f-(n-1)a^n\mathbbm{1}_{B},0\}\in \mathcal{L}^\uparrow.$$
        $$ \inf\{t^{-1}f, \mathbbm{1}_B\}^n \in \mathcal{L}^\uparrow, \forall t>0,\ \forall n\in \NN_{\ge 1}.$$
        $\left(\mathbbm{1}_{B}-\inf\{t^{-1}f,\mathbbm{1}_{B}\}^n\right)_{n\in \NN_{\ge 1}}$ is increasing and converges to $\mathbbm{1}_{\{0<f<t\}}$.
        So, $\{0<f<t\}\in \mathcal{A}$.

(4)
Let $$f_n = \sum_{k=1}^{n2^n-1} \frac{k}{2^n} \mathbbm{1}_{\{ \frac{k}{2^n} \leq f < \frac{k+1}{2^n} \}}+ n \mathbbm{1}_{\{ f \geq n \}}=\sum_{k=1}^{n2^n-1} \frac{k}{2^n} \mathbbm{1}_{A_{\frac{k+1}{2}}\backslash A_{\frac{k}{2^n}}} + n \mathbbm{1}_{B\backslash A}\in \mathcal{L}^\uparrow. $$
Since \( f \) is the limit of \( (f_n)_{n \in \mathbb{N}} \), we obtain \( f \in \mathcal{L}^\uparrow \).
\end{proofenv}
\begin{remark}
    Let $\Omega$ be a set, $S$ be a Riesz space on $\Omega$. $I: S\longrightarrow \RR$ be an integral operator.
    If $(f,g)\in \mathcal{L}^1(I)\times \mathcal{L}^{1}(I)^\uparrow$,
    then
    $$ \inf\{f,g\}\in \mathcal{L}^1(I)^{\uparrow}\text{ and } I\left(\inf\{f,g\}\right)\le I(f)<+\infty.$$
    So, $\inf\{f,g\}\in \mathcal{L}^{1}(I)$ by Beppo Levi's theorem.
\end{remark}

\section{Measure Space}
\begin{definitionenv}
    Let $(\Omega,\mathcal{A})$ be a measurable space. We call \textbf{measure} on $(\Omega,\mathcal{A})$ any mapping $\mu:\mathcal{A}\longrightarrow [0,+\infty]$ such that
    \newline
    (1) $\mu(\varnothing)=0$.
    \newline
    (2) $\mu$ is $\sigma$-additive.
    \newline
    $(\Omega,\mathcal{A},\mu)$ is called a \textbf{measure space}.
\end{definitionenv}
\begin{remark}
    Let $(\Omega,\mathcal{A},\mu)$ be a measure space. Let $(A_n)_{n\in\NN}$ be an increasing sequence in $\mathcal{A}$ and $\dis A=\bigcup_{n\in \NN}A_n$.
    Then 
    $$ \mu(A)=\lim_{n\rightarrow +\infty} \mu(A_n).$$
    In fact, if we let $B_n=A_n\backslash A_{n-1}, (n\ge 1)$, $B_0,\cdots,B_n$ are pairwise disjoint and $B_0=A_0$.
    $$ A_n=B_0\cup \cdots \cup B_n, \mu(A_n)=\mu(B_0)+\cdots+ \mu(B_n).$$
    So,
    $$ \lim_{n\rightarrow +\infty} \mu(A_n)=\sum_{n\in \NN} \mu(B_n)=\mu\left(\bigcup_{n\in\NN}B_n\right)=\mu(A).$$
\end{remark}
\begin{propositionenv}
    Let $(\Omega,\mathcal{A},\mu)$ be a measure space. Let 
    $$\mathcal{C}\coloneq \{A\in \mathcal{A}\mid \mu(A)<+\infty\}.$$
    Then $\mathcal{C}$ is a semiring and $\mu|_\mathcal{C}$ is $\sigma$-additive.
\end{propositionenv}
\begin{proofenv}
    $\mu(\varnothing)=0\Rightarrow \varnothing\in \mathcal{C}$.
    If $(A,B)\in \mathcal{C}\times \mathcal{C}$, $\mu(A\cap B)\le \mu (A)$ ($\mu(A)=\mu(A\cap B)+\mu(A\backslash B)$).
    So $A\cap B\in \mathcal{C}$, $A\backslash B\in \mathcal{C}$.
\end{proofenv}
\begin{definitionenv}
    Let $(\Omega,\mathcal{A},\mu)$ be a measure space, 
    $$\mathcal{C}=\{A\in \mathcal{A}\mid \mu(A)<+\infty\},$$
    $$S=\mathrm{Vect}_{\RR}\left(\{\mathbbm{1}_{A}\mid A\in \mathcal{C}\}\right)\subseteq \RR^{\Omega},$$
    $$ I: S\longrightarrow \RR, I(\mathbbm{1}_{A})=\mu(A).$$
    We denote by $\mathcal{L}^1(\Omega,\mathcal{A},\mu)$ the set of $\mathcal{A}$-measurable mappings from $\Omega$ to $\RR$ such that belongs to $\mathcal{L}^1(I)$.
    If $f\in \mathcal{L}^1(\Omega, \mathcal{A},\mu)$, we denote by 
    $$ \int_{\Omega} f \dd\mu \text{ or } \int_{\Omega} f(\omega) \mu(\dd\omega)$$
    the valued of $I(f)$. If 
    $$ f:\Omega\longrightarrow \RR\cup\{+\infty\}$$
    is $\mathcal{A}$-measurable and bounded from below by the same element of $\mathcal{L}^1(\Omega,\mathcal{A},\mu)$.
    When $f\notin \mathcal{L}^1(\Omega,\mathcal{A},\mu)$. By convention, $\int_{\Omega} f \dd\mu\coloneq +\infty$.
\end{definitionenv}
\begin{propositionenv}
    For any $A\in \mathcal{A}$, $\dis \int_{\Omega} \mathbbm{1}_A \dd\mu=\mu(A)$.
\end{propositionenv}
\begin{proofenv}
    This is true by definition when $\mu(A)<+\infty$.
    We assume that $\mu(A)=+\infty$. We reason by contradiction that $\mathbbm{1}_{A}\in \mathcal{L}^1(\Omega,\mathcal{A},\mu)$.

    \quad We first show that $\forall f\in S$, $f>0$, one has $\inf\{\mathbbm{1}_{A},f\}\in S$.
    If
    $$f=\sum_{i=1}^{n}a_i\mathbbm{1}_{B_i},$$
    then 
    $$\inf\{\mathbbm{1}_A,f\}=\sum_{i=1}^{n}\min\{b_i,1\}\mathbbm{1}_{A\cap B_i}\in S.$$
    Since $\mathbbm{1}_A\in \mathcal{L}^{1}(\Omega,A,\mu)$, $\exists g\in S^\uparrow$, $\mathbbm{1}_{A}\le g$. $I(g)<+\infty$.
    Let $(f_n)_{n\in \NN}\in S^\NN$ increasing such that $g=\lim_{n\rightarrow +\infty} f_n$ (we may assume).
    Then $\left(\inf\{f_n,\mathbbm{1}_A\}\right)_{n\in \NN}\in S^\NN$ is increasing and converges to $\mathbbm{1}_A$. Let $A_n=\{\omega\in A\mid f_n(\omega)\ge 1\}\in \mathcal{C}$, $(A_n)_{n\in \NN}$ is increasing. $A=\bigcup_{n\in \NN}A_n$.
    $$ \mu(A)=\lim_{n\rightarrow +\infty} \mu(A_n)\le \int_{\Omega} g \dd\mu<+\infty.$$
    Contradiction.
\end{proofenv}
\begin{propositionenv}
    Let $(\Omega,\mathcal{A},\mu)$ be a measure space,
    \newline
    (1) Let $f:\Omega\longrightarrow \RR_{\ge 0}$ be an $\mathcal{A}$-measurable mapping.
    If $f\in \mathcal{L}^1(\Omega,\mathcal{A},\mu)$, then $\forall t\in \RR_{\ge 0}$, $\mu(\{\omega\in \Omega \mid f(\omega)\ge t\})<+\infty$.
    \newline
    (2) Let $f,g$ be measurable mappings from $\Omega$ to $\RR$ such that $f\le g$. If $g\in \mathcal{L}^1(\Omega,\mathcal{A},\mu)$ then $f\in \mathcal{L}^1(\Omega,\mathcal{A},\mu)$
    \newline
    (3) An $\mathcal{A}$-measurable mapping $f:\Omega \longrightarrow \RR$ belongs to $\mathcal{L}^1(\Omega,\mathcal{A},\mu)$ if and only if $|f|\in \mathcal{L}^{1}(\Omega,\mathcal{A},\mu)$.
\end{propositionenv}
\begin{proofenv}
    Apply monotone class theorem to $\mathcal{L}^1(I)$.
    \newline
    (1) $\mathbbm{1}_{0< f<x}\in\mathcal{L}^1(I)^\uparrow$, $\forall x>0$. If $0<t<x$,
    $$ \mathbbm{1}_{\{t\le f<x\}}=\mathbbm{1}_{\{0<f<x\}}-\mathbbm{1}_{\{0<f<t\}}\in \mathcal{L}^1(I)^\uparrow.$$
    So, 
    $$ \mathbbm{1}_{f\ge t}=\lim_{n\rightarrow+\infty} \mathbbm{1}_{\{t\le f<t+n\}}\in \mathcal{L}^1(I)^\uparrow.$$
    Since $t\mathbbm{1}_{t\le f}\le f$.
    So, $t \mu(\{t\le f\})=I(t \mathbbm{1}_{\{t\le f\}})\le I(f)<+\infty$. So $\mu(\{f\ge t\}<+\infty)$.
    \newline
    (2) By (1), $\forall a>0$, $\mu(\{g\ge a\})<+\infty$. So $\forall a>0$, $\mu(\{f\ge a\})<+\infty$.
    So,
    $$ \mathbbm{1}_{\{f>0\}}=\sup_{a\in \QQ_{>0}}\{\mathbbm{1}_{f\ge a}\}\in \mathcal{L}^1(I)^\uparrow.$$
    For $0<a<t$, $\mu(\{a\le f<t\})\le \mu(\{f\ge a\})<+\infty$.
    So $\mathbbm{1}_{\{0<f<t\}}\in \mathcal{L}^1(I)$. 
    Therefore, (by monotone class theorem) $f\in \mathcal{L}^1(I)^\uparrow$.
    Hence $I(f)\le I(g)<+\infty$. So $f\in \mathcal{L}^(I)$.
    \newline
    (3) $\Rightarrow$: $|f|=\sup\{f,0\}=\inf\{f,0\}\in \mathcal{L}^1(I)$, if $f\in \mathcal{L}^1(I)$.

    $\Leftarrow$: $$ 0\le \sup\{f,0\}\le |f|,\ 0\le \sup\{-f,0\}\le |f|.$$
    So $\sup\{f,0\}$, $\sup\{-f,0\}\in \mathcal{L}^1(\Omega,\mathcal{A},\mu).$
    Hence, 
    $$ f=\sup\{f,0\}-\sup\{-f,0\}\in \mathcal{L}^{1}(\Omega,\mathcal{A},\mu).$$


\end{proofenv}
\begin{notationenv}
    Let $(\Omega,\mathcal{A},\mu)$ be a measure space, $f\in \mathcal{L}^1(\Omega,\mathcal{A},\mu)$. For any $A\in \mathcal{A}$, $|\mathbbm{1}_{A} f|\le |f|$.
    So $\mathbbm{1}_{A}f\in \mathcal{L}^1(\Omega,\mathcal{A},\mu)$. If $\mu=\mu_{\varphi}$, where $\varphi:[a,b]\longrightarrow \RR$  increasing and right continuous,
    $$ \int_{a}^{b} f \dd\mu_{\varphi} \text{ is written as } \int_{a}^{b} f(t)\dd(\varphi(t)).$$
    If $\varphi(t)=t$, it is also written as 
    $$ \int_{a}^{b} f(t) \dd t.$$
\end{notationenv}
\begin{definitionenv}
    Let $(\Omega,\mathcal{A},\mu)$ be measure space. If there exists $(A_n)_{n\in \NN}\in \mathcal{A}^\NN$ such that
    \newline
    (1) $\forall n\in \NN$, $\mu(A_n)<+\infty$.
    \newline
    (2) $\Omega=\bigcup_{n\in \NN}A_n$, then we say that $(\Omega,\mathcal{A}, \mu)$ is $\sigma$-finite.
\end{definitionenv}
\begin{remark}
    If $(\Omega,\mathcal{A},\mu)$ is $\sigma$-finite, then $\mathbbm{1}_{\Omega}\in \mathcal{L}^1(I)^\uparrow$.
\end{remark}
\begin{theoremenv}[Carathéodory]
    Let $\Omega$ be a set, $\mathcal{C}$ be a semiring on $\Omega$, $\mu:\mathcal{C}\longrightarrow \RR_{\le 0}$ be a $\sigma$-additive mapping. If there exists an increasing sequence $(A_n)_{n\in \NN}$ in $\mathcal{A}$ such that
    \newline
    (1) $\forall n\in \NN$, $\mu(A_n)<+\infty$.
    \newline
    (2) $\dis \Omega=\bigcup_{n\in \NN}A_n.$
    \newline
    Then $\mu$ extends in a unique way to a $\sigma$-finite measure on $\sigma(\mathcal{C})$.
\end{theoremenv}
\begin{proofenv}
    Let $S=\mathrm{Span}_{\RR}\left(\{\mathbbm{1}_A\mid A\in \mathcal{C}\}\right)$,
    $$\begin{array}{rrcl}
        I:&S&\longrightarrow &\RR\\
        & \mathbbm{1}_A &\longmapsto &\mu(A)
    \end{array}$$
    is an integral operator.
    $$ \mathcal{A}=\{A\in \mathscr{P}(\Omega)\mid \mathbbm{1}_{A} \in \mathcal{L}^1(I)^\uparrow\}$$
    is a $\sigma$-algebra since $\mathbbm{1}_{\Omega}\in S^\uparrow \subseteq \mathcal{L}^1(I)^\uparrow$ and $(A\in \mathcal{A})\longmapsto I(\mathbbm{1}_A)$ is a measure extending $\mu$.
    If $\nu$ is a measure on $\sigma(\mathcal{C})$ extending $\mu$.
    Let $S'=\mathrm{Span}_{\RR}\{\mathbbm{1}_A\mid A\in \sigma(\mathcal{C}), \nu(A)<+\infty\}$. One has $S\subseteq S'\subseteq \mathcal{L}^1(I)$.
    By Beppo Levi's theorem, the restriction of $I_{\nu}$ to $S^\uparrow$ and $S^\downarrow$ coincides with $I$. Therefore, $I=I_\nu$ and hence $\nu=\mu$ on $\sigma(\mathcal{C})$.
\end{proofenv}
\begin{box2}
\textbf{Application}
\newline
Let $\varphi:I\longrightarrow \RR$ be an increasing right continuous mapping, where $I\subseteq \RR$ is an interval. Let $\mathcal{C}=\{\interval[open left]{a}{b}\mid (a,b)\in I^2, a<b\}$.
$\mu_\varphi:\mathcal{C}\longrightarrow \RR_{\ge 0}$, $\interval[open left]{a}{b}\longmapsto \varphi(b)-\varphi(a)$ is $\sigma$-additive. So it extends to a measure $\mu_\varphi$ on $\sigma(\mathcal{C})$ (Borel $\sigma$-algebra on $I$).
\end{box2}

\begin{definitionenv}
    Let $(\Omega,\mathcal{A},\mu)$ be a measure space. We say that $A\in \mathscr{P}(\Omega)$ is $\mu$-negligeable if there exists $B\in \mathcal{A}$, $\mu(B)=0$ and $A\subseteq B$.
    (If $A\in\mathcal{A}$ and is $\mu$-negligeable, then $\mu(A)=0$).
\end{definitionenv}
\begin{propositionenv}
    If $f\in \mathcal{L}^1(\Omega,\mathcal{A}, \mu)$ and $A\in \mathcal{A}$, $\mu(A)=0$, then $\dis \int_{\Omega} \mathbbm{1}_A f\dd\mu=0$.
\end{propositionenv}
\begin{proofenv}
    We may assume that $f\in S=\mathrm{Span}_{\RR}\{\mathbbm{1}_{B}\mid B\in \mathcal{A},\mu(B)<+\infty\}$. When $f=\mathbbm{1}_B$, $B\in \mathcal{A}$, $\mu(B)=<+\infty$.
    $$ \mathbbm{1}_{A}\cdot f=\mathbbm{1}_{A\cap B},\ \int_{\Omega} \mathbbm{1}_{A} f \dd\mu =\mu(A\cap B)=0.$$
    ($I_A: S\longrightarrow \RR$, $f\longmapsto \int_{\Omega} \mathbbm{1}_A  f \dd\mu$ is a integral operator.)
\end{proofenv}
\begin{theoremenv}
    Let $Y$ be a metric space, $(X,\mathcal{A}, \mu)$ be a measure space. $f:X\times Y\longrightarrow \RR$ be a mapping, $g\in \mathcal{L}^1(X,\mathcal{A},\mu)$ and $p\in y$. Assume that
    \newline
    (1) There exists $\mu$-negligeable set $S\in \mathscr{P}(\Omega)$ such that $\forall x\in X\backslash A$, $f(x,\cdot): Y\longrightarrow \RR$, $y\longmapsto f(x,y)$ is continuous at $p$.
    \newline
    (2) For any $y\in Y$, $f(\cdot,y): X\longrightarrow \RR,\ x\longmapsto f(x,y)$ is $\mathcal{A}$-measurable.
    \newline
    (3) $\forall y\in Y$, $\exists A_y\in \mathscr{P}(\Omega)$ $\mu$-negligeable, such that $\forall x\in X\backslash A_y$, $|f(x,y)|\le g(x)$. Then $(y\in Y)\longmapsto \int_{X}f(x,y)\mu(\dd x)$ is continuous at $p$.
\end{theoremenv}
\begin{proofenv}
    Let $(y_n)_{n\in\NN}$ be a sequence in $Y$ that converges to $p$. For any $n\in \NN$, let $f_n:X\longrightarrow \RR$, $f_n(x)\coloneq f(x,y_n).$
    Let $\dis B=A\cup\left(\bigcup_{n\in \NN} A_{y_n}\right)$, $\mu(B)=0$.
    On $X\backslash B$, $f_n$ converges pointwise to $f(\cdot,p)$ and $|f_n(x)|\le g(x)$.
    By dominated convergence theorem,
    $$ \lim_{n\rightarrow +\infty} \int_{X\backslash B} f(x,y_n)\mu(\dd x)=\int_{X\backslash B} f(x,p)\mu(\dd x).$$
\end{proofenv}
\begin{corollaryenv}
    Let $(a,b)\in \RR^2$, $a<b$, $\mu$ be a Borel measure on $[a,b]$, and $f:[a,b]\longrightarrow \RR$ be a Borel measurable mapping. Let $F:[a,b]\longrightarrow \RR$,
    $$ F(x)\coloneq \int_{a}^{x} f(t) \mu(\dd t). $$
    If $x_0\in [a,b]$ is such that $\mu(\{x_0\})=0$, then $F$ is continuous at $x_0$.
\end{corollaryenv}
\begin{proofenv}
    By definition,
    $$ \int_{a}^{x} f(t)\mu(\dd t)\coloneq \int_{\interval[open left]{a}{x}} f(t)\mu(\dd t)=\int_{[a,b]}\mathbbm{1}_{\interval[open left]{a}{x}}(t)f(t)\mu(\dd t).$$
    For any $x_0\in [a,b]$. If $t\neq x_0$, then $x\longmapsto \mathbbm{1}_{\interval[open left]{a}{x}}(t)$ is continuous at $x_0$.
    Since $|\mathbbm{1}_{\interval[open left]{a}{x}}f|\le |f|$, we obtain that $F$ is continuous at $x_0$.
\end{proofenv}
\begin{theoremenv}
    Let $(a,b)\in \RR^2$, $a<b$ and $f:[a,b]\longrightarrow\RR$ be a Borel measurable mapping which is Lebesgue integrable ($f\in \mathcal{L}^1([a,b])$).

    Let $F:[a,b]\longrightarrow \RR$, $F(x)\coloneq \int_{a}^{x} f(t)\dd t$.
    \newline
    (1) Let $x_0\in \interval[open]{a}{b}$. If $f$ is continuous at $x_0$, then $F$ is differentiable at $x_0$ and $F'(x_0)=f(x_0)$.
    \newline
    (2) Suppose that $f$ is continuous and $G:[a,b]\longrightarrow \RR$ is a continuous mapping such that
    $$ \forall x\in \interval[open]{a}{b}, G'(x) =f (x).$$
    Then,
    $$ \int_{a}^{b} f(t)\dd t =G (b)-G(a).$$
    Moreover, $ \exists \xi\in \interval[open]{a}{b}$, $\dis \int_{a}^{b} f(t)\dd t=f(\xi)(b-a)$.
\end{theoremenv}
\begin{proofenv}
    \ \newline
    (1) For $h<0$, sufficiently small, let
    $$\mu_{h}=\sup_{t\in [x_0-h,x_0+h]}|f(t)-f(x+0)|.$$
    $$ |F(x_0+h)-F(x_0)-f(x_0)h|=\left| \int_{x_0}^{x_0+h} (f(t)-f(x_0)) \dd t\right|\le \mu_{h}\cdot h=o(h), h\rightarrow 0.$$
    Similarly,
    $$ |F(x_0-h)-F(x_0)+f(x_0)h|\le \mu_h\cdot h=o(h), h\rightarrow 0.$$
    (2) By (1), $F$ is a primitive function of $f$ on $[a,b]$, so $F-G$ is a constant mapping. Hence
    $$ \int_{a}^{b}f(t)\dd t=F(b)-F(a)=G(b)-G(a).$$
    By mean value theorem, $\exists \xi \in \interval[open]{a}{b}$, $F(b)-F(a)=F'(\xi )(b-a)$.
    That is 
    $$ \int_{a}^{b} f(t)\dd t=f(\xi)(b-a).$$
\end{proofenv}

\section{Product Measure}
We fix $(X,\mathcal{A}_X,\mu_X)$ and $(Y,\mathcal{A}_Y, \mu_Y)$ two $\sigma$-finite measure space.

We equip $X\times Y$ the product $\sigma$-algebra
$$ \mathcal{A}_X \otimes \mathcal{A}_Y=\sigma\left(\{A\times B\mid A\in \mathcal{A}_X,\ B\in \mathcal{A}_Y\}\right).$$
\begin{propositionenv}
    Let $f:X\times Y\longrightarrow \RR$ be a measurable mapping. For any $x\in X$, $f(x,\cdot): Y\longrightarrow \RR$, $y\longmapsto f(x,y)$ is $\mathcal{A}_Y$ measurable.
\end{propositionenv}
\begin{proofenv}
    \ \newline
    (1) $f=\sup\{f,0\}-\sup\{-f,0\}$. We may assume that $f$ is non-negative.
    \newline
    (2) $$f=\lim_{n\rightarrow +\infty}\sum_{k=0}^{n\cdot2^n-1} \frac{k}{2^n}\mathbbm{1}_{\{\frac{k}{2^n}\le f\le \frac{k+1}{2^n}\}}+n\mathbbm{1}_{\{f\ge n\}}.$$
    We may assume that $f$ is bounded. Let 
    $$\mathcal{H}=\{f:X\times Y\longrightarrow \RR_{\ge 0}\mid f \text{ bounded }, f(x,\cdot) \text{ is } \mathcal{A}_Y\text{-measurable}, \forall x\in X\}.$$
    Then $\mathcal{H}$ is a $\lambda$-family.

    If $f:X\times Y\longrightarrow\RR_{\ge 0}$ is of the form $f(x,y)=f_1(x)f_2(y)$, where $f_1$ and $f_2$ are non-negative, measurable and bounded, $f\in \mathcal{H}$.
    Let $\mathcal{C}$ be the set of such mappings. Then $\mathcal{C}\subseteq \mathcal{H}$  and $\mathcal{C}$ is stable by multiplication. By monotone class theorem $\mathcal{H}$ contains all bounded non-negative $\mathcal{A}_X\otimes \mathcal{A}_Y=\sigma(\mathcal{C})$-measurable functions.
    (if $(A,B)\in \mathcal{A}_X\otimes \mathcal{A}_Y$, then $\left((x,y)\longmapsto \mathbbm{1}_{A}(x)\mathbbm{1}_B(y)\right)\in \mathcal{H}$.)
\end{proofenv}
\begin{theoremenv}[Fubini-Tonelli]
    Let $f:X\times Y\longrightarrow [0,+\infty]$ be an $\mathcal{A}_X\otimes \mathcal{A}_Y$-measurable mapping. Then the mapping
    $$ (x\in X)\longmapsto \int_{Y} f(x,y)\mu_Y(\dd y)$$
    is $\mathcal{A}_X$ measurable. Moreover, there exists a unique $\sigma$-finite measure $\mu_X\otimes \mu_Y$ on $X\times Y$ such that
    \begin{align*}
        \int_{X\times Y} f(x,y)(\mu_X\otimes \mu_Y)\left(\dd(x,y)\right)&=\int_{X}\int_{Y} f(x,y) \mu_Y(\dd y)\mu_X(\dd x)\\
        &=\int_{Y}\int_{X} f(x,y)\mu_X(\dd x)\mu_Y (\dd y).
    \end{align*}
\end{theoremenv}
\begin{proofenv}
    \ \newline
    (1)Let
    $$\mathcal{H}=\{\substack{\text{bounded }\mathcal{A}_{X}\otimes \mathcal{A}_{Y}\text{-measurable }\\ f:X\times Y\longrightarrow \RR_{\ge 0}}\mid (x\in X)\longmapsto \int_{Y} f(x,y)\mu_Y(\dd y) \text{ is }\mathcal{A}_X\text{-measurable}\}$$
    Then $\mathcal{H}$ is a $\lambda$-family and $\mathcal{H}$ contains $\mathcal{C}$.
    So $\mathcal{H}$ contains all $\mathcal{A}_X\otimes \mathcal{A}_Y$ measurable bounded non-negative mappings.
    \newline
    (2) Let $\mathcal{A}_0=\{A\times B\mid (A,B)\in \mathcal{A}_X\times \mathcal{A}_Y\}$. $\mathcal{A}_0$ is a semiring on $X\times Y$, $\mu_X(A)<+\infty$, $\mu_Y(B)<+\infty$. $\mu: \mathcal{A}_0\longrightarrow \interval[open right]{0}{+\infty}$.
    $$ \nu(A\times B)\longmapsto \mu_X(A)\mu_Y(B)=\int_X\int_{Y} \mathbbm{1}_{A\times B}(x,y)\mu_{Y}(\dd y)\mu_X(\dd x)$$
    is a $\sigma$-additive mapping (by Beppo Levi).
    So $\nu$ extends in a unique way to a $\sigma$-finite measure on $\sigma(\mathcal{A}_0)=\mathcal{A}_X\otimes \mathcal{A}_Y$. 
    For $f\in \mathcal{C}$,
    $$ \int_{X\times Y} f(x,y)\dd \nu \int_{X}\int_{Y}f(x,y)\mu_Y(\dd y)\mu_{X}(\dd x).$$
    So the same equality holds for any bounded non-negative $\mathcal{A}_X\otimes \mathcal{A}_Y$-measurable mappings $X\times Y\longrightarrow \RR_{\ge 0}$.
    For general case, take an increasing limit.
\end{proofenv}
\begin{theoremenv}[Fubini]
    Let $f:X\times Y\longrightarrow \RR$  be an $\mathcal{A}_X\otimes \mathcal{A}_Y$-measurable mapping. $f$ is $\mu_X\otimes \mu_Y$-integrable if and only if
    $$ \int_X\int_{Y}|f(x,y)|\mu_Y(\dd y)\mu_X(\dd  x)<+\infty.$$
    Moreover, when $f$ is $\mu_X\otimes \mu_Y$ integrable, one has
    $$ \int_{X\times Y} f(x,y)\dd (\mu_X\otimes \mu_Y)=\int_{X}\int_{Y} f(x,y)\mu_{Y}(\dd y)\mu_{X}(\dd x).$$
\end{theoremenv}
\begin{proofenv}
    \ \newline
    (1) $f$ integrable $\Leftrightarrow$ $|f|$ is integrable.
    \newline
    (2) $$ f=\sup\{f,0\}-\sup\{-f,0\}.$$
    Apply Fubini Tonelli to $\sup\{f,0\}$, $\sup\{-f,0\}$.
\end{proofenv}
\begin{definitionenv}
    Let $(a,b)\in \RR^2$, $a<b$, $\varphi:[a,b]\longrightarrow \RR$. Assume that $\varphi$ is the difference of two right continuous increasing mappings $\varphi_1$ and $\varphi_2$.
    For $x\in \interval[open left]{a}{b}$, let
    $$\varphi(x-)\coloneq \lim_{h>0,h\rightarrow 0} f(x-h).$$
    Let $\Delta \varphi(x)=\varphi(x)-\varphi(x-)$.
    If $f:[a,b]\longrightarrow \RR$ bounded Borel measurable, let 
    $$ \int_{a}^{b}f(t)\dd \varphi (t) \coloneq \int_{a}^{b} f(t)\dd \varphi_1(t) -\int_{a}^{b} f(t) \dd\varphi_2(t).$$
\end{definitionenv}
\begin{theoremenv}
    Let $(a,b)\in \RR^2$, $a<b$. Let $\varphi$ and $\psi$ be mappings from $[a,b]$ to $\RR$, that can be written as difference of increasing right continuous mappings.
    Then
    $$ \int_{a}^{b} \varphi (t)\dd \psi(t) +\int_{a}^{b} \psi(t) \dd\varphi(t)= \varphi(b)\psi(b)-\varphi(a)\psi(a)+\sum_{t\in \interval[open left]{a}{b}}\Delta\varphi(t)\Delta\psi(t).$$
\end{theoremenv}
\begin{proofenv}
    We assume that $\varphi$ and $\psi$ are increasing.
    $$ \int_{a}^{b}\left(\varphi(y)-\varphi(a)\right)\dd \psi(y)=\int_{a}^{b}\int_{a}^{y}\dd\psi(x) \dd \psi (y)=\int_{\interval[open left]{a}{b}^2}\mathbbm{1}_{\{(x,y)\in \interval[open left]{a}{b}^2\mid x\le y\}}\dd\varphi\otimes\dd \psi.$$
    Taking the sum, get
    \begin{align*}
        &\int_{a}^{b}\varphi(y)\dd\psi(y)-\varphi(a)\left(\psi(b)-\psi(a)\right)+\int_{a}^{b}\psi(x)\dd \varphi(x)-\psi(a)\left(\varphi(b)-\varphi(a)\right)\\
        =& \int_{\interval[open left]{a}{b}^2}\dd\varphi\otimes \dd\psi +\int_{\interval[open left]{a}{b}^2}\mathbbm{1}_{D} \dd\varphi\otimes \dd\psi,
    \end{align*}
    where $D=\{(x,x)\mid x\in \interval[open left]{a}{b}\}$. The first term is $(\varphi(b)-\varphi(a))(\psi(b)-\psi(a))$, and the second term is 
    $$ \int_{\interval[open left]{a}{b}}\int_{\interval[open left]{a}{b}}\mathbbm{1}_{D}(x,y)\dd\varphi(x)\dd\psi(y)=\sum_{x\in \interval[open left]{a}{b}}\Delta\varphi(x)\Delta\psi(x).$$



% Pattern Info
 
\tikzset{
pattern size/.store in=\mcSize, 
pattern size = 5pt,
pattern thickness/.store in=\mcThickness, 
pattern thickness = 0.3pt,
pattern radius/.store in=\mcRadius, 
pattern radius = 1pt}
\makeatletter
\pgfutil@ifundefined{pgf@pattern@name@_xjg1mrdmp}{
\pgfdeclarepatternformonly[\mcThickness,\mcSize]{_xjg1mrdmp}
{\pgfqpoint{0pt}{0pt}}
{\pgfpoint{\mcSize+\mcThickness}{\mcSize+\mcThickness}}
{\pgfpoint{\mcSize}{\mcSize}}
{
\pgfsetcolor{\tikz@pattern@color}
\pgfsetlinewidth{\mcThickness}
\pgfpathmoveto{\pgfqpoint{0pt}{0pt}}
\pgfpathlineto{\pgfpoint{\mcSize+\mcThickness}{\mcSize+\mcThickness}}
\pgfusepath{stroke}
}}
\makeatother

% Pattern Info
 
\tikzset{
pattern size/.store in=\mcSize, 
pattern size = 5pt,
pattern thickness/.store in=\mcThickness, 
pattern thickness = 0.3pt,
pattern radius/.store in=\mcRadius, 
pattern radius = 1pt}
\makeatletter
\pgfutil@ifundefined{pgf@pattern@name@_w1ewpsbcl}{
\pgfdeclarepatternformonly[\mcThickness,\mcSize]{_w1ewpsbcl}
{\pgfqpoint{0pt}{-\mcThickness}}
{\pgfpoint{\mcSize}{\mcSize}}
{\pgfpoint{\mcSize}{\mcSize}}
{
\pgfsetcolor{\tikz@pattern@color}
\pgfsetlinewidth{\mcThickness}
\pgfpathmoveto{\pgfqpoint{0pt}{\mcSize}}
\pgfpathlineto{\pgfpoint{\mcSize+\mcThickness}{-\mcThickness}}
\pgfusepath{stroke}
}}
\makeatother
\tikzset{every picture/.style={line width=0.75pt}} %set default line width to 0.75pt        
\begin{center}
    \begin{tikzpicture}[x=0.75pt,y=0.75pt,yscale=-0.75,xscale=0.75]
    %uncomment if require: \path (0,300); %set diagram left start at 0, and has height of 300
    
    %Shape: Axis 2D [id:dp4395752479182208] 
    \draw  (132,205) -- (331.5,205)(151.95,61) -- (151.95,221) (324.5,200) -- (331.5,205) -- (324.5,210) (146.95,68) -- (151.95,61) -- (156.95,68)  ;
    %Shape: Right Triangle [id:dp1860128403661172] 
    \draw  [pattern=_xjg1mrdmp,pattern size=6pt,pattern thickness=0.75pt,pattern radius=0pt, pattern color={rgb, 255:red, 0; green, 0; blue, 0}] (254,102) -- (188.5,164) -- (188.5,102) -- cycle ;
    %Shape: Right Triangle [id:dp5223162572273542] 
    \draw  [pattern=_w1ewpsbcl,pattern size=6pt,pattern thickness=0.75pt,pattern radius=0pt, pattern color={rgb, 255:red, 0; green, 0; blue, 0}] (188.5,164) -- (254,102) -- (254,164) -- cycle ;
    %Straight Lines [id:da9933023519665637] 
    \draw  [dash pattern={on 0.84pt off 2.51pt}]  (188.5,102) -- (188.5,206) ;
    %Straight Lines [id:da35236584806497295] 
    \draw  [dash pattern={on 0.84pt off 2.51pt}]  (254,102) -- (254,206) ;
    %Straight Lines [id:da18838545752167135] 
    \draw  [dash pattern={on 0.84pt off 2.51pt}]  (254,164) -- (152.5,164) ;
    %Straight Lines [id:da7402149972421611] 
    \draw  [dash pattern={on 0.84pt off 2.51pt}]  (254,102) -- (152.5,102) ;
    
    % Text Node
    \draw (183,211.4) node [anchor=north west][inner sep=0.75pt]    {$a$};
    % Text Node
    \draw (135,155.4) node [anchor=north west][inner sep=0.75pt]    {$a$};
    % Text Node
    \draw (134,95.4) node [anchor=north west][inner sep=0.75pt]    {$b$};
    % Text Node
    \draw (248,210.4) node [anchor=north west][inner sep=0.75pt]    {$b$};
    % Text Node
    \draw (339,196.4) node [anchor=north west][inner sep=0.75pt]    {$x$};
    % Text Node
    \draw (159,48.4) node [anchor=north west][inner sep=0.75pt]    {$y$};
    
    
    \end{tikzpicture}
\end{center}
\end{proofenv}
