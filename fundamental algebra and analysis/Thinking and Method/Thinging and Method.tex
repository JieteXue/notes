\documentclass{article}
\newcommand{\singleline}{\hrule height 0.8pt}
\newcommand{\mydate}{Otctober 7, 2025 - \today}
\newcommand{\mytitle}{Thinking and Method of FAA}
\title{\textbf{\mytitle}}
\author{Jiete XUE}
\date{\mydate}
\usepackage{fancyhdr}
\pagestyle{fancy}
\fancyhf{}
\fancyhead[L]{\leftmark}
\fancyfoot[C]{\thepage}
\usepackage[perpage]{footmisc}
\usepackage{tocloft}
\renewcommand{\cftsecleader}{\cftdotfill{\cftdotsep}}
\usepackage{tcolorbox}
\definecolor{mlv}{RGB}{40, 137, 124}

\usepackage{amsthm}
\usepackage{amsmath}
\usepackage{amssymb}
\usepackage{mathrsfs}
\usepackage{tikz}
\usepackage{tcolorbox}
\usepackage{mathrsfs}
\usepackage{tikz-cd}

\usepackage{pdfpages}
\usepackage{xr-hyper}
\usepackage[colorlinks=true, linkcolor=black, citecolor=red, urlcolor=cyan, filecolor=mlv]{hyperref}
\externaldocument[main]{../fundamental algebra and analysis}

\newtheoremstyle{1}{}{}{}{}{\bfseries}{}{\newline}{}
\newtheoremstyle{2}{}{}{}{}{\bfseries}{.}{\newline}{}
\theoremstyle{1}
\newtheorem{problem}{Problem}[section]
\newtheorem{theorem}[problem]{Theorem}
\newtheorem{definition}{Definition}[section]
\usepackage{chngcntr}
\counterwithin{equation}{section}
\newcommand{\pa}{\partial}
\newcommand{\CC}{\mathbb{C}} % 黑板粗体C
\newcommand{\QQ}{\mathbb{Q}} % 黑板粗体Q
\newcommand{\RR}{\mathbb{R}} % 黑板粗体R
\newcommand{\ZZ}{\mathbb{Z}} % 黑板粗体Z
\newcommand{\NN}{\mathbb{N}} % 黑板粗体N

\begin{document}
\maketitle
\thispagestyle{empty}
\newpage
\pagenumbering{roman}
\setcounter{page}{1}
\tableofcontents
\newpage
\pagenumbering{arabic}
\setcounter{page}{1}

\section{Basic Logic}
Iff. $P=Q=\neg R=$True, $P\Rightarrow\left(Q\Rightarrow R\right)$ is False. This is equivalent to $\left(P\wedge Q\right)\Rightarrow R$. So in \texttt{LEAN 4}, you can see a goal in the form
$$a\rightarrow b\rightarrow c\rightarrow\dots,$$
then you can use \textit{intro} to get props. They have the relation \textit{and} logically.

\section{Set Theory}
Definition \ref{main2.4.1} defines quantifiers, by 
\ref{maincorollary2.6.1} and \ref{main2.7.4} , we can use set to understand quantifiers.
Let us first consider 
\begin{equation}
    \forall x\in X, \forall y\in Y, P(x,y).
\end{equation}
That is 
\begin{equation}
    X=\{x\in X \mid \forall y\in Y, P(x,y)\}=\bigcap_{y\in Y}\{x\in X \mid P(x,y)\}.
\end{equation}
That means 
\begin{equation}
    \forall y\in Y,\ X\subseteq\{x\in X \mid P(x,y)\}.
\end{equation}
Thus, 
\begin{equation}
    \forall y\in Y, X=\{x\in X\mid P(x,y)\},
\end{equation}
equivalent to 
\begin{equation}
    \forall y\in Y,\forall x\in X ,P(x,y).
\end{equation}
But if we consider
\begin{equation}
    \forall x\in X, \exists y\in Y, P(x,y),
\end{equation}
the situation becomes
\begin{equation}
    X=\bigcup_{y\in Y}\{x\in X\mid P(x,y)\}
\end{equation}
The union equals to $X$ does not give enough information. Similarly, $\exists,\forall,\dots$ can't go farther, too\footnote{The intersection is not empty leads to any sets is not empty, but it is not equivalent, $\exists x\in X,\forall y\in Y,P(x,y)\Rightarrow\forall y\in Y,x\in X,P(x,y).$}. But
\begin{equation}
    \exists x\in X, \exists y\in Y, P(x,y)
\end{equation}
is equivalent to 
\begin{equation}
    \bigcup_{y\in Y}\{x\in X\mid P(x,y)\}\not=\varnothing.
\end{equation}
That means 
\begin{equation}
    \exists y\in Y, \{x\in X \mid P(x,y)\}\not=\varnothing.
\end{equation}
Thus, 
\begin{equation}
    \exists y\in Y,\exists x\in X, P(x,y).
\end{equation}
\section{Correspondence}
For the similar reason, if $f$ is a correspondence, then
\begin{equation}
    f\left(\bigcup_{i\in I}A_i \right)=\bigcup_{i\in I}f\left(A_i\right),
\end{equation}
\begin{equation}
    f\left(\bigcap_{i\in I}A_i \right)\subseteq\bigcap_{i\in I}f\left(A_i\right).
\end{equation}
If in addition, $f$ is injective, then
\begin{equation}
    f\left(\bigcap_{i\in I}A_i \right)=\bigcap_{i\in I}f\left(A_i\right).
\end{equation}
\singleline
A conclusion: Let $f,g$ be correspondences, if $f\circ g=\mathrm{Id},\ g\circ f=\mathrm{Id}$, then $f$ is a bijection and $f^{-1}=g$.
\singleline

\section{Ordering}
Forgettable concepts: Well-ordered set \ref{main4.7.1}, Order-complete \ref{main4.8.1}
\singleline
\begin{problem}[Eg.]
$$m:=\inf(A^\mathrm{u})\in A^\mathrm{u}.$$
\end{problem}
\singleline
\begin{proof}
    
 By definition, we only need to prove $\forall x\in A, \ x\le m$. $m$ is the max element in $\left(A^\mathrm{u}\right)^l$, then we only need to prove $\forall x\in A,\ x\in \left(A^\mathrm{u}\right)^\mathrm{l}$. It is easy to check.

\end{proof}
\singleline
The power set with $\subseteq$ forms a order-complete partially ordered set. If we want to construct a order-complete partially ordered set, we may consider build a relation between them. Knaster-Tarski fixed point theorem tell us a property of monotonic functions,  and Dedekind-MacNeille theorem tell us how to do in detail.

\newpage
\section{Rings and Modules}
\begin{definition}[Unitary Ring]
    A set $A$ with ``$+$''\footnote{``$+$'' usually equipped with communicative law. So we say a communicative unitary ring means the ``$*$'' is communicative, in addition.} (communicative group), ``$*$'' (monoid\footnote{``Unitary'' refer to the unitary element.}), and distributivity forms a unitary ring.
\end{definition}
\ \newline
The homomorphism of unitary rings is the combination of groups and monoids.
\begin{definition}[Division Ring \& Field]
    Let $K$ be a unitary ring. We denote by $K^\times$ the invertible elements of $(K, \cdot)$. If $K^\times=K\backslash\{0\}$ then we say that $K$ is a division ring. If in addition,  $K$ is commutative,  then we say that $K$ is a \textbf{field}.
\end{definition}
\begin{definition}[Actions]
    Set $X$, monoid $G$, We call \textbf{left action} of $G$ on $X$ any mapping 
    $$\phi:G\times X\rightarrow X, $$
    such that 
    \newline
    (1) $\phi(e, x)=x$,  for any $x\in X$.
    \newline
    (2) $\forall (a, b)\in G\times G, \forall x\in X$, 
    $$\phi(a*b, x)=\phi(a, \phi(b, x)).$$
\end{definition}
If we let $G$ be a group, then we get a equivalent relation like orbit\footnote{Denote as $\mathrm{orb}_\phi(x)$.}.

\begin{definition}[Modules]
    $K$: unitary ring. $(V, +)$: abelian group. We call a \textbf{left K-module structure} any left action of $(K, \cdot)$ on $V$. 
    $$\phi:K\times V\longrightarrow V$$
    (1) $\forall (a, b)\in K\times K, \forall x\in V$, $$\phi(a+b, x)=\phi(a, x)+\phi(b, x).$$
    (2) $\forall a\in K$,  $\forall (x, y)\in V\times V$, $$\phi(a, x+y)=\phi(a, x)+\phi(a, y).$$
    $(V, +)$ equipped with a left $K$-module structure is called a \textbf{left K-module}. If $K$ is communitative,  left and right $K$-modules structures have the same axioms: $K$-module structures. Left and right $K$-modules structures: $K$-modules.
    If $K$ is a field,  a $K$-module is called a \textbf{vector space} over $K$.
\end{definition}
\begin{definition}[Sub-K-modules]
    $V$: left $K$-module,  we call \textbf{left sub-K-module} of $V$ any subgroup $W$ of $(V, +)$ if $\forall(a, x)\in K\times W, ax\in W$. (resp. right.)
\end{definition}
\begin{definition}[Homomorphism]
    $E$, $F$ be left-K-modules. We call \textbf{homomorphism of left K-modules from $E$ to $F$} any mapping $f:E\rightarrow F$,  such that 
    \newline
    (1) $f$ is a homomorphism of groups from $(E, +)$ to $(F, +)$.
    \newline
    (2) For any $(a, x)\in K\times E, f(ax)=af(x)$.
    \newline
    If $K$ is communitative, also called a \textbf{$K$-linear mapping}.
\end{definition}
\begin{definition}[Ideal]
    Let $A$ be a unitary ring. If a subset $I$ of $A$ is a left sub-$A$-module of $A$ and a right sub-$A$-module of $A$, then we call $I$ a \textbf{ideal} of $A$. If $I$ is an ideal of $A$, then the composition laws of $A$ define by passing to quotient a structure of unitary ring on the quotient mapping $A/I$. So that $A/I$ becomes a quotient ring of $A$.
\end{definition}
\begin{definition}[Principal Ideal]
    Let $A$ be a communitative unitary ring. If an ideal of $A$ is of the form 
    $$Ax:\{ax\mid a\in A\}\text{ with }x\in A.$$
    We say that it is a \textbf{principal ideal}. If all ideals of $A$ are principal, we say that $A$ is a \textbf{principal ideal ring}. 
\end{definition}
\begin{definition}[]
    Let $V$ be a left K-module. For any family $\underline{x}:=\left(x_i\right)_{i\in I}\in V^I$, we denote by 
    $$\varphi_{\underline{x}}:K^{\oplus I}\longrightarrow V$$
    the homomorphism sending $(a_i)_{i\in I}$ to $\sum_{i\in I}a_ix_i.$
    \newline
    (1) $\mathrm{Im}\left(\varphi_{\underline{x}}\right)$ is a left $K$-submodule of $V$, called the \textbf{left sub-K-module generated by }$\underline{x}$, denote as $\mathrm{Span}_{K}\left((x_i)_{i\in I}\right)$. If $\varphi_{x}$ is surjective, we say that $(x_i)_{i\in I}$ is a \textbf{system of generators} of $V$.
    ($\forall y\in V,\exists (a_i)_{i\in I}\in K^{\oplus I}, y=\sum_{i\in I}a_ix_i$) Elements of $\mathrm{Span}_K\left((x_i)_{i\in I}\right)$ are called \textbf{K-linear combinations} of $(x_i)_{i\in I}$.
    \newline
    (2) If $\varphi_{\underline{x}}$ is injective, we say that $(x_i)_{i\in I}$ is \textbf{K-linearly independent}. ($\forall (a_i)_{i\in I}\in K^{\oplus I},\sum_{i\in I}a_ix_i=0\rightarrow a_i=0,\forall i\in I$)
    \newline
    (3) If $\varphi_{\underline{x}}$ is an isomorphism, we say $(x_i)_{i\in I}$ is a \textbf{basis} of $V$. If $V$ has at least a basis, we say that $V$ is a \textbf{free left K-module}. If $V$ has a system of generators $(x_i)_{i\in I}$ such that $I$ is finite, we say that $V$ is \textbf{finitely generated}, or is \textbf{finite types}.
\end{definition}
\begin{definition}[Rank]
    Let $K$ be a division ring and $V$ is a left $K$-module of finite type. We denote by $\mathrm{rk}(V)$ the least cardinality of the bases $V$, called the \textbf{rank} of $V$. If $K$ is a field, then $\mathrm{rk}(V)$ is also denoted as $\dim(V)$, called the \textbf{dimension} of $V$. If $f:W\longrightarrow V$ is a homomorphism of left $K$-modules, the rank of $f$ is defined as the rank of $\mathrm{Im}(f)$, denoted as $\mathrm{rk}(f)$.
\end{definition}
\begin{definition}[Algebra]
    Let $K$ be a communicative unitary ring. If $A$ is a $K$-module equipped with a  composition law 
        $$A\times A\longrightarrow A,$$
    $$(a,b)\longmapsto a b.$$
    such that $(A,+,\cdot)$ forms a unitary ring, such that
     $$\forall \lambda\in K,\forall (a,b)\in A\times A, \ \lambda\left(ab\right)=\left(\lambda a\right)b=a\left(\lambda b\right).$$
     Then we say that $A$ is a \textbf{K-Algebra}.
\end{definition}
\begin{definition}[Sub-algebra]
    Let $A$ be a K-algebra. If $B$ is a subset of $A$ which is a sub-K-module and a unitary subring of $A$, we say that $B$ is a \textbf{sub-K-algebra} of $A$.
\end{definition}
\begin{theorem}[Rank–Nullity Theorem]
    $A:V\longrightarrow W$, $\dim(V)=n,\ \dim(W)=m$, $A\in M_{m,n}$, 
    $$n=\dim(\ker(A))+\dim(\mathrm{Im}(A)).$$
\end{theorem}






\section{Filters}
\begin{definition}
    Let $X$ be a set. We call \textbf{filter} on $X$ any non-empty subset $\mathcal{F}$ of $\wp(X)$ this satisfies:
    \newline
    (1) $\forall(V_1,V_2)\in \mathcal{F}^2, V_1\cap V_2\in \mathcal{F}$.
    \newline
    (2) $ \forall V\in \mathcal{F}, \forall W\in \wp(X)$, if $V\subseteq W$, then $W\in \mathcal{F}$.
\end{definition}
\begin{definition}
    Let $S$ be a subset of $\wp(X)$. We denote by $\mathcal{F}_S$ the intersection of all filters containing $S$. It is thus the least filter containing $S$. We call it the filter generated by $S$.
\end{definition}
\begin{definition}
    We say that a subset $S$ of $\wp(X)$ is a \textbf{filter basis} if, for any $(A,B)\in S\times S$, there exists $C\in S$, such that $C\subseteq A\cap B$.\footnote{If $n\in \NN_{\ge 1}$ and $(A_1,\dots,A_n)\in S^n,\exists C\in S$ such that $C\subseteq A_1\cap \dots\cap A_n$.}
\end{definition}

    If $S$ is a filter basis, then 
    $$\mathcal{F}_S=\{U\in \wp(X)\mid \exists A\in S, A\subseteq U\}.$$
    
    If $S$ is a subset of $\wp(X)$, then 
    $$\mathcal{B}_S:=\{A_1\cap \cdots \cap A_n\mid n\in \NN,\ (A_1,\dots,A_n)\in S^n\}$$
    is a filter basis containing $S$. Moreover, $\mathcal{F}_S=\mathcal{F}_{\mathcal{B}_S}$.
\begin{definition}
    Let $X$ be a set and $f:X\longrightarrow G$ be a mapping. For any $U\in \wp(X)$, we define 
    $$f^s(U):=\sup_{x\in U}f(x)=\sup f(U).$$
    $$f^i(U):=\inf_{x\in U}f(x)=\inf f(U).$$
    If $U\not=\varnothing$, $f^s(U)\ge f^i(U)$. Let $\mathcal{F}$ be a filter on $X$. We define 
    $$\limsup_\mathcal{F} f:=\inf_{U\in \mathcal{F}} f^s(U).$$
    $$\liminf_\mathcal{F} f:=\sup_{U\in \mathcal{F}} f^i(U).$$
    They are called the \textbf{superior limit} and the \textbf{inferior limit} of $f$ along $\mathcal{F}$. If 
    $$\liminf_\mathcal{F} f=\limsup_\mathcal{F} f,$$
    we say that $f$ has a limit along $\mathcal{F}$, and we denote $\displaystyle \lim_{\mathcal{F}}f$ this value.
\end{definition}
\begin{definition}
    Let $(G,*)$ be a group, and $\le $ be a partial order on $G$. If 
    $$\forall (a,b,c)\in G^3,a<b\Rightarrow a*c<b*c \text{ and } c*a<c*b,$$
    we say that $(G,*,\le)$ is a \textbf{partially ordered group}. If in addition $\le $ is a total order, we say that $(G,*,\le)$ is a \textbf{totally ordered group}. (Resp. semigroup, monoid.)
\end{definition}
\end{document}

