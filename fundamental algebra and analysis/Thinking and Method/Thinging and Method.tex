\documentclass{article}
\newcommand{\singleline}{\hrule height 0.8pt}
\newcommand{\mydate}{Otctober 7, 2025 - \today}
\newcommand{\mytitle}{Thinking and Method of FAA}
\title{\textbf{\mytitle}}
\author{Jiete XUE}
\date{\mydate}
\usepackage{fancyhdr}
\pagestyle{fancy}
\fancyhf{}
\fancyhead[L]{\leftmark}
\fancyfoot[C]{\thepage}
\usepackage[perpage]{footmisc}
\usepackage{tocloft}
\renewcommand{\cftsecleader}{\cftdotfill{\cftdotsep}}
\usepackage{tcolorbox}
\definecolor{mlv}{RGB}{40, 137, 124}

\usepackage{amsthm}
\usepackage{amsmath}
\usepackage{amssymb}
\usepackage{mathrsfs}
\usepackage{tikz}
\usepackage{tcolorbox}
\usepackage{mathrsfs}
\usepackage{tikz-cd}

\usepackage{pdfpages}
\usepackage{xr-hyper}
\usepackage[colorlinks=true, linkcolor=black, citecolor=red, urlcolor=cyan, filecolor=mlv]{hyperref}
\externaldocument[main]{../fundamental algebra and analysis}

\newtheoremstyle{1}{}{}{}{}{\bfseries}{}{\newline}{}
\newtheoremstyle{2}{}{}{}{}{\bfseries}{.}{\newline}{}
\theoremstyle{1}
\newtheorem{problem}{Problem}[section]
\newtheorem{theorem}[problem]{Theorem}
\usepackage{chngcntr}
\counterwithin{equation}{section}
\newcommand{\pa}{\partial}
\newcommand{\CC}{\mathbb{C}} % 黑板粗体C
\newcommand{\QQ}{\mathbb{Q}} % 黑板粗体Q
\newcommand{\RR}{\mathbb{R}} % 黑板粗体R
\newcommand{\ZZ}{\mathbb{Z}} % 黑板粗体Z
\newcommand{\NN}{\mathbb{N}} % 黑板粗体N

\begin{document}
\maketitle
\thispagestyle{empty}
\newpage
\pagenumbering{roman}
\setcounter{page}{1}
\tableofcontents
\newpage
\pagenumbering{arabic}
\setcounter{page}{1}

\section{Basic Logic}
Iff. $P=Q=\neg R=$True, $P\Rightarrow\left(Q\Rightarrow R\right)$ is False. This is equivalent to $\left(P\wedge Q\right)\Rightarrow R$. So in \texttt{LEAN 4}, you can see a goal in the form
$$a\rightarrow b\rightarrow c\rightarrow\dots,$$
then you can use \textit{intro} to get props. They have the relation \textit{and} logically.

\section{Set Theory}
Definition \ref{main2.4.1} defines quantifiers, by 
\ref{maincorollary2.6.1} and \ref{main2.7.4} , we can use set to understand quantifiers.
Let us first consider 
\begin{equation}
    \forall x\in X, \forall y\in Y, P(x,y).
\end{equation}
That is 
\begin{equation}
    X=\{x\in X \mid \forall y\in Y, P(x,y)\}=\bigcap_{y\in Y}\{x\in X \mid P(x,y)\}.
\end{equation}
That means 
\begin{equation}
    \forall y\in Y,\ X\subseteq\{x\in X \mid P(x,y)\}.
\end{equation}
Thus, 
\begin{equation}
    \forall y\in Y, X=\{x\in X\mid P(x,y)\},
\end{equation}
equivalent to 
\begin{equation}
    \forall y\in Y,\forall x\in X ,P(x,y).
\end{equation}
But if we consider
\begin{equation}
    \forall x\in X, \exists y\in Y, P(x,y),
\end{equation}
the situation becomes
\begin{equation}
    X=\bigcup_{y\in Y}\{x\in X\mid P(x,y)\}
\end{equation}
The union equals to $X$ does not give enough information. Similarly, $\exists,\forall,\dots$ can't go farther, too\footnote{The intersection is not empty leads to any sets is not empty, but it is not equivalent, $\exists x\in X,\forall y\in Y,P(x,y)\Rightarrow\forall y\in Y,x\in X,P(x,y).$}. But
\begin{equation}
    \exists x\in X, \exists y\in Y, P(x,y)
\end{equation}
is equivalent to 
\begin{equation}
    \bigcup_{y\in Y}\{x\in X\mid P(x,y)\}\not=\varnothing.
\end{equation}
That means 
\begin{equation}
    \exists y\in Y, \{x\in X \mid P(x,y)\}\not=\varnothing.
\end{equation}
Thus, 
\begin{equation}
    \exists y\in Y,\exists x\in X, P(x,y).
\end{equation}
\section{Correspondence}
For the similar reason, if $f$ is a correspondence, then
\begin{equation}
    f\left(\bigcup_{i\in I}A_i \right)=\bigcup_{i\in I}f\left(A_i\right),
\end{equation}
\begin{equation}
    f\left(\bigcap_{i\in I}A_i \right)\subseteq\bigcap_{i\in I}f\left(A_i\right).
\end{equation}
If in addition, $f$ is injective, then
\begin{equation}
    f\left(\bigcap_{i\in I}A_i \right)=\bigcap_{i\in I}f\left(A_i\right).
\end{equation}
\singleline
A conclusion: Let $f,g$ be correspondences, if $f\circ g=\mathrm{Id},\ g\circ f=\mathrm{Id}$, then $f$ is a bijection and $f^{-1}=g$.
\singleline

\section{Ordering}
Forgettable concepts: Well-ordered set \ref{main4.7.1}, Order-complete \ref{main4.8.1}
\singleline
\begin{problem}[Eg.]
$$m:=\inf(A^\mathrm{u})\in A^\mathrm{u}.$$
\end{problem}
\singleline
\begin{proof}
    
 By definition, we only need to prove $\forall x\in A, \ x\le m$. $m$ is the max element in $\left(A^\mathrm{u}\right)^l$, then we only need to prove $\forall x\in A,\ x\in \left(A^\mathrm{u}\right)^\mathrm{l}$. It is easy to check.

\end{proof}
\singleline
The power set with $\subseteq$ forms a order-complete partially ordered set. If we want to construct a order-complete partially ordered set, we may consider build a relation between them. Knaster-Tarski fixed point theorem tell us a property of monotonic functions,  and Dedekind-MacNeille theorem tell us how to do in detail.

\end{document}

