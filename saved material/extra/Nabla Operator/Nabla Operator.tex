\documentclass{article}
\usepackage{amsmath}
\usepackage{physics}
\usepackage{amsthm}
\usepackage{amssymb}
\newtheoremstyle{1}{}{}{}{}{\bfseries}{.}{\newline}{}
\newtheoremstyle{2}{0}{0}{}{}{\texttt}{.}{\newline}{}
\theoremstyle{1}
\newtheorem{definition}{Definition}
\newtheorem{property}{Property}
\theoremstyle{2}
\newtheorem*{proof_env}{Proof}
\newcommand{\pa}{\partial}
\begin{document}
\begin{center}
    \huge{\textbf{Nabla Opertor}}
\end{center}

This article aim to give a proof\footnote{I will use tensor notation.}  of the expression\footnote{Mainly for 3D vectors.} of nabla operator in different coordinate system and explain its ``meaning''. And this will be a cheat sheet.
\section{The Nabla Operator $\nabla$}
Nabla operator has three action modules: ``association'', ``inner product'', ``cross product''\footnote{In my words.}. Or we can say we denote three different actions as a same notation .
They have different meanings, and so have the different expressions. 


\section{Physical Component and Orthogonal Basis}
$\mathbf{g}^i,\mathbf{g}_i$ as the natural basis have different units. That is inconvenient for physics calculation. So we construct another covariant basis:
\begin{equation}
    \mathbf{g}_{(i)}=\frac{\mathbf{g}_i}{\sqrt{g_{\underline{i}\underline{i}}}}=\beta^j_{(i)}\mathbf{g}_j,
\end{equation}
where, underline means do not take summation, and 
\begin{equation}
    \beta^j_{(i)}=\frac{\delta^j_i}{\sqrt{\mathbf{g}_{\underline{i}}\cdot\mathbf{g}_{\underline{i}}}}.
\end{equation}
Then the physical component can be written as
\begin{equation}
    v^{(i)}=\sqrt{g_{\underline{i}\underline{i}}}v^i,\quad v_{(i)}=\frac{1}{\sqrt{g_{\underline{i}\underline{i}}}}v_i.
\end{equation}
If $\mathbf{g}_i$ is orthogonal, let 
\begin{equation}
    \left|\mathbf{g}_i\right|=A_i,
\end{equation}
which are called the \textbf{Lamé coefficient}, then
\begin{equation}
    g_{ij}=\left\{
    \begin{matrix}
        0,&i\not=j\\
        A_i^2,&i=j.
    \end{matrix}
    \right.
\end{equation}
Let 
\begin{equation}
    \mathbf{e}_i=\frac{\mathbf{g}_i}{A_i},\quad \mathbf{e}^i=A_i\mathbf{g}^i,
\end{equation}
then,
\begin{equation}
    \mathbf{e}_i=\mathbf{e}^i=\mathbf{e}(i).
\end{equation}
\begin{equation}
    \Gamma_{ij}^k=0,(i\not=j\not=k),\quad \Gamma_{ij}^i=\frac{1}{A_i}\frac{\pa A_i}{\pa x^j}, \quad \Gamma_{ii}^j=-\frac{A_i}{A_j^2}\pdv{A_i}{x^j},(i\not=j).
\end{equation}



\section{Gradient}
Let $\mathbf{T}$ be a tensor, then we define the gradient of the tensor as
\begin{equation}
    \nabla \mathbf{T}=\mathbf{g}^i\pdv{\mathbf{T}}{x^i}=\frac{\mathbf{e}^i}{A_i}\pdv{\mathbf{T}}{x^i}.
\end{equation}



\section{Divergence}
Note that 
\begin{equation}
    \frac{\pa \sqrt{g}}{\pa x^i}=\Gamma_{ji}^i\sqrt{g},
\end{equation}
we have 
\begin{equation}
    \nabla\cdot\mathbf{F}=\pa_i F^i+F^m\Gamma_{im}^i=\pa_i F^i+ F^m\frac{1}{\sqrt{g}}\pa _m\sqrt{g}=\frac{1}{\sqrt{g}}\pa_i\left(\sqrt{g}F^i\right).
\end{equation}
In orthogonal coordinate system, we have
\begin{equation}
    \nabla\cdot\mathbf{F}=\sum_{i=1}^3\frac{1}{A_1A_2A_3}\pa_i\left(\frac{A_1A_3A_3}{A_i}F(i)\right).
\end{equation}



\section{Curl}
\begin{align}
    \nabla\times \mathbf{F}&=\epsilon^{ijk}\nabla_i F_j\mathbf{g}_k=\epsilon^{ijk}\left(\pa_i F_j-F_m\Gamma_{ij}^m\right)\mathbf{g}^k=\epsilon^{ijk}\pa_iF_j\mathbf{g}^k\\
    &=\frac{1}{\sqrt{g}}\begin{vmatrix}
 \mathbf{g}_1  & \mathbf{g}_1 &\mathbf{g}_1 \\
 \partial _1 & \partial _2 &\partial _3 \\
 F_1 & F_2 &F_3
\end{vmatrix}.
\end{align}
In orthogonal coordinate system, we have
\begin{equation}
    \nabla\times \mathbf{F}=\frac{1}{A_1A_2A_3} \begin{vmatrix}
 \mathbf{e}_1  & \mathbf{e}_1 &\mathbf{e}_1 \\
 \partial _1 & \partial _2 &\partial _3 \\
 A_1F(1) & A_2F_(2) &A_3F(3)
\end{vmatrix}.
\end{equation}


\section{Laplacian}
For a scalar function $f$, we have
\begin{equation}
    \nabla^2f=\nabla\cdot\nabla f=\frac{1}{A_1A_2A_3}\sum_{i=1}^{3}\pa_i\left(\frac{A_1A_2A_3}{A_i^2}\pa_i f\right).
\end{equation}
\end{document}