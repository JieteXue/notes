\documentclass{article}
\usepackage{amsmath}
\usepackage{physics}
\usepackage{amsthm}
\usepackage{amssymb}
\newcommand{\pa}{\partial}
\newtheoremstyle{1}{}{}{}{}{\bfseries}{.}{\newline}{}
\newtheoremstyle{2}{0}{0}{}{}{\texttt}{.}{\newline}{}
\theoremstyle{1}
\newtheorem{definition}{Definition}
\newtheorem{property}{Property}
\theoremstyle{2}
\newtheorem*{proof_env}{Proof}
\begin{document}
\begin{center}
    \huge{\textbf{Dirac Delta Function}}
\end{center}
\section{One-Dimentiinal Case}
\begin{definition}
    $\delta(x)$ is a generalized function satisfies:
    \begin{equation}
        \delta(x):=\left\{\begin{matrix}
            0,&x\not=0\\
            +\infty,&x=0
        \end{matrix} \right. ,
    \end{equation}
    \begin{equation}
        \int_{-\epsilon}^{+\epsilon}\delta(x)\, \dd x=1,\quad\epsilon>0.
    \end{equation}
    
\end{definition}
All the equalities should be understand under integration.
\begin{property}
        \begin{equation}\label{3}
            \delta(-x)=\delta(x).
        \end{equation}
\end{property}
\begin{proof_env}
    \begin{equation}
        \delta(-x)=\left\{\begin{matrix}
           0,&x\not=0\\
            +\infty,&x=0 
        \end{matrix}\right. ,
    \end{equation}
    \begin{equation}
       \int_{-\epsilon}^{+\epsilon}\delta(-x)\, \dd x=-\int_{+\epsilon}^{-\epsilon}\delta(-x)\, \dd (-x)=\int_{-\epsilon}^{+\epsilon}\delta(x)\, \dd x,\quad\epsilon>0.
    \end{equation}
\end{proof_env}
\begin{property}\label{p2}
    \begin{equation}
            \delta(ax)=\frac{1}{\left|a\right|}\delta(x)\quad (a\not=0).
    \end{equation}
\end{property}
\begin{proof_env}
    \begin{equation}
        \delta(ax)=\left\{\begin{matrix}
           0,&x\not=0\\
            +\infty,&x=0 
        \end{matrix}\right. =\frac{\delta(x)}{\left|a\right|},
    \end{equation}
    \begin{equation}
       \int_{-\epsilon}^{+\epsilon}\delta(ax)\, \dd x=\left\{\begin{matrix}
        \frac{1}{a}\int_{-\epsilon}^{+\epsilon}\delta(ax)\, \dd (ax),\, a>0\\
        \frac{1}{a}\int_{+\epsilon}^{-\epsilon}\delta(ax)\, \dd (ax),\, a<0
       \end{matrix}\right. =\frac{1}{\left|a\right|}\int_{-\epsilon}^{+\epsilon}\delta(x)\, \dd x .
    \end{equation}
\end{proof_env}
\begin{property}
    \begin{equation}
        f(x)\delta(x-a)=f(a)\delta(x-a).
    \end{equation}
\end{property}
\begin{property}
    \begin{equation}
        \int\delta(x-y)\delta(y-a)\, \dd y=\delta(x-a).
    \end{equation}   
\end{property}
\begin{property}
    \begin{equation}
        \delta(x)=\frac{1}{2\pi}\int_{-\infty}^{+\infty}e^{ikx}\, \dd k.
    \end{equation}    
\end{property}
\begin{proof_env}
    Add a convergence factor to soft cutoff the divergent integration.
    \begin{equation}
       \frac{1}{2\pi} \int_{-\infty}^{+\infty}e^{ikx}e^{-\epsilon k^2}\, \dd k=\frac{e^{-\frac{x^2}{4\epsilon}}}{2\sqrt{\pi\epsilon}}.
    \end{equation}
    Then check it like above proof.
\end{proof_env}
\begin{property}
    \begin{equation}
        \delta[g(x)]=\sum_{n}\frac{\delta(x-x_n)}{\left|g'(x_n)\right|},\ g(x_n)=0,g'(x_n)\not=0.
    \end{equation}
\end{property}
\begin{proof_env}
    Around $x=x_n$,
    \begin{equation}
        g(x)=g(x_n)+g'(x_n)(x-x_n)=g'(x_n)(x-x_n)
    \end{equation}
    Thus, 
    \begin{equation}
        \delta[g(x)]=\delta(\sum_{n}g'(x_n)(x-x_n))=\sum_{n}\delta[g'(x_n)(x-x_n)].
    \end{equation}
    By Property \ref{p2},
    \begin{equation}
        \delta[g(x)]=\sum_{n}\frac{1}{\left|g'(x_n)\right|}\delta(x-x_n),\ g(x_n)=0,g'(x_n)\not=0.
    \end{equation}
    For those $g(x_n)=0,g'(x_n)=0$, take $ 2\le m=\min\left(\{m\in \mathbb{N}\mid g^{m}(x_n)\not=0\}\right)$. Then we induct on $m$ that 
    \begin{equation}
        \delta[g^{m}(x_n)(x-x_n)^m]=0.
    \end{equation}
    For $m=2$,
    \begin{equation}
        \delta[g''(x_n)(x-x_n)^2]=
    \end{equation}
\end{proof_env}
\section{High-Dimentiinal Case}
\begin{definition}
    \begin{equation}
        \delta^n(\mathbf{x-x_0})=\prod_{i=1}^{n}\delta(x_i-x_{0i})
    \end{equation}
\end{definition}
\begin{property}[Normalization]
    \begin{equation}
           \int_{\mathbb{R}^n}\delta^n(\mathbf{x-x_0})\, \dd x=\prod_{i=1}^{n}\int_{-\infty}^{+\infty}\delta(x_i-x_{0i})\, \dd x=1. 
    \end{equation}
\end{property}
\begin{property}[Sifting Property]
    \begin{equation}
    \int_{\mathbb{R}^n}f(\mathbf{x})\delta^n(\mathbf{x-x_0})\, \dd x=f(\mathbf{x_0}).
    \end{equation}
\end{property}
\begin{property}[Coordinate Transformations]
    For transformation $\mathbf{y}=\mathbf{y}(\mathbf{x})$ with Jacobian $J=\left|\frac{\pa \mathbf{y}}{\pa \mathbf{x}}\right|$,
    \begin{equation}
        \delta^n(\mathbf{x-x_0})=\frac{1}{\left|J\right|}\delta^n(\mathbf{y(x)-y(x_0)}).
    \end{equation}
    Easy to check the integration.
\end{property}
\begin{property}[Integration by Parts Generalization]
    \begin{equation}
        \int f(\mathbf{x})\nabla\delta(\mathbf{x-x_0})\dd^n \mathbf{x}=-\nabla f(\mathbf{x_0}).
    \end{equation}
\end{property}
\begin{property}[Relationship with Laplacian Operator]
    \begin{equation}
        \nabla^2\left(\frac{1}{\left|\mathbf{x-x_0}\right|^{n-2}}\right)=-(n-2)S_n\delta^n(\mathbf{x-x_0}),\ n\not=2,
    \end{equation}
    where 
    \begin{equation}
        S_n=\frac{2\pi^{\frac{n}{2}}}{\Gamma\left(\frac{n}{2}\right)}
    \end{equation}
    is the surface area of the n-dimensional unit sphere. In particular, $n=3$:
    \begin{equation}
        \nabla^2\left(\frac{1}{\left|\mathbf{r-r_0}\right|^2}\right)=-4\pi\delta^3(\mathbf{r-r_0}).
    \end{equation}
\end{property}
\begin{property}[Scaling Property]
    \begin{equation}
        \delta^n(\alpha \mathbf{x})=\frac{1}{\left|\alpha\right|^n}\delta^n(\mathbf{x}).
    \end{equation}
\end{property}
\begin{property}[Composition with Functions]
    For a function $f:\mathbb{R}^n\longrightarrow \mathbb{R}$ with simple zeros at $\mathbf{x}_k$:
    \begin{equation}
        \delta(f(\mathbf{x}))=\sum_{k}\frac{\delta(\mathbf{x-x}_k)}{\left|\nabla f(\mathbf{x}_k)\right|}.
    \end{equation}
\end{property}
\end{document}