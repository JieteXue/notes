\documentclass{article}
\usepackage{amsmath}
\usepackage{physics}
\usepackage{amsthm}
\usepackage{amssymb}
\newtheoremstyle{1}{}{}{}{}{\bfseries}{.}{\newline}{}
\newtheoremstyle{2}{0}{0}{}{}{\texttt}{.}{\newline}{}
\theoremstyle{1}
\newtheorem{definition}{Definition}
\newtheorem{property}{Property}
\theoremstyle{2}
\newtheorem*{proof_env}{Proof}
\newcommand{\pa}{\partial}
\begin{document}
\begin{center}
    \huge{\textbf{Hermite Ploynomial}}
\end{center}

\section{Explicit Expression}
\begin{equation}
    H_n(z)=\sum_{k=0}^{\lfloor n/2 \rfloor}\frac{(-1)^k n!}{k!(n-2k)!}(2z)^{n-2k}.
\end{equation}
\section{Generating Function}
\begin{equation}
    e^{2xt-t^2}=\sum_{n=0}^{\infty}H_n(x)\frac{t^n}{n!}.
\end{equation}
\section{Hermite Equation}
\begin{equation}
    H''_n(x)-2xH_n'(x)+2nH_n(x)=0.
\end{equation}
This is a Sturm-Liouville type equation with weight function $w(x)=e^{-x^2}$.
\newline
\textbf{Proof.}
    \begin{equation}
        \frac{\pa G}{\pa s}=\sum_{n=0}^{\infty}\frac{1}{n!}H_{n+1}(z)s^n.
    \end{equation}
    \begin{equation}
        2sG=\sum_{n=1}^{\infty}2n\frac{1}{n!}H_{n-1}s^n.
    \end{equation}
    Compare the coefficients of $s^n$,
    \begin{equation}
        \boxed{H_{n+1}(z)-2zH_n(z)+2nH_{n-1}(z)=0.}
    \end{equation}
\section{Recurrence Relations}
\begin{equation}
        \frac{\pa G}{\pa z}=\sum_{n=0}^{\infty}\frac{1}{n!}\frac{\dd}{\dd z}H_n(z)s^n.
    \end{equation}
    Hence,
    \begin{equation}
        \boxed{\frac{\dd}{\dd{z}}H_n=2nH_{n-1}.}
    \end{equation}
\section{Rodrigues Formula}
\begin{equation}
        e^{-(s-z)^2}=\sum_{n=0}^{\infty}\frac{H_n(z)e^{-z^2}}{n!}s^n.
    \end{equation}
    \begin{equation}
        H_n(z)e^{-z^2}=\left.\frac{\dd^n}{\dd{s}^n}e^{-(s-z)^2}\right|_{s=0}.
    \end{equation}
    $\dd{s}=-\dd{(z-s)}$, hence
    \begin{equation}
        \boxed{H_n(z)=(-)^ne^{z^2}\frac{\dd ^n}{\dd{z}^n}e^{-z^2}.}
    \end{equation}
\section{Parity Property}
\begin{equation}
    H_n(-x)=(-)^nH_n(x).
\end{equation}
\section{Special Values}
\begin{equation}
    H_{2m}(0)=(-)^m\frac{(2m)!}{m!}.
\end{equation}
\begin{equation}
    H_{2m+1}(0)=0.
\end{equation}
\section{Orthogonality Relation}
\begin{equation}
    \int_{-\infty}^{\infty} H_m(x) H_n(x) e^{-x^2} dx = \sqrt{\pi} 2^n n! \delta_{mn}
\end{equation}
\textbf{Proof.}
\begin{equation}
        \int_{-\infty}^{+\infty}G_1(s,z)G_2(t,z)\dd{z}=e^{-(z-(s+t))^2}e^{2st}=\Gamma(\frac{1}{2})e^{2st}.
    \end{equation}
    Hence,
    \begin{equation}
        \int_{-\infty}^{+\infty}G_1(s,z)G_2(t,z)\dd{z}=\sqrt{\pi}e^{2st}.
    \end{equation}
    \begin{equation}
        G_1(s,z)G_2(t,z)=\sum_{(n,m)\in \mathbb{N}^2}\frac{1}{n!m!}H_nH_ms^nt^m.
    \end{equation}
    \begin{equation}
        e^{2st}=\sum_{n=0}^{+\infty}\frac{(2st)^n}{n!}.
    \end{equation}
    Therefore,
    \begin{equation}
        \boxed{\int_{-\infty}^{+\infty}H_n(z)H_m(z)e^{-z^2}\dd{z}=\delta_{nm}2^nn!\sqrt{\pi}.}
    \end{equation}
\section{The First Several Hermite Polynomials}
\begin{eqnarray*}
    &H_0(x)=1\\
    &H_1(x)=2x\\
    &H_2(x)=4x^2-2\\
    &H_3(x)=8x^3-12x\\
    &H_4(x)=16x^4-48x^2+12\\
\end{eqnarray*}
\end{document}