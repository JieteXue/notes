\documentclass{article}
\usepackage{amsmath}
\usepackage{physics}
\usepackage{amsthm}
\usepackage{amssymb}
\newtheoremstyle{1}{}{}{}{}{\bfseries}{.}{\newline}{}
\newtheoremstyle{2}{0}{0}{}{}{\texttt}{.}{\newline}{}
\theoremstyle{1}
\newtheorem{definition}{Definition}
\newtheorem{property}{Property}
\theoremstyle{2}
\newtheorem*{proof_env}{Proof}
\newcommand{\pa}{\partial}
\newcommand{\ee}{\mathrm{e}}
\newcommand{\ii}{\mathrm{i}}
\begin{document}
\begin{center}
    \huge{\textbf{Spherical Functions}}
\end{center}

\section{Legendre Polynomial}
\begin{equation}\label{1}
    P_l(x)=\sum_{n=0}^l\frac{1}{\left(n!\right)^2}\frac{\left(l+n\right)!}{\left(l-n\right)!}\left(\frac{x-1}{2}\right)^n.
\end{equation}
Where $l$ is a natural number.
\subsection{Rodrigues Formula:}
\begin{equation}\label{2}
    P_l\left(x\right)=\frac{1}{2^ll!}\frac{\dd^l}{\dd{x^l}}\left(x^2-1\right)^l.
\end{equation}
\subsection{Differential Equation:}
\begin{equation}
    \frac{\dd}{\dd{x}}\left[\left(1-x^2\right)\frac{\dd{P_l(x)}}{\dd{x}}\right]+l(l+1)P_l(x)=0.
\end{equation}
We can get 
\begin{equation}
    P_l(1)=1
\end{equation}
via \eqref{1} and obtain 
\begin{equation}
    P_l(-x)=(-)^lP_l(x)
\end{equation}
from \eqref{2}.

\subsection{Explicit Expression:}
\begin{equation}
    P_l(x)=\sum_{k=0}^{\left[l/2\right]}(-)^k\frac{\left(2l-2k\right)!}{2^lk!\left(l-k\right)!\left(l-2k\right)!}x^{l-2k}.
\end{equation}
So,
\begin{equation}
    P_{2l}(0)=(-)^l\frac{(2l)!}{\left(2^ll!\right)^2},\quad P_{2l+1}(0)=0.
\end{equation}
\subsection{Orthogonal Completeness}
\begin{equation}
    \int_{-1}^{1}P_k(x)P_l(x)\dd{x}=\frac{2}{2l+1}\delta_{kl}.
\end{equation}
\subsection{Generating Function}
\begin{equation}\label{9}
    \frac{1}{\sqrt{1-2xt+t^2}}=\sum_{l=0}^{\infty}P_l(x)t^l,\quad \left|t\right|<\min\left|x\pm\sqrt{x^2-1}\right|.
\end{equation}
\subsection{Recursive Relation}
Use \eqref{9} to get
\begin{equation}
    (2l+1)xP_l(x)=(l+1)P_{l+1}(x)+lP_{l-1}(x),
\end{equation}
and 
\begin{equation}
    P_{l}(x)=P_{l+1}'(x)-2xP_l'(x)+P_{l-1}(x),
\end{equation}
\begin{equation}
    P_{l+1}'(x)=xP_l'(x)+\left(l+1\right)P_l(x),
\end{equation}


\section{Associated Legendre Function}
\subsection{Differential Equation}
\begin{equation}
    \frac{\dd}{\dd{x}}\left[(1-x^2)\frac{\dd}{\dd{x}}P_l^m(x)\right]+\left[l(l+1)-\frac{m^2}{1-x^2}\right]P_l^m(x)=0,
\end{equation}
where, $m\le l$ are natural numbers. We have the solution
\begin{equation}
    P_l^m(x)=(-)^m(1-x^2)^\frac{m}{2}\frac{\dd^m}{\dd{x}}P_l(x),
\end{equation}
\begin{equation}
    P_l^{-m}(x)=(-)^m\frac{(l-m)!}{(l+m)!}P_l^m(x).
\end{equation}
\subsection{Orthogonal Relation}
\begin{equation}
    \int_{-1}^{1}P_l^m(x)P_k^m(x)\dd{x}=\frac{(l+m)!}{(l-m)!}\frac{2}{2l+1}\delta_{lk},
\end{equation}
\section{Spherical Harmonics}
Special Harmonic functions are the normalized associated Legendre functions.
\begin{equation}
    Y_l^m\left(\theta,\phi\right)=\sqrt{\frac{2l+1}{4\pi}\frac{\left(l-m\right)!}{\left(l+m\right)!}}P_l^m\left(\cos \theta\right) \ee^{\ii m\phi}.
\end{equation}
\subsection{Orthogonal Relation}
\begin{equation}
    \int_{0}^{2\pi}\dd{\phi}\int_{0}^{\pi}Y_{l'}^{m'}\left(\theta,\phi\right){Y^m_l}^*\left(\theta,\phi\right)\sin\theta\dd{\theta}=\delta_{ll'}\delta_{mm'},
\end{equation}
\subsection{Parity}
Under the transformation of $\mathbf{r}\rightarrow -\mathbf{r}$, 
\begin{equation}
    Y_l^m\left(\pi-\theta,\phi+\pi\right)=\left(-1\right)^mY_l^m\left(\theta,\phi\right).
\end{equation}
\subsection{Addition Theorem}
\begin{equation}
    P_l\left(\hat{\mathbf{n}}\cdot\hat{\mathbf{n}}'\right)=\frac{4\pi}{2l+1}\sum_{m=-l}^{l}{Y_l^m}^*\left(\hat{\mathbf{n}}\right)Y_l^m\left(\hat{\mathbf{n}}'\right).
\end{equation}
\end{document}